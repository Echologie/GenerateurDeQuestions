\documentclass[oneside,twoside]{book}
\usepackage[T1]{fontenc}
\usepackage[utf8]{inputenc}
\usepackage{geometry}
\geometry{verbose,tmargin=2cm,bmargin=2cm,lmargin=2cm,rmargin=2cm}
\setcounter{secnumdepth}{3}
\setcounter{tocdepth}{3}
\usepackage{amsmath}
\usepackage{amssymb}
\usepackage{stmaryrd}
\PassOptionsToPackage{version=3}{mhchem}
\usepackage{mhchem}
\usepackage{xcolor}

\makeatletter
%%%%%%%%%%%%%%%%%%%%%%%%%%%%%% User specified LaTeX commands.
\usepackage{amsfonts}
\usepackage{bbold}

\raggedbottom

\AtBeginDocument{
  \def\labelitemi{\(\triangleright\)}
  \def\labelitemii{\(\triangleright\)}
  \def\labelitemiii{\(\triangleright\)}
  \def\labelitemiv{\(\triangleright\)}
}

\makeatother

\begin{document}

\chapter*{Contrôle de Mathématiques}

%\let\MauvaiseReponse\null
%\let\BonneReponse\null
\def\MauvaiseReponse#1\par{\textcolor{red}{#1}}
\def\BonneReponse#1\par{\textcolor{green}{#1}}


{\bf Merci de répondre sur la grille fournie et de rendre le sujet avec la grille.}
\vspace{2em}


\begin{enumerate}


\item L'ensemble des solutions de l'inéquation $-4 x+8<9$ est :

\begin{enumerate}

\item\MauvaiseReponse $\mathbb{R}$


\item\MauvaiseReponse $]-\infty, \frac{1}{4}]$

\item\BonneReponse $]-\frac{1}{4},+\infty[$

\end{enumerate}






\item L'ensemble des solutions de l'inéquation $-5x+4 \geqslant 3$

\begin{enumerate}

\item\MauvaiseReponse $\mathbb{R}$

\item\BonneReponse $]-\infty,\frac{1}{5}]$

\item\MauvaiseReponse $[\frac{1}{5},+\infty[$


\end{enumerate}





\item Le nombre $\frac{1}{4}$

\begin{enumerate}

\item\MauvaiseReponse est solution de l'équation $4x+1=0$

\item\BonneReponse est solution de l'inéquation $4x+5>0$

\item\MauvaiseReponse est solution de l'équation $x+4=0$


\end{enumerate}



\item Le nombre $\sqrt{7}$

\begin{enumerate}

\item\BonneReponse est solution de l'équation $x^{3}-7x=0$

\item\MauvaiseReponse est solution de l'équation $x^{2}+7=0$


\item\MauvaiseReponse est solution de l'inéquation $-5x+3>0$

\end{enumerate}



\item Le nombre $\frac{1}{10}$

\begin{enumerate}

\item\MauvaiseReponse est solution de l'équation $9x+1=0$


\item\MauvaiseReponse est solution de l'inéquation $2x+3<0$

\item\BonneReponse est solution de l'équation $10x-1=0$

\end{enumerate}

\newpage

\item Le couple solution du système $\left\{\begin{array}{c}3x+5y=-24 \\ x-6y=15\end{array}\right.$ est

\begin{enumerate}

\item\MauvaiseReponse $(6;-3)$

\item\BonneReponse $(-3 ;-3)$

\item\MauvaiseReponse $(-\frac{3}{2} ; -3)$


\end{enumerate}




\item L'ensemble des solutions de l'équation $x^{2}+3 x-40=0$ est :

\begin{enumerate}

\item\MauvaiseReponse $\emptyset$

\item\BonneReponse $\{5 ;-8\}$

\item\MauvaiseReponse $\{5 ; 8\}$

\end{enumerate}



\item $\frac{6}{7} - \frac{5}{4} \times2=$

\begin{enumerate}

\item\BonneReponse $-\frac{23}{14}$

\item\MauvaiseReponse $-\frac{11}{14}$

\item\MauvaiseReponse $-\frac{11}{28}$


\end{enumerate}



\item $\displaystyle\frac{\frac{13}{2}-1}{\frac{1}{8}+1}=$

\begin{enumerate}


\item\MauvaiseReponse $52$

\item\BonneReponse $\frac{44}{9}$

\item\MauvaiseReponse $\frac{99}{16}$

\end{enumerate}




\item Dans un triangle $ABC$ rectangle en $A$, si $AB=4$ et $BC=20$, alors

\begin{enumerate}

\item\MauvaiseReponse $\sin (\hat{B})=\frac{1}{5}$


\item\MauvaiseReponse $\cos (\hat{C})=\frac{1}{5}$

\item\BonneReponse $AC=\sqrt{384}$

\end{enumerate}



\item Dans un triangle $ABC$ rectangle en $A$, si $AB=6$ et $BC=10$, alors

\begin{enumerate}

\item\BonneReponse $\cos (\hat{B})=\frac{3}{5}$

\item\MauvaiseReponse $\tan (\hat{C})=\frac{3}{5}$

\item\MauvaiseReponse $AC=\sqrt{136}$


\end{enumerate}



\item Dans un triangle $ABC$ rectangle en $A$, si $AB=4\mathrm{cm}$ et $AC=35\mathrm{mm}$, alors $\hat{B}=$ :

\begin{enumerate}

\item\MauvaiseReponse $40\times\arctan\left(35\right)$


\item\MauvaiseReponse $\frac{\arctan\left(\frac{7}{2}\right)}{4}$

\item\BonneReponse $\arctan\left(\frac{35}{40}\right)$

\end{enumerate}



\item Dans un triangle $ABC$ rectangle en $B$, si $\widehat{A}=30^{\circ}$ alors

\begin{enumerate}

\item\MauvaiseReponse $\widehat{C}=70^{\circ}$

\item\BonneReponse $\widehat{C}=60^{\circ}$

\item\MauvaiseReponse $\widehat{C}=75^{\circ}$


\end{enumerate}




\item Dans un triangle $ABC$ rectangle en $B$, si $\widehat{A}=75^{\circ}$ alors

\begin{enumerate}

\item\MauvaiseReponse $\widehat{C}=25^{\circ}$

\item\MauvaiseReponse $\widehat{C}=30^{\circ}$

\item\BonneReponse $\widehat{C}=15^{\circ}$

\end{enumerate}



\item Si $ABC$ est un triangle rectangle en $B$ tel que $AB=63\mathrm{cm}$ et $BC=16\mathrm{cm}$, alors le segment $\left[AC\right]$ mesure :

\begin{enumerate}

\item\BonneReponse $65\mathrm{cm}$

\item\MauvaiseReponse $79\mathrm{cm}$

\item\MauvaiseReponse $47\mathrm{cm}$

\end{enumerate}



\item Si $ABC$ est un triangle rectangle en $B$ tel que $AB=19\mathrm{cm}$ et $AC=181\mathrm{cm}$, alors le segment $\left[BC\right]$ mesure :

\begin{enumerate}


\item\MauvaiseReponse $98\mathrm{cm}$

\item\MauvaiseReponse $200\mathrm{cm}$

\item\BonneReponse $180\mathrm{cm}$

\end{enumerate}



\item Dans quel cas le triangle $ABC$ est-il rectangle ?

\begin{enumerate}


\item\MauvaiseReponse $AB=99\mathrm{cm}$, $AC=119\mathrm{cm}$ et $BC=20\mathrm{cm}$

\item\BonneReponse $AB=99\mathrm{cm}$, $AC=101\mathrm{cm}$ et $BC=20\mathrm{cm}$

\item\MauvaiseReponse $AB=99\mathrm{cm}$, $AC=79\mathrm{cm}$ et $BC=20\mathrm{cm}$

\end{enumerate}



\item On considère deux triangles non plat $ABC$ et $A^\prime B^\prime C^\prime$ tels que $\left(AB\right)//\left(A^{\prime}B^{\prime}\right)$, $\left(AC\right)//\left(A^{\prime}C^{\prime}\right)$ et $\left(CB\right)//\left(C^{\prime}B^{\prime}\right)$. Si on a $AB=30\mathrm{cm}$, $AC=12\mathrm{cm}$ et $A^{\prime}B^{\prime}=45\mathrm{mm}$, alors $A^{\prime}C^{\prime}=$

\begin{enumerate}


\item\MauvaiseReponse $20\mathrm{cm}$

\item\MauvaiseReponse $8\mathrm{cm}$

\item\BonneReponse $18\mathrm{mm}$

\end{enumerate}




\item On considère deux triangles non plat $ABC$ et $A^\prime B^\prime C^\prime$ tels que $\left(AB\right)//\left(A^{\prime}B^{\prime}\right)$, $\left(AC\right)//\left(A^{\prime}C^{\prime}\right)$ et $\left(CB\right)//\left(C^{\prime}B^{\prime}\right)$. Si on a $AB=15\mathrm{mm}$, $AC=20\mathrm{mm}$ et $A^{\prime}B^{\prime}=6\mathrm{cm}$, alors $A^{\prime}C^{\prime}=$

\begin{enumerate}


\item\MauvaiseReponse $50\mathrm{mm}$

\item\BonneReponse $80\mathrm{mm}$

\item\MauvaiseReponse $50\mathrm{cm}$

\end{enumerate}





\item On considère deux triangles non plat $ABC$ et $A^\prime B^\prime C^\prime$ tels que $\left(AB\right)//\left(A^{\prime}B^{\prime}\right)$ et $\left(CB\right)//\left(C^{\prime}B^{\prime}\right)$. On a $\left(AC\right)//\left(A^{\prime}C^{\prime}\right)$ si on a :

\begin{enumerate}


\item\MauvaiseReponse $AB=21\mathrm{m}$, $AC=42\mathrm{m}$, $A^{\prime}B^{\prime}=49\mathrm{cm}$ et $A^{\prime}C^{\prime}=18\mathrm{cm}$

\item\MauvaiseReponse $AB=21\mathrm{m}$, $AC=42\mathrm{m}$, $A^{\prime}B^{\prime}=49\mathrm{cm}$ et $A^{\prime}C^{\prime}=9\mathrm{cm}$

\item\BonneReponse $AB=21\mathrm{m}$, $AC=42\mathrm{m}$, $A^{\prime}B^{\prime}=49\mathrm{cm}$ et $A^{\prime}C^{\prime}=98\mathrm{cm}$

\end{enumerate}


\end{enumerate}

\end{document}




