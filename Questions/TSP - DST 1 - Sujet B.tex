\documentclass[oneside,twoside]{book}
\usepackage[T1]{fontenc}
\usepackage[utf8]{inputenc}
\usepackage{geometry}
\geometry{verbose,tmargin=2cm,bmargin=2cm,lmargin=2cm,rmargin=2cm}
\setcounter{secnumdepth}{3}
\setcounter{tocdepth}{3}
\usepackage{amsmath}
\usepackage{amssymb}
\usepackage{stmaryrd}
\PassOptionsToPackage{version=3}{mhchem}
\usepackage{mhchem}

\makeatletter
%%%%%%%%%%%%%%%%%%%%%%%%%%%%%% User specified LaTeX commands.
\usepackage{amsfonts}
\usepackage{bbold}

\raggedbottom

\AtBeginDocument{
  \def\labelitemi{\(\triangleright\)}
  \def\labelitemii{\(\triangleright\)}
  \def\labelitemiii{\(\triangleright\)}
  \def\labelitemiv{\(\triangleright\)}
}

\makeatother

\begin{document}

\chapter*{Contrôle de Mathématiques}

\let\MauvaiseReponse\null
\let\BonneReponse\null

{\bf Merci de répondre sur la grille fournie et de rendre le sujet avec la grille.}
\vspace{2em}
\begin{enumerate}




\item L'ensemble des solutions de l'inéquation $-3 x+2<5$ est :

\begin{enumerate}

\item\MauvaiseReponse $\mathbb{R}$

\item\BonneReponse $]-1,+\infty[$

\item\MauvaiseReponse $]-\infty, 1]$

\end{enumerate}

\item L'ensemble des solutions de l'inéquation $-3x+8 \geqslant 5$

\begin{enumerate}

\item\MauvaiseReponse $\mathbb{R}$

\item\MauvaiseReponse $[1,+\infty[$

\item\BonneReponse $]-\infty,1]$

\end{enumerate}





\item Le nombre $\frac{1}{6}$

\begin{enumerate}

\item\MauvaiseReponse est solution de l'équation $6x+1=0$

\item\MauvaiseReponse est solution de l'équation $x+6=0$

\item\BonneReponse est solution de l'inéquation $3x+3>0$

\end{enumerate}



\item Le nombre $\sqrt{6}$

\begin{enumerate}

\item\MauvaiseReponse est solution de l'équation $x^{2}+6=0$

\item\BonneReponse est solution de l'équation $x^{3}-6x=0$

\item\MauvaiseReponse est solution de l'inéquation $-6x+2>0$

\end{enumerate}

\newpage

\item Le nombre $\frac{1}{8}$

\begin{enumerate}

\item\BonneReponse est solution de l'équation $8x-1=0$
\item\MauvaiseReponse est solution de l'équation $7x+1=0$


\item\MauvaiseReponse est solution de l'inéquation $8x+3<0$

\end{enumerate}



\item Le couple solution du système $\left\{\begin{array}{c}5x+2y=-7 \\ x-8y=-35\end{array}\right.$ est

\begin{enumerate}

\item\MauvaiseReponse $(6;4)$

\item\BonneReponse $(-3 ;4)$
\item\MauvaiseReponse $(2 ; -3)$


\end{enumerate}



\item L'ensemble des solutions de l'équation $x^{2}+2 x-24=0$ est :

\begin{enumerate}

\item\BonneReponse $\{4 ;-6\}$
\item\MauvaiseReponse $\emptyset$


\item\MauvaiseReponse $\{4 ; 6\}$

\end{enumerate}



\item $\frac{12}{5} - \frac{5}{4} \times2=$

\begin{enumerate}

\item\MauvaiseReponse $\frac{23}{10}$
\item\BonneReponse $-\frac{1}{10}$

\item\MauvaiseReponse $\frac{23}{20}$


\end{enumerate}



\item $\displaystyle\frac{\frac{9}{5}-1}{\frac{1}{8}+1}=$

\begin{enumerate}


\item\MauvaiseReponse $\frac{72}{5}$
\item\BonneReponse $\frac{32}{45}$

\item\MauvaiseReponse $\frac{9}{10}$

\end{enumerate}



\item Dans un triangle $ABC$ rectangle en $A$, si $AB=4$ et $BC=10$, alors

\begin{enumerate}

\item\MauvaiseReponse $\sin (\hat{B})=\frac{2}{5}$


\item\MauvaiseReponse $\cos (\hat{C})=\frac{2}{5}$
\item\BonneReponse $AC=\sqrt{84}$

\end{enumerate}



\item Dans un triangle $ABC$ rectangle en $A$, si $AB=4$ et $BC=14$, alors

\begin{enumerate}

\item\MauvaiseReponse $\tan (\hat{C})=\frac{2}{7}$

\item\BonneReponse $\cos (\hat{B})=\frac{2}{7}$
\item\MauvaiseReponse $AC=\sqrt{212}$


\end{enumerate}



\item Dans un triangle $ABC$ rectangle en $A$, si $AB=2\mathrm{cm}$ et $AC=35\mathrm{mm}$, alors $\hat{B}=$ :

\begin{enumerate}

\item\BonneReponse $\arctan\left(\frac{35}{20}\right)$
\item\MauvaiseReponse $20\times\arctan\left(35\right)$


\item\MauvaiseReponse $\frac{\arctan\left(\frac{7}{2}\right)}{2}$

\end{enumerate}



\item Dans un triangle $ABC$ rectangle en $B$, si $\widehat{A}=20^{\circ}$ alors

\begin{enumerate}

\item\MauvaiseReponse $\widehat{C}=80^{\circ}$

\item\MauvaiseReponse $\widehat{C}=65^{\circ}$

\item\BonneReponse $\widehat{C}=70^{\circ}$

\end{enumerate}



\item Dans un triangle $ABC$ rectangle en $B$, si $\widehat{A}=60^{\circ}$ alors

\begin{enumerate}

\item\MauvaiseReponse $\widehat{C}=40^{\circ}$

\item\BonneReponse $\widehat{C}=30^{\circ}$
\item\MauvaiseReponse $\widehat{C}=15^{\circ}$


\end{enumerate}



\item Si $ABC$ est un triangle rectangle en $B$ tel que $AB=35\mathrm{cm}$ et $BC=12\mathrm{cm}$, alors le segment $\left[AC\right]$ mesure :

\begin{enumerate}


\item\MauvaiseReponse $47\mathrm{cm}$

\item\BonneReponse $37\mathrm{cm}$
\item\MauvaiseReponse $23\mathrm{cm}$

\end{enumerate}



\item Si $ABC$ est un triangle rectangle en $B$ tel que $AB=13\mathrm{cm}$ et $AC=85\mathrm{cm}$, alors le segment $\left[BC\right]$ mesure :

\begin{enumerate}


\item\MauvaiseReponse $47\mathrm{cm}$

\item\MauvaiseReponse $98\mathrm{cm}$
\item\BonneReponse $84\mathrm{cm}$

\end{enumerate}

\newpage

\item Dans quel cas le triangle $ABC$ est-il rectangle ?

\begin{enumerate}

\item\BonneReponse $AB=48\mathrm{cm}$, $AC=50\mathrm{cm}$ et $BC=14\mathrm{cm}$

\item\MauvaiseReponse $AB=48\mathrm{cm}$, $AC=62\mathrm{cm}$ et $BC=14\mathrm{cm}$

\item\MauvaiseReponse $AB=48\mathrm{cm}$, $AC=34\mathrm{cm}$ et $BC=14\mathrm{cm}$

\end{enumerate}



\item On considère deux triangles non plat $ABC$ et $A^\prime B^\prime C^\prime$ tels que $\left(AB\right)//\left(A^{\prime}B^{\prime}\right)$, $\left(AC\right)//\left(A^{\prime}C^{\prime}\right)$ et $\left(CB\right)//\left(C^{\prime}B^{\prime}\right)$. Si on a $AB=6\mathrm{cm}$, $AC=36\mathrm{cm}$ et $A^{\prime}B^{\prime}=9\mathrm{mm}$, alors $A^{\prime}C^{\prime}=$

\begin{enumerate}

\item\BonneReponse $54\mathrm{mm}$

\item\MauvaiseReponse $4\mathrm{cm}$

\item\MauvaiseReponse $24\mathrm{cm}$

\end{enumerate}



\item On considère deux triangles non plat $ABC$ et $A^\prime B^\prime C^\prime$ tels que $\left(AB\right)//\left(A^{\prime}B^{\prime}\right)$, $\left(AC\right)//\left(A^{\prime}C^{\prime}\right)$ et $\left(CB\right)//\left(C^{\prime}B^{\prime}\right)$. Si on a $AB=5\mathrm{mm}$, $AC=30\mathrm{mm}$ et $A^{\prime}B^{\prime}=\frac{25}{2}\mathrm{cm}$, alors $A^{\prime}C^{\prime}=$

\begin{enumerate}


\item\MauvaiseReponse $12\mathrm{mm}$
\item\BonneReponse $750\mathrm{mm}$

\item\MauvaiseReponse $12\mathrm{cm}$

\end{enumerate}





\item On considère deux triangles non plat $ABC$ et $A^\prime B^\prime C^\prime$ tels que $\left(AB\right)//\left(A^{\prime}B^{\prime}\right)$ et $\left(CB\right)//\left(C^{\prime}B^{\prime}\right)$. On a $\left(AC\right)//\left(A^{\prime}C^{\prime}\right)$ si on a :

\begin{enumerate}


\item\MauvaiseReponse $AB=21\mathrm{m}$, $AC=84\mathrm{m}$, $A^{\prime}B^{\prime}=49\mathrm{cm}$ et $A^{\prime}C^{\prime}=36\mathrm{cm}$
\item\BonneReponse $AB=21\mathrm{m}$, $AC=84\mathrm{m}$, $A^{\prime}B^{\prime}=49\mathrm{cm}$ et $A^{\prime}C^{\prime}=196\mathrm{cm}$

\item\MauvaiseReponse $AB=21\mathrm{m}$, $AC=84\mathrm{m}$, $A^{\prime}B^{\prime}=49\mathrm{cm}$ et $A^{\prime}C^{\prime}=9\mathrm{cm}$

\end{enumerate}



\end{enumerate}

\end{document}
