\documentclass[oneside,twoside]{book}
\usepackage[T1]{fontenc}
\usepackage[utf8]{inputenc}
\usepackage{geometry}
\geometry{verbose,tmargin=2cm,bmargin=2cm,lmargin=2cm,rmargin=2cm}
\setcounter{secnumdepth}{3}
\setcounter{tocdepth}{3}
\usepackage{amsmath}
\usepackage{amssymb}
\usepackage{stmaryrd}
\PassOptionsToPackage{version=3}{mhchem}
\usepackage{mhchem}

\makeatletter
%%%%%%%%%%%%%%%%%%%%%%%%%%%%%% User specified LaTeX commands.
\usepackage{amsfonts}
\usepackage{bbold}

\raggedbottom

\AtBeginDocument{
  \def\labelitemi{\(\triangleright\)}
  \def\labelitemii{\(\triangleright\)}
  \def\labelitemiii{\(\triangleright\)}
  \def\labelitemiv{\(\triangleright\)}
}

\makeatother

\begin{document}

\chapter*{Contrôle de Mathématiques}

\let\MauvaiseReponse\null
\let\BonneReponse\null

{\bf Merci de répondre sur la grille fournie et de rendre le sujet avec la grille.}
\vspace{2em}

\begin{enumerate}





\item L'ensemble des solutions de l'inéquation $-2 x+2<1$ est :
\begin{enumerate}
\item\BonneReponse $]\frac{1}{2},+\infty[$
\item\MauvaiseReponse $\mathbb{R}$
\item\MauvaiseReponse $]-\infty, -\frac{1}{2}]$
\end{enumerate}


\item L'ensemble des solutions de l'inéquation $-2x+8 \geqslant 9$

\begin{enumerate}

\item\MauvaiseReponse $\mathbb{R}$


\item\BonneReponse $]-\infty,-\frac{1}{2}]$
\item\MauvaiseReponse $[-\frac{1}{2},+\infty[$

\end{enumerate}





\item Le nombre $\frac{1}{2}$

\begin{enumerate}

\item\MauvaiseReponse est solution de l'équation $x+2=0$
\item\MauvaiseReponse est solution de l'équation $2x+1=0$


\item\BonneReponse est solution de l'inéquation $3x+7>0$

\end{enumerate}



\item Le nombre $\sqrt{5}$

\begin{enumerate}


\item\BonneReponse est solution de l'équation $x^{3}-5x=0$

\item\MauvaiseReponse est solution de l'inéquation $-5x+2>0$
\item\MauvaiseReponse est solution de l'équation $x^{2}+5=0$

\end{enumerate}



\item Le nombre $\frac{1}{8}$

\begin{enumerate}


\item\BonneReponse est solution de l'équation $8x-1=0$
\item\MauvaiseReponse est solution de l'équation $7x+1=0$

\item\MauvaiseReponse est solution de l'inéquation $2x+7<0$

\end{enumerate}


\newpage

\item Le couple solution du système $\left\{\begin{array}{c}5x+5y=10 \\ x-8y=-34\end{array}\right.$ est

\begin{enumerate}

\item\MauvaiseReponse $(4;4)$

\item\BonneReponse $(-2 ;4)$
\item\MauvaiseReponse $(2 ; -2)$


\end{enumerate}



\item L'ensemble des solutions de l'équation $x^{2}+4 x-32=0$ est :

\begin{enumerate}

\item\MauvaiseReponse $\emptyset$

\item\BonneReponse $\{4 ;-8\}$

\item\MauvaiseReponse $\{4 ; 8\}$

\end{enumerate}


















\item $\frac{15}{2} - \frac{3}{8} \times3=$

\begin{enumerate}

\item\MauvaiseReponse $\frac{171}{8}$

\item\MauvaiseReponse $\frac{57}{8}$

\item\BonneReponse $\frac{51}{8}$

\end{enumerate}



\item $\displaystyle\frac{\frac{5}{7}-1}{\frac{1}{8}+1}=$

\begin{enumerate}

\item\BonneReponse $-\frac{16}{63}$

\item\MauvaiseReponse $\frac{40}{7}$

\item\MauvaiseReponse $-\frac{9}{28}$

\end{enumerate}



\item Dans un triangle $ABC$ rectangle en $A$, si $AB=2$ et $BC=14$, alors

\begin{enumerate}

\item\MauvaiseReponse $\sin (\hat{B})=\frac{1}{7}$

\item\BonneReponse $AC=\sqrt{192}$

\item\MauvaiseReponse $\cos (\hat{C})=\frac{1}{7}$

\end{enumerate}



\item Dans un triangle $ABC$ rectangle en $A$, si $AB=2$ et $BC=20$, alors

\begin{enumerate}

\item\MauvaiseReponse $\tan (\hat{C})=\frac{1}{10}$

\item\MauvaiseReponse $AC=\sqrt{404}$

\item\BonneReponse $\cos (\hat{B})=\frac{1}{10}$

\end{enumerate}



\item Dans un triangle $ABC$ rectangle en $A$, si $AB=2\mathrm{cm}$ et $AC=20\mathrm{mm}$, alors $\hat{B}=$ :

\begin{enumerate}

\item\MauvaiseReponse $20\times\arctan\left(20\right)$

\item\BonneReponse $\arctan\left(\frac{20}{20}\right)$

\item\MauvaiseReponse $\frac{\arctan\left(2\right)}{2}$

\end{enumerate}


\item Dans un triangle $ABC$ rectangle en $B$, si $\widehat{A}=10^{\circ}$ alors

\begin{enumerate}

\item\MauvaiseReponse $\widehat{C}=90^{\circ}$

\item\MauvaiseReponse $\widehat{C}=55^{\circ}$

\item\BonneReponse $\widehat{C}=80^{\circ}$

\end{enumerate}


\item Dans un triangle $ABC$ rectangle en $B$, si $\widehat{A}=50^{\circ}$ alors

\begin{enumerate}

\item\MauvaiseReponse $\widehat{C}=50^{\circ}$

\item\MauvaiseReponse $\widehat{C}=5^{\circ}$

\item\BonneReponse $\widehat{C}=40^{\circ}$

\end{enumerate}



\item Si $ABC$ est un triangle rectangle en $B$ tel que $AB=15\mathrm{cm}$ et $BC=8\mathrm{cm}$, alors le segment $\left[AC\right]$ mesure :

\begin{enumerate}

\item\BonneReponse $17\mathrm{cm}$

\item\MauvaiseReponse $23\mathrm{cm}$

\item\MauvaiseReponse $7\mathrm{cm}$

\end{enumerate}



\item Si $ABC$ est un triangle rectangle en $B$ tel que $AB=9\mathrm{cm}$ et $AC=41\mathrm{cm}$, alors le segment $\left[BC\right]$ mesure :

\begin{enumerate}


\item\MauvaiseReponse $23\mathrm{cm}$
\item\BonneReponse $40\mathrm{cm}$

\item\MauvaiseReponse $50\mathrm{cm}$

\end{enumerate}



\item Dans quel cas le triangle $ABC$ est-il rectangle ?

\begin{enumerate}


\item\MauvaiseReponse $AB=24\mathrm{cm}$, $AC=34\mathrm{cm}$ et $BC=10\mathrm{cm}$

\item\MauvaiseReponse $AB=24\mathrm{cm}$, $AC=14\mathrm{cm}$ et $BC=10\mathrm{cm}$
\item\BonneReponse $AB=24\mathrm{cm}$, $AC=26\mathrm{cm}$ et $BC=10\mathrm{cm}$

\end{enumerate}



\item On considère deux triangles non plat $ABC$ et $A^\prime B^\prime C^\prime$ tels que $\left(AB\right)//\left(A^{\prime}B^{\prime}\right)$, $\left(AC\right)//\left(A^{\prime}C^{\prime}\right)$ et $\left(CB\right)//\left(C^{\prime}B^{\prime}\right)$. Si on a $AB=6\mathrm{cm}$, $AC=24\mathrm{cm}$ et $A^{\prime}B^{\prime}=4\mathrm{mm}$, alors $A^{\prime}C^{\prime}=$

\begin{enumerate}


\item\MauvaiseReponse $9\mathrm{cm}$

\item\BonneReponse $16\mathrm{mm}$
\item\MauvaiseReponse $36\mathrm{cm}$

\end{enumerate}



\item On considère deux triangles non plat $ABC$ et $A^\prime B^\prime C^\prime$ tels que $\left(AB\right)//\left(A^{\prime}B^{\prime}\right)$, $\left(AC\right)//\left(A^{\prime}C^{\prime}\right)$ et $\left(CB\right)//\left(C^{\prime}B^{\prime}\right)$. Si on a $AB=5\mathrm{mm}$, $AC=20\mathrm{mm}$ et $A^{\prime}B^{\prime}=2\mathrm{cm}$, alors $A^{\prime}C^{\prime}=$

\begin{enumerate}


\item\MauvaiseReponse $50\mathrm{mm}$
\item\BonneReponse $80\mathrm{mm}$

\item\MauvaiseReponse $50\mathrm{cm}$

\end{enumerate}






\item On considère deux triangles non plat $ABC$ et $A^\prime B^\prime C^\prime$ tels que $\left(AB\right)//\left(A^{\prime}B^{\prime}\right)$ et $\left(CB\right)//\left(C^{\prime}B^{\prime}\right)$. On a $\left(AC\right)//\left(A^{\prime}C^{\prime}\right)$ si on a :

\begin{enumerate}

\item\BonneReponse $AB=63\mathrm{m}$, $AC=42\mathrm{m}$, $A^{\prime}B^{\prime}=147\mathrm{cm}$ et $A^{\prime}C^{\prime}=98\mathrm{cm}$

\item\MauvaiseReponse $AB=63\mathrm{m}$, $AC=42\mathrm{m}$, $A^{\prime}B^{\prime}=147\mathrm{cm}$ et $A^{\prime}C^{\prime}=18\mathrm{cm}$

\item\MauvaiseReponse $AB=63\mathrm{m}$, $AC=42\mathrm{m}$, $A^{\prime}B^{\prime}=147\mathrm{cm}$ et $A^{\prime}C^{\prime}=27\mathrm{cm}$

\end{enumerate}


\end{enumerate}

\end{document}
