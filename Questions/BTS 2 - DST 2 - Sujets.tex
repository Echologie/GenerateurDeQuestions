\documentclass[oneside,twocolumn,landscape]{book}
\usepackage[T1]{fontenc}
\usepackage[utf8]{inputenc}
\usepackage{geometry}
\geometry{verbose,tmargin=1cm,bmargin=1cm,lmargin=2cm,rmargin=2cm}
\setcounter{secnumdepth}{3}
\setcounter{tocdepth}{3}
\usepackage{mathrsfs}
\usepackage{amsmath}
\usepackage{amssymb}
\usepackage{amsfonts}
\usepackage{bbold}
\usepackage{xcolor}
\pagestyle{empty}

\newcounter{NumeroDuSujet}
\setcounter{NumeroDuSujet}{10}

\newenvironment{Sujet}[1][]
  {
    \refstepcounter{NumeroDuSujet}
    \section*{Numéro du sujet :~\theNumeroDuSujet}
    \par #1
  }
  {
    \newpage
  }

\newcounter{NumeroDeLaQuestion}[NumeroDuSujet]

\newenvironment{VraiFaux}[1][]
  {
    \begin{enumerate}
      \setcounter{enumi}{\theNumeroDeLaQuestion}
      #1
  }
  {
    \end{enumerate}
    \stepcounter{NumeroDeLaQuestion}
  }

\newenvironment{QCM}[1][]
  {
    \begin{enumerate}
      \setcounter{enumi}{\theNumeroDeLaQuestion}
      \item #1
  }
  {
    \end{enumerate}
    \stepcounter{NumeroDeLaQuestion}
  }

%\let\Vrai\item
%\let\Faux\item

\def\Vrai#1{\item{\color{green}#1}}
\def\Faux#1{\item{\color{red}#1}}

\begin{document}


\begin{Sujet}

\subsection*{Exercice 1}

Pour chacune des propositions de cet exercice, cochez A si elle est vraie\par et B si elle est fausse.

On s'intéresse à la fonction f définie par $f:x\mapsto\left(3x+3\right)e^{2x}$.

\begin{VraiFaux}

\Vrai{ L'ensemble de définition de f est $\left]-\infty,+\infty\right[$.}

\end{VraiFaux}

\begin{VraiFaux}

\Faux{ La dérivée de f est donnée par $f^{\prime}:x\mapsto\left(6\right)e^{cx}$.}

\end{VraiFaux}

On admet que son approximation quadratique est donné par\par $\qquad f\left(x\right)\approx 3 + 12 x + \frac{45}{2} x^2$.

\begin{VraiFaux}

\Faux{ L'équation de la tangente $T$ à la courbe représentative $\mathscr C$ de $f$ en son point d'abscisse $0$ est donnée par $y=3x + 12 $.}

\end{VraiFaux}

\begin{VraiFaux}

\Faux{ $T$ est au-dessus de $\mathscr C$.}

\end{VraiFaux}

\subsection*{Exercice 2}

Cet exercice doit être rédigé au verso de la grille réponse.

\subsection*{Exercice 3}

\begin{QCM}

Parmi ces fonctions, laquelle est une primitive de la fonction $f$\par définie par $ f(x)=x \cdot \cos (3 x+1)$ ?

\begin{enumerate}

\Faux{ $F:x \mapsto \cos (3 x+1)-3 x \cdot \sin (3 x+1)$}

\Faux{ $G:x \mapsto x \cdot \sin (3 x+1)$}

\Vrai{ $H:x \mapsto \frac{x \cdot \sin (3 x+1)}{3}+\frac{\cos (3 x+1)}{9}$}

\end{enumerate}

\end{QCM}

\begin{QCM}

$\int\limits_{0}^{2}\left(5 t^{3}+3 t^{2}-t+7\right) d t=$

\begin{enumerate}

\Faux{ 71}

\Vrai{ 40}

\Faux{ $15 t^{2}+6 t-1$}

\end{enumerate}

\end{QCM}

\begin{QCM}

La valeur moyenne de la fonction $g: x \rightarrow e^{5 x} $ entre 0 et 2 est de :

\begin{enumerate}

\Faux{environ $4405,1$}

\Vrai{environ $2 202,5$}

\Faux{environ $22 025,5$}

\end{enumerate}

\end{QCM}

\newpage

\subsection*{Problème}

Alain Raffletou cherche à connaitre le nombre moyen de clients journaliers dans une de ses boutiques sur l'année en cours. Pour cela, il modélise le nombre de clients par une fonction donnant le nombre de clients $c\left(t\right)$ en fonction du temps $t$, exprimé en jours. On prends donc $t\in\left[0;365\right]$.

\begin{QCM}

En remarquant que le nombre de clients ne peut être négatif, déterminé, parmi les propositions ci-dessous, laquelle donne la formule de $c\left(t\right)$ en fonction de $t$ :

\begin{enumerate}

\Faux{$c\left(t\right)=3\cos\left(\frac{8\pi t}{365}-30\right)$}

\Vrai{$c\left(t\right)=3\cos\left(\frac{8\pi t}{365}\right)+30$}

\Faux{$c\left(t\right)=3\sin\left(\frac{8\pi t}{365}\right)-30$}

\Faux{$c\left(t\right)=3\sin\left(\frac{8\pi t}{365}+30\right)$}

\end{enumerate}

\end{QCM}

\begin{QCM}

On a alors que les valeurs de $c\left(t\right)$ sont comprises entre :

\begin{enumerate}

\Faux{$0$ et $3$ }

\Faux{$0$ et $30$}

\Vrai{$27$ et $33$}

\Faux{$0$ et $33$ }

\end{enumerate}

\end{QCM}

Dès lors, il peut calculer le nombre moyen de clients en appliquant la formule

$\frac{1}{365}\int\limits _{0}^{365}C\left(t\right)dt$ , où $C$ désigne une primitive de $c$.

\begin{QCM}

Parmi les propositions ci-dessous, laquelle donne une formule possible C~?

\begin{enumerate}

\Faux{$C\left(t\right)=\frac{24}{365}\pi\sin\left(\frac{8\pi t}{365}\right)-30t+25$}

\Faux{$C\left(t\right)=\frac{24}{365}\pi\cos\left(\frac{8\pi t}{365}\right)-30t-51$}

\Faux{$C\left(t\right)=-3\sin\left(\frac{8\pi t}{365}\right)$}

\Vrai{$C\left(t\right)=\frac{1095}{2}\pi\sin\left(\frac{8\pi t}{365}\right)+30t$}

\end{enumerate}

\end{QCM}

\begin{QCM}

On obtient ainsi que le nombre moyen de clients journaliers sur l'année en cours est donné par :

\begin{enumerate}

\Faux{$3\cos\left(8\pi\right)$}

\Vrai{$30$}

\Faux{$30\sin\left(8\pi\right)$ }

\Faux{$3$}

\end{enumerate}

\end{QCM}

\end{Sujet}

\begin{Sujet}

\subsection*{Exercice 1}

Pour chacune des propositions de cet exercice, cochez A si elle est vraie\par et B si elle est fausse.

On s'intéresse à la fonction f définie par $f:x\mapsto\left(3x+4\right)e^{2x}$.

\begin{VraiFaux}

\Faux{ L'ensemble de définition de f est $\left]0,+\infty\right[$.}

\end{VraiFaux}

\begin{VraiFaux}

\Vrai{ La dérivée de f est donnée par $f^{\prime}:x\mapsto\left(6x+11\right)e^{2x}$.}

\end{VraiFaux}

On admet que son approximation quadratique est donné par\par $\qquad f\left(x\right)\approx 4 + 15 x + 27 x^2$.

\begin{VraiFaux}

\Faux{ L'équation de la tangente $T$ à la courbe représentative $\mathscr C$ de $f$ en son point d'abscisse $0$ est donnée par $y=4x + 15 $.}

\end{VraiFaux}

\begin{VraiFaux}

\Faux{ $T$ est au-dessus de $\mathscr C$.}

\end{VraiFaux}

\subsection*{Exercice 2}

Cet exercice doit être rédigé au verso de la grille réponse.

\subsection*{Exercice 3}

\begin{QCM}

Parmi ces fonctions, laquelle est une primitive de la fonction $f$\par définie par $ f(x)=x \cdot \cos (3 x+1)$ ?

\begin{enumerate}

\Faux{ $G:x \mapsto x \cdot \sin (3 x+1)$}

\Vrai{ $H:x \mapsto \frac{x \cdot \sin (3 x+1)}{3}+\frac{\cos (3 x+1)}{9}$}

\Faux{ $F:x \mapsto \cos (3 x+1)-3 x \cdot \sin (3 x+1)$}

\end{enumerate}

\end{QCM}

\begin{QCM}

$\int\limits_{0}^{2}\left(5 t^{3}+3 t^{2}-t+7\right) d t=$

\begin{enumerate}

\Faux{ $15 t^{2}+6 t-1$}

\Faux{ 71}

\Vrai{ 40}

\end{enumerate}

\end{QCM}

\begin{QCM}

La valeur moyenne de la fonction $g: x \rightarrow e^{5 x} $ entre 0 et 2 est de :

\begin{enumerate}

\Faux{environ $4405,1$}

\Vrai{environ $2 202,5$}

\Faux{environ $22 025,5$}

\end{enumerate}

\end{QCM}

\newpage

\subsection*{Problème}

Alain Raffletou cherche à connaitre le nombre moyen de clients journaliers dans une de ses boutiques sur l'année en cours. Pour cela, il modélise le nombre de clients par une fonction donnant le nombre de clients $c\left(t\right)$ en fonction du temps $t$, exprimé en jours. On prends donc $t\in\left[0;365\right]$.

\begin{QCM}

En remarquant que le nombre de clients ne peut être négatif, déterminé, parmi les propositions ci-dessous, laquelle donne la formule de $c\left(t\right)$ en fonction de $t$ :

\begin{enumerate}

\Faux{$c\left(t\right)=5\sin\left(\frac{8\pi t}{365}+22\right)$}

\Faux{$c\left(t\right)=5\cos\left(\frac{8\pi t}{365}-22\right)$}

\Faux{$c\left(t\right)=5\sin\left(\frac{8\pi t}{365}\right)-22$}

\Vrai{$c\left(t\right)=5\cos\left(\frac{8\pi t}{365}\right)+22$}

\end{enumerate}

\end{QCM}

\begin{QCM}

On a alors que les valeurs de $c\left(t\right)$ sont comprises entre :

\begin{enumerate}

\Faux{$0$ et $27$ }

\Faux{$0$ et $5$ }

\Vrai{$17$ et $27$}

\Faux{$0$ et $22$}

\end{enumerate}

\end{QCM}

Dès lors, il peut calculer le nombre moyen de clients en appliquant la formule

$\frac{1}{365}\int\limits _{0}^{365}C\left(t\right)dt$ , où $C$ désigne une primitive de $c$.

\begin{QCM}

Parmi les propositions ci-dessous, laquelle donne une formule possible C~?

\begin{enumerate}

\Vrai{$C\left(t\right)=\frac{1825}{2}\pi\sin\left(\frac{8\pi t}{365}\right)+22t$}

\Faux{$C\left(t\right)=\frac{40}{365}\pi\sin\left(\frac{8\pi t}{365}\right)-22t+13$}

\Faux{$C\left(t\right)=\frac{40}{365}\pi\cos\left(\frac{8\pi t}{365}\right)-22t-29$}

\Faux{$C\left(t\right)=-5\sin\left(\frac{8\pi t}{365}\right)$}

\end{enumerate}

\end{QCM}

\begin{QCM}

On obtient ainsi que le nombre moyen de clients journaliers sur l'année en cours est donné par :

\begin{enumerate}

\Faux{$22\sin\left(8\pi\right)$ }

\Faux{$5$}

\Faux{$5\cos\left(8\pi\right)$}

\Vrai{$22$}

\end{enumerate}

\end{QCM}

\end{Sujet}

\begin{Sujet}

\subsection*{Exercice 1}

Pour chacune des propositions de cet exercice, cochez A si elle est vraie\par et B si elle est fausse.

On s'intéresse à la fonction f définie par $f:x\mapsto\left(3x+7\right)e^{6x}$.

\begin{VraiFaux}

\Vrai{ L'ensemble de définition de f est $\left]-\infty,+\infty\right[$.}

\end{VraiFaux}

\begin{VraiFaux}

\Faux{ La dérivée de f est donnée par $f^{\prime}:x\mapsto\left(18\right)e^{cx}$.}

\end{VraiFaux}

On admet que son approximation quadratique est donné par\par $\qquad f\left(x\right)\approx 7 + 24 x + \frac{81}{2} x^2$.

\begin{VraiFaux}

\Vrai{ L'équation de la tangente $T$ à la courbe représentative $\mathscr C$ de $f$ en son point d'abscisse $0$ est donnée par $y=24 x +7 $.}

\end{VraiFaux}

\begin{VraiFaux}

\Vrai{ $T$ est au-dessous de $\mathscr C$.}

\end{VraiFaux}

\subsection*{Exercice 2}

Cet exercice doit être rédigé au verso de la grille réponse.

\subsection*{Exercice 3}

\begin{QCM}

Parmi ces fonctions, laquelle est une primitive de la fonction $f$\par définie par $ f(x)=x \cdot \cos (3 x+1)$ ?

\begin{enumerate}

\Faux{ $F:x \mapsto \cos (3 x+1)-3 x \cdot \sin (3 x+1)$}

\Faux{ $G:x \mapsto x \cdot \sin (3 x+1)$}

\Vrai{ $H:x \mapsto \frac{x \cdot \sin (3 x+1)}{3}+\frac{\cos (3 x+1)}{9}$}

\end{enumerate}

\end{QCM}

\begin{QCM}

$\int\limits_{0}^{2}\left(5 t^{3}+3 t^{2}-t+7\right) d t=$

\begin{enumerate}

\Faux{ 71}

\Faux{ $15 t^{2}+6 t-1$}

\Vrai{ 40}

\end{enumerate}

\end{QCM}

\begin{QCM}

La valeur moyenne de la fonction $g: x \rightarrow e^{5 x} $ entre 0 et 2 est de :

\begin{enumerate}

\Faux{environ $22 025,5$}

\Faux{environ $4405,1$}

\Vrai{environ $2 202,5$}

\end{enumerate}

\end{QCM}

\newpage

\subsection*{Problème}

Alain Raffletou cherche à connaitre le nombre moyen de clients journaliers dans une de ses boutiques sur l'année en cours. Pour cela, il modélise le nombre de clients par une fonction donnant le nombre de clients $c\left(t\right)$ en fonction du temps $t$, exprimé en jours. On prends donc $t\in\left[0;365\right]$.

\begin{QCM}

En remarquant que le nombre de clients ne peut être négatif, déterminé, parmi les propositions ci-dessous, laquelle donne la formule de $c\left(t\right)$ en fonction de $t$ :

\begin{enumerate}

\Faux{$c\left(t\right)=4\sin\left(\frac{4\pi t}{365}+16\right)$}

\Vrai{$c\left(t\right)=4\cos\left(\frac{4\pi t}{365}\right)+16$}

\Faux{$c\left(t\right)=4\sin\left(\frac{4\pi t}{365}\right)-16$}

\Faux{$c\left(t\right)=4\cos\left(\frac{4\pi t}{365}-16\right)$}

\end{enumerate}

\end{QCM}

\begin{QCM}

On a alors que les valeurs de $c\left(t\right)$ sont comprises entre :

\begin{enumerate}

\Vrai{$12$ et $20$}

\Faux{$0$ et $16$}

\Faux{$0$ et $20$ }

\Faux{$0$ et $4$ }

\end{enumerate}

\end{QCM}

Dès lors, il peut calculer le nombre moyen de clients en appliquant la formule

$\frac{1}{365}\int\limits _{0}^{365}C\left(t\right)dt$ , où $C$ désigne une primitive de $c$.

\begin{QCM}

Parmi les propositions ci-dessous, laquelle donne une formule possible C~?

\begin{enumerate}

\Faux{$C\left(t\right)=\frac{16}{365}\pi\sin\left(\frac{4\pi t}{365}\right)-16t+9$}

\Faux{$C\left(t\right)=\frac{16}{365}\pi\cos\left(\frac{4\pi t}{365}\right)-16t-20$}

\Faux{$C\left(t\right)=-4\sin\left(\frac{4\pi t}{365}\right)$}

\Vrai{$C\left(t\right)=\frac{1460}{2}\pi\sin\left(\frac{4\pi t}{365}\right)+16t$}

\end{enumerate}

\end{QCM}

\begin{QCM}

On obtient ainsi que le nombre moyen de clients journaliers sur l'année en cours est donné par :

\begin{enumerate}

\Faux{$4$}

\Faux{$16\sin\left(4\pi\right)$ }

\Faux{$4\cos\left(4\pi\right)$}

\Vrai{$16$}

\end{enumerate}

\end{QCM}

\end{Sujet}

\begin{Sujet}

\subsection*{Exercice 1}

Pour chacune des propositions de cet exercice, cochez A si elle est vraie\par et B si elle est fausse.

On s'intéresse à la fonction f définie par $f:x\mapsto\left(2x+7\right)e^{7x}$.

\begin{VraiFaux}

\Faux{ L'ensemble de définition de f est $\left]0,+\infty\right[$.}

\end{VraiFaux}

\begin{VraiFaux}

\Vrai{ La dérivée de f est donnée par $f^{\prime}:x\mapsto\left(14x+51\right)e^{7x}$.}

\end{VraiFaux}

On admet que son approximation quadratique est donné par\par $\qquad f\left(x\right)\approx 7 + 16 x + 18 x^2$.

\begin{VraiFaux}

\Vrai{ L'équation de la tangente $T$ à la courbe représentative $\mathscr C$ de $f$ en son point d'abscisse $0$ est donnée par $y=16 x +7 $.}

\end{VraiFaux}

\begin{VraiFaux}

\Faux{ $T$ est au-dessus de $\mathscr C$.}

\end{VraiFaux}

\subsection*{Exercice 2}

Cet exercice doit être rédigé au verso de la grille réponse.

\subsection*{Exercice 3}

\begin{QCM}

Parmi ces fonctions, laquelle est une primitive de la fonction $f$\par définie par $ f(x)=x \cdot \cos (3 x+1)$ ?

\begin{enumerate}

\Vrai{ $H:x \mapsto \frac{x \cdot \sin (3 x+1)}{3}+\frac{\cos (3 x+1)}{9}$}

\Faux{ $G:x \mapsto x \cdot \sin (3 x+1)$}

\Faux{ $F:x \mapsto \cos (3 x+1)-3 x \cdot \sin (3 x+1)$}

\end{enumerate}

\end{QCM}

\begin{QCM}

$\int\limits_{0}^{2}\left(5 t^{3}+3 t^{2}-t+7\right) d t=$

\begin{enumerate}

\Vrai{ 40}

\Faux{ 71}

\Faux{ $15 t^{2}+6 t-1$}

\end{enumerate}

\end{QCM}

\begin{QCM}

La valeur moyenne de la fonction $g: x \rightarrow e^{5 x} $ entre 0 et 2 est de :

\begin{enumerate}

\Faux{environ $4405,1$}

\Faux{environ $22 025,5$}

\Vrai{environ $2 202,5$}

\end{enumerate}

\end{QCM}

\newpage

\subsection*{Problème}

Alain Raffletou cherche à connaitre le nombre moyen de clients journaliers dans une de ses boutiques sur l'année en cours. Pour cela, il modélise le nombre de clients par une fonction donnant le nombre de clients $c\left(t\right)$ en fonction du temps $t$, exprimé en jours. On prends donc $t\in\left[0;365\right]$.

\begin{QCM}

En remarquant que le nombre de clients ne peut être négatif, déterminé, parmi les propositions ci-dessous, laquelle donne la formule de $c\left(t\right)$ en fonction de $t$ :

\begin{enumerate}

\Faux{$c\left(t\right)=4\cos\left(\frac{4\pi t}{365}-10\right)$}

\Vrai{$c\left(t\right)=4\cos\left(\frac{4\pi t}{365}\right)+10$}

\Faux{$c\left(t\right)=4\sin\left(\frac{4\pi t}{365}+10\right)$}

\Faux{$c\left(t\right)=4\sin\left(\frac{4\pi t}{365}\right)-10$}

\end{enumerate}

\end{QCM}

\begin{QCM}

On a alors que les valeurs de $c\left(t\right)$ sont comprises entre :

\begin{enumerate}

\Vrai{$6$ et $14$}

\Faux{$0$ et $10$}

\Faux{$0$ et $14$ }

\Faux{$0$ et $4$ }

\end{enumerate}

\end{QCM}

Dès lors, il peut calculer le nombre moyen de clients en appliquant la formule

$\frac{1}{365}\int\limits _{0}^{365}C\left(t\right)dt$ , où $C$ désigne une primitive de $c$.

\begin{QCM}

Parmi les propositions ci-dessous, laquelle donne une formule possible C~?

\begin{enumerate}

\Faux{$C\left(t\right)=\frac{16}{365}\pi\cos\left(\frac{4\pi t}{365}\right)-10t-8$}

\Faux{$C\left(t\right)=-4\sin\left(\frac{4\pi t}{365}\right)$}

\Faux{$C\left(t\right)=\frac{16}{365}\pi\sin\left(\frac{4\pi t}{365}\right)-10t+3$}

\Vrai{$C\left(t\right)=\frac{1460}{2}\pi\sin\left(\frac{4\pi t}{365}\right)+10t$}

\end{enumerate}

\end{QCM}

\begin{QCM}

On obtient ainsi que le nombre moyen de clients journaliers sur l'année en cours est donné par :

\begin{enumerate}

\Faux{$4\cos\left(4\pi\right)$}

\Vrai{$10$}

\Faux{$10\sin\left(4\pi\right)$ }

\Faux{$4$}

\end{enumerate}

\end{QCM}

\end{Sujet}

\begin{Sujet}

\subsection*{Exercice 1}

Pour chacune des propositions de cet exercice, cochez A si elle est vraie\par et B si elle est fausse.

On s'intéresse à la fonction f définie par $f:x\mapsto\left(4x+5\right)e^{5x}$.

\begin{VraiFaux}

\Faux{ L'ensemble de définition de f est $\left]0,+\infty\right[$.}

\end{VraiFaux}

\begin{VraiFaux}

\Vrai{ La dérivée de f est donnée par $f^{\prime}:x\mapsto\left(20x+29\right)e^{5x}$.}

\end{VraiFaux}

On admet que son approximation quadratique est donné par\par $\qquad f\left(x\right)\approx 5 + 24 x + 56 x^2$.

\begin{VraiFaux}

\Vrai{ L'équation de la tangente $T$ à la courbe représentative $\mathscr C$ de $f$ en son point d'abscisse $0$ est donnée par $y=24 x +5 $.}

\end{VraiFaux}

\begin{VraiFaux}

\Vrai{ $T$ est au-dessous de $\mathscr C$.}

\end{VraiFaux}

\subsection*{Exercice 2}

Cet exercice doit être rédigé au verso de la grille réponse.

\subsection*{Exercice 3}

\begin{QCM}

Parmi ces fonctions, laquelle est une primitive de la fonction $f$\par définie par $ f(x)=x \cdot \cos (3 x+1)$ ?

\begin{enumerate}

\Faux{ $F:x \mapsto \cos (3 x+1)-3 x \cdot \sin (3 x+1)$}

\Faux{ $G:x \mapsto x \cdot \sin (3 x+1)$}

\Vrai{ $H:x \mapsto \frac{x \cdot \sin (3 x+1)}{3}+\frac{\cos (3 x+1)}{9}$}

\end{enumerate}

\end{QCM}

\begin{QCM}

$\int\limits_{0}^{2}\left(5 t^{3}+3 t^{2}-t+7\right) d t=$

\begin{enumerate}

\Faux{ 71}

\Faux{ $15 t^{2}+6 t-1$}

\Vrai{ 40}

\end{enumerate}

\end{QCM}

\begin{QCM}

La valeur moyenne de la fonction $g: x \rightarrow e^{5 x} $ entre 0 et 2 est de :

\begin{enumerate}

\Faux{environ $22 025,5$}

\Vrai{environ $2 202,5$}

\Faux{environ $4405,1$}

\end{enumerate}

\end{QCM}

\newpage

\subsection*{Problème}

Alain Raffletou cherche à connaitre le nombre moyen de clients journaliers dans une de ses boutiques sur l'année en cours. Pour cela, il modélise le nombre de clients par une fonction donnant le nombre de clients $c\left(t\right)$ en fonction du temps $t$, exprimé en jours. On prends donc $t\in\left[0;365\right]$.

\begin{QCM}

En remarquant que le nombre de clients ne peut être négatif, déterminé, parmi les propositions ci-dessous, laquelle donne la formule de $c\left(t\right)$ en fonction de $t$ :

\begin{enumerate}

\Vrai{$c\left(t\right)=5\cos\left(\frac{8\pi t}{365}\right)+17$}

\Faux{$c\left(t\right)=5\cos\left(\frac{8\pi t}{365}-17\right)$}

\Faux{$c\left(t\right)=5\sin\left(\frac{8\pi t}{365}\right)-17$}

\Faux{$c\left(t\right)=5\sin\left(\frac{8\pi t}{365}+17\right)$}

\end{enumerate}

\end{QCM}

\begin{QCM}

On a alors que les valeurs de $c\left(t\right)$ sont comprises entre :

\begin{enumerate}

\Vrai{$12$ et $22$}

\Faux{$0$ et $22$ }

\Faux{$0$ et $17$}

\Faux{$0$ et $5$ }

\end{enumerate}

\end{QCM}

Dès lors, il peut calculer le nombre moyen de clients en appliquant la formule

$\frac{1}{365}\int\limits _{0}^{365}C\left(t\right)dt$ , où $C$ désigne une primitive de $c$.

\begin{QCM}

Parmi les propositions ci-dessous, laquelle donne une formule possible C~?

\begin{enumerate}

\Vrai{$C\left(t\right)=\frac{1825}{2}\pi\sin\left(\frac{8\pi t}{365}\right)+17t$}

\Faux{$C\left(t\right)=\frac{40}{365}\pi\cos\left(\frac{8\pi t}{365}\right)-17t-19$}

\Faux{$C\left(t\right)=\frac{40}{365}\pi\sin\left(\frac{8\pi t}{365}\right)-17t+8$}

\Faux{$C\left(t\right)=-5\sin\left(\frac{8\pi t}{365}\right)$}

\end{enumerate}

\end{QCM}

\begin{QCM}

On obtient ainsi que le nombre moyen de clients journaliers sur l'année en cours est donné par :

\begin{enumerate}

\Faux{$5\cos\left(8\pi\right)$}

\Faux{$17\sin\left(8\pi\right)$ }

\Vrai{$17$}

\Faux{$5$}

\end{enumerate}

\end{QCM}

\end{Sujet}

\begin{Sujet}

\subsection*{Exercice 1}

Pour chacune des propositions de cet exercice, cochez A si elle est vraie\par et B si elle est fausse.

On s'intéresse à la fonction f définie par $f:x\mapsto\left(5x+2\right)e^{3x}$.

\begin{VraiFaux}

\Vrai{ L'ensemble de définition de f est $\left]-\infty,+\infty\right[$.}

\end{VraiFaux}

\begin{VraiFaux}

\Faux{ La dérivée de f est donnée par $f^{\prime}:x\mapsto\left(15\right)e^{cx}$.}

\end{VraiFaux}

On admet que son approximation quadratique est donné par\par $\qquad f\left(x\right)\approx 2 + 15 x + 50 x^2$.

\begin{VraiFaux}

\Faux{ L'équation de la tangente $T$ à la courbe représentative $\mathscr C$ de $f$ en son point d'abscisse $0$ est donnée par $y=2x + 15 $.}

\end{VraiFaux}

\begin{VraiFaux}

\Vrai{ $T$ est au-dessous de $\mathscr C$.}

\end{VraiFaux}

\subsection*{Exercice 2}

Cet exercice doit être rédigé au verso de la grille réponse.

\subsection*{Exercice 3}

\begin{QCM}

Parmi ces fonctions, laquelle est une primitive de la fonction $f$\par définie par $ f(x)=x \cdot \cos (3 x+1)$ ?

\begin{enumerate}

\Vrai{ $H:x \mapsto \frac{x \cdot \sin (3 x+1)}{3}+\frac{\cos (3 x+1)}{9}$}

\Faux{ $F:x \mapsto \cos (3 x+1)-3 x \cdot \sin (3 x+1)$}

\Faux{ $G:x \mapsto x \cdot \sin (3 x+1)$}

\end{enumerate}

\end{QCM}

\begin{QCM}

$\int\limits_{0}^{2}\left(5 t^{3}+3 t^{2}-t+7\right) d t=$

\begin{enumerate}

\Faux{ $15 t^{2}+6 t-1$}

\Vrai{ 40}

\Faux{ 71}

\end{enumerate}

\end{QCM}

\begin{QCM}

La valeur moyenne de la fonction $g: x \rightarrow e^{5 x} $ entre 0 et 2 est de :

\begin{enumerate}

\Vrai{environ $2 202,5$}

\Faux{environ $4405,1$}

\Faux{environ $22 025,5$}

\end{enumerate}

\end{QCM}

\newpage

\subsection*{Problème}

Alain Raffletou cherche à connaitre le nombre moyen de clients journaliers dans une de ses boutiques sur l'année en cours. Pour cela, il modélise le nombre de clients par une fonction donnant le nombre de clients $c\left(t\right)$ en fonction du temps $t$, exprimé en jours. On prends donc $t\in\left[0;365\right]$.

\begin{QCM}

En remarquant que le nombre de clients ne peut être négatif, déterminé, parmi les propositions ci-dessous, laquelle donne la formule de $c\left(t\right)$ en fonction de $t$ :

\begin{enumerate}

\Faux{$c\left(t\right)=5\sin\left(\frac{8\pi t}{365}+13\right)$}

\Faux{$c\left(t\right)=5\sin\left(\frac{8\pi t}{365}\right)-13$}

\Vrai{$c\left(t\right)=5\cos\left(\frac{8\pi t}{365}\right)+13$}

\Faux{$c\left(t\right)=5\cos\left(\frac{8\pi t}{365}-13\right)$}

\end{enumerate}

\end{QCM}

\begin{QCM}

On a alors que les valeurs de $c\left(t\right)$ sont comprises entre :

\begin{enumerate}

\Faux{$0$ et $13$}

\Faux{$0$ et $5$ }

\Vrai{$8$ et $18$}

\Faux{$0$ et $18$ }

\end{enumerate}

\end{QCM}

Dès lors, il peut calculer le nombre moyen de clients en appliquant la formule

$\frac{1}{365}\int\limits _{0}^{365}C\left(t\right)dt$ , où $C$ désigne une primitive de $c$.

\begin{QCM}

Parmi les propositions ci-dessous, laquelle donne une formule possible C~?

\begin{enumerate}

\Vrai{$C\left(t\right)=\frac{1825}{2}\pi\sin\left(\frac{8\pi t}{365}\right)+13t$}

\Faux{$C\left(t\right)=\frac{40}{365}\pi\cos\left(\frac{8\pi t}{365}\right)-13t-11$}

\Faux{$C\left(t\right)=-5\sin\left(\frac{8\pi t}{365}\right)$}

\Faux{$C\left(t\right)=\frac{40}{365}\pi\sin\left(\frac{8\pi t}{365}\right)-13t+4$}

\end{enumerate}

\end{QCM}

\begin{QCM}

On obtient ainsi que le nombre moyen de clients journaliers sur l'année en cours est donné par :

\begin{enumerate}

\Faux{$13\sin\left(8\pi\right)$ }

\Faux{$5\cos\left(8\pi\right)$}

\Vrai{$13$}

\Faux{$5$}

\end{enumerate}

\end{QCM}

\end{Sujet}

\begin{Sujet}

\subsection*{Exercice 1}

Pour chacune des propositions de cet exercice, cochez A si elle est vraie\par et B si elle est fausse.

On s'intéresse à la fonction f définie par $f:x\mapsto\left(2x+2\right)e^{4x}$.

\begin{VraiFaux}

\Faux{ L'ensemble de définition de f est $\left]0,+\infty\right[$.}

\end{VraiFaux}

\begin{VraiFaux}

\Faux{ La dérivée de f est donnée par $f^{\prime}:x\mapsto\left(8\right)e^{cx}$.}

\end{VraiFaux}

On admet que son approximation quadratique est donné par\par $\qquad f\left(x\right)\approx 2 + 6 x + 8 x^2$.

\begin{VraiFaux}

\Faux{ L'équation de la tangente $T$ à la courbe représentative $\mathscr C$ de $f$ en son point d'abscisse $0$ est donnée par $y=2x + 6 $.}

\end{VraiFaux}

\begin{VraiFaux}

\Faux{ $T$ est au-dessus de $\mathscr C$.}

\end{VraiFaux}

\subsection*{Exercice 2}

Cet exercice doit être rédigé au verso de la grille réponse.

\subsection*{Exercice 3}

\begin{QCM}

Parmi ces fonctions, laquelle est une primitive de la fonction $f$\par définie par $ f(x)=x \cdot \cos (3 x+1)$ ?

\begin{enumerate}

\Vrai{ $H:x \mapsto \frac{x \cdot \sin (3 x+1)}{3}+\frac{\cos (3 x+1)}{9}$}

\Faux{ $G:x \mapsto x \cdot \sin (3 x+1)$}

\Faux{ $F:x \mapsto \cos (3 x+1)-3 x \cdot \sin (3 x+1)$}

\end{enumerate}

\end{QCM}

\begin{QCM}

$\int\limits_{0}^{2}\left(5 t^{3}+3 t^{2}-t+7\right) d t=$

\begin{enumerate}

\Faux{ 71}

\Faux{ $15 t^{2}+6 t-1$}

\Vrai{ 40}

\end{enumerate}

\end{QCM}

\begin{QCM}

La valeur moyenne de la fonction $g: x \rightarrow e^{5 x} $ entre 0 et 2 est de :

\begin{enumerate}

\Faux{environ $22 025,5$}

\Faux{environ $4405,1$}

\Vrai{environ $2 202,5$}

\end{enumerate}

\end{QCM}

\newpage

\subsection*{Problème}

Alain Raffletou cherche à connaitre le nombre moyen de clients journaliers dans une de ses boutiques sur l'année en cours. Pour cela, il modélise le nombre de clients par une fonction donnant le nombre de clients $c\left(t\right)$ en fonction du temps $t$, exprimé en jours. On prends donc $t\in\left[0;365\right]$.

\begin{QCM}

En remarquant que le nombre de clients ne peut être négatif, déterminé, parmi les propositions ci-dessous, laquelle donne la formule de $c\left(t\right)$ en fonction de $t$ :

\begin{enumerate}

\Vrai{$c\left(t\right)=5\cos\left(\frac{4\pi t}{365}\right)+22$}

\Faux{$c\left(t\right)=5\cos\left(\frac{4\pi t}{365}-22\right)$}

\Faux{$c\left(t\right)=5\sin\left(\frac{4\pi t}{365}+22\right)$}

\Faux{$c\left(t\right)=5\sin\left(\frac{4\pi t}{365}\right)-22$}

\end{enumerate}

\end{QCM}

\begin{QCM}

On a alors que les valeurs de $c\left(t\right)$ sont comprises entre :

\begin{enumerate}

\Faux{$0$ et $5$ }

\Faux{$0$ et $22$}

\Faux{$0$ et $27$ }

\Vrai{$17$ et $27$}

\end{enumerate}

\end{QCM}

Dès lors, il peut calculer le nombre moyen de clients en appliquant la formule

$\frac{1}{365}\int\limits _{0}^{365}C\left(t\right)dt$ , où $C$ désigne une primitive de $c$.

\begin{QCM}

Parmi les propositions ci-dessous, laquelle donne une formule possible C~?

\begin{enumerate}

\Vrai{$C\left(t\right)=\frac{1825}{2}\pi\sin\left(\frac{4\pi t}{365}\right)+22t$}

\Faux{$C\left(t\right)=\frac{20}{365}\pi\cos\left(\frac{4\pi t}{365}\right)-22t-29$}

\Faux{$C\left(t\right)=\frac{20}{365}\pi\sin\left(\frac{4\pi t}{365}\right)-22t+13$}

\Faux{$C\left(t\right)=-5\sin\left(\frac{4\pi t}{365}\right)$}

\end{enumerate}

\end{QCM}

\begin{QCM}

On obtient ainsi que le nombre moyen de clients journaliers sur l'année en cours est donné par :

\begin{enumerate}

\Faux{$22\sin\left(4\pi\right)$ }

\Faux{$5$}

\Vrai{$22$}

\Faux{$5\cos\left(4\pi\right)$}

\end{enumerate}

\end{QCM}

\end{Sujet}

\begin{Sujet}

\subsection*{Exercice 1}

Pour chacune des propositions de cet exercice, cochez A si elle est vraie\par et B si elle est fausse.

On s'intéresse à la fonction f définie par $f:x\mapsto\left(2x+4\right)e^{2x}$.

\begin{VraiFaux}

\Vrai{ L'ensemble de définition de f est $\left]-\infty,+\infty\right[$.}

\end{VraiFaux}

\begin{VraiFaux}

\Vrai{ La dérivée de f est donnée par $f^{\prime}:x\mapsto\left(4x+10\right)e^{2x}$.}

\end{VraiFaux}

On admet que son approximation quadratique est donné par\par $\qquad f\left(x\right)\approx 4 + 10 x + 12 x^2$.

\begin{VraiFaux}

\Vrai{ L'équation de la tangente $T$ à la courbe représentative $\mathscr C$ de $f$ en son point d'abscisse $0$ est donnée par $y=10 x +4 $.}

\end{VraiFaux}

\begin{VraiFaux}

\Faux{ $T$ est au-dessus de $\mathscr C$.}

\end{VraiFaux}

\subsection*{Exercice 2}

Cet exercice doit être rédigé au verso de la grille réponse.

\subsection*{Exercice 3}

\begin{QCM}

Parmi ces fonctions, laquelle est une primitive de la fonction $f$\par définie par $ f(x)=x \cdot \cos (3 x+1)$ ?

\begin{enumerate}

\Faux{ $G:x \mapsto x \cdot \sin (3 x+1)$}

\Faux{ $F:x \mapsto \cos (3 x+1)-3 x \cdot \sin (3 x+1)$}

\Vrai{ $H:x \mapsto \frac{x \cdot \sin (3 x+1)}{3}+\frac{\cos (3 x+1)}{9}$}

\end{enumerate}

\end{QCM}

\begin{QCM}

$\int\limits_{0}^{2}\left(5 t^{3}+3 t^{2}-t+7\right) d t=$

\begin{enumerate}

\Faux{ 71}

\Faux{ $15 t^{2}+6 t-1$}

\Vrai{ 40}

\end{enumerate}

\end{QCM}

\begin{QCM}

La valeur moyenne de la fonction $g: x \rightarrow e^{5 x} $ entre 0 et 2 est de :

\begin{enumerate}

\Faux{environ $4405,1$}

\Vrai{environ $2 202,5$}

\Faux{environ $22 025,5$}

\end{enumerate}

\end{QCM}

\newpage

\subsection*{Problème}

Alain Raffletou cherche à connaitre le nombre moyen de clients journaliers dans une de ses boutiques sur l'année en cours. Pour cela, il modélise le nombre de clients par une fonction donnant le nombre de clients $c\left(t\right)$ en fonction du temps $t$, exprimé en jours. On prends donc $t\in\left[0;365\right]$.

\begin{QCM}

En remarquant que le nombre de clients ne peut être négatif, déterminé, parmi les propositions ci-dessous, laquelle donne la formule de $c\left(t\right)$ en fonction de $t$ :

\begin{enumerate}

\Faux{$c\left(t\right)=4\sin\left(\frac{8\pi t}{365}+16\right)$}

\Faux{$c\left(t\right)=4\cos\left(\frac{8\pi t}{365}-16\right)$}

\Faux{$c\left(t\right)=4\sin\left(\frac{8\pi t}{365}\right)-16$}

\Vrai{$c\left(t\right)=4\cos\left(\frac{8\pi t}{365}\right)+16$}

\end{enumerate}

\end{QCM}

\begin{QCM}

On a alors que les valeurs de $c\left(t\right)$ sont comprises entre :

\begin{enumerate}

\Faux{$0$ et $16$}

\Vrai{$12$ et $20$}

\Faux{$0$ et $20$ }

\Faux{$0$ et $4$ }

\end{enumerate}

\end{QCM}

Dès lors, il peut calculer le nombre moyen de clients en appliquant la formule

$\frac{1}{365}\int\limits _{0}^{365}C\left(t\right)dt$ , où $C$ désigne une primitive de $c$.

\begin{QCM}

Parmi les propositions ci-dessous, laquelle donne une formule possible C~?

\begin{enumerate}

\Faux{$C\left(t\right)=\frac{32}{365}\pi\cos\left(\frac{8\pi t}{365}\right)-16t-20$}

\Vrai{$C\left(t\right)=\frac{1460}{2}\pi\sin\left(\frac{8\pi t}{365}\right)+16t$}

\Faux{$C\left(t\right)=\frac{32}{365}\pi\sin\left(\frac{8\pi t}{365}\right)-16t+9$}

\Faux{$C\left(t\right)=-4\sin\left(\frac{8\pi t}{365}\right)$}

\end{enumerate}

\end{QCM}

\begin{QCM}

On obtient ainsi que le nombre moyen de clients journaliers sur l'année en cours est donné par :

\begin{enumerate}

\Faux{$16\sin\left(8\pi\right)$ }

\Vrai{$16$}

\Faux{$4\cos\left(8\pi\right)$}

\Faux{$4$}

\end{enumerate}

\end{QCM}

\end{Sujet}

\begin{Sujet}

\subsection*{Exercice 1}

Pour chacune des propositions de cet exercice, cochez A si elle est vraie\par et B si elle est fausse.

On s'intéresse à la fonction f définie par $f:x\mapsto\left(2x+4\right)e^{4x}$.

\begin{VraiFaux}

\Faux{ L'ensemble de définition de f est $\left]0,+\infty\right[$.}

\end{VraiFaux}

\begin{VraiFaux}

\Vrai{ La dérivée de f est donnée par $f^{\prime}:x\mapsto\left(8x+18\right)e^{4x}$.}

\end{VraiFaux}

On admet que son approximation quadratique est donné par\par $\qquad f\left(x\right)\approx 4 + 10 x + 12 x^2$.

\begin{VraiFaux}

\Faux{ L'équation de la tangente $T$ à la courbe représentative $\mathscr C$ de $f$ en son point d'abscisse $0$ est donnée par $y=4x + 10 $.}

\end{VraiFaux}

\begin{VraiFaux}

\Faux{ $T$ est au-dessus de $\mathscr C$.}

\end{VraiFaux}

\subsection*{Exercice 2}

Cet exercice doit être rédigé au verso de la grille réponse.

\subsection*{Exercice 3}

\begin{QCM}

Parmi ces fonctions, laquelle est une primitive de la fonction $f$\par définie par $ f(x)=x \cdot \cos (3 x+1)$ ?

\begin{enumerate}

\Faux{ $G:x \mapsto x \cdot \sin (3 x+1)$}

\Faux{ $F:x \mapsto \cos (3 x+1)-3 x \cdot \sin (3 x+1)$}

\Vrai{ $H:x \mapsto \frac{x \cdot \sin (3 x+1)}{3}+\frac{\cos (3 x+1)}{9}$}

\end{enumerate}

\end{QCM}

\begin{QCM}

$\int\limits_{0}^{2}\left(5 t^{3}+3 t^{2}-t+7\right) d t=$

\begin{enumerate}

\Vrai{ 40}

\Faux{ 71}

\Faux{ $15 t^{2}+6 t-1$}

\end{enumerate}

\end{QCM}

\begin{QCM}

La valeur moyenne de la fonction $g: x \rightarrow e^{5 x} $ entre 0 et 2 est de :

\begin{enumerate}

\Vrai{environ $2 202,5$}

\Faux{environ $22 025,5$}

\Faux{environ $4405,1$}

\end{enumerate}

\end{QCM}

\newpage

\subsection*{Problème}

Alain Raffletou cherche à connaitre le nombre moyen de clients journaliers dans une de ses boutiques sur l'année en cours. Pour cela, il modélise le nombre de clients par une fonction donnant le nombre de clients $c\left(t\right)$ en fonction du temps $t$, exprimé en jours. On prends donc $t\in\left[0;365\right]$.

\begin{QCM}

En remarquant que le nombre de clients ne peut être négatif, déterminé, parmi les propositions ci-dessous, laquelle donne la formule de $c\left(t\right)$ en fonction de $t$ :

\begin{enumerate}

\Faux{$c\left(t\right)=3\sin\left(\frac{6\pi t}{365}+14\right)$}

\Faux{$c\left(t\right)=3\cos\left(\frac{6\pi t}{365}-14\right)$}

\Faux{$c\left(t\right)=3\sin\left(\frac{6\pi t}{365}\right)-14$}

\Vrai{$c\left(t\right)=3\cos\left(\frac{6\pi t}{365}\right)+14$}

\end{enumerate}

\end{QCM}

\begin{QCM}

On a alors que les valeurs de $c\left(t\right)$ sont comprises entre :

\begin{enumerate}

\Faux{$0$ et $3$ }

\Vrai{$11$ et $17$}

\Faux{$0$ et $17$ }

\Faux{$0$ et $14$}

\end{enumerate}

\end{QCM}

Dès lors, il peut calculer le nombre moyen de clients en appliquant la formule

$\frac{1}{365}\int\limits _{0}^{365}C\left(t\right)dt$ , où $C$ désigne une primitive de $c$.

\begin{QCM}

Parmi les propositions ci-dessous, laquelle donne une formule possible C~?

\begin{enumerate}

\Faux{$C\left(t\right)=\frac{18}{365}\pi\cos\left(\frac{6\pi t}{365}\right)-14t-19$}

\Faux{$C\left(t\right)=-3\sin\left(\frac{6\pi t}{365}\right)$}

\Faux{$C\left(t\right)=\frac{18}{365}\pi\sin\left(\frac{6\pi t}{365}\right)-14t+9$}

\Vrai{$C\left(t\right)=\frac{1095}{2}\pi\sin\left(\frac{6\pi t}{365}\right)+14t$}

\end{enumerate}

\end{QCM}

\begin{QCM}

On obtient ainsi que le nombre moyen de clients journaliers sur l'année en cours est donné par :

\begin{enumerate}

\Faux{$3\cos\left(6\pi\right)$}

\Faux{$3$}

\Vrai{$14$}

\Faux{$14\sin\left(6\pi\right)$ }

\end{enumerate}

\end{QCM}

\end{Sujet}

\begin{Sujet}

\subsection*{Exercice 1}

Pour chacune des propositions de cet exercice, cochez A si elle est vraie\par et B si elle est fausse.

On s'intéresse à la fonction f définie par $f:x\mapsto\left(7x+7\right)e^{6x}$.

\begin{VraiFaux}

\Vrai{ L'ensemble de définition de f est $\left]-\infty,+\infty\right[$.}

\end{VraiFaux}

\begin{VraiFaux}

\Vrai{ La dérivée de f est donnée par $f^{\prime}:x\mapsto\left(42x+49\right)e^{6x}$.}

\end{VraiFaux}

On admet que son approximation quadratique est donné par\par $\qquad f\left(x\right)\approx 7 + 56 x + \frac{441}{2} x^2$.

\begin{VraiFaux}

\Vrai{ L'équation de la tangente $T$ à la courbe représentative $\mathscr C$ de $f$ en son point d'abscisse $0$ est donnée par $y=56 x +7 $.}

\end{VraiFaux}

\begin{VraiFaux}

\Vrai{ $T$ est au-dessous de $\mathscr C$.}

\end{VraiFaux}

\subsection*{Exercice 2}

Cet exercice doit être rédigé au verso de la grille réponse.

\subsection*{Exercice 3}

\begin{QCM}

Parmi ces fonctions, laquelle est une primitive de la fonction $f$\par définie par $ f(x)=x \cdot \cos (3 x+1)$ ?

\begin{enumerate}

\Vrai{ $H:x \mapsto \frac{x \cdot \sin (3 x+1)}{3}+\frac{\cos (3 x+1)}{9}$}

\Faux{ $G:x \mapsto x \cdot \sin (3 x+1)$}

\Faux{ $F:x \mapsto \cos (3 x+1)-3 x \cdot \sin (3 x+1)$}

\end{enumerate}

\end{QCM}

\begin{QCM}

$\int\limits_{0}^{2}\left(5 t^{3}+3 t^{2}-t+7\right) d t=$

\begin{enumerate}

\Vrai{ 40}

\Faux{ $15 t^{2}+6 t-1$}

\Faux{ 71}

\end{enumerate}

\end{QCM}

\begin{QCM}

La valeur moyenne de la fonction $g: x \rightarrow e^{5 x} $ entre 0 et 2 est de :

\begin{enumerate}

\Faux{environ $4405,1$}

\Faux{environ $22 025,5$}

\Vrai{environ $2 202,5$}

\end{enumerate}

\end{QCM}

\newpage

\subsection*{Problème}

Alain Raffletou cherche à connaitre le nombre moyen de clients journaliers dans une de ses boutiques sur l'année en cours. Pour cela, il modélise le nombre de clients par une fonction donnant le nombre de clients $c\left(t\right)$ en fonction du temps $t$, exprimé en jours. On prends donc $t\in\left[0;365\right]$.

\begin{QCM}

En remarquant que le nombre de clients ne peut être négatif, déterminé, parmi les propositions ci-dessous, laquelle donne la formule de $c\left(t\right)$ en fonction de $t$ :

\begin{enumerate}

\Faux{$c\left(t\right)=5\sin\left(\frac{8\pi t}{365}\right)-23$}

\Faux{$c\left(t\right)=5\sin\left(\frac{8\pi t}{365}+23\right)$}

\Vrai{$c\left(t\right)=5\cos\left(\frac{8\pi t}{365}\right)+23$}

\Faux{$c\left(t\right)=5\cos\left(\frac{8\pi t}{365}-23\right)$}

\end{enumerate}

\end{QCM}

\begin{QCM}

On a alors que les valeurs de $c\left(t\right)$ sont comprises entre :

\begin{enumerate}

\Faux{$0$ et $23$}

\Faux{$0$ et $5$ }

\Faux{$0$ et $28$ }

\Vrai{$18$ et $28$}

\end{enumerate}

\end{QCM}

Dès lors, il peut calculer le nombre moyen de clients en appliquant la formule

$\frac{1}{365}\int\limits _{0}^{365}C\left(t\right)dt$ , où $C$ désigne une primitive de $c$.

\begin{QCM}

Parmi les propositions ci-dessous, laquelle donne une formule possible C~?

\begin{enumerate}

\Faux{$C\left(t\right)=-5\sin\left(\frac{8\pi t}{365}\right)$}

\Vrai{$C\left(t\right)=\frac{1825}{2}\pi\sin\left(\frac{8\pi t}{365}\right)+23t$}

\Faux{$C\left(t\right)=\frac{40}{365}\pi\sin\left(\frac{8\pi t}{365}\right)-23t+14$}

\Faux{$C\left(t\right)=\frac{40}{365}\pi\cos\left(\frac{8\pi t}{365}\right)-23t-31$}

\end{enumerate}

\end{QCM}

\begin{QCM}

On obtient ainsi que le nombre moyen de clients journaliers sur l'année en cours est donné par :

\begin{enumerate}

\Faux{$23\sin\left(8\pi\right)$ }

\Faux{$5$}

\Faux{$5\cos\left(8\pi\right)$}

\Vrai{$23$}

\end{enumerate}

\end{QCM}

\end{Sujet}

\begin{Sujet}

\subsection*{Exercice 1}

Pour chacune des propositions de cet exercice, cochez A si elle est vraie\par et B si elle est fausse.

On s'intéresse à la fonction f définie par $f:x\mapsto\left(2x+6\right)e^{3x}$.

\begin{VraiFaux}

\Faux{ L'ensemble de définition de f est $\left]0,+\infty\right[$.}

\end{VraiFaux}

\begin{VraiFaux}

\Faux{ La dérivée de f est donnée par $f^{\prime}:x\mapsto\left(6\right)e^{cx}$.}

\end{VraiFaux}

On admet que son approximation quadratique est donné par\par $\qquad f\left(x\right)\approx 6 + 14 x + 16 x^2$.

\begin{VraiFaux}

\Faux{ L'équation de la tangente $T$ à la courbe représentative $\mathscr C$ de $f$ en son point d'abscisse $0$ est donnée par $y=6x + 14 $.}

\end{VraiFaux}

\begin{VraiFaux}

\Vrai{ $T$ est au-dessous de $\mathscr C$.}

\end{VraiFaux}

\subsection*{Exercice 2}

Cet exercice doit être rédigé au verso de la grille réponse.

\subsection*{Exercice 3}

\begin{QCM}

Parmi ces fonctions, laquelle est une primitive de la fonction $f$\par définie par $ f(x)=x \cdot \cos (3 x+1)$ ?

\begin{enumerate}

\Faux{ $G:x \mapsto x \cdot \sin (3 x+1)$}

\Vrai{ $H:x \mapsto \frac{x \cdot \sin (3 x+1)}{3}+\frac{\cos (3 x+1)}{9}$}

\Faux{ $F:x \mapsto \cos (3 x+1)-3 x \cdot \sin (3 x+1)$}

\end{enumerate}

\end{QCM}

\begin{QCM}

$\int\limits_{0}^{2}\left(5 t^{3}+3 t^{2}-t+7\right) d t=$

\begin{enumerate}

\Faux{ $15 t^{2}+6 t-1$}

\Faux{ 71}

\Vrai{ 40}

\end{enumerate}

\end{QCM}

\begin{QCM}

La valeur moyenne de la fonction $g: x \rightarrow e^{5 x} $ entre 0 et 2 est de :

\begin{enumerate}

\Faux{environ $4405,1$}

\Vrai{environ $2 202,5$}

\Faux{environ $22 025,5$}

\end{enumerate}

\end{QCM}

\newpage

\subsection*{Problème}

Alain Raffletou cherche à connaitre le nombre moyen de clients journaliers dans une de ses boutiques sur l'année en cours. Pour cela, il modélise le nombre de clients par une fonction donnant le nombre de clients $c\left(t\right)$ en fonction du temps $t$, exprimé en jours. On prends donc $t\in\left[0;365\right]$.

\begin{QCM}

En remarquant que le nombre de clients ne peut être négatif, déterminé, parmi les propositions ci-dessous, laquelle donne la formule de $c\left(t\right)$ en fonction de $t$ :

\begin{enumerate}

\Faux{$c\left(t\right)=3\sin\left(\frac{6\pi t}{365}\right)-13$}

\Faux{$c\left(t\right)=3\cos\left(\frac{6\pi t}{365}-13\right)$}

\Faux{$c\left(t\right)=3\sin\left(\frac{6\pi t}{365}+13\right)$}

\Vrai{$c\left(t\right)=3\cos\left(\frac{6\pi t}{365}\right)+13$}

\end{enumerate}

\end{QCM}

\begin{QCM}

On a alors que les valeurs de $c\left(t\right)$ sont comprises entre :

\begin{enumerate}

\Faux{$0$ et $3$ }

\Faux{$0$ et $16$ }

\Faux{$0$ et $13$}

\Vrai{$10$ et $16$}

\end{enumerate}

\end{QCM}

Dès lors, il peut calculer le nombre moyen de clients en appliquant la formule

$\frac{1}{365}\int\limits _{0}^{365}C\left(t\right)dt$ , où $C$ désigne une primitive de $c$.

\begin{QCM}

Parmi les propositions ci-dessous, laquelle donne une formule possible C~?

\begin{enumerate}

\Faux{$C\left(t\right)=\frac{18}{365}\pi\cos\left(\frac{6\pi t}{365}\right)-13t-17$}

\Vrai{$C\left(t\right)=\frac{1095}{2}\pi\sin\left(\frac{6\pi t}{365}\right)+13t$}

\Faux{$C\left(t\right)=\frac{18}{365}\pi\sin\left(\frac{6\pi t}{365}\right)-13t+8$}

\Faux{$C\left(t\right)=-3\sin\left(\frac{6\pi t}{365}\right)$}

\end{enumerate}

\end{QCM}

\begin{QCM}

On obtient ainsi que le nombre moyen de clients journaliers sur l'année en cours est donné par :

\begin{enumerate}

\Faux{$3\cos\left(6\pi\right)$}

\Faux{$3$}

\Vrai{$13$}

\Faux{$13\sin\left(6\pi\right)$ }

\end{enumerate}

\end{QCM}

\end{Sujet}

\begin{Sujet}

\subsection*{Exercice 1}

Pour chacune des propositions de cet exercice, cochez A si elle est vraie\par et B si elle est fausse.

On s'intéresse à la fonction f définie par $f:x\mapsto\left(2x+3\right)e^{2x}$.

\begin{VraiFaux}

\Vrai{ L'ensemble de définition de f est $\left]-\infty,+\infty\right[$.}

\end{VraiFaux}

\begin{VraiFaux}

\Faux{ La dérivée de f est donnée par $f^{\prime}:x\mapsto\left(4\right)e^{cx}$.}

\end{VraiFaux}

On admet que son approximation quadratique est donné par\par $\qquad f\left(x\right)\approx 3 + 8 x + 10 x^2$.

\begin{VraiFaux}

\Faux{ L'équation de la tangente $T$ à la courbe représentative $\mathscr C$ de $f$ en son point d'abscisse $0$ est donnée par $y=3x + 8 $.}

\end{VraiFaux}

\begin{VraiFaux}

\Faux{ $T$ est au-dessus de $\mathscr C$.}

\end{VraiFaux}

\subsection*{Exercice 2}

Cet exercice doit être rédigé au verso de la grille réponse.

\subsection*{Exercice 3}

\begin{QCM}

Parmi ces fonctions, laquelle est une primitive de la fonction $f$\par définie par $ f(x)=x \cdot \cos (3 x+1)$ ?

\begin{enumerate}

\Faux{ $F:x \mapsto \cos (3 x+1)-3 x \cdot \sin (3 x+1)$}

\Vrai{ $H:x \mapsto \frac{x \cdot \sin (3 x+1)}{3}+\frac{\cos (3 x+1)}{9}$}

\Faux{ $G:x \mapsto x \cdot \sin (3 x+1)$}

\end{enumerate}

\end{QCM}

\begin{QCM}

$\int\limits_{0}^{2}\left(5 t^{3}+3 t^{2}-t+7\right) d t=$

\begin{enumerate}

\Vrai{ 40}

\Faux{ $15 t^{2}+6 t-1$}

\Faux{ 71}

\end{enumerate}

\end{QCM}

\begin{QCM}

La valeur moyenne de la fonction $g: x \rightarrow e^{5 x} $ entre 0 et 2 est de :

\begin{enumerate}

\Faux{environ $4405,1$}

\Faux{environ $22 025,5$}

\Vrai{environ $2 202,5$}

\end{enumerate}

\end{QCM}

\newpage

\subsection*{Problème}

Alain Raffletou cherche à connaitre le nombre moyen de clients journaliers dans une de ses boutiques sur l'année en cours. Pour cela, il modélise le nombre de clients par une fonction donnant le nombre de clients $c\left(t\right)$ en fonction du temps $t$, exprimé en jours. On prends donc $t\in\left[0;365\right]$.

\begin{QCM}

En remarquant que le nombre de clients ne peut être négatif, déterminé, parmi les propositions ci-dessous, laquelle donne la formule de $c\left(t\right)$ en fonction de $t$ :

\begin{enumerate}

\Faux{$c\left(t\right)=3\sin\left(\frac{8\pi t}{365}\right)-28$}

\Faux{$c\left(t\right)=3\cos\left(\frac{8\pi t}{365}-28\right)$}

\Vrai{$c\left(t\right)=3\cos\left(\frac{8\pi t}{365}\right)+28$}

\Faux{$c\left(t\right)=3\sin\left(\frac{8\pi t}{365}+28\right)$}

\end{enumerate}

\end{QCM}

\begin{QCM}

On a alors que les valeurs de $c\left(t\right)$ sont comprises entre :

\begin{enumerate}

\Faux{$0$ et $3$ }

\Faux{$0$ et $28$}

\Vrai{$25$ et $31$}

\Faux{$0$ et $31$ }

\end{enumerate}

\end{QCM}

Dès lors, il peut calculer le nombre moyen de clients en appliquant la formule

$\frac{1}{365}\int\limits _{0}^{365}C\left(t\right)dt$ , où $C$ désigne une primitive de $c$.

\begin{QCM}

Parmi les propositions ci-dessous, laquelle donne une formule possible C~?

\begin{enumerate}

\Faux{$C\left(t\right)=\frac{24}{365}\pi\cos\left(\frac{8\pi t}{365}\right)-28t-47$}

\Faux{$C\left(t\right)=\frac{24}{365}\pi\sin\left(\frac{8\pi t}{365}\right)-28t+23$}

\Faux{$C\left(t\right)=-3\sin\left(\frac{8\pi t}{365}\right)$}

\Vrai{$C\left(t\right)=\frac{1095}{2}\pi\sin\left(\frac{8\pi t}{365}\right)+28t$}

\end{enumerate}

\end{QCM}

\begin{QCM}

On obtient ainsi que le nombre moyen de clients journaliers sur l'année en cours est donné par :

\begin{enumerate}

\Vrai{$28$}

\Faux{$28\sin\left(8\pi\right)$ }

\Faux{$3\cos\left(8\pi\right)$}

\Faux{$3$}

\end{enumerate}

\end{QCM}

\end{Sujet}

\begin{Sujet}

\subsection*{Exercice 1}

Pour chacune des propositions de cet exercice, cochez A si elle est vraie\par et B si elle est fausse.

On s'intéresse à la fonction f définie par $f:x\mapsto\left(5x+4\right)e^{4x}$.

\begin{VraiFaux}

\Vrai{ L'ensemble de définition de f est $\left]-\infty,+\infty\right[$.}

\end{VraiFaux}

\begin{VraiFaux}

\Vrai{ La dérivée de f est donnée par $f^{\prime}:x\mapsto\left(20x+21\right)e^{4x}$.}

\end{VraiFaux}

On admet que son approximation quadratique est donné par\par $\qquad f\left(x\right)\approx 4 + 25 x + 75 x^2$.

\begin{VraiFaux}

\Faux{ L'équation de la tangente $T$ à la courbe représentative $\mathscr C$ de $f$ en son point d'abscisse $0$ est donnée par $y=4x + 25 $.}

\end{VraiFaux}

\begin{VraiFaux}

\Vrai{ $T$ est au-dessous de $\mathscr C$.}

\end{VraiFaux}

\subsection*{Exercice 2}

Cet exercice doit être rédigé au verso de la grille réponse.

\subsection*{Exercice 3}

\begin{QCM}

Parmi ces fonctions, laquelle est une primitive de la fonction $f$\par définie par $ f(x)=x \cdot \cos (3 x+1)$ ?

\begin{enumerate}

\Vrai{ $H:x \mapsto \frac{x \cdot \sin (3 x+1)}{3}+\frac{\cos (3 x+1)}{9}$}

\Faux{ $F:x \mapsto \cos (3 x+1)-3 x \cdot \sin (3 x+1)$}

\Faux{ $G:x \mapsto x \cdot \sin (3 x+1)$}

\end{enumerate}

\end{QCM}

\begin{QCM}

$\int\limits_{0}^{2}\left(5 t^{3}+3 t^{2}-t+7\right) d t=$

\begin{enumerate}

\Faux{ $15 t^{2}+6 t-1$}

\Vrai{ 40}

\Faux{ 71}

\end{enumerate}

\end{QCM}

\begin{QCM}

La valeur moyenne de la fonction $g: x \rightarrow e^{5 x} $ entre 0 et 2 est de :

\begin{enumerate}

\Faux{environ $22 025,5$}

\Faux{environ $4405,1$}

\Vrai{environ $2 202,5$}

\end{enumerate}

\end{QCM}

\newpage

\subsection*{Problème}

Alain Raffletou cherche à connaitre le nombre moyen de clients journaliers dans une de ses boutiques sur l'année en cours. Pour cela, il modélise le nombre de clients par une fonction donnant le nombre de clients $c\left(t\right)$ en fonction du temps $t$, exprimé en jours. On prends donc $t\in\left[0;365\right]$.

\begin{QCM}

En remarquant que le nombre de clients ne peut être négatif, déterminé, parmi les propositions ci-dessous, laquelle donne la formule de $c\left(t\right)$ en fonction de $t$ :

\begin{enumerate}

\Faux{$c\left(t\right)=5\sin\left(\frac{4\pi t}{365}\right)-11$}

\Vrai{$c\left(t\right)=5\cos\left(\frac{4\pi t}{365}\right)+11$}

\Faux{$c\left(t\right)=5\sin\left(\frac{4\pi t}{365}+11\right)$}

\Faux{$c\left(t\right)=5\cos\left(\frac{4\pi t}{365}-11\right)$}

\end{enumerate}

\end{QCM}

\begin{QCM}

On a alors que les valeurs de $c\left(t\right)$ sont comprises entre :

\begin{enumerate}

\Faux{$0$ et $5$ }

\Faux{$0$ et $16$ }

\Vrai{$6$ et $16$}

\Faux{$0$ et $11$}

\end{enumerate}

\end{QCM}

Dès lors, il peut calculer le nombre moyen de clients en appliquant la formule

$\frac{1}{365}\int\limits _{0}^{365}C\left(t\right)dt$ , où $C$ désigne une primitive de $c$.

\begin{QCM}

Parmi les propositions ci-dessous, laquelle donne une formule possible C~?

\begin{enumerate}

\Vrai{$C\left(t\right)=\frac{1825}{2}\pi\sin\left(\frac{4\pi t}{365}\right)+11t$}

\Faux{$C\left(t\right)=\frac{20}{365}\pi\cos\left(\frac{4\pi t}{365}\right)-11t-7$}

\Faux{$C\left(t\right)=\frac{20}{365}\pi\sin\left(\frac{4\pi t}{365}\right)-11t+2$}

\Faux{$C\left(t\right)=-5\sin\left(\frac{4\pi t}{365}\right)$}

\end{enumerate}

\end{QCM}

\begin{QCM}

On obtient ainsi que le nombre moyen de clients journaliers sur l'année en cours est donné par :

\begin{enumerate}

\Faux{$11\sin\left(4\pi\right)$ }

\Faux{$5$}

\Faux{$5\cos\left(4\pi\right)$}

\Vrai{$11$}

\end{enumerate}

\end{QCM}

\end{Sujet}

\begin{Sujet}

\subsection*{Exercice 1}

Pour chacune des propositions de cet exercice, cochez A si elle est vraie\par et B si elle est fausse.

On s'intéresse à la fonction f définie par $f:x\mapsto\left(7x+7\right)e^{7x}$.

\begin{VraiFaux}

\Faux{ L'ensemble de définition de f est $\left]0,+\infty\right[$.}

\end{VraiFaux}

\begin{VraiFaux}

\Vrai{ La dérivée de f est donnée par $f^{\prime}:x\mapsto\left(49x+56\right)e^{7x}$.}

\end{VraiFaux}

On admet que son approximation quadratique est donné par\par $\qquad f\left(x\right)\approx 7 + 56 x + \frac{441}{2} x^2$.

\begin{VraiFaux}

\Vrai{ L'équation de la tangente $T$ à la courbe représentative $\mathscr C$ de $f$ en son point d'abscisse $0$ est donnée par $y=56 x +7 $.}

\end{VraiFaux}

\begin{VraiFaux}

\Vrai{ $T$ est au-dessous de $\mathscr C$.}

\end{VraiFaux}

\subsection*{Exercice 2}

Cet exercice doit être rédigé au verso de la grille réponse.

\subsection*{Exercice 3}

\begin{QCM}

Parmi ces fonctions, laquelle est une primitive de la fonction $f$\par définie par $ f(x)=x \cdot \cos (3 x+1)$ ?

\begin{enumerate}

\Faux{ $G:x \mapsto x \cdot \sin (3 x+1)$}

\Vrai{ $H:x \mapsto \frac{x \cdot \sin (3 x+1)}{3}+\frac{\cos (3 x+1)}{9}$}

\Faux{ $F:x \mapsto \cos (3 x+1)-3 x \cdot \sin (3 x+1)$}

\end{enumerate}

\end{QCM}

\begin{QCM}

$\int\limits_{0}^{2}\left(5 t^{3}+3 t^{2}-t+7\right) d t=$

\begin{enumerate}

\Vrai{ 40}

\Faux{ 71}

\Faux{ $15 t^{2}+6 t-1$}

\end{enumerate}

\end{QCM}

\begin{QCM}

La valeur moyenne de la fonction $g: x \rightarrow e^{5 x} $ entre 0 et 2 est de :

\begin{enumerate}

\Faux{environ $4405,1$}

\Faux{environ $22 025,5$}

\Vrai{environ $2 202,5$}

\end{enumerate}

\end{QCM}

\newpage

\subsection*{Problème}

Alain Raffletou cherche à connaitre le nombre moyen de clients journaliers dans une de ses boutiques sur l'année en cours. Pour cela, il modélise le nombre de clients par une fonction donnant le nombre de clients $c\left(t\right)$ en fonction du temps $t$, exprimé en jours. On prends donc $t\in\left[0;365\right]$.

\begin{QCM}

En remarquant que le nombre de clients ne peut être négatif, déterminé, parmi les propositions ci-dessous, laquelle donne la formule de $c\left(t\right)$ en fonction de $t$ :

\begin{enumerate}

\Faux{$c\left(t\right)=4\sin\left(\frac{8\pi t}{365}\right)-14$}

\Faux{$c\left(t\right)=4\cos\left(\frac{8\pi t}{365}-14\right)$}

\Vrai{$c\left(t\right)=4\cos\left(\frac{8\pi t}{365}\right)+14$}

\Faux{$c\left(t\right)=4\sin\left(\frac{8\pi t}{365}+14\right)$}

\end{enumerate}

\end{QCM}

\begin{QCM}

On a alors que les valeurs de $c\left(t\right)$ sont comprises entre :

\begin{enumerate}

\Faux{$0$ et $14$}

\Faux{$0$ et $4$ }

\Vrai{$10$ et $18$}

\Faux{$0$ et $18$ }

\end{enumerate}

\end{QCM}

Dès lors, il peut calculer le nombre moyen de clients en appliquant la formule

$\frac{1}{365}\int\limits _{0}^{365}C\left(t\right)dt$ , où $C$ désigne une primitive de $c$.

\begin{QCM}

Parmi les propositions ci-dessous, laquelle donne une formule possible C~?

\begin{enumerate}

\Faux{$C\left(t\right)=\frac{32}{365}\pi\cos\left(\frac{8\pi t}{365}\right)-14t-16$}

\Faux{$C\left(t\right)=-4\sin\left(\frac{8\pi t}{365}\right)$}

\Faux{$C\left(t\right)=\frac{32}{365}\pi\sin\left(\frac{8\pi t}{365}\right)-14t+7$}

\Vrai{$C\left(t\right)=\frac{1460}{2}\pi\sin\left(\frac{8\pi t}{365}\right)+14t$}

\end{enumerate}

\end{QCM}

\begin{QCM}

On obtient ainsi que le nombre moyen de clients journaliers sur l'année en cours est donné par :

\begin{enumerate}

\Faux{$4\cos\left(8\pi\right)$}

\Vrai{$14$}

\Faux{$14\sin\left(8\pi\right)$ }

\Faux{$4$}

\end{enumerate}

\end{QCM}

\end{Sujet}

\begin{Sujet}

\subsection*{Exercice 1}

Pour chacune des propositions de cet exercice, cochez A si elle est vraie\par et B si elle est fausse.

On s'intéresse à la fonction f définie par $f:x\mapsto\left(2x+2\right)e^{2x}$.

\begin{VraiFaux}

\Faux{ L'ensemble de définition de f est $\left]0,+\infty\right[$.}

\end{VraiFaux}

\begin{VraiFaux}

\Faux{ La dérivée de f est donnée par $f^{\prime}:x\mapsto\left(4\right)e^{cx}$.}

\end{VraiFaux}

On admet que son approximation quadratique est donné par\par $\qquad f\left(x\right)\approx 2 + 6 x + 8 x^2$.

\begin{VraiFaux}

\Vrai{ L'équation de la tangente $T$ à la courbe représentative $\mathscr C$ de $f$ en son point d'abscisse $0$ est donnée par $y=6 x +2 $.}

\end{VraiFaux}

\begin{VraiFaux}

\Faux{ $T$ est au-dessus de $\mathscr C$.}

\end{VraiFaux}

\subsection*{Exercice 2}

Cet exercice doit être rédigé au verso de la grille réponse.

\subsection*{Exercice 3}

\begin{QCM}

Parmi ces fonctions, laquelle est une primitive de la fonction $f$\par définie par $ f(x)=x \cdot \cos (3 x+1)$ ?

\begin{enumerate}

\Faux{ $G:x \mapsto x \cdot \sin (3 x+1)$}

\Vrai{ $H:x \mapsto \frac{x \cdot \sin (3 x+1)}{3}+\frac{\cos (3 x+1)}{9}$}

\Faux{ $F:x \mapsto \cos (3 x+1)-3 x \cdot \sin (3 x+1)$}

\end{enumerate}

\end{QCM}

\begin{QCM}

$\int\limits_{0}^{2}\left(5 t^{3}+3 t^{2}-t+7\right) d t=$

\begin{enumerate}

\Faux{ $15 t^{2}+6 t-1$}

\Faux{ 71}

\Vrai{ 40}

\end{enumerate}

\end{QCM}

\begin{QCM}

La valeur moyenne de la fonction $g: x \rightarrow e^{5 x} $ entre 0 et 2 est de :

\begin{enumerate}

\Vrai{environ $2 202,5$}

\Faux{environ $4405,1$}

\Faux{environ $22 025,5$}

\end{enumerate}

\end{QCM}

\newpage

\subsection*{Problème}

Alain Raffletou cherche à connaitre le nombre moyen de clients journaliers dans une de ses boutiques sur l'année en cours. Pour cela, il modélise le nombre de clients par une fonction donnant le nombre de clients $c\left(t\right)$ en fonction du temps $t$, exprimé en jours. On prends donc $t\in\left[0;365\right]$.

\begin{QCM}

En remarquant que le nombre de clients ne peut être négatif, déterminé, parmi les propositions ci-dessous, laquelle donne la formule de $c\left(t\right)$ en fonction de $t$ :

\begin{enumerate}

\Vrai{$c\left(t\right)=4\cos\left(\frac{6\pi t}{365}\right)+16$}

\Faux{$c\left(t\right)=4\sin\left(\frac{6\pi t}{365}+16\right)$}

\Faux{$c\left(t\right)=4\sin\left(\frac{6\pi t}{365}\right)-16$}

\Faux{$c\left(t\right)=4\cos\left(\frac{6\pi t}{365}-16\right)$}

\end{enumerate}

\end{QCM}

\begin{QCM}

On a alors que les valeurs de $c\left(t\right)$ sont comprises entre :

\begin{enumerate}

\Faux{$0$ et $20$ }

\Faux{$0$ et $16$}

\Vrai{$12$ et $20$}

\Faux{$0$ et $4$ }

\end{enumerate}

\end{QCM}

Dès lors, il peut calculer le nombre moyen de clients en appliquant la formule

$\frac{1}{365}\int\limits _{0}^{365}C\left(t\right)dt$ , où $C$ désigne une primitive de $c$.

\begin{QCM}

Parmi les propositions ci-dessous, laquelle donne une formule possible C~?

\begin{enumerate}

\Faux{$C\left(t\right)=\frac{24}{365}\pi\sin\left(\frac{6\pi t}{365}\right)-16t+9$}

\Faux{$C\left(t\right)=\frac{24}{365}\pi\cos\left(\frac{6\pi t}{365}\right)-16t-20$}

\Vrai{$C\left(t\right)=\frac{1460}{2}\pi\sin\left(\frac{6\pi t}{365}\right)+16t$}

\Faux{$C\left(t\right)=-4\sin\left(\frac{6\pi t}{365}\right)$}

\end{enumerate}

\end{QCM}

\begin{QCM}

On obtient ainsi que le nombre moyen de clients journaliers sur l'année en cours est donné par :

\begin{enumerate}

\Faux{$4$}

\Faux{$4\cos\left(6\pi\right)$}

\Vrai{$16$}

\Faux{$16\sin\left(6\pi\right)$ }

\end{enumerate}

\end{QCM}

\end{Sujet}

\begin{Sujet}

\subsection*{Exercice 1}

Pour chacune des propositions de cet exercice, cochez A si elle est vraie\par et B si elle est fausse.

On s'intéresse à la fonction f définie par $f:x\mapsto\left(5x+3\right)e^{3x}$.

\begin{VraiFaux}

\Faux{ L'ensemble de définition de f est $\left]0,+\infty\right[$.}

\end{VraiFaux}

\begin{VraiFaux}

\Vrai{ La dérivée de f est donnée par $f^{\prime}:x\mapsto\left(15x+14\right)e^{3x}$.}

\end{VraiFaux}

On admet que son approximation quadratique est donné par\par $\qquad f\left(x\right)\approx 3 + 20 x + \frac{125}{2} x^2$.

\begin{VraiFaux}

\Faux{ L'équation de la tangente $T$ à la courbe représentative $\mathscr C$ de $f$ en son point d'abscisse $0$ est donnée par $y=3x + 20 $.}

\end{VraiFaux}

\begin{VraiFaux}

\Faux{ $T$ est au-dessus de $\mathscr C$.}

\end{VraiFaux}

\subsection*{Exercice 2}

Cet exercice doit être rédigé au verso de la grille réponse.

\subsection*{Exercice 3}

\begin{QCM}

Parmi ces fonctions, laquelle est une primitive de la fonction $f$\par définie par $ f(x)=x \cdot \cos (3 x+1)$ ?

\begin{enumerate}

\Faux{ $F:x \mapsto \cos (3 x+1)-3 x \cdot \sin (3 x+1)$}

\Vrai{ $H:x \mapsto \frac{x \cdot \sin (3 x+1)}{3}+\frac{\cos (3 x+1)}{9}$}

\Faux{ $G:x \mapsto x \cdot \sin (3 x+1)$}

\end{enumerate}

\end{QCM}

\begin{QCM}

$\int\limits_{0}^{2}\left(5 t^{3}+3 t^{2}-t+7\right) d t=$

\begin{enumerate}

\Faux{ 71}

\Vrai{ 40}

\Faux{ $15 t^{2}+6 t-1$}

\end{enumerate}

\end{QCM}

\begin{QCM}

La valeur moyenne de la fonction $g: x \rightarrow e^{5 x} $ entre 0 et 2 est de :

\begin{enumerate}

\Faux{environ $4405,1$}

\Faux{environ $22 025,5$}

\Vrai{environ $2 202,5$}

\end{enumerate}

\end{QCM}

\newpage

\subsection*{Problème}

Alain Raffletou cherche à connaitre le nombre moyen de clients journaliers dans une de ses boutiques sur l'année en cours. Pour cela, il modélise le nombre de clients par une fonction donnant le nombre de clients $c\left(t\right)$ en fonction du temps $t$, exprimé en jours. On prends donc $t\in\left[0;365\right]$.

\begin{QCM}

En remarquant que le nombre de clients ne peut être négatif, déterminé, parmi les propositions ci-dessous, laquelle donne la formule de $c\left(t\right)$ en fonction de $t$ :

\begin{enumerate}

\Faux{$c\left(t\right)=5\sin\left(\frac{6\pi t}{365}\right)-20$}

\Faux{$c\left(t\right)=5\sin\left(\frac{6\pi t}{365}+20\right)$}

\Faux{$c\left(t\right)=5\cos\left(\frac{6\pi t}{365}-20\right)$}

\Vrai{$c\left(t\right)=5\cos\left(\frac{6\pi t}{365}\right)+20$}

\end{enumerate}

\end{QCM}

\begin{QCM}

On a alors que les valeurs de $c\left(t\right)$ sont comprises entre :

\begin{enumerate}

\Faux{$0$ et $25$ }

\Faux{$0$ et $20$}

\Vrai{$15$ et $25$}

\Faux{$0$ et $5$ }

\end{enumerate}

\end{QCM}

Dès lors, il peut calculer le nombre moyen de clients en appliquant la formule

$\frac{1}{365}\int\limits _{0}^{365}C\left(t\right)dt$ , où $C$ désigne une primitive de $c$.

\begin{QCM}

Parmi les propositions ci-dessous, laquelle donne une formule possible C~?

\begin{enumerate}

\Faux{$C\left(t\right)=-5\sin\left(\frac{6\pi t}{365}\right)$}

\Faux{$C\left(t\right)=\frac{30}{365}\pi\sin\left(\frac{6\pi t}{365}\right)-20t+11$}

\Vrai{$C\left(t\right)=\frac{1825}{2}\pi\sin\left(\frac{6\pi t}{365}\right)+20t$}

\Faux{$C\left(t\right)=\frac{30}{365}\pi\cos\left(\frac{6\pi t}{365}\right)-20t-25$}

\end{enumerate}

\end{QCM}

\begin{QCM}

On obtient ainsi que le nombre moyen de clients journaliers sur l'année en cours est donné par :

\begin{enumerate}

\Vrai{$20$}

\Faux{$5$}

\Faux{$20\sin\left(6\pi\right)$ }

\Faux{$5\cos\left(6\pi\right)$}

\end{enumerate}

\end{QCM}

\end{Sujet}

\begin{Sujet}

\subsection*{Exercice 1}

Pour chacune des propositions de cet exercice, cochez A si elle est vraie\par et B si elle est fausse.

On s'intéresse à la fonction f définie par $f:x\mapsto\left(2x+7\right)e^{6x}$.

\begin{VraiFaux}

\Vrai{ L'ensemble de définition de f est $\left]-\infty,+\infty\right[$.}

\end{VraiFaux}

\begin{VraiFaux}

\Vrai{ La dérivée de f est donnée par $f^{\prime}:x\mapsto\left(12x+44\right)e^{6x}$.}

\end{VraiFaux}

On admet que son approximation quadratique est donné par\par $\qquad f\left(x\right)\approx 7 + 16 x + 18 x^2$.

\begin{VraiFaux}

\Faux{ L'équation de la tangente $T$ à la courbe représentative $\mathscr C$ de $f$ en son point d'abscisse $0$ est donnée par $y=7x + 16 $.}

\end{VraiFaux}

\begin{VraiFaux}

\Faux{ $T$ est au-dessus de $\mathscr C$.}

\end{VraiFaux}

\subsection*{Exercice 2}

Cet exercice doit être rédigé au verso de la grille réponse.

\subsection*{Exercice 3}

\begin{QCM}

Parmi ces fonctions, laquelle est une primitive de la fonction $f$\par définie par $ f(x)=x \cdot \cos (3 x+1)$ ?

\begin{enumerate}

\Faux{ $F:x \mapsto \cos (3 x+1)-3 x \cdot \sin (3 x+1)$}

\Faux{ $G:x \mapsto x \cdot \sin (3 x+1)$}

\Vrai{ $H:x \mapsto \frac{x \cdot \sin (3 x+1)}{3}+\frac{\cos (3 x+1)}{9}$}

\end{enumerate}

\end{QCM}

\begin{QCM}

$\int\limits_{0}^{2}\left(5 t^{3}+3 t^{2}-t+7\right) d t=$

\begin{enumerate}

\Faux{ $15 t^{2}+6 t-1$}

\Faux{ 71}

\Vrai{ 40}

\end{enumerate}

\end{QCM}

\begin{QCM}

La valeur moyenne de la fonction $g: x \rightarrow e^{5 x} $ entre 0 et 2 est de :

\begin{enumerate}

\Faux{environ $22 025,5$}

\Vrai{environ $2 202,5$}

\Faux{environ $4405,1$}

\end{enumerate}

\end{QCM}

\newpage

\subsection*{Problème}

Alain Raffletou cherche à connaitre le nombre moyen de clients journaliers dans une de ses boutiques sur l'année en cours. Pour cela, il modélise le nombre de clients par une fonction donnant le nombre de clients $c\left(t\right)$ en fonction du temps $t$, exprimé en jours. On prends donc $t\in\left[0;365\right]$.

\begin{QCM}

En remarquant que le nombre de clients ne peut être négatif, déterminé, parmi les propositions ci-dessous, laquelle donne la formule de $c\left(t\right)$ en fonction de $t$ :

\begin{enumerate}

\Vrai{$c\left(t\right)=5\cos\left(\frac{4\pi t}{365}\right)+17$}

\Faux{$c\left(t\right)=5\sin\left(\frac{4\pi t}{365}+17\right)$}

\Faux{$c\left(t\right)=5\sin\left(\frac{4\pi t}{365}\right)-17$}

\Faux{$c\left(t\right)=5\cos\left(\frac{4\pi t}{365}-17\right)$}

\end{enumerate}

\end{QCM}

\begin{QCM}

On a alors que les valeurs de $c\left(t\right)$ sont comprises entre :

\begin{enumerate}

\Faux{$0$ et $5$ }

\Faux{$0$ et $17$}

\Vrai{$12$ et $22$}

\Faux{$0$ et $22$ }

\end{enumerate}

\end{QCM}

Dès lors, il peut calculer le nombre moyen de clients en appliquant la formule

$\frac{1}{365}\int\limits _{0}^{365}C\left(t\right)dt$ , où $C$ désigne une primitive de $c$.

\begin{QCM}

Parmi les propositions ci-dessous, laquelle donne une formule possible C~?

\begin{enumerate}

\Faux{$C\left(t\right)=\frac{20}{365}\pi\sin\left(\frac{4\pi t}{365}\right)-17t+8$}

\Faux{$C\left(t\right)=\frac{20}{365}\pi\cos\left(\frac{4\pi t}{365}\right)-17t-19$}

\Faux{$C\left(t\right)=-5\sin\left(\frac{4\pi t}{365}\right)$}

\Vrai{$C\left(t\right)=\frac{1825}{2}\pi\sin\left(\frac{4\pi t}{365}\right)+17t$}

\end{enumerate}

\end{QCM}

\begin{QCM}

On obtient ainsi que le nombre moyen de clients journaliers sur l'année en cours est donné par :

\begin{enumerate}

\Faux{$17\sin\left(4\pi\right)$ }

\Vrai{$17$}

\Faux{$5\cos\left(4\pi\right)$}

\Faux{$5$}

\end{enumerate}

\end{QCM}

\end{Sujet}

\begin{Sujet}

\subsection*{Exercice 1}

Pour chacune des propositions de cet exercice, cochez A si elle est vraie\par et B si elle est fausse.

On s'intéresse à la fonction f définie par $f:x\mapsto\left(3x+7\right)e^{7x}$.

\begin{VraiFaux}

\Faux{ L'ensemble de définition de f est $\left]0,+\infty\right[$.}

\end{VraiFaux}

\begin{VraiFaux}

\Faux{ La dérivée de f est donnée par $f^{\prime}:x\mapsto\left(21\right)e^{cx}$.}

\end{VraiFaux}

On admet que son approximation quadratique est donné par\par $\qquad f\left(x\right)\approx 7 + 24 x + \frac{81}{2} x^2$.

\begin{VraiFaux}

\Vrai{ L'équation de la tangente $T$ à la courbe représentative $\mathscr C$ de $f$ en son point d'abscisse $0$ est donnée par $y=24 x +7 $.}

\end{VraiFaux}

\begin{VraiFaux}

\Faux{ $T$ est au-dessus de $\mathscr C$.}

\end{VraiFaux}

\subsection*{Exercice 2}

Cet exercice doit être rédigé au verso de la grille réponse.

\subsection*{Exercice 3}

\begin{QCM}

Parmi ces fonctions, laquelle est une primitive de la fonction $f$\par définie par $ f(x)=x \cdot \cos (3 x+1)$ ?

\begin{enumerate}

\Vrai{ $H:x \mapsto \frac{x \cdot \sin (3 x+1)}{3}+\frac{\cos (3 x+1)}{9}$}

\Faux{ $G:x \mapsto x \cdot \sin (3 x+1)$}

\Faux{ $F:x \mapsto \cos (3 x+1)-3 x \cdot \sin (3 x+1)$}

\end{enumerate}

\end{QCM}

\begin{QCM}

$\int\limits_{0}^{2}\left(5 t^{3}+3 t^{2}-t+7\right) d t=$

\begin{enumerate}

\Faux{ 71}

\Faux{ $15 t^{2}+6 t-1$}

\Vrai{ 40}

\end{enumerate}

\end{QCM}

\begin{QCM}

La valeur moyenne de la fonction $g: x \rightarrow e^{5 x} $ entre 0 et 2 est de :

\begin{enumerate}

\Vrai{environ $2 202,5$}

\Faux{environ $22 025,5$}

\Faux{environ $4405,1$}

\end{enumerate}

\end{QCM}

\newpage

\subsection*{Problème}

Alain Raffletou cherche à connaitre le nombre moyen de clients journaliers dans une de ses boutiques sur l'année en cours. Pour cela, il modélise le nombre de clients par une fonction donnant le nombre de clients $c\left(t\right)$ en fonction du temps $t$, exprimé en jours. On prends donc $t\in\left[0;365\right]$.

\begin{QCM}

En remarquant que le nombre de clients ne peut être négatif, déterminé, parmi les propositions ci-dessous, laquelle donne la formule de $c\left(t\right)$ en fonction de $t$ :

\begin{enumerate}

\Vrai{$c\left(t\right)=4\cos\left(\frac{4\pi t}{365}\right)+12$}

\Faux{$c\left(t\right)=4\sin\left(\frac{4\pi t}{365}\right)-12$}

\Faux{$c\left(t\right)=4\sin\left(\frac{4\pi t}{365}+12\right)$}

\Faux{$c\left(t\right)=4\cos\left(\frac{4\pi t}{365}-12\right)$}

\end{enumerate}

\end{QCM}

\begin{QCM}

On a alors que les valeurs de $c\left(t\right)$ sont comprises entre :

\begin{enumerate}

\Faux{$0$ et $16$ }

\Vrai{$8$ et $16$}

\Faux{$0$ et $4$ }

\Faux{$0$ et $12$}

\end{enumerate}

\end{QCM}

Dès lors, il peut calculer le nombre moyen de clients en appliquant la formule

$\frac{1}{365}\int\limits _{0}^{365}C\left(t\right)dt$ , où $C$ désigne une primitive de $c$.

\begin{QCM}

Parmi les propositions ci-dessous, laquelle donne une formule possible C~?

\begin{enumerate}

\Faux{$C\left(t\right)=\frac{16}{365}\pi\cos\left(\frac{4\pi t}{365}\right)-12t-12$}

\Vrai{$C\left(t\right)=\frac{1460}{2}\pi\sin\left(\frac{4\pi t}{365}\right)+12t$}

\Faux{$C\left(t\right)=-4\sin\left(\frac{4\pi t}{365}\right)$}

\Faux{$C\left(t\right)=\frac{16}{365}\pi\sin\left(\frac{4\pi t}{365}\right)-12t+5$}

\end{enumerate}

\end{QCM}

\begin{QCM}

On obtient ainsi que le nombre moyen de clients journaliers sur l'année en cours est donné par :

\begin{enumerate}

\Faux{$4\cos\left(4\pi\right)$}

\Faux{$4$}

\Vrai{$12$}

\Faux{$12\sin\left(4\pi\right)$ }

\end{enumerate}

\end{QCM}

\end{Sujet}

\begin{Sujet}

\subsection*{Exercice 1}

Pour chacune des propositions de cet exercice, cochez A si elle est vraie\par et B si elle est fausse.

On s'intéresse à la fonction f définie par $f:x\mapsto\left(4x+7\right)e^{3x}$.

\begin{VraiFaux}

\Vrai{ L'ensemble de définition de f est $\left]-\infty,+\infty\right[$.}

\end{VraiFaux}

\begin{VraiFaux}

\Vrai{ La dérivée de f est donnée par $f^{\prime}:x\mapsto\left(12x+25\right)e^{3x}$.}

\end{VraiFaux}

On admet que son approximation quadratique est donné par\par $\qquad f\left(x\right)\approx 7 + 32 x + 72 x^2$.

\begin{VraiFaux}

\Vrai{ L'équation de la tangente $T$ à la courbe représentative $\mathscr C$ de $f$ en son point d'abscisse $0$ est donnée par $y=32 x +7 $.}

\end{VraiFaux}

\begin{VraiFaux}

\Vrai{ $T$ est au-dessous de $\mathscr C$.}

\end{VraiFaux}

\subsection*{Exercice 2}

Cet exercice doit être rédigé au verso de la grille réponse.

\subsection*{Exercice 3}

\begin{QCM}

Parmi ces fonctions, laquelle est une primitive de la fonction $f$\par définie par $ f(x)=x \cdot \cos (3 x+1)$ ?

\begin{enumerate}

\Faux{ $G:x \mapsto x \cdot \sin (3 x+1)$}

\Faux{ $F:x \mapsto \cos (3 x+1)-3 x \cdot \sin (3 x+1)$}

\Vrai{ $H:x \mapsto \frac{x \cdot \sin (3 x+1)}{3}+\frac{\cos (3 x+1)}{9}$}

\end{enumerate}

\end{QCM}

\begin{QCM}

$\int\limits_{0}^{2}\left(5 t^{3}+3 t^{2}-t+7\right) d t=$

\begin{enumerate}

\Faux{ $15 t^{2}+6 t-1$}

\Vrai{ 40}

\Faux{ 71}

\end{enumerate}

\end{QCM}

\begin{QCM}

La valeur moyenne de la fonction $g: x \rightarrow e^{5 x} $ entre 0 et 2 est de :

\begin{enumerate}

\Faux{environ $22 025,5$}

\Vrai{environ $2 202,5$}

\Faux{environ $4405,1$}

\end{enumerate}

\end{QCM}

\newpage

\subsection*{Problème}

Alain Raffletou cherche à connaitre le nombre moyen de clients journaliers dans une de ses boutiques sur l'année en cours. Pour cela, il modélise le nombre de clients par une fonction donnant le nombre de clients $c\left(t\right)$ en fonction du temps $t$, exprimé en jours. On prends donc $t\in\left[0;365\right]$.

\begin{QCM}

En remarquant que le nombre de clients ne peut être négatif, déterminé, parmi les propositions ci-dessous, laquelle donne la formule de $c\left(t\right)$ en fonction de $t$ :

\begin{enumerate}

\Vrai{$c\left(t\right)=3\cos\left(\frac{6\pi t}{365}\right)+21$}

\Faux{$c\left(t\right)=3\sin\left(\frac{6\pi t}{365}+21\right)$}

\Faux{$c\left(t\right)=3\sin\left(\frac{6\pi t}{365}\right)-21$}

\Faux{$c\left(t\right)=3\cos\left(\frac{6\pi t}{365}-21\right)$}

\end{enumerate}

\end{QCM}

\begin{QCM}

On a alors que les valeurs de $c\left(t\right)$ sont comprises entre :

\begin{enumerate}

\Vrai{$18$ et $24$}

\Faux{$0$ et $3$ }

\Faux{$0$ et $21$}

\Faux{$0$ et $24$ }

\end{enumerate}

\end{QCM}

Dès lors, il peut calculer le nombre moyen de clients en appliquant la formule

$\frac{1}{365}\int\limits _{0}^{365}C\left(t\right)dt$ , où $C$ désigne une primitive de $c$.

\begin{QCM}

Parmi les propositions ci-dessous, laquelle donne une formule possible C~?

\begin{enumerate}

\Vrai{$C\left(t\right)=\frac{1095}{2}\pi\sin\left(\frac{6\pi t}{365}\right)+21t$}

\Faux{$C\left(t\right)=\frac{18}{365}\pi\sin\left(\frac{6\pi t}{365}\right)-21t+16$}

\Faux{$C\left(t\right)=\frac{18}{365}\pi\cos\left(\frac{6\pi t}{365}\right)-21t-33$}

\Faux{$C\left(t\right)=-3\sin\left(\frac{6\pi t}{365}\right)$}

\end{enumerate}

\end{QCM}

\begin{QCM}

On obtient ainsi que le nombre moyen de clients journaliers sur l'année en cours est donné par :

\begin{enumerate}

\Vrai{$21$}

\Faux{$3$}

\Faux{$21\sin\left(6\pi\right)$ }

\Faux{$3\cos\left(6\pi\right)$}

\end{enumerate}

\end{QCM}

\end{Sujet}

\begin{Sujet}

\subsection*{Exercice 1}

Pour chacune des propositions de cet exercice, cochez A si elle est vraie\par et B si elle est fausse.

On s'intéresse à la fonction f définie par $f:x\mapsto\left(4x+4\right)e^{5x}$.

\begin{VraiFaux}

\Vrai{ L'ensemble de définition de f est $\left]-\infty,+\infty\right[$.}

\end{VraiFaux}

\begin{VraiFaux}

\Vrai{ La dérivée de f est donnée par $f^{\prime}:x\mapsto\left(20x+24\right)e^{5x}$.}

\end{VraiFaux}

On admet que son approximation quadratique est donné par\par $\qquad f\left(x\right)\approx 4 + 20 x + 48 x^2$.

\begin{VraiFaux}

\Vrai{ L'équation de la tangente $T$ à la courbe représentative $\mathscr C$ de $f$ en son point d'abscisse $0$ est donnée par $y=20 x +4 $.}

\end{VraiFaux}

\begin{VraiFaux}

\Vrai{ $T$ est au-dessous de $\mathscr C$.}

\end{VraiFaux}

\subsection*{Exercice 2}

Cet exercice doit être rédigé au verso de la grille réponse.

\subsection*{Exercice 3}

\begin{QCM}

Parmi ces fonctions, laquelle est une primitive de la fonction $f$\par définie par $ f(x)=x \cdot \cos (3 x+1)$ ?

\begin{enumerate}

\Faux{ $G:x \mapsto x \cdot \sin (3 x+1)$}

\Vrai{ $H:x \mapsto \frac{x \cdot \sin (3 x+1)}{3}+\frac{\cos (3 x+1)}{9}$}

\Faux{ $F:x \mapsto \cos (3 x+1)-3 x \cdot \sin (3 x+1)$}

\end{enumerate}

\end{QCM}

\begin{QCM}

$\int\limits_{0}^{2}\left(5 t^{3}+3 t^{2}-t+7\right) d t=$

\begin{enumerate}

\Faux{ $15 t^{2}+6 t-1$}

\Faux{ 71}

\Vrai{ 40}

\end{enumerate}

\end{QCM}

\begin{QCM}

La valeur moyenne de la fonction $g: x \rightarrow e^{5 x} $ entre 0 et 2 est de :

\begin{enumerate}

\Faux{environ $22 025,5$}

\Faux{environ $4405,1$}

\Vrai{environ $2 202,5$}

\end{enumerate}

\end{QCM}

\newpage

\subsection*{Problème}

Alain Raffletou cherche à connaitre le nombre moyen de clients journaliers dans une de ses boutiques sur l'année en cours. Pour cela, il modélise le nombre de clients par une fonction donnant le nombre de clients $c\left(t\right)$ en fonction du temps $t$, exprimé en jours. On prends donc $t\in\left[0;365\right]$.

\begin{QCM}

En remarquant que le nombre de clients ne peut être négatif, déterminé, parmi les propositions ci-dessous, laquelle donne la formule de $c\left(t\right)$ en fonction de $t$ :

\begin{enumerate}

\Faux{$c\left(t\right)=5\sin\left(\frac{6\pi t}{365}+25\right)$}

\Faux{$c\left(t\right)=5\sin\left(\frac{6\pi t}{365}\right)-25$}

\Faux{$c\left(t\right)=5\cos\left(\frac{6\pi t}{365}-25\right)$}

\Vrai{$c\left(t\right)=5\cos\left(\frac{6\pi t}{365}\right)+25$}

\end{enumerate}

\end{QCM}

\begin{QCM}

On a alors que les valeurs de $c\left(t\right)$ sont comprises entre :

\begin{enumerate}

\Faux{$0$ et $25$}

\Faux{$0$ et $5$ }

\Vrai{$20$ et $30$}

\Faux{$0$ et $30$ }

\end{enumerate}

\end{QCM}

Dès lors, il peut calculer le nombre moyen de clients en appliquant la formule

$\frac{1}{365}\int\limits _{0}^{365}C\left(t\right)dt$ , où $C$ désigne une primitive de $c$.

\begin{QCM}

Parmi les propositions ci-dessous, laquelle donne une formule possible C~?

\begin{enumerate}

\Faux{$C\left(t\right)=\frac{30}{365}\pi\cos\left(\frac{6\pi t}{365}\right)-25t-35$}

\Faux{$C\left(t\right)=\frac{30}{365}\pi\sin\left(\frac{6\pi t}{365}\right)-25t+16$}

\Vrai{$C\left(t\right)=\frac{1825}{2}\pi\sin\left(\frac{6\pi t}{365}\right)+25t$}

\Faux{$C\left(t\right)=-5\sin\left(\frac{6\pi t}{365}\right)$}

\end{enumerate}

\end{QCM}

\begin{QCM}

On obtient ainsi que le nombre moyen de clients journaliers sur l'année en cours est donné par :

\begin{enumerate}

\Faux{$25\sin\left(6\pi\right)$ }

\Vrai{$25$}

\Faux{$5\cos\left(6\pi\right)$}

\Faux{$5$}

\end{enumerate}

\end{QCM}

\end{Sujet}

\begin{Sujet}

\subsection*{Exercice 1}

Pour chacune des propositions de cet exercice, cochez A si elle est vraie\par et B si elle est fausse.

On s'intéresse à la fonction f définie par $f:x\mapsto\left(3x+6\right)e^{6x}$.

\begin{VraiFaux}

\Vrai{ L'ensemble de définition de f est $\left]-\infty,+\infty\right[$.}

\end{VraiFaux}

\begin{VraiFaux}

\Vrai{ La dérivée de f est donnée par $f^{\prime}:x\mapsto\left(18x+39\right)e^{6x}$.}

\end{VraiFaux}

On admet que son approximation quadratique est donné par\par $\qquad f\left(x\right)\approx 6 + 21 x + 36 x^2$.

\begin{VraiFaux}

\Faux{ L'équation de la tangente $T$ à la courbe représentative $\mathscr C$ de $f$ en son point d'abscisse $0$ est donnée par $y=6x + 21 $.}

\end{VraiFaux}

\begin{VraiFaux}

\Faux{ $T$ est au-dessus de $\mathscr C$.}

\end{VraiFaux}

\subsection*{Exercice 2}

Cet exercice doit être rédigé au verso de la grille réponse.

\subsection*{Exercice 3}

\begin{QCM}

Parmi ces fonctions, laquelle est une primitive de la fonction $f$\par définie par $ f(x)=x \cdot \cos (3 x+1)$ ?

\begin{enumerate}

\Faux{ $G:x \mapsto x \cdot \sin (3 x+1)$}

\Vrai{ $H:x \mapsto \frac{x \cdot \sin (3 x+1)}{3}+\frac{\cos (3 x+1)}{9}$}

\Faux{ $F:x \mapsto \cos (3 x+1)-3 x \cdot \sin (3 x+1)$}

\end{enumerate}

\end{QCM}

\begin{QCM}

$\int\limits_{0}^{2}\left(5 t^{3}+3 t^{2}-t+7\right) d t=$

\begin{enumerate}

\Faux{ $15 t^{2}+6 t-1$}

\Faux{ 71}

\Vrai{ 40}

\end{enumerate}

\end{QCM}

\begin{QCM}

La valeur moyenne de la fonction $g: x \rightarrow e^{5 x} $ entre 0 et 2 est de :

\begin{enumerate}

\Faux{environ $4405,1$}

\Faux{environ $22 025,5$}

\Vrai{environ $2 202,5$}

\end{enumerate}

\end{QCM}

\newpage

\subsection*{Problème}

Alain Raffletou cherche à connaitre le nombre moyen de clients journaliers dans une de ses boutiques sur l'année en cours. Pour cela, il modélise le nombre de clients par une fonction donnant le nombre de clients $c\left(t\right)$ en fonction du temps $t$, exprimé en jours. On prends donc $t\in\left[0;365\right]$.

\begin{QCM}

En remarquant que le nombre de clients ne peut être négatif, déterminé, parmi les propositions ci-dessous, laquelle donne la formule de $c\left(t\right)$ en fonction de $t$ :

\begin{enumerate}

\Faux{$c\left(t\right)=4\sin\left(\frac{4\pi t}{365}\right)-12$}

\Vrai{$c\left(t\right)=4\cos\left(\frac{4\pi t}{365}\right)+12$}

\Faux{$c\left(t\right)=4\cos\left(\frac{4\pi t}{365}-12\right)$}

\Faux{$c\left(t\right)=4\sin\left(\frac{4\pi t}{365}+12\right)$}

\end{enumerate}

\end{QCM}

\begin{QCM}

On a alors que les valeurs de $c\left(t\right)$ sont comprises entre :

\begin{enumerate}

\Faux{$0$ et $12$}

\Faux{$0$ et $4$ }

\Faux{$0$ et $16$ }

\Vrai{$8$ et $16$}

\end{enumerate}

\end{QCM}

Dès lors, il peut calculer le nombre moyen de clients en appliquant la formule

$\frac{1}{365}\int\limits _{0}^{365}C\left(t\right)dt$ , où $C$ désigne une primitive de $c$.

\begin{QCM}

Parmi les propositions ci-dessous, laquelle donne une formule possible C~?

\begin{enumerate}

\Vrai{$C\left(t\right)=\frac{1460}{2}\pi\sin\left(\frac{4\pi t}{365}\right)+12t$}

\Faux{$C\left(t\right)=\frac{16}{365}\pi\cos\left(\frac{4\pi t}{365}\right)-12t-12$}

\Faux{$C\left(t\right)=\frac{16}{365}\pi\sin\left(\frac{4\pi t}{365}\right)-12t+5$}

\Faux{$C\left(t\right)=-4\sin\left(\frac{4\pi t}{365}\right)$}

\end{enumerate}

\end{QCM}

\begin{QCM}

On obtient ainsi que le nombre moyen de clients journaliers sur l'année en cours est donné par :

\begin{enumerate}

\Vrai{$12$}

\Faux{$4\cos\left(4\pi\right)$}

\Faux{$12\sin\left(4\pi\right)$ }

\Faux{$4$}

\end{enumerate}

\end{QCM}

\end{Sujet}

\begin{Sujet}

\subsection*{Exercice 1}

Pour chacune des propositions de cet exercice, cochez A si elle est vraie\par et B si elle est fausse.

On s'intéresse à la fonction f définie par $f:x\mapsto\left(7x+7\right)e^{6x}$.

\begin{VraiFaux}

\Faux{ L'ensemble de définition de f est $\left]0,+\infty\right[$.}

\end{VraiFaux}

\begin{VraiFaux}

\Faux{ La dérivée de f est donnée par $f^{\prime}:x\mapsto\left(42\right)e^{cx}$.}

\end{VraiFaux}

On admet que son approximation quadratique est donné par\par $\qquad f\left(x\right)\approx 7 + 56 x + \frac{441}{2} x^2$.

\begin{VraiFaux}

\Vrai{ L'équation de la tangente $T$ à la courbe représentative $\mathscr C$ de $f$ en son point d'abscisse $0$ est donnée par $y=56 x +7 $.}

\end{VraiFaux}

\begin{VraiFaux}

\Vrai{ $T$ est au-dessous de $\mathscr C$.}

\end{VraiFaux}

\subsection*{Exercice 2}

Cet exercice doit être rédigé au verso de la grille réponse.

\subsection*{Exercice 3}

\begin{QCM}

Parmi ces fonctions, laquelle est une primitive de la fonction $f$\par définie par $ f(x)=x \cdot \cos (3 x+1)$ ?

\begin{enumerate}

\Faux{ $G:x \mapsto x \cdot \sin (3 x+1)$}

\Faux{ $F:x \mapsto \cos (3 x+1)-3 x \cdot \sin (3 x+1)$}

\Vrai{ $H:x \mapsto \frac{x \cdot \sin (3 x+1)}{3}+\frac{\cos (3 x+1)}{9}$}

\end{enumerate}

\end{QCM}

\begin{QCM}

$\int\limits_{0}^{2}\left(5 t^{3}+3 t^{2}-t+7\right) d t=$

\begin{enumerate}

\Faux{ 71}

\Faux{ $15 t^{2}+6 t-1$}

\Vrai{ 40}

\end{enumerate}

\end{QCM}

\begin{QCM}

La valeur moyenne de la fonction $g: x \rightarrow e^{5 x} $ entre 0 et 2 est de :

\begin{enumerate}

\Faux{environ $22 025,5$}

\Faux{environ $4405,1$}

\Vrai{environ $2 202,5$}

\end{enumerate}

\end{QCM}

\newpage

\subsection*{Problème}

Alain Raffletou cherche à connaitre le nombre moyen de clients journaliers dans une de ses boutiques sur l'année en cours. Pour cela, il modélise le nombre de clients par une fonction donnant le nombre de clients $c\left(t\right)$ en fonction du temps $t$, exprimé en jours. On prends donc $t\in\left[0;365\right]$.

\begin{QCM}

En remarquant que le nombre de clients ne peut être négatif, déterminé, parmi les propositions ci-dessous, laquelle donne la formule de $c\left(t\right)$ en fonction de $t$ :

\begin{enumerate}

\Faux{$c\left(t\right)=4\sin\left(\frac{4\pi t}{365}+28\right)$}

\Vrai{$c\left(t\right)=4\cos\left(\frac{4\pi t}{365}\right)+28$}

\Faux{$c\left(t\right)=4\sin\left(\frac{4\pi t}{365}\right)-28$}

\Faux{$c\left(t\right)=4\cos\left(\frac{4\pi t}{365}-28\right)$}

\end{enumerate}

\end{QCM}

\begin{QCM}

On a alors que les valeurs de $c\left(t\right)$ sont comprises entre :

\begin{enumerate}

\Vrai{$24$ et $32$}

\Faux{$0$ et $32$ }

\Faux{$0$ et $4$ }

\Faux{$0$ et $28$}

\end{enumerate}

\end{QCM}

Dès lors, il peut calculer le nombre moyen de clients en appliquant la formule

$\frac{1}{365}\int\limits _{0}^{365}C\left(t\right)dt$ , où $C$ désigne une primitive de $c$.

\begin{QCM}

Parmi les propositions ci-dessous, laquelle donne une formule possible C~?

\begin{enumerate}

\Faux{$C\left(t\right)=\frac{16}{365}\pi\sin\left(\frac{4\pi t}{365}\right)-28t+21$}

\Faux{$C\left(t\right)=-4\sin\left(\frac{4\pi t}{365}\right)$}

\Faux{$C\left(t\right)=\frac{16}{365}\pi\cos\left(\frac{4\pi t}{365}\right)-28t-44$}

\Vrai{$C\left(t\right)=\frac{1460}{2}\pi\sin\left(\frac{4\pi t}{365}\right)+28t$}

\end{enumerate}

\end{QCM}

\begin{QCM}

On obtient ainsi que le nombre moyen de clients journaliers sur l'année en cours est donné par :

\begin{enumerate}

\Faux{$4$}

\Faux{$4\cos\left(4\pi\right)$}

\Vrai{$28$}

\Faux{$28\sin\left(4\pi\right)$ }

\end{enumerate}

\end{QCM}

\end{Sujet}

\begin{Sujet}

\subsection*{Exercice 1}

Pour chacune des propositions de cet exercice, cochez A si elle est vraie\par et B si elle est fausse.

On s'intéresse à la fonction f définie par $f:x\mapsto\left(6x+2\right)e^{7x}$.

\begin{VraiFaux}

\Faux{ L'ensemble de définition de f est $\left]0,+\infty\right[$.}

\end{VraiFaux}

\begin{VraiFaux}

\Faux{ La dérivée de f est donnée par $f^{\prime}:x\mapsto\left(42\right)e^{cx}$.}

\end{VraiFaux}

On admet que son approximation quadratique est donné par\par $\qquad f\left(x\right)\approx 2 + 18 x + 72 x^2$.

\begin{VraiFaux}

\Faux{ L'équation de la tangente $T$ à la courbe représentative $\mathscr C$ de $f$ en son point d'abscisse $0$ est donnée par $y=2x + 18 $.}

\end{VraiFaux}

\begin{VraiFaux}

\Faux{ $T$ est au-dessus de $\mathscr C$.}

\end{VraiFaux}

\subsection*{Exercice 2}

Cet exercice doit être rédigé au verso de la grille réponse.

\subsection*{Exercice 3}

\begin{QCM}

Parmi ces fonctions, laquelle est une primitive de la fonction $f$\par définie par $ f(x)=x \cdot \cos (3 x+1)$ ?

\begin{enumerate}

\Faux{ $G:x \mapsto x \cdot \sin (3 x+1)$}

\Vrai{ $H:x \mapsto \frac{x \cdot \sin (3 x+1)}{3}+\frac{\cos (3 x+1)}{9}$}

\Faux{ $F:x \mapsto \cos (3 x+1)-3 x \cdot \sin (3 x+1)$}

\end{enumerate}

\end{QCM}

\begin{QCM}

$\int\limits_{0}^{2}\left(5 t^{3}+3 t^{2}-t+7\right) d t=$

\begin{enumerate}

\Faux{ 71}

\Faux{ $15 t^{2}+6 t-1$}

\Vrai{ 40}

\end{enumerate}

\end{QCM}

\begin{QCM}

La valeur moyenne de la fonction $g: x \rightarrow e^{5 x} $ entre 0 et 2 est de :

\begin{enumerate}

\Faux{environ $22 025,5$}

\Vrai{environ $2 202,5$}

\Faux{environ $4405,1$}

\end{enumerate}

\end{QCM}

\newpage

\subsection*{Problème}

Alain Raffletou cherche à connaitre le nombre moyen de clients journaliers dans une de ses boutiques sur l'année en cours. Pour cela, il modélise le nombre de clients par une fonction donnant le nombre de clients $c\left(t\right)$ en fonction du temps $t$, exprimé en jours. On prends donc $t\in\left[0;365\right]$.

\begin{QCM}

En remarquant que le nombre de clients ne peut être négatif, déterminé, parmi les propositions ci-dessous, laquelle donne la formule de $c\left(t\right)$ en fonction de $t$ :

\begin{enumerate}

\Vrai{$c\left(t\right)=3\cos\left(\frac{6\pi t}{365}\right)+27$}

\Faux{$c\left(t\right)=3\cos\left(\frac{6\pi t}{365}-27\right)$}

\Faux{$c\left(t\right)=3\sin\left(\frac{6\pi t}{365}+27\right)$}

\Faux{$c\left(t\right)=3\sin\left(\frac{6\pi t}{365}\right)-27$}

\end{enumerate}

\end{QCM}

\begin{QCM}

On a alors que les valeurs de $c\left(t\right)$ sont comprises entre :

\begin{enumerate}

\Vrai{$24$ et $30$}

\Faux{$0$ et $30$ }

\Faux{$0$ et $27$}

\Faux{$0$ et $3$ }

\end{enumerate}

\end{QCM}

Dès lors, il peut calculer le nombre moyen de clients en appliquant la formule

$\frac{1}{365}\int\limits _{0}^{365}C\left(t\right)dt$ , où $C$ désigne une primitive de $c$.

\begin{QCM}

Parmi les propositions ci-dessous, laquelle donne une formule possible C~?

\begin{enumerate}

\Faux{$C\left(t\right)=\frac{18}{365}\pi\sin\left(\frac{6\pi t}{365}\right)-27t+22$}

\Vrai{$C\left(t\right)=\frac{1095}{2}\pi\sin\left(\frac{6\pi t}{365}\right)+27t$}

\Faux{$C\left(t\right)=-3\sin\left(\frac{6\pi t}{365}\right)$}

\Faux{$C\left(t\right)=\frac{18}{365}\pi\cos\left(\frac{6\pi t}{365}\right)-27t-45$}

\end{enumerate}

\end{QCM}

\begin{QCM}

On obtient ainsi que le nombre moyen de clients journaliers sur l'année en cours est donné par :

\begin{enumerate}

\Faux{$3\cos\left(6\pi\right)$}

\Faux{$3$}

\Vrai{$27$}

\Faux{$27\sin\left(6\pi\right)$ }

\end{enumerate}

\end{QCM}

\end{Sujet}

\begin{Sujet}

\subsection*{Exercice 1}

Pour chacune des propositions de cet exercice, cochez A si elle est vraie\par et B si elle est fausse.

On s'intéresse à la fonction f définie par $f:x\mapsto\left(2x+5\right)e^{2x}$.

\begin{VraiFaux}

\Vrai{ L'ensemble de définition de f est $\left]-\infty,+\infty\right[$.}

\end{VraiFaux}

\begin{VraiFaux}

\Vrai{ La dérivée de f est donnée par $f^{\prime}:x\mapsto\left(4x+12\right)e^{2x}$.}

\end{VraiFaux}

On admet que son approximation quadratique est donné par\par $\qquad f\left(x\right)\approx 5 + 12 x + 14 x^2$.

\begin{VraiFaux}

\Vrai{ L'équation de la tangente $T$ à la courbe représentative $\mathscr C$ de $f$ en son point d'abscisse $0$ est donnée par $y=12 x +5 $.}

\end{VraiFaux}

\begin{VraiFaux}

\Vrai{ $T$ est au-dessous de $\mathscr C$.}

\end{VraiFaux}

\subsection*{Exercice 2}

Cet exercice doit être rédigé au verso de la grille réponse.

\subsection*{Exercice 3}

\begin{QCM}

Parmi ces fonctions, laquelle est une primitive de la fonction $f$\par définie par $ f(x)=x \cdot \cos (3 x+1)$ ?

\begin{enumerate}

\Faux{ $F:x \mapsto \cos (3 x+1)-3 x \cdot \sin (3 x+1)$}

\Vrai{ $H:x \mapsto \frac{x \cdot \sin (3 x+1)}{3}+\frac{\cos (3 x+1)}{9}$}

\Faux{ $G:x \mapsto x \cdot \sin (3 x+1)$}

\end{enumerate}

\end{QCM}

\begin{QCM}

$\int\limits_{0}^{2}\left(5 t^{3}+3 t^{2}-t+7\right) d t=$

\begin{enumerate}

\Vrai{ 40}

\Faux{ 71}

\Faux{ $15 t^{2}+6 t-1$}

\end{enumerate}

\end{QCM}

\begin{QCM}

La valeur moyenne de la fonction $g: x \rightarrow e^{5 x} $ entre 0 et 2 est de :

\begin{enumerate}

\Faux{environ $22 025,5$}

\Vrai{environ $2 202,5$}

\Faux{environ $4405,1$}

\end{enumerate}

\end{QCM}

\newpage

\subsection*{Problème}

Alain Raffletou cherche à connaitre le nombre moyen de clients journaliers dans une de ses boutiques sur l'année en cours. Pour cela, il modélise le nombre de clients par une fonction donnant le nombre de clients $c\left(t\right)$ en fonction du temps $t$, exprimé en jours. On prends donc $t\in\left[0;365\right]$.

\begin{QCM}

En remarquant que le nombre de clients ne peut être négatif, déterminé, parmi les propositions ci-dessous, laquelle donne la formule de $c\left(t\right)$ en fonction de $t$ :

\begin{enumerate}

\Faux{$c\left(t\right)=3\sin\left(\frac{4\pi t}{365}+29\right)$}

\Faux{$c\left(t\right)=3\cos\left(\frac{4\pi t}{365}-29\right)$}

\Vrai{$c\left(t\right)=3\cos\left(\frac{4\pi t}{365}\right)+29$}

\Faux{$c\left(t\right)=3\sin\left(\frac{4\pi t}{365}\right)-29$}

\end{enumerate}

\end{QCM}

\begin{QCM}

On a alors que les valeurs de $c\left(t\right)$ sont comprises entre :

\begin{enumerate}

\Vrai{$26$ et $32$}

\Faux{$0$ et $32$ }

\Faux{$0$ et $3$ }

\Faux{$0$ et $29$}

\end{enumerate}

\end{QCM}

Dès lors, il peut calculer le nombre moyen de clients en appliquant la formule

$\frac{1}{365}\int\limits _{0}^{365}C\left(t\right)dt$ , où $C$ désigne une primitive de $c$.

\begin{QCM}

Parmi les propositions ci-dessous, laquelle donne une formule possible C~?

\begin{enumerate}

\Faux{$C\left(t\right)=\frac{12}{365}\pi\cos\left(\frac{4\pi t}{365}\right)-29t-49$}

\Faux{$C\left(t\right)=\frac{12}{365}\pi\sin\left(\frac{4\pi t}{365}\right)-29t+24$}

\Faux{$C\left(t\right)=-3\sin\left(\frac{4\pi t}{365}\right)$}

\Vrai{$C\left(t\right)=\frac{1095}{2}\pi\sin\left(\frac{4\pi t}{365}\right)+29t$}

\end{enumerate}

\end{QCM}

\begin{QCM}

On obtient ainsi que le nombre moyen de clients journaliers sur l'année en cours est donné par :

\begin{enumerate}

\Faux{$29\sin\left(4\pi\right)$ }

\Vrai{$29$}

\Faux{$3\cos\left(4\pi\right)$}

\Faux{$3$}

\end{enumerate}

\end{QCM}

\end{Sujet}

\begin{Sujet}

\subsection*{Exercice 1}

Pour chacune des propositions de cet exercice, cochez A si elle est vraie\par et B si elle est fausse.

On s'intéresse à la fonction f définie par $f:x\mapsto\left(4x+5\right)e^{6x}$.

\begin{VraiFaux}

\Vrai{ L'ensemble de définition de f est $\left]-\infty,+\infty\right[$.}

\end{VraiFaux}

\begin{VraiFaux}

\Faux{ La dérivée de f est donnée par $f^{\prime}:x\mapsto\left(24\right)e^{cx}$.}

\end{VraiFaux}

On admet que son approximation quadratique est donné par\par $\qquad f\left(x\right)\approx 5 + 24 x + 56 x^2$.

\begin{VraiFaux}

\Faux{ L'équation de la tangente $T$ à la courbe représentative $\mathscr C$ de $f$ en son point d'abscisse $0$ est donnée par $y=5x + 24 $.}

\end{VraiFaux}

\begin{VraiFaux}

\Faux{ $T$ est au-dessus de $\mathscr C$.}

\end{VraiFaux}

\subsection*{Exercice 2}

Cet exercice doit être rédigé au verso de la grille réponse.

\subsection*{Exercice 3}

\begin{QCM}

Parmi ces fonctions, laquelle est une primitive de la fonction $f$\par définie par $ f(x)=x \cdot \cos (3 x+1)$ ?

\begin{enumerate}

\Vrai{ $H:x \mapsto \frac{x \cdot \sin (3 x+1)}{3}+\frac{\cos (3 x+1)}{9}$}

\Faux{ $F:x \mapsto \cos (3 x+1)-3 x \cdot \sin (3 x+1)$}

\Faux{ $G:x \mapsto x \cdot \sin (3 x+1)$}

\end{enumerate}

\end{QCM}

\begin{QCM}

$\int\limits_{0}^{2}\left(5 t^{3}+3 t^{2}-t+7\right) d t=$

\begin{enumerate}

\Vrai{ 40}

\Faux{ 71}

\Faux{ $15 t^{2}+6 t-1$}

\end{enumerate}

\end{QCM}

\begin{QCM}

La valeur moyenne de la fonction $g: x \rightarrow e^{5 x} $ entre 0 et 2 est de :

\begin{enumerate}

\Vrai{environ $2 202,5$}

\Faux{environ $4405,1$}

\Faux{environ $22 025,5$}

\end{enumerate}

\end{QCM}

\newpage

\subsection*{Problème}

Alain Raffletou cherche à connaitre le nombre moyen de clients journaliers dans une de ses boutiques sur l'année en cours. Pour cela, il modélise le nombre de clients par une fonction donnant le nombre de clients $c\left(t\right)$ en fonction du temps $t$, exprimé en jours. On prends donc $t\in\left[0;365\right]$.

\begin{QCM}

En remarquant que le nombre de clients ne peut être négatif, déterminé, parmi les propositions ci-dessous, laquelle donne la formule de $c\left(t\right)$ en fonction de $t$ :

\begin{enumerate}

\Faux{$c\left(t\right)=3\sin\left(\frac{4\pi t}{365}+16\right)$}

\Faux{$c\left(t\right)=3\sin\left(\frac{4\pi t}{365}\right)-16$}

\Faux{$c\left(t\right)=3\cos\left(\frac{4\pi t}{365}-16\right)$}

\Vrai{$c\left(t\right)=3\cos\left(\frac{4\pi t}{365}\right)+16$}

\end{enumerate}

\end{QCM}

\begin{QCM}

On a alors que les valeurs de $c\left(t\right)$ sont comprises entre :

\begin{enumerate}

\Faux{$0$ et $16$}

\Faux{$0$ et $3$ }

\Faux{$0$ et $19$ }

\Vrai{$13$ et $19$}

\end{enumerate}

\end{QCM}

Dès lors, il peut calculer le nombre moyen de clients en appliquant la formule

$\frac{1}{365}\int\limits _{0}^{365}C\left(t\right)dt$ , où $C$ désigne une primitive de $c$.

\begin{QCM}

Parmi les propositions ci-dessous, laquelle donne une formule possible C~?

\begin{enumerate}

\Vrai{$C\left(t\right)=\frac{1095}{2}\pi\sin\left(\frac{4\pi t}{365}\right)+16t$}

\Faux{$C\left(t\right)=\frac{12}{365}\pi\sin\left(\frac{4\pi t}{365}\right)-16t+11$}

\Faux{$C\left(t\right)=\frac{12}{365}\pi\cos\left(\frac{4\pi t}{365}\right)-16t-23$}

\Faux{$C\left(t\right)=-3\sin\left(\frac{4\pi t}{365}\right)$}

\end{enumerate}

\end{QCM}

\begin{QCM}

On obtient ainsi que le nombre moyen de clients journaliers sur l'année en cours est donné par :

\begin{enumerate}

\Faux{$16\sin\left(4\pi\right)$ }

\Faux{$3\cos\left(4\pi\right)$}

\Faux{$3$}

\Vrai{$16$}

\end{enumerate}

\end{QCM}

\end{Sujet}

\begin{Sujet}

\subsection*{Exercice 1}

Pour chacune des propositions de cet exercice, cochez A si elle est vraie\par et B si elle est fausse.

On s'intéresse à la fonction f définie par $f:x\mapsto\left(7x+6\right)e^{6x}$.

\begin{VraiFaux}

\Vrai{ L'ensemble de définition de f est $\left]-\infty,+\infty\right[$.}

\end{VraiFaux}

\begin{VraiFaux}

\Vrai{ La dérivée de f est donnée par $f^{\prime}:x\mapsto\left(42x+43\right)e^{6x}$.}

\end{VraiFaux}

On admet que son approximation quadratique est donné par\par $\qquad f\left(x\right)\approx 6 + 49 x + 196 x^2$.

\begin{VraiFaux}

\Vrai{ L'équation de la tangente $T$ à la courbe représentative $\mathscr C$ de $f$ en son point d'abscisse $0$ est donnée par $y=49 x +6 $.}

\end{VraiFaux}

\begin{VraiFaux}

\Vrai{ $T$ est au-dessous de $\mathscr C$.}

\end{VraiFaux}

\subsection*{Exercice 2}

Cet exercice doit être rédigé au verso de la grille réponse.

\subsection*{Exercice 3}

\begin{QCM}

Parmi ces fonctions, laquelle est une primitive de la fonction $f$\par définie par $ f(x)=x \cdot \cos (3 x+1)$ ?

\begin{enumerate}

\Vrai{ $H:x \mapsto \frac{x \cdot \sin (3 x+1)}{3}+\frac{\cos (3 x+1)}{9}$}

\Faux{ $F:x \mapsto \cos (3 x+1)-3 x \cdot \sin (3 x+1)$}

\Faux{ $G:x \mapsto x \cdot \sin (3 x+1)$}

\end{enumerate}

\end{QCM}

\begin{QCM}

$\int\limits_{0}^{2}\left(5 t^{3}+3 t^{2}-t+7\right) d t=$

\begin{enumerate}

\Vrai{ 40}

\Faux{ 71}

\Faux{ $15 t^{2}+6 t-1$}

\end{enumerate}

\end{QCM}

\begin{QCM}

La valeur moyenne de la fonction $g: x \rightarrow e^{5 x} $ entre 0 et 2 est de :

\begin{enumerate}

\Faux{environ $4405,1$}

\Vrai{environ $2 202,5$}

\Faux{environ $22 025,5$}

\end{enumerate}

\end{QCM}

\newpage

\subsection*{Problème}

Alain Raffletou cherche à connaitre le nombre moyen de clients journaliers dans une de ses boutiques sur l'année en cours. Pour cela, il modélise le nombre de clients par une fonction donnant le nombre de clients $c\left(t\right)$ en fonction du temps $t$, exprimé en jours. On prends donc $t\in\left[0;365\right]$.

\begin{QCM}

En remarquant que le nombre de clients ne peut être négatif, déterminé, parmi les propositions ci-dessous, laquelle donne la formule de $c\left(t\right)$ en fonction de $t$ :

\begin{enumerate}

\Faux{$c\left(t\right)=5\sin\left(\frac{4\pi t}{365}\right)-17$}

\Faux{$c\left(t\right)=5\sin\left(\frac{4\pi t}{365}+17\right)$}

\Faux{$c\left(t\right)=5\cos\left(\frac{4\pi t}{365}-17\right)$}

\Vrai{$c\left(t\right)=5\cos\left(\frac{4\pi t}{365}\right)+17$}

\end{enumerate}

\end{QCM}

\begin{QCM}

On a alors que les valeurs de $c\left(t\right)$ sont comprises entre :

\begin{enumerate}

\Faux{$0$ et $5$ }

\Faux{$0$ et $17$}

\Faux{$0$ et $22$ }

\Vrai{$12$ et $22$}

\end{enumerate}

\end{QCM}

Dès lors, il peut calculer le nombre moyen de clients en appliquant la formule

$\frac{1}{365}\int\limits _{0}^{365}C\left(t\right)dt$ , où $C$ désigne une primitive de $c$.

\begin{QCM}

Parmi les propositions ci-dessous, laquelle donne une formule possible C~?

\begin{enumerate}

\Faux{$C\left(t\right)=\frac{20}{365}\pi\sin\left(\frac{4\pi t}{365}\right)-17t+8$}

\Vrai{$C\left(t\right)=\frac{1825}{2}\pi\sin\left(\frac{4\pi t}{365}\right)+17t$}

\Faux{$C\left(t\right)=\frac{20}{365}\pi\cos\left(\frac{4\pi t}{365}\right)-17t-19$}

\Faux{$C\left(t\right)=-5\sin\left(\frac{4\pi t}{365}\right)$}

\end{enumerate}

\end{QCM}

\begin{QCM}

On obtient ainsi que le nombre moyen de clients journaliers sur l'année en cours est donné par :

\begin{enumerate}

\Vrai{$17$}

\Faux{$5$}

\Faux{$5\cos\left(4\pi\right)$}

\Faux{$17\sin\left(4\pi\right)$ }

\end{enumerate}

\end{QCM}

\end{Sujet}

\begin{Sujet}

\subsection*{Exercice 1}

Pour chacune des propositions de cet exercice, cochez A si elle est vraie\par et B si elle est fausse.

On s'intéresse à la fonction f définie par $f:x\mapsto\left(4x+2\right)e^{3x}$.

\begin{VraiFaux}

\Faux{ L'ensemble de définition de f est $\left]0,+\infty\right[$.}

\end{VraiFaux}

\begin{VraiFaux}

\Faux{ La dérivée de f est donnée par $f^{\prime}:x\mapsto\left(12\right)e^{cx}$.}

\end{VraiFaux}

On admet que son approximation quadratique est donné par\par $\qquad f\left(x\right)\approx 2 + 12 x + 32 x^2$.

\begin{VraiFaux}

\Vrai{ L'équation de la tangente $T$ à la courbe représentative $\mathscr C$ de $f$ en son point d'abscisse $0$ est donnée par $y=12 x +2 $.}

\end{VraiFaux}

\begin{VraiFaux}

\Vrai{ $T$ est au-dessous de $\mathscr C$.}

\end{VraiFaux}

\subsection*{Exercice 2}

Cet exercice doit être rédigé au verso de la grille réponse.

\subsection*{Exercice 3}

\begin{QCM}

Parmi ces fonctions, laquelle est une primitive de la fonction $f$\par définie par $ f(x)=x \cdot \cos (3 x+1)$ ?

\begin{enumerate}

\Vrai{ $H:x \mapsto \frac{x \cdot \sin (3 x+1)}{3}+\frac{\cos (3 x+1)}{9}$}

\Faux{ $F:x \mapsto \cos (3 x+1)-3 x \cdot \sin (3 x+1)$}

\Faux{ $G:x \mapsto x \cdot \sin (3 x+1)$}

\end{enumerate}

\end{QCM}

\begin{QCM}

$\int\limits_{0}^{2}\left(5 t^{3}+3 t^{2}-t+7\right) d t=$

\begin{enumerate}

\Vrai{ 40}

\Faux{ 71}

\Faux{ $15 t^{2}+6 t-1$}

\end{enumerate}

\end{QCM}

\begin{QCM}

La valeur moyenne de la fonction $g: x \rightarrow e^{5 x} $ entre 0 et 2 est de :

\begin{enumerate}

\Faux{environ $22 025,5$}

\Faux{environ $4405,1$}

\Vrai{environ $2 202,5$}

\end{enumerate}

\end{QCM}

\newpage

\subsection*{Problème}

Alain Raffletou cherche à connaitre le nombre moyen de clients journaliers dans une de ses boutiques sur l'année en cours. Pour cela, il modélise le nombre de clients par une fonction donnant le nombre de clients $c\left(t\right)$ en fonction du temps $t$, exprimé en jours. On prends donc $t\in\left[0;365\right]$.

\begin{QCM}

En remarquant que le nombre de clients ne peut être négatif, déterminé, parmi les propositions ci-dessous, laquelle donne la formule de $c\left(t\right)$ en fonction de $t$ :

\begin{enumerate}

\Faux{$c\left(t\right)=3\cos\left(\frac{6\pi t}{365}-25\right)$}

\Faux{$c\left(t\right)=3\sin\left(\frac{6\pi t}{365}+25\right)$}

\Vrai{$c\left(t\right)=3\cos\left(\frac{6\pi t}{365}\right)+25$}

\Faux{$c\left(t\right)=3\sin\left(\frac{6\pi t}{365}\right)-25$}

\end{enumerate}

\end{QCM}

\begin{QCM}

On a alors que les valeurs de $c\left(t\right)$ sont comprises entre :

\begin{enumerate}

\Faux{$0$ et $25$}

\Faux{$0$ et $28$ }

\Vrai{$22$ et $28$}

\Faux{$0$ et $3$ }

\end{enumerate}

\end{QCM}

Dès lors, il peut calculer le nombre moyen de clients en appliquant la formule

$\frac{1}{365}\int\limits _{0}^{365}C\left(t\right)dt$ , où $C$ désigne une primitive de $c$.

\begin{QCM}

Parmi les propositions ci-dessous, laquelle donne une formule possible C~?

\begin{enumerate}

\Faux{$C\left(t\right)=\frac{18}{365}\pi\cos\left(\frac{6\pi t}{365}\right)-25t-41$}

\Faux{$C\left(t\right)=-3\sin\left(\frac{6\pi t}{365}\right)$}

\Faux{$C\left(t\right)=\frac{18}{365}\pi\sin\left(\frac{6\pi t}{365}\right)-25t+20$}

\Vrai{$C\left(t\right)=\frac{1095}{2}\pi\sin\left(\frac{6\pi t}{365}\right)+25t$}

\end{enumerate}

\end{QCM}

\begin{QCM}

On obtient ainsi que le nombre moyen de clients journaliers sur l'année en cours est donné par :

\begin{enumerate}

\Vrai{$25$}

\Faux{$3\cos\left(6\pi\right)$}

\Faux{$3$}

\Faux{$25\sin\left(6\pi\right)$ }

\end{enumerate}

\end{QCM}

\end{Sujet}

\begin{Sujet}

\subsection*{Exercice 1}

Pour chacune des propositions de cet exercice, cochez A si elle est vraie\par et B si elle est fausse.

On s'intéresse à la fonction f définie par $f:x\mapsto\left(5x+7\right)e^{6x}$.

\begin{VraiFaux}

\Vrai{ L'ensemble de définition de f est $\left]-\infty,+\infty\right[$.}

\end{VraiFaux}

\begin{VraiFaux}

\Faux{ La dérivée de f est donnée par $f^{\prime}:x\mapsto\left(30\right)e^{cx}$.}

\end{VraiFaux}

On admet que son approximation quadratique est donné par\par $\qquad f\left(x\right)\approx 7 + 40 x + \frac{225}{2} x^2$.

\begin{VraiFaux}

\Faux{ L'équation de la tangente $T$ à la courbe représentative $\mathscr C$ de $f$ en son point d'abscisse $0$ est donnée par $y=7x + 40 $.}

\end{VraiFaux}

\begin{VraiFaux}

\Faux{ $T$ est au-dessus de $\mathscr C$.}

\end{VraiFaux}

\subsection*{Exercice 2}

Cet exercice doit être rédigé au verso de la grille réponse.

\subsection*{Exercice 3}

\begin{QCM}

Parmi ces fonctions, laquelle est une primitive de la fonction $f$\par définie par $ f(x)=x \cdot \cos (3 x+1)$ ?

\begin{enumerate}

\Faux{ $F:x \mapsto \cos (3 x+1)-3 x \cdot \sin (3 x+1)$}

\Faux{ $G:x \mapsto x \cdot \sin (3 x+1)$}

\Vrai{ $H:x \mapsto \frac{x \cdot \sin (3 x+1)}{3}+\frac{\cos (3 x+1)}{9}$}

\end{enumerate}

\end{QCM}

\begin{QCM}

$\int\limits_{0}^{2}\left(5 t^{3}+3 t^{2}-t+7\right) d t=$

\begin{enumerate}

\Vrai{ 40}

\Faux{ 71}

\Faux{ $15 t^{2}+6 t-1$}

\end{enumerate}

\end{QCM}

\begin{QCM}

La valeur moyenne de la fonction $g: x \rightarrow e^{5 x} $ entre 0 et 2 est de :

\begin{enumerate}

\Vrai{environ $2 202,5$}

\Faux{environ $4405,1$}

\Faux{environ $22 025,5$}

\end{enumerate}

\end{QCM}

\newpage

\subsection*{Problème}

Alain Raffletou cherche à connaitre le nombre moyen de clients journaliers dans une de ses boutiques sur l'année en cours. Pour cela, il modélise le nombre de clients par une fonction donnant le nombre de clients $c\left(t\right)$ en fonction du temps $t$, exprimé en jours. On prends donc $t\in\left[0;365\right]$.

\begin{QCM}

En remarquant que le nombre de clients ne peut être négatif, déterminé, parmi les propositions ci-dessous, laquelle donne la formule de $c\left(t\right)$ en fonction de $t$ :

\begin{enumerate}

\Faux{$c\left(t\right)=5\sin\left(\frac{4\pi t}{365}\right)-17$}

\Vrai{$c\left(t\right)=5\cos\left(\frac{4\pi t}{365}\right)+17$}

\Faux{$c\left(t\right)=5\cos\left(\frac{4\pi t}{365}-17\right)$}

\Faux{$c\left(t\right)=5\sin\left(\frac{4\pi t}{365}+17\right)$}

\end{enumerate}

\end{QCM}

\begin{QCM}

On a alors que les valeurs de $c\left(t\right)$ sont comprises entre :

\begin{enumerate}

\Vrai{$12$ et $22$}

\Faux{$0$ et $17$}

\Faux{$0$ et $5$ }

\Faux{$0$ et $22$ }

\end{enumerate}

\end{QCM}

Dès lors, il peut calculer le nombre moyen de clients en appliquant la formule

$\frac{1}{365}\int\limits _{0}^{365}C\left(t\right)dt$ , où $C$ désigne une primitive de $c$.

\begin{QCM}

Parmi les propositions ci-dessous, laquelle donne une formule possible C~?

\begin{enumerate}

\Faux{$C\left(t\right)=-5\sin\left(\frac{4\pi t}{365}\right)$}

\Vrai{$C\left(t\right)=\frac{1825}{2}\pi\sin\left(\frac{4\pi t}{365}\right)+17t$}

\Faux{$C\left(t\right)=\frac{20}{365}\pi\cos\left(\frac{4\pi t}{365}\right)-17t-19$}

\Faux{$C\left(t\right)=\frac{20}{365}\pi\sin\left(\frac{4\pi t}{365}\right)-17t+8$}

\end{enumerate}

\end{QCM}

\begin{QCM}

On obtient ainsi que le nombre moyen de clients journaliers sur l'année en cours est donné par :

\begin{enumerate}

\Faux{$5\cos\left(4\pi\right)$}

\Faux{$17\sin\left(4\pi\right)$ }

\Faux{$5$}

\Vrai{$17$}

\end{enumerate}

\end{QCM}

\end{Sujet}

\begin{Sujet}

\subsection*{Exercice 1}

Pour chacune des propositions de cet exercice, cochez A si elle est vraie\par et B si elle est fausse.

On s'intéresse à la fonction f définie par $f:x\mapsto\left(5x+6\right)e^{3x}$.

\begin{VraiFaux}

\Faux{ L'ensemble de définition de f est $\left]0,+\infty\right[$.}

\end{VraiFaux}

\begin{VraiFaux}

\Vrai{ La dérivée de f est donnée par $f^{\prime}:x\mapsto\left(15x+23\right)e^{3x}$.}

\end{VraiFaux}

On admet que son approximation quadratique est donné par\par $\qquad f\left(x\right)\approx 6 + 35 x + 100 x^2$.

\begin{VraiFaux}

\Faux{ L'équation de la tangente $T$ à la courbe représentative $\mathscr C$ de $f$ en son point d'abscisse $0$ est donnée par $y=6x + 35 $.}

\end{VraiFaux}

\begin{VraiFaux}

\Faux{ $T$ est au-dessus de $\mathscr C$.}

\end{VraiFaux}

\subsection*{Exercice 2}

Cet exercice doit être rédigé au verso de la grille réponse.

\subsection*{Exercice 3}

\begin{QCM}

Parmi ces fonctions, laquelle est une primitive de la fonction $f$\par définie par $ f(x)=x \cdot \cos (3 x+1)$ ?

\begin{enumerate}

\Faux{ $G:x \mapsto x \cdot \sin (3 x+1)$}

\Faux{ $F:x \mapsto \cos (3 x+1)-3 x \cdot \sin (3 x+1)$}

\Vrai{ $H:x \mapsto \frac{x \cdot \sin (3 x+1)}{3}+\frac{\cos (3 x+1)}{9}$}

\end{enumerate}

\end{QCM}

\begin{QCM}

$\int\limits_{0}^{2}\left(5 t^{3}+3 t^{2}-t+7\right) d t=$

\begin{enumerate}

\Faux{ 71}

\Vrai{ 40}

\Faux{ $15 t^{2}+6 t-1$}

\end{enumerate}

\end{QCM}

\begin{QCM}

La valeur moyenne de la fonction $g: x \rightarrow e^{5 x} $ entre 0 et 2 est de :

\begin{enumerate}

\Faux{environ $22 025,5$}

\Faux{environ $4405,1$}

\Vrai{environ $2 202,5$}

\end{enumerate}

\end{QCM}

\newpage

\subsection*{Problème}

Alain Raffletou cherche à connaitre le nombre moyen de clients journaliers dans une de ses boutiques sur l'année en cours. Pour cela, il modélise le nombre de clients par une fonction donnant le nombre de clients $c\left(t\right)$ en fonction du temps $t$, exprimé en jours. On prends donc $t\in\left[0;365\right]$.

\begin{QCM}

En remarquant que le nombre de clients ne peut être négatif, déterminé, parmi les propositions ci-dessous, laquelle donne la formule de $c\left(t\right)$ en fonction de $t$ :

\begin{enumerate}

\Faux{$c\left(t\right)=3\cos\left(\frac{6\pi t}{365}-28\right)$}

\Vrai{$c\left(t\right)=3\cos\left(\frac{6\pi t}{365}\right)+28$}

\Faux{$c\left(t\right)=3\sin\left(\frac{6\pi t}{365}\right)-28$}

\Faux{$c\left(t\right)=3\sin\left(\frac{6\pi t}{365}+28\right)$}

\end{enumerate}

\end{QCM}

\begin{QCM}

On a alors que les valeurs de $c\left(t\right)$ sont comprises entre :

\begin{enumerate}

\Vrai{$25$ et $31$}

\Faux{$0$ et $31$ }

\Faux{$0$ et $28$}

\Faux{$0$ et $3$ }

\end{enumerate}

\end{QCM}

Dès lors, il peut calculer le nombre moyen de clients en appliquant la formule

$\frac{1}{365}\int\limits _{0}^{365}C\left(t\right)dt$ , où $C$ désigne une primitive de $c$.

\begin{QCM}

Parmi les propositions ci-dessous, laquelle donne une formule possible C~?

\begin{enumerate}

\Faux{$C\left(t\right)=\frac{18}{365}\pi\cos\left(\frac{6\pi t}{365}\right)-28t-47$}

\Vrai{$C\left(t\right)=\frac{1095}{2}\pi\sin\left(\frac{6\pi t}{365}\right)+28t$}

\Faux{$C\left(t\right)=\frac{18}{365}\pi\sin\left(\frac{6\pi t}{365}\right)-28t+23$}

\Faux{$C\left(t\right)=-3\sin\left(\frac{6\pi t}{365}\right)$}

\end{enumerate}

\end{QCM}

\begin{QCM}

On obtient ainsi que le nombre moyen de clients journaliers sur l'année en cours est donné par :

\begin{enumerate}

\Vrai{$28$}

\Faux{$3\cos\left(6\pi\right)$}

\Faux{$3$}

\Faux{$28\sin\left(6\pi\right)$ }

\end{enumerate}

\end{QCM}

\end{Sujet}

\begin{Sujet}

\subsection*{Exercice 1}

Pour chacune des propositions de cet exercice, cochez A si elle est vraie\par et B si elle est fausse.

On s'intéresse à la fonction f définie par $f:x\mapsto\left(2x+2\right)e^{7x}$.

\begin{VraiFaux}

\Vrai{ L'ensemble de définition de f est $\left]-\infty,+\infty\right[$.}

\end{VraiFaux}

\begin{VraiFaux}

\Vrai{ La dérivée de f est donnée par $f^{\prime}:x\mapsto\left(14x+16\right)e^{7x}$.}

\end{VraiFaux}

On admet que son approximation quadratique est donné par\par $\qquad f\left(x\right)\approx 2 + 6 x + 8 x^2$.

\begin{VraiFaux}

\Vrai{ L'équation de la tangente $T$ à la courbe représentative $\mathscr C$ de $f$ en son point d'abscisse $0$ est donnée par $y=6 x +2 $.}

\end{VraiFaux}

\begin{VraiFaux}

\Faux{ $T$ est au-dessus de $\mathscr C$.}

\end{VraiFaux}

\subsection*{Exercice 2}

Cet exercice doit être rédigé au verso de la grille réponse.

\subsection*{Exercice 3}

\begin{QCM}

Parmi ces fonctions, laquelle est une primitive de la fonction $f$\par définie par $ f(x)=x \cdot \cos (3 x+1)$ ?

\begin{enumerate}

\Faux{ $F:x \mapsto \cos (3 x+1)-3 x \cdot \sin (3 x+1)$}

\Vrai{ $H:x \mapsto \frac{x \cdot \sin (3 x+1)}{3}+\frac{\cos (3 x+1)}{9}$}

\Faux{ $G:x \mapsto x \cdot \sin (3 x+1)$}

\end{enumerate}

\end{QCM}

\begin{QCM}

$\int\limits_{0}^{2}\left(5 t^{3}+3 t^{2}-t+7\right) d t=$

\begin{enumerate}

\Vrai{ 40}

\Faux{ $15 t^{2}+6 t-1$}

\Faux{ 71}

\end{enumerate}

\end{QCM}

\begin{QCM}

La valeur moyenne de la fonction $g: x \rightarrow e^{5 x} $ entre 0 et 2 est de :

\begin{enumerate}

\Faux{environ $22 025,5$}

\Vrai{environ $2 202,5$}

\Faux{environ $4405,1$}

\end{enumerate}

\end{QCM}

\newpage

\subsection*{Problème}

Alain Raffletou cherche à connaitre le nombre moyen de clients journaliers dans une de ses boutiques sur l'année en cours. Pour cela, il modélise le nombre de clients par une fonction donnant le nombre de clients $c\left(t\right)$ en fonction du temps $t$, exprimé en jours. On prends donc $t\in\left[0;365\right]$.

\begin{QCM}

En remarquant que le nombre de clients ne peut être négatif, déterminé, parmi les propositions ci-dessous, laquelle donne la formule de $c\left(t\right)$ en fonction de $t$ :

\begin{enumerate}

\Faux{$c\left(t\right)=3\cos\left(\frac{6\pi t}{365}-25\right)$}

\Vrai{$c\left(t\right)=3\cos\left(\frac{6\pi t}{365}\right)+25$}

\Faux{$c\left(t\right)=3\sin\left(\frac{6\pi t}{365}+25\right)$}

\Faux{$c\left(t\right)=3\sin\left(\frac{6\pi t}{365}\right)-25$}

\end{enumerate}

\end{QCM}

\begin{QCM}

On a alors que les valeurs de $c\left(t\right)$ sont comprises entre :

\begin{enumerate}

\Faux{$0$ et $25$}

\Faux{$0$ et $3$ }

\Vrai{$22$ et $28$}

\Faux{$0$ et $28$ }

\end{enumerate}

\end{QCM}

Dès lors, il peut calculer le nombre moyen de clients en appliquant la formule

$\frac{1}{365}\int\limits _{0}^{365}C\left(t\right)dt$ , où $C$ désigne une primitive de $c$.

\begin{QCM}

Parmi les propositions ci-dessous, laquelle donne une formule possible C~?

\begin{enumerate}

\Faux{$C\left(t\right)=\frac{18}{365}\pi\cos\left(\frac{6\pi t}{365}\right)-25t-41$}

\Faux{$C\left(t\right)=\frac{18}{365}\pi\sin\left(\frac{6\pi t}{365}\right)-25t+20$}

\Vrai{$C\left(t\right)=\frac{1095}{2}\pi\sin\left(\frac{6\pi t}{365}\right)+25t$}

\Faux{$C\left(t\right)=-3\sin\left(\frac{6\pi t}{365}\right)$}

\end{enumerate}

\end{QCM}

\begin{QCM}

On obtient ainsi que le nombre moyen de clients journaliers sur l'année en cours est donné par :

\begin{enumerate}

\Vrai{$25$}

\Faux{$3$}

\Faux{$25\sin\left(6\pi\right)$ }

\Faux{$3\cos\left(6\pi\right)$}

\end{enumerate}

\end{QCM}

\end{Sujet}

\begin{Sujet}

\subsection*{Exercice 1}

Pour chacune des propositions de cet exercice, cochez A si elle est vraie\par et B si elle est fausse.

On s'intéresse à la fonction f définie par $f:x\mapsto\left(7x+2\right)e^{7x}$.

\begin{VraiFaux}

\Faux{ L'ensemble de définition de f est $\left]0,+\infty\right[$.}

\end{VraiFaux}

\begin{VraiFaux}

\Faux{ La dérivée de f est donnée par $f^{\prime}:x\mapsto\left(49\right)e^{cx}$.}

\end{VraiFaux}

On admet que son approximation quadratique est donné par\par $\qquad f\left(x\right)\approx 2 + 21 x + 98 x^2$.

\begin{VraiFaux}

\Faux{ L'équation de la tangente $T$ à la courbe représentative $\mathscr C$ de $f$ en son point d'abscisse $0$ est donnée par $y=2x + 21 $.}

\end{VraiFaux}

\begin{VraiFaux}

\Vrai{ $T$ est au-dessous de $\mathscr C$.}

\end{VraiFaux}

\subsection*{Exercice 2}

Cet exercice doit être rédigé au verso de la grille réponse.

\subsection*{Exercice 3}

\begin{QCM}

Parmi ces fonctions, laquelle est une primitive de la fonction $f$\par définie par $ f(x)=x \cdot \cos (3 x+1)$ ?

\begin{enumerate}

\Vrai{ $H:x \mapsto \frac{x \cdot \sin (3 x+1)}{3}+\frac{\cos (3 x+1)}{9}$}

\Faux{ $G:x \mapsto x \cdot \sin (3 x+1)$}

\Faux{ $F:x \mapsto \cos (3 x+1)-3 x \cdot \sin (3 x+1)$}

\end{enumerate}

\end{QCM}

\begin{QCM}

$\int\limits_{0}^{2}\left(5 t^{3}+3 t^{2}-t+7\right) d t=$

\begin{enumerate}

\Faux{ $15 t^{2}+6 t-1$}

\Faux{ 71}

\Vrai{ 40}

\end{enumerate}

\end{QCM}

\begin{QCM}

La valeur moyenne de la fonction $g: x \rightarrow e^{5 x} $ entre 0 et 2 est de :

\begin{enumerate}

\Vrai{environ $2 202,5$}

\Faux{environ $22 025,5$}

\Faux{environ $4405,1$}

\end{enumerate}

\end{QCM}

\newpage

\subsection*{Problème}

Alain Raffletou cherche à connaitre le nombre moyen de clients journaliers dans une de ses boutiques sur l'année en cours. Pour cela, il modélise le nombre de clients par une fonction donnant le nombre de clients $c\left(t\right)$ en fonction du temps $t$, exprimé en jours. On prends donc $t\in\left[0;365\right]$.

\begin{QCM}

En remarquant que le nombre de clients ne peut être négatif, déterminé, parmi les propositions ci-dessous, laquelle donne la formule de $c\left(t\right)$ en fonction de $t$ :

\begin{enumerate}

\Vrai{$c\left(t\right)=3\cos\left(\frac{6\pi t}{365}\right)+10$}

\Faux{$c\left(t\right)=3\cos\left(\frac{6\pi t}{365}-10\right)$}

\Faux{$c\left(t\right)=3\sin\left(\frac{6\pi t}{365}\right)-10$}

\Faux{$c\left(t\right)=3\sin\left(\frac{6\pi t}{365}+10\right)$}

\end{enumerate}

\end{QCM}

\begin{QCM}

On a alors que les valeurs de $c\left(t\right)$ sont comprises entre :

\begin{enumerate}

\Vrai{$7$ et $13$}

\Faux{$0$ et $10$}

\Faux{$0$ et $3$ }

\Faux{$0$ et $13$ }

\end{enumerate}

\end{QCM}

Dès lors, il peut calculer le nombre moyen de clients en appliquant la formule

$\frac{1}{365}\int\limits _{0}^{365}C\left(t\right)dt$ , où $C$ désigne une primitive de $c$.

\begin{QCM}

Parmi les propositions ci-dessous, laquelle donne une formule possible C~?

\begin{enumerate}

\Faux{$C\left(t\right)=\frac{18}{365}\pi\cos\left(\frac{6\pi t}{365}\right)-10t-11$}

\Vrai{$C\left(t\right)=\frac{1095}{2}\pi\sin\left(\frac{6\pi t}{365}\right)+10t$}

\Faux{$C\left(t\right)=\frac{18}{365}\pi\sin\left(\frac{6\pi t}{365}\right)-10t+5$}

\Faux{$C\left(t\right)=-3\sin\left(\frac{6\pi t}{365}\right)$}

\end{enumerate}

\end{QCM}

\begin{QCM}

On obtient ainsi que le nombre moyen de clients journaliers sur l'année en cours est donné par :

\begin{enumerate}

\Faux{$3$}

\Faux{$10\sin\left(6\pi\right)$ }

\Faux{$3\cos\left(6\pi\right)$}

\Vrai{$10$}

\end{enumerate}

\end{QCM}

\end{Sujet}

\begin{Sujet}

\subsection*{Exercice 1}

Pour chacune des propositions de cet exercice, cochez A si elle est vraie\par et B si elle est fausse.

On s'intéresse à la fonction f définie par $f:x\mapsto\left(5x+4\right)e^{3x}$.

\begin{VraiFaux}

\Vrai{ L'ensemble de définition de f est $\left]-\infty,+\infty\right[$.}

\end{VraiFaux}

\begin{VraiFaux}

\Vrai{ La dérivée de f est donnée par $f^{\prime}:x\mapsto\left(15x+17\right)e^{3x}$.}

\end{VraiFaux}

On admet que son approximation quadratique est donné par\par $\qquad f\left(x\right)\approx 4 + 25 x + 75 x^2$.

\begin{VraiFaux}

\Vrai{ L'équation de la tangente $T$ à la courbe représentative $\mathscr C$ de $f$ en son point d'abscisse $0$ est donnée par $y=25 x +4 $.}

\end{VraiFaux}

\begin{VraiFaux}

\Vrai{ $T$ est au-dessous de $\mathscr C$.}

\end{VraiFaux}

\subsection*{Exercice 2}

Cet exercice doit être rédigé au verso de la grille réponse.

\subsection*{Exercice 3}

\begin{QCM}

Parmi ces fonctions, laquelle est une primitive de la fonction $f$\par définie par $ f(x)=x \cdot \cos (3 x+1)$ ?

\begin{enumerate}

\Faux{ $F:x \mapsto \cos (3 x+1)-3 x \cdot \sin (3 x+1)$}

\Faux{ $G:x \mapsto x \cdot \sin (3 x+1)$}

\Vrai{ $H:x \mapsto \frac{x \cdot \sin (3 x+1)}{3}+\frac{\cos (3 x+1)}{9}$}

\end{enumerate}

\end{QCM}

\begin{QCM}

$\int\limits_{0}^{2}\left(5 t^{3}+3 t^{2}-t+7\right) d t=$

\begin{enumerate}

\Faux{ 71}

\Vrai{ 40}

\Faux{ $15 t^{2}+6 t-1$}

\end{enumerate}

\end{QCM}

\begin{QCM}

La valeur moyenne de la fonction $g: x \rightarrow e^{5 x} $ entre 0 et 2 est de :

\begin{enumerate}

\Vrai{environ $2 202,5$}

\Faux{environ $22 025,5$}

\Faux{environ $4405,1$}

\end{enumerate}

\end{QCM}

\newpage

\subsection*{Problème}

Alain Raffletou cherche à connaitre le nombre moyen de clients journaliers dans une de ses boutiques sur l'année en cours. Pour cela, il modélise le nombre de clients par une fonction donnant le nombre de clients $c\left(t\right)$ en fonction du temps $t$, exprimé en jours. On prends donc $t\in\left[0;365\right]$.

\begin{QCM}

En remarquant que le nombre de clients ne peut être négatif, déterminé, parmi les propositions ci-dessous, laquelle donne la formule de $c\left(t\right)$ en fonction de $t$ :

\begin{enumerate}

\Vrai{$c\left(t\right)=3\cos\left(\frac{8\pi t}{365}\right)+25$}

\Faux{$c\left(t\right)=3\sin\left(\frac{8\pi t}{365}\right)-25$}

\Faux{$c\left(t\right)=3\sin\left(\frac{8\pi t}{365}+25\right)$}

\Faux{$c\left(t\right)=3\cos\left(\frac{8\pi t}{365}-25\right)$}

\end{enumerate}

\end{QCM}

\begin{QCM}

On a alors que les valeurs de $c\left(t\right)$ sont comprises entre :

\begin{enumerate}

\Faux{$0$ et $3$ }

\Vrai{$22$ et $28$}

\Faux{$0$ et $25$}

\Faux{$0$ et $28$ }

\end{enumerate}

\end{QCM}

Dès lors, il peut calculer le nombre moyen de clients en appliquant la formule

$\frac{1}{365}\int\limits _{0}^{365}C\left(t\right)dt$ , où $C$ désigne une primitive de $c$.

\begin{QCM}

Parmi les propositions ci-dessous, laquelle donne une formule possible C~?

\begin{enumerate}

\Faux{$C\left(t\right)=-3\sin\left(\frac{8\pi t}{365}\right)$}

\Vrai{$C\left(t\right)=\frac{1095}{2}\pi\sin\left(\frac{8\pi t}{365}\right)+25t$}

\Faux{$C\left(t\right)=\frac{24}{365}\pi\cos\left(\frac{8\pi t}{365}\right)-25t-41$}

\Faux{$C\left(t\right)=\frac{24}{365}\pi\sin\left(\frac{8\pi t}{365}\right)-25t+20$}

\end{enumerate}

\end{QCM}

\begin{QCM}

On obtient ainsi que le nombre moyen de clients journaliers sur l'année en cours est donné par :

\begin{enumerate}

\Faux{$25\sin\left(8\pi\right)$ }

\Faux{$3$}

\Vrai{$25$}

\Faux{$3\cos\left(8\pi\right)$}

\end{enumerate}

\end{QCM}

\end{Sujet}

\begin{Sujet}

\subsection*{Exercice 1}

Pour chacune des propositions de cet exercice, cochez A si elle est vraie\par et B si elle est fausse.

On s'intéresse à la fonction f définie par $f:x\mapsto\left(2x+4\right)e^{5x}$.

\begin{VraiFaux}

\Vrai{ L'ensemble de définition de f est $\left]-\infty,+\infty\right[$.}

\end{VraiFaux}

\begin{VraiFaux}

\Vrai{ La dérivée de f est donnée par $f^{\prime}:x\mapsto\left(10x+22\right)e^{5x}$.}

\end{VraiFaux}

On admet que son approximation quadratique est donné par\par $\qquad f\left(x\right)\approx 4 + 10 x + 12 x^2$.

\begin{VraiFaux}

\Faux{ L'équation de la tangente $T$ à la courbe représentative $\mathscr C$ de $f$ en son point d'abscisse $0$ est donnée par $y=4x + 10 $.}

\end{VraiFaux}

\begin{VraiFaux}

\Vrai{ $T$ est au-dessous de $\mathscr C$.}

\end{VraiFaux}

\subsection*{Exercice 2}

Cet exercice doit être rédigé au verso de la grille réponse.

\subsection*{Exercice 3}

\begin{QCM}

Parmi ces fonctions, laquelle est une primitive de la fonction $f$\par définie par $ f(x)=x \cdot \cos (3 x+1)$ ?

\begin{enumerate}

\Vrai{ $H:x \mapsto \frac{x \cdot \sin (3 x+1)}{3}+\frac{\cos (3 x+1)}{9}$}

\Faux{ $G:x \mapsto x \cdot \sin (3 x+1)$}

\Faux{ $F:x \mapsto \cos (3 x+1)-3 x \cdot \sin (3 x+1)$}

\end{enumerate}

\end{QCM}

\begin{QCM}

$\int\limits_{0}^{2}\left(5 t^{3}+3 t^{2}-t+7\right) d t=$

\begin{enumerate}

\Faux{ $15 t^{2}+6 t-1$}

\Faux{ 71}

\Vrai{ 40}

\end{enumerate}

\end{QCM}

\begin{QCM}

La valeur moyenne de la fonction $g: x \rightarrow e^{5 x} $ entre 0 et 2 est de :

\begin{enumerate}

\Faux{environ $22 025,5$}

\Vrai{environ $2 202,5$}

\Faux{environ $4405,1$}

\end{enumerate}

\end{QCM}

\newpage

\subsection*{Problème}

Alain Raffletou cherche à connaitre le nombre moyen de clients journaliers dans une de ses boutiques sur l'année en cours. Pour cela, il modélise le nombre de clients par une fonction donnant le nombre de clients $c\left(t\right)$ en fonction du temps $t$, exprimé en jours. On prends donc $t\in\left[0;365\right]$.

\begin{QCM}

En remarquant que le nombre de clients ne peut être négatif, déterminé, parmi les propositions ci-dessous, laquelle donne la formule de $c\left(t\right)$ en fonction de $t$ :

\begin{enumerate}

\Faux{$c\left(t\right)=5\sin\left(\frac{4\pi t}{365}\right)-15$}

\Faux{$c\left(t\right)=5\sin\left(\frac{4\pi t}{365}+15\right)$}

\Vrai{$c\left(t\right)=5\cos\left(\frac{4\pi t}{365}\right)+15$}

\Faux{$c\left(t\right)=5\cos\left(\frac{4\pi t}{365}-15\right)$}

\end{enumerate}

\end{QCM}

\begin{QCM}

On a alors que les valeurs de $c\left(t\right)$ sont comprises entre :

\begin{enumerate}

\Faux{$0$ et $15$}

\Faux{$0$ et $20$ }

\Faux{$0$ et $5$ }

\Vrai{$10$ et $20$}

\end{enumerate}

\end{QCM}

Dès lors, il peut calculer le nombre moyen de clients en appliquant la formule

$\frac{1}{365}\int\limits _{0}^{365}C\left(t\right)dt$ , où $C$ désigne une primitive de $c$.

\begin{QCM}

Parmi les propositions ci-dessous, laquelle donne une formule possible C~?

\begin{enumerate}

\Faux{$C\left(t\right)=\frac{20}{365}\pi\sin\left(\frac{4\pi t}{365}\right)-15t+6$}

\Faux{$C\left(t\right)=-5\sin\left(\frac{4\pi t}{365}\right)$}

\Vrai{$C\left(t\right)=\frac{1825}{2}\pi\sin\left(\frac{4\pi t}{365}\right)+15t$}

\Faux{$C\left(t\right)=\frac{20}{365}\pi\cos\left(\frac{4\pi t}{365}\right)-15t-15$}

\end{enumerate}

\end{QCM}

\begin{QCM}

On obtient ainsi que le nombre moyen de clients journaliers sur l'année en cours est donné par :

\begin{enumerate}

\Vrai{$15$}

\Faux{$15\sin\left(4\pi\right)$ }

\Faux{$5$}

\Faux{$5\cos\left(4\pi\right)$}

\end{enumerate}

\end{QCM}

\end{Sujet}

\begin{Sujet}

\subsection*{Exercice 1}

Pour chacune des propositions de cet exercice, cochez A si elle est vraie\par et B si elle est fausse.

On s'intéresse à la fonction f définie par $f:x\mapsto\left(4x+6\right)e^{7x}$.

\begin{VraiFaux}

\Faux{ L'ensemble de définition de f est $\left]0,+\infty\right[$.}

\end{VraiFaux}

\begin{VraiFaux}

\Faux{ La dérivée de f est donnée par $f^{\prime}:x\mapsto\left(28\right)e^{cx}$.}

\end{VraiFaux}

On admet que son approximation quadratique est donné par\par $\qquad f\left(x\right)\approx 6 + 28 x + 64 x^2$.

\begin{VraiFaux}

\Vrai{ L'équation de la tangente $T$ à la courbe représentative $\mathscr C$ de $f$ en son point d'abscisse $0$ est donnée par $y=28 x +6 $.}

\end{VraiFaux}

\begin{VraiFaux}

\Vrai{ $T$ est au-dessous de $\mathscr C$.}

\end{VraiFaux}

\subsection*{Exercice 2}

Cet exercice doit être rédigé au verso de la grille réponse.

\subsection*{Exercice 3}

\begin{QCM}

Parmi ces fonctions, laquelle est une primitive de la fonction $f$\par définie par $ f(x)=x \cdot \cos (3 x+1)$ ?

\begin{enumerate}

\Faux{ $F:x \mapsto \cos (3 x+1)-3 x \cdot \sin (3 x+1)$}

\Vrai{ $H:x \mapsto \frac{x \cdot \sin (3 x+1)}{3}+\frac{\cos (3 x+1)}{9}$}

\Faux{ $G:x \mapsto x \cdot \sin (3 x+1)$}

\end{enumerate}

\end{QCM}

\begin{QCM}

$\int\limits_{0}^{2}\left(5 t^{3}+3 t^{2}-t+7\right) d t=$

\begin{enumerate}

\Faux{ 71}

\Vrai{ 40}

\Faux{ $15 t^{2}+6 t-1$}

\end{enumerate}

\end{QCM}

\begin{QCM}

La valeur moyenne de la fonction $g: x \rightarrow e^{5 x} $ entre 0 et 2 est de :

\begin{enumerate}

\Faux{environ $22 025,5$}

\Faux{environ $4405,1$}

\Vrai{environ $2 202,5$}

\end{enumerate}

\end{QCM}

\newpage

\subsection*{Problème}

Alain Raffletou cherche à connaitre le nombre moyen de clients journaliers dans une de ses boutiques sur l'année en cours. Pour cela, il modélise le nombre de clients par une fonction donnant le nombre de clients $c\left(t\right)$ en fonction du temps $t$, exprimé en jours. On prends donc $t\in\left[0;365\right]$.

\begin{QCM}

En remarquant que le nombre de clients ne peut être négatif, déterminé, parmi les propositions ci-dessous, laquelle donne la formule de $c\left(t\right)$ en fonction de $t$ :

\begin{enumerate}

\Faux{$c\left(t\right)=4\sin\left(\frac{4\pi t}{365}\right)-17$}

\Faux{$c\left(t\right)=4\sin\left(\frac{4\pi t}{365}+17\right)$}

\Faux{$c\left(t\right)=4\cos\left(\frac{4\pi t}{365}-17\right)$}

\Vrai{$c\left(t\right)=4\cos\left(\frac{4\pi t}{365}\right)+17$}

\end{enumerate}

\end{QCM}

\begin{QCM}

On a alors que les valeurs de $c\left(t\right)$ sont comprises entre :

\begin{enumerate}

\Faux{$0$ et $4$ }

\Vrai{$13$ et $21$}

\Faux{$0$ et $17$}

\Faux{$0$ et $21$ }

\end{enumerate}

\end{QCM}

Dès lors, il peut calculer le nombre moyen de clients en appliquant la formule

$\frac{1}{365}\int\limits _{0}^{365}C\left(t\right)dt$ , où $C$ désigne une primitive de $c$.

\begin{QCM}

Parmi les propositions ci-dessous, laquelle donne une formule possible C~?

\begin{enumerate}

\Faux{$C\left(t\right)=\frac{16}{365}\pi\cos\left(\frac{4\pi t}{365}\right)-17t-22$}

\Vrai{$C\left(t\right)=\frac{1460}{2}\pi\sin\left(\frac{4\pi t}{365}\right)+17t$}

\Faux{$C\left(t\right)=\frac{16}{365}\pi\sin\left(\frac{4\pi t}{365}\right)-17t+10$}

\Faux{$C\left(t\right)=-4\sin\left(\frac{4\pi t}{365}\right)$}

\end{enumerate}

\end{QCM}

\begin{QCM}

On obtient ainsi que le nombre moyen de clients journaliers sur l'année en cours est donné par :

\begin{enumerate}

\Faux{$4$}

\Faux{$4\cos\left(4\pi\right)$}

\Vrai{$17$}

\Faux{$17\sin\left(4\pi\right)$ }

\end{enumerate}

\end{QCM}

\end{Sujet}

\begin{Sujet}

\subsection*{Exercice 1}

Pour chacune des propositions de cet exercice, cochez A si elle est vraie\par et B si elle est fausse.

On s'intéresse à la fonction f définie par $f:x\mapsto\left(4x+7\right)e^{3x}$.

\begin{VraiFaux}

\Vrai{ L'ensemble de définition de f est $\left]-\infty,+\infty\right[$.}

\end{VraiFaux}

\begin{VraiFaux}

\Faux{ La dérivée de f est donnée par $f^{\prime}:x\mapsto\left(12\right)e^{cx}$.}

\end{VraiFaux}

On admet que son approximation quadratique est donné par\par $\qquad f\left(x\right)\approx 7 + 32 x + 72 x^2$.

\begin{VraiFaux}

\Vrai{ L'équation de la tangente $T$ à la courbe représentative $\mathscr C$ de $f$ en son point d'abscisse $0$ est donnée par $y=32 x +7 $.}

\end{VraiFaux}

\begin{VraiFaux}

\Vrai{ $T$ est au-dessous de $\mathscr C$.}

\end{VraiFaux}

\subsection*{Exercice 2}

Cet exercice doit être rédigé au verso de la grille réponse.

\subsection*{Exercice 3}

\begin{QCM}

Parmi ces fonctions, laquelle est une primitive de la fonction $f$\par définie par $ f(x)=x \cdot \cos (3 x+1)$ ?

\begin{enumerate}

\Faux{ $G:x \mapsto x \cdot \sin (3 x+1)$}

\Faux{ $F:x \mapsto \cos (3 x+1)-3 x \cdot \sin (3 x+1)$}

\Vrai{ $H:x \mapsto \frac{x \cdot \sin (3 x+1)}{3}+\frac{\cos (3 x+1)}{9}$}

\end{enumerate}

\end{QCM}

\begin{QCM}

$\int\limits_{0}^{2}\left(5 t^{3}+3 t^{2}-t+7\right) d t=$

\begin{enumerate}

\Faux{ $15 t^{2}+6 t-1$}

\Faux{ 71}

\Vrai{ 40}

\end{enumerate}

\end{QCM}

\begin{QCM}

La valeur moyenne de la fonction $g: x \rightarrow e^{5 x} $ entre 0 et 2 est de :

\begin{enumerate}

\Faux{environ $4405,1$}

\Vrai{environ $2 202,5$}

\Faux{environ $22 025,5$}

\end{enumerate}

\end{QCM}

\newpage

\subsection*{Problème}

Alain Raffletou cherche à connaitre le nombre moyen de clients journaliers dans une de ses boutiques sur l'année en cours. Pour cela, il modélise le nombre de clients par une fonction donnant le nombre de clients $c\left(t\right)$ en fonction du temps $t$, exprimé en jours. On prends donc $t\in\left[0;365\right]$.

\begin{QCM}

En remarquant que le nombre de clients ne peut être négatif, déterminé, parmi les propositions ci-dessous, laquelle donne la formule de $c\left(t\right)$ en fonction de $t$ :

\begin{enumerate}

\Faux{$c\left(t\right)=5\sin\left(\frac{4\pi t}{365}\right)-16$}

\Vrai{$c\left(t\right)=5\cos\left(\frac{4\pi t}{365}\right)+16$}

\Faux{$c\left(t\right)=5\sin\left(\frac{4\pi t}{365}+16\right)$}

\Faux{$c\left(t\right)=5\cos\left(\frac{4\pi t}{365}-16\right)$}

\end{enumerate}

\end{QCM}

\begin{QCM}

On a alors que les valeurs de $c\left(t\right)$ sont comprises entre :

\begin{enumerate}

\Vrai{$11$ et $21$}

\Faux{$0$ et $5$ }

\Faux{$0$ et $16$}

\Faux{$0$ et $21$ }

\end{enumerate}

\end{QCM}

Dès lors, il peut calculer le nombre moyen de clients en appliquant la formule

$\frac{1}{365}\int\limits _{0}^{365}C\left(t\right)dt$ , où $C$ désigne une primitive de $c$.

\begin{QCM}

Parmi les propositions ci-dessous, laquelle donne une formule possible C~?

\begin{enumerate}

\Vrai{$C\left(t\right)=\frac{1825}{2}\pi\sin\left(\frac{4\pi t}{365}\right)+16t$}

\Faux{$C\left(t\right)=\frac{20}{365}\pi\cos\left(\frac{4\pi t}{365}\right)-16t-17$}

\Faux{$C\left(t\right)=\frac{20}{365}\pi\sin\left(\frac{4\pi t}{365}\right)-16t+7$}

\Faux{$C\left(t\right)=-5\sin\left(\frac{4\pi t}{365}\right)$}

\end{enumerate}

\end{QCM}

\begin{QCM}

On obtient ainsi que le nombre moyen de clients journaliers sur l'année en cours est donné par :

\begin{enumerate}

\Vrai{$16$}

\Faux{$5$}

\Faux{$16\sin\left(4\pi\right)$ }

\Faux{$5\cos\left(4\pi\right)$}

\end{enumerate}

\end{QCM}

\end{Sujet}

\begin{Sujet}

\subsection*{Exercice 1}

Pour chacune des propositions de cet exercice, cochez A si elle est vraie\par et B si elle est fausse.

On s'intéresse à la fonction f définie par $f:x\mapsto\left(3x+3\right)e^{4x}$.

\begin{VraiFaux}

\Faux{ L'ensemble de définition de f est $\left]0,+\infty\right[$.}

\end{VraiFaux}

\begin{VraiFaux}

\Vrai{ La dérivée de f est donnée par $f^{\prime}:x\mapsto\left(12x+15\right)e^{4x}$.}

\end{VraiFaux}

On admet que son approximation quadratique est donné par\par $\qquad f\left(x\right)\approx 3 + 12 x + \frac{45}{2} x^2$.

\begin{VraiFaux}

\Vrai{ L'équation de la tangente $T$ à la courbe représentative $\mathscr C$ de $f$ en son point d'abscisse $0$ est donnée par $y=12 x +3 $.}

\end{VraiFaux}

\begin{VraiFaux}

\Vrai{ $T$ est au-dessous de $\mathscr C$.}

\end{VraiFaux}

\subsection*{Exercice 2}

Cet exercice doit être rédigé au verso de la grille réponse.

\subsection*{Exercice 3}

\begin{QCM}

Parmi ces fonctions, laquelle est une primitive de la fonction $f$\par définie par $ f(x)=x \cdot \cos (3 x+1)$ ?

\begin{enumerate}

\Faux{ $F:x \mapsto \cos (3 x+1)-3 x \cdot \sin (3 x+1)$}

\Faux{ $G:x \mapsto x \cdot \sin (3 x+1)$}

\Vrai{ $H:x \mapsto \frac{x \cdot \sin (3 x+1)}{3}+\frac{\cos (3 x+1)}{9}$}

\end{enumerate}

\end{QCM}

\begin{QCM}

$\int\limits_{0}^{2}\left(5 t^{3}+3 t^{2}-t+7\right) d t=$

\begin{enumerate}

\Faux{ 71}

\Faux{ $15 t^{2}+6 t-1$}

\Vrai{ 40}

\end{enumerate}

\end{QCM}

\begin{QCM}

La valeur moyenne de la fonction $g: x \rightarrow e^{5 x} $ entre 0 et 2 est de :

\begin{enumerate}

\Vrai{environ $2 202,5$}

\Faux{environ $22 025,5$}

\Faux{environ $4405,1$}

\end{enumerate}

\end{QCM}

\newpage

\subsection*{Problème}

Alain Raffletou cherche à connaitre le nombre moyen de clients journaliers dans une de ses boutiques sur l'année en cours. Pour cela, il modélise le nombre de clients par une fonction donnant le nombre de clients $c\left(t\right)$ en fonction du temps $t$, exprimé en jours. On prends donc $t\in\left[0;365\right]$.

\begin{QCM}

En remarquant que le nombre de clients ne peut être négatif, déterminé, parmi les propositions ci-dessous, laquelle donne la formule de $c\left(t\right)$ en fonction de $t$ :

\begin{enumerate}

\Vrai{$c\left(t\right)=3\cos\left(\frac{8\pi t}{365}\right)+11$}

\Faux{$c\left(t\right)=3\cos\left(\frac{8\pi t}{365}-11\right)$}

\Faux{$c\left(t\right)=3\sin\left(\frac{8\pi t}{365}\right)-11$}

\Faux{$c\left(t\right)=3\sin\left(\frac{8\pi t}{365}+11\right)$}

\end{enumerate}

\end{QCM}

\begin{QCM}

On a alors que les valeurs de $c\left(t\right)$ sont comprises entre :

\begin{enumerate}

\Faux{$0$ et $14$ }

\Faux{$0$ et $11$}

\Faux{$0$ et $3$ }

\Vrai{$8$ et $14$}

\end{enumerate}

\end{QCM}

Dès lors, il peut calculer le nombre moyen de clients en appliquant la formule

$\frac{1}{365}\int\limits _{0}^{365}C\left(t\right)dt$ , où $C$ désigne une primitive de $c$.

\begin{QCM}

Parmi les propositions ci-dessous, laquelle donne une formule possible C~?

\begin{enumerate}

\Faux{$C\left(t\right)=-3\sin\left(\frac{8\pi t}{365}\right)$}

\Faux{$C\left(t\right)=\frac{24}{365}\pi\cos\left(\frac{8\pi t}{365}\right)-11t-13$}

\Vrai{$C\left(t\right)=\frac{1095}{2}\pi\sin\left(\frac{8\pi t}{365}\right)+11t$}

\Faux{$C\left(t\right)=\frac{24}{365}\pi\sin\left(\frac{8\pi t}{365}\right)-11t+6$}

\end{enumerate}

\end{QCM}

\begin{QCM}

On obtient ainsi que le nombre moyen de clients journaliers sur l'année en cours est donné par :

\begin{enumerate}

\Faux{$11\sin\left(8\pi\right)$ }

\Vrai{$11$}

\Faux{$3$}

\Faux{$3\cos\left(8\pi\right)$}

\end{enumerate}

\end{QCM}

\end{Sujet}

\begin{Sujet}

\subsection*{Exercice 1}

Pour chacune des propositions de cet exercice, cochez A si elle est vraie\par et B si elle est fausse.

On s'intéresse à la fonction f définie par $f:x\mapsto\left(7x+4\right)e^{4x}$.

\begin{VraiFaux}

\Faux{ L'ensemble de définition de f est $\left]0,+\infty\right[$.}

\end{VraiFaux}

\begin{VraiFaux}

\Faux{ La dérivée de f est donnée par $f^{\prime}:x\mapsto\left(28\right)e^{cx}$.}

\end{VraiFaux}

On admet que son approximation quadratique est donné par\par $\qquad f\left(x\right)\approx 4 + 35 x + 147 x^2$.

\begin{VraiFaux}

\Vrai{ L'équation de la tangente $T$ à la courbe représentative $\mathscr C$ de $f$ en son point d'abscisse $0$ est donnée par $y=35 x +4 $.}

\end{VraiFaux}

\begin{VraiFaux}

\Faux{ $T$ est au-dessus de $\mathscr C$.}

\end{VraiFaux}

\subsection*{Exercice 2}

Cet exercice doit être rédigé au verso de la grille réponse.

\subsection*{Exercice 3}

\begin{QCM}

Parmi ces fonctions, laquelle est une primitive de la fonction $f$\par définie par $ f(x)=x \cdot \cos (3 x+1)$ ?

\begin{enumerate}

\Vrai{ $H:x \mapsto \frac{x \cdot \sin (3 x+1)}{3}+\frac{\cos (3 x+1)}{9}$}

\Faux{ $F:x \mapsto \cos (3 x+1)-3 x \cdot \sin (3 x+1)$}

\Faux{ $G:x \mapsto x \cdot \sin (3 x+1)$}

\end{enumerate}

\end{QCM}

\begin{QCM}

$\int\limits_{0}^{2}\left(5 t^{3}+3 t^{2}-t+7\right) d t=$

\begin{enumerate}

\Vrai{ 40}

\Faux{ 71}

\Faux{ $15 t^{2}+6 t-1$}

\end{enumerate}

\end{QCM}

\begin{QCM}

La valeur moyenne de la fonction $g: x \rightarrow e^{5 x} $ entre 0 et 2 est de :

\begin{enumerate}

\Faux{environ $22 025,5$}

\Faux{environ $4405,1$}

\Vrai{environ $2 202,5$}

\end{enumerate}

\end{QCM}

\newpage

\subsection*{Problème}

Alain Raffletou cherche à connaitre le nombre moyen de clients journaliers dans une de ses boutiques sur l'année en cours. Pour cela, il modélise le nombre de clients par une fonction donnant le nombre de clients $c\left(t\right)$ en fonction du temps $t$, exprimé en jours. On prends donc $t\in\left[0;365\right]$.

\begin{QCM}

En remarquant que le nombre de clients ne peut être négatif, déterminé, parmi les propositions ci-dessous, laquelle donne la formule de $c\left(t\right)$ en fonction de $t$ :

\begin{enumerate}

\Vrai{$c\left(t\right)=4\cos\left(\frac{6\pi t}{365}\right)+17$}

\Faux{$c\left(t\right)=4\cos\left(\frac{6\pi t}{365}-17\right)$}

\Faux{$c\left(t\right)=4\sin\left(\frac{6\pi t}{365}\right)-17$}

\Faux{$c\left(t\right)=4\sin\left(\frac{6\pi t}{365}+17\right)$}

\end{enumerate}

\end{QCM}

\begin{QCM}

On a alors que les valeurs de $c\left(t\right)$ sont comprises entre :

\begin{enumerate}

\Faux{$0$ et $21$ }

\Vrai{$13$ et $21$}

\Faux{$0$ et $4$ }

\Faux{$0$ et $17$}

\end{enumerate}

\end{QCM}

Dès lors, il peut calculer le nombre moyen de clients en appliquant la formule

$\frac{1}{365}\int\limits _{0}^{365}C\left(t\right)dt$ , où $C$ désigne une primitive de $c$.

\begin{QCM}

Parmi les propositions ci-dessous, laquelle donne une formule possible C~?

\begin{enumerate}

\Vrai{$C\left(t\right)=\frac{1460}{2}\pi\sin\left(\frac{6\pi t}{365}\right)+17t$}

\Faux{$C\left(t\right)=-4\sin\left(\frac{6\pi t}{365}\right)$}

\Faux{$C\left(t\right)=\frac{24}{365}\pi\cos\left(\frac{6\pi t}{365}\right)-17t-22$}

\Faux{$C\left(t\right)=\frac{24}{365}\pi\sin\left(\frac{6\pi t}{365}\right)-17t+10$}

\end{enumerate}

\end{QCM}

\begin{QCM}

On obtient ainsi que le nombre moyen de clients journaliers sur l'année en cours est donné par :

\begin{enumerate}

\Faux{$4$}

\Faux{$4\cos\left(6\pi\right)$}

\Faux{$17\sin\left(6\pi\right)$ }

\Vrai{$17$}

\end{enumerate}

\end{QCM}

\end{Sujet}

\begin{Sujet}

\subsection*{Exercice 1}

Pour chacune des propositions de cet exercice, cochez A si elle est vraie\par et B si elle est fausse.

On s'intéresse à la fonction f définie par $f:x\mapsto\left(3x+5\right)e^{4x}$.

\begin{VraiFaux}

\Faux{ L'ensemble de définition de f est $\left]0,+\infty\right[$.}

\end{VraiFaux}

\begin{VraiFaux}

\Vrai{ La dérivée de f est donnée par $f^{\prime}:x\mapsto\left(12x+23\right)e^{4x}$.}

\end{VraiFaux}

On admet que son approximation quadratique est donné par\par $\qquad f\left(x\right)\approx 5 + 18 x + \frac{63}{2} x^2$.

\begin{VraiFaux}

\Vrai{ L'équation de la tangente $T$ à la courbe représentative $\mathscr C$ de $f$ en son point d'abscisse $0$ est donnée par $y=18 x +5 $.}

\end{VraiFaux}

\begin{VraiFaux}

\Faux{ $T$ est au-dessus de $\mathscr C$.}

\end{VraiFaux}

\subsection*{Exercice 2}

Cet exercice doit être rédigé au verso de la grille réponse.

\subsection*{Exercice 3}

\begin{QCM}

Parmi ces fonctions, laquelle est une primitive de la fonction $f$\par définie par $ f(x)=x \cdot \cos (3 x+1)$ ?

\begin{enumerate}

\Faux{ $F:x \mapsto \cos (3 x+1)-3 x \cdot \sin (3 x+1)$}

\Vrai{ $H:x \mapsto \frac{x \cdot \sin (3 x+1)}{3}+\frac{\cos (3 x+1)}{9}$}

\Faux{ $G:x \mapsto x \cdot \sin (3 x+1)$}

\end{enumerate}

\end{QCM}

\begin{QCM}

$\int\limits_{0}^{2}\left(5 t^{3}+3 t^{2}-t+7\right) d t=$

\begin{enumerate}

\Faux{ $15 t^{2}+6 t-1$}

\Vrai{ 40}

\Faux{ 71}

\end{enumerate}

\end{QCM}

\begin{QCM}

La valeur moyenne de la fonction $g: x \rightarrow e^{5 x} $ entre 0 et 2 est de :

\begin{enumerate}

\Faux{environ $4405,1$}

\Faux{environ $22 025,5$}

\Vrai{environ $2 202,5$}

\end{enumerate}

\end{QCM}

\newpage

\subsection*{Problème}

Alain Raffletou cherche à connaitre le nombre moyen de clients journaliers dans une de ses boutiques sur l'année en cours. Pour cela, il modélise le nombre de clients par une fonction donnant le nombre de clients $c\left(t\right)$ en fonction du temps $t$, exprimé en jours. On prends donc $t\in\left[0;365\right]$.

\begin{QCM}

En remarquant que le nombre de clients ne peut être négatif, déterminé, parmi les propositions ci-dessous, laquelle donne la formule de $c\left(t\right)$ en fonction de $t$ :

\begin{enumerate}

\Vrai{$c\left(t\right)=5\cos\left(\frac{8\pi t}{365}\right)+13$}

\Faux{$c\left(t\right)=5\sin\left(\frac{8\pi t}{365}+13\right)$}

\Faux{$c\left(t\right)=5\cos\left(\frac{8\pi t}{365}-13\right)$}

\Faux{$c\left(t\right)=5\sin\left(\frac{8\pi t}{365}\right)-13$}

\end{enumerate}

\end{QCM}

\begin{QCM}

On a alors que les valeurs de $c\left(t\right)$ sont comprises entre :

\begin{enumerate}

\Faux{$0$ et $5$ }

\Vrai{$8$ et $18$}

\Faux{$0$ et $18$ }

\Faux{$0$ et $13$}

\end{enumerate}

\end{QCM}

Dès lors, il peut calculer le nombre moyen de clients en appliquant la formule

$\frac{1}{365}\int\limits _{0}^{365}C\left(t\right)dt$ , où $C$ désigne une primitive de $c$.

\begin{QCM}

Parmi les propositions ci-dessous, laquelle donne une formule possible C~?

\begin{enumerate}

\Faux{$C\left(t\right)=\frac{40}{365}\pi\cos\left(\frac{8\pi t}{365}\right)-13t-11$}

\Faux{$C\left(t\right)=-5\sin\left(\frac{8\pi t}{365}\right)$}

\Faux{$C\left(t\right)=\frac{40}{365}\pi\sin\left(\frac{8\pi t}{365}\right)-13t+4$}

\Vrai{$C\left(t\right)=\frac{1825}{2}\pi\sin\left(\frac{8\pi t}{365}\right)+13t$}

\end{enumerate}

\end{QCM}

\begin{QCM}

On obtient ainsi que le nombre moyen de clients journaliers sur l'année en cours est donné par :

\begin{enumerate}

\Faux{$13\sin\left(8\pi\right)$ }

\Faux{$5$}

\Faux{$5\cos\left(8\pi\right)$}

\Vrai{$13$}

\end{enumerate}

\end{QCM}

\end{Sujet}

\begin{Sujet}

\subsection*{Exercice 1}

Pour chacune des propositions de cet exercice, cochez A si elle est vraie\par et B si elle est fausse.

On s'intéresse à la fonction f définie par $f:x\mapsto\left(5x+2\right)e^{7x}$.

\begin{VraiFaux}

\Faux{ L'ensemble de définition de f est $\left]0,+\infty\right[$.}

\end{VraiFaux}

\begin{VraiFaux}

\Faux{ La dérivée de f est donnée par $f^{\prime}:x\mapsto\left(35\right)e^{cx}$.}

\end{VraiFaux}

On admet que son approximation quadratique est donné par\par $\qquad f\left(x\right)\approx 2 + 15 x + 50 x^2$.

\begin{VraiFaux}

\Faux{ L'équation de la tangente $T$ à la courbe représentative $\mathscr C$ de $f$ en son point d'abscisse $0$ est donnée par $y=2x + 15 $.}

\end{VraiFaux}

\begin{VraiFaux}

\Faux{ $T$ est au-dessus de $\mathscr C$.}

\end{VraiFaux}

\subsection*{Exercice 2}

Cet exercice doit être rédigé au verso de la grille réponse.

\subsection*{Exercice 3}

\begin{QCM}

Parmi ces fonctions, laquelle est une primitive de la fonction $f$\par définie par $ f(x)=x \cdot \cos (3 x+1)$ ?

\begin{enumerate}

\Vrai{ $H:x \mapsto \frac{x \cdot \sin (3 x+1)}{3}+\frac{\cos (3 x+1)}{9}$}

\Faux{ $F:x \mapsto \cos (3 x+1)-3 x \cdot \sin (3 x+1)$}

\Faux{ $G:x \mapsto x \cdot \sin (3 x+1)$}

\end{enumerate}

\end{QCM}

\begin{QCM}

$\int\limits_{0}^{2}\left(5 t^{3}+3 t^{2}-t+7\right) d t=$

\begin{enumerate}

\Vrai{ 40}

\Faux{ $15 t^{2}+6 t-1$}

\Faux{ 71}

\end{enumerate}

\end{QCM}

\begin{QCM}

La valeur moyenne de la fonction $g: x \rightarrow e^{5 x} $ entre 0 et 2 est de :

\begin{enumerate}

\Faux{environ $4405,1$}

\Vrai{environ $2 202,5$}

\Faux{environ $22 025,5$}

\end{enumerate}

\end{QCM}

\newpage

\subsection*{Problème}

Alain Raffletou cherche à connaitre le nombre moyen de clients journaliers dans une de ses boutiques sur l'année en cours. Pour cela, il modélise le nombre de clients par une fonction donnant le nombre de clients $c\left(t\right)$ en fonction du temps $t$, exprimé en jours. On prends donc $t\in\left[0;365\right]$.

\begin{QCM}

En remarquant que le nombre de clients ne peut être négatif, déterminé, parmi les propositions ci-dessous, laquelle donne la formule de $c\left(t\right)$ en fonction de $t$ :

\begin{enumerate}

\Vrai{$c\left(t\right)=4\cos\left(\frac{4\pi t}{365}\right)+24$}

\Faux{$c\left(t\right)=4\sin\left(\frac{4\pi t}{365}\right)-24$}

\Faux{$c\left(t\right)=4\sin\left(\frac{4\pi t}{365}+24\right)$}

\Faux{$c\left(t\right)=4\cos\left(\frac{4\pi t}{365}-24\right)$}

\end{enumerate}

\end{QCM}

\begin{QCM}

On a alors que les valeurs de $c\left(t\right)$ sont comprises entre :

\begin{enumerate}

\Faux{$0$ et $28$ }

\Faux{$0$ et $24$}

\Vrai{$20$ et $28$}

\Faux{$0$ et $4$ }

\end{enumerate}

\end{QCM}

Dès lors, il peut calculer le nombre moyen de clients en appliquant la formule

$\frac{1}{365}\int\limits _{0}^{365}C\left(t\right)dt$ , où $C$ désigne une primitive de $c$.

\begin{QCM}

Parmi les propositions ci-dessous, laquelle donne une formule possible C~?

\begin{enumerate}

\Faux{$C\left(t\right)=\frac{16}{365}\pi\cos\left(\frac{4\pi t}{365}\right)-24t-36$}

\Faux{$C\left(t\right)=\frac{16}{365}\pi\sin\left(\frac{4\pi t}{365}\right)-24t+17$}

\Vrai{$C\left(t\right)=\frac{1460}{2}\pi\sin\left(\frac{4\pi t}{365}\right)+24t$}

\Faux{$C\left(t\right)=-4\sin\left(\frac{4\pi t}{365}\right)$}

\end{enumerate}

\end{QCM}

\begin{QCM}

On obtient ainsi que le nombre moyen de clients journaliers sur l'année en cours est donné par :

\begin{enumerate}

\Vrai{$24$}

\Faux{$24\sin\left(4\pi\right)$ }

\Faux{$4\cos\left(4\pi\right)$}

\Faux{$4$}

\end{enumerate}

\end{QCM}

\end{Sujet}

\begin{Sujet}

\subsection*{Exercice 1}

Pour chacune des propositions de cet exercice, cochez A si elle est vraie\par et B si elle est fausse.

On s'intéresse à la fonction f définie par $f:x\mapsto\left(6x+7\right)e^{6x}$.

\begin{VraiFaux}

\Vrai{ L'ensemble de définition de f est $\left]-\infty,+\infty\right[$.}

\end{VraiFaux}

\begin{VraiFaux}

\Vrai{ La dérivée de f est donnée par $f^{\prime}:x\mapsto\left(36x+48\right)e^{6x}$.}

\end{VraiFaux}

On admet que son approximation quadratique est donné par\par $\qquad f\left(x\right)\approx 7 + 48 x + 162 x^2$.

\begin{VraiFaux}

\Faux{ L'équation de la tangente $T$ à la courbe représentative $\mathscr C$ de $f$ en son point d'abscisse $0$ est donnée par $y=7x + 48 $.}

\end{VraiFaux}

\begin{VraiFaux}

\Faux{ $T$ est au-dessus de $\mathscr C$.}

\end{VraiFaux}

\subsection*{Exercice 2}

Cet exercice doit être rédigé au verso de la grille réponse.

\subsection*{Exercice 3}

\begin{QCM}

Parmi ces fonctions, laquelle est une primitive de la fonction $f$\par définie par $ f(x)=x \cdot \cos (3 x+1)$ ?

\begin{enumerate}

\Faux{ $F:x \mapsto \cos (3 x+1)-3 x \cdot \sin (3 x+1)$}

\Faux{ $G:x \mapsto x \cdot \sin (3 x+1)$}

\Vrai{ $H:x \mapsto \frac{x \cdot \sin (3 x+1)}{3}+\frac{\cos (3 x+1)}{9}$}

\end{enumerate}

\end{QCM}

\begin{QCM}

$\int\limits_{0}^{2}\left(5 t^{3}+3 t^{2}-t+7\right) d t=$

\begin{enumerate}

\Faux{ 71}

\Faux{ $15 t^{2}+6 t-1$}

\Vrai{ 40}

\end{enumerate}

\end{QCM}

\begin{QCM}

La valeur moyenne de la fonction $g: x \rightarrow e^{5 x} $ entre 0 et 2 est de :

\begin{enumerate}

\Vrai{environ $2 202,5$}

\Faux{environ $4405,1$}

\Faux{environ $22 025,5$}

\end{enumerate}

\end{QCM}

\newpage

\subsection*{Problème}

Alain Raffletou cherche à connaitre le nombre moyen de clients journaliers dans une de ses boutiques sur l'année en cours. Pour cela, il modélise le nombre de clients par une fonction donnant le nombre de clients $c\left(t\right)$ en fonction du temps $t$, exprimé en jours. On prends donc $t\in\left[0;365\right]$.

\begin{QCM}

En remarquant que le nombre de clients ne peut être négatif, déterminé, parmi les propositions ci-dessous, laquelle donne la formule de $c\left(t\right)$ en fonction de $t$ :

\begin{enumerate}

\Faux{$c\left(t\right)=3\sin\left(\frac{8\pi t}{365}+12\right)$}

\Faux{$c\left(t\right)=3\cos\left(\frac{8\pi t}{365}-12\right)$}

\Faux{$c\left(t\right)=3\sin\left(\frac{8\pi t}{365}\right)-12$}

\Vrai{$c\left(t\right)=3\cos\left(\frac{8\pi t}{365}\right)+12$}

\end{enumerate}

\end{QCM}

\begin{QCM}

On a alors que les valeurs de $c\left(t\right)$ sont comprises entre :

\begin{enumerate}

\Faux{$0$ et $3$ }

\Vrai{$9$ et $15$}

\Faux{$0$ et $15$ }

\Faux{$0$ et $12$}

\end{enumerate}

\end{QCM}

Dès lors, il peut calculer le nombre moyen de clients en appliquant la formule

$\frac{1}{365}\int\limits _{0}^{365}C\left(t\right)dt$ , où $C$ désigne une primitive de $c$.

\begin{QCM}

Parmi les propositions ci-dessous, laquelle donne une formule possible C~?

\begin{enumerate}

\Vrai{$C\left(t\right)=\frac{1095}{2}\pi\sin\left(\frac{8\pi t}{365}\right)+12t$}

\Faux{$C\left(t\right)=\frac{24}{365}\pi\sin\left(\frac{8\pi t}{365}\right)-12t+7$}

\Faux{$C\left(t\right)=-3\sin\left(\frac{8\pi t}{365}\right)$}

\Faux{$C\left(t\right)=\frac{24}{365}\pi\cos\left(\frac{8\pi t}{365}\right)-12t-15$}

\end{enumerate}

\end{QCM}

\begin{QCM}

On obtient ainsi que le nombre moyen de clients journaliers sur l'année en cours est donné par :

\begin{enumerate}

\Faux{$3$}

\Faux{$12\sin\left(8\pi\right)$ }

\Vrai{$12$}

\Faux{$3\cos\left(8\pi\right)$}

\end{enumerate}

\end{QCM}

\end{Sujet}

\begin{Sujet}

\subsection*{Exercice 1}

Pour chacune des propositions de cet exercice, cochez A si elle est vraie\par et B si elle est fausse.

On s'intéresse à la fonction f définie par $f:x\mapsto\left(4x+3\right)e^{4x}$.

\begin{VraiFaux}

\Faux{ L'ensemble de définition de f est $\left]0,+\infty\right[$.}

\end{VraiFaux}

\begin{VraiFaux}

\Vrai{ La dérivée de f est donnée par $f^{\prime}:x\mapsto\left(16x+16\right)e^{4x}$.}

\end{VraiFaux}

On admet que son approximation quadratique est donné par\par $\qquad f\left(x\right)\approx 3 + 16 x + 40 x^2$.

\begin{VraiFaux}

\Faux{ L'équation de la tangente $T$ à la courbe représentative $\mathscr C$ de $f$ en son point d'abscisse $0$ est donnée par $y=3x + 16 $.}

\end{VraiFaux}

\begin{VraiFaux}

\Faux{ $T$ est au-dessus de $\mathscr C$.}

\end{VraiFaux}

\subsection*{Exercice 2}

Cet exercice doit être rédigé au verso de la grille réponse.

\subsection*{Exercice 3}

\begin{QCM}

Parmi ces fonctions, laquelle est une primitive de la fonction $f$\par définie par $ f(x)=x \cdot \cos (3 x+1)$ ?

\begin{enumerate}

\Faux{ $G:x \mapsto x \cdot \sin (3 x+1)$}

\Vrai{ $H:x \mapsto \frac{x \cdot \sin (3 x+1)}{3}+\frac{\cos (3 x+1)}{9}$}

\Faux{ $F:x \mapsto \cos (3 x+1)-3 x \cdot \sin (3 x+1)$}

\end{enumerate}

\end{QCM}

\begin{QCM}

$\int\limits_{0}^{2}\left(5 t^{3}+3 t^{2}-t+7\right) d t=$

\begin{enumerate}

\Faux{ 71}

\Vrai{ 40}

\Faux{ $15 t^{2}+6 t-1$}

\end{enumerate}

\end{QCM}

\begin{QCM}

La valeur moyenne de la fonction $g: x \rightarrow e^{5 x} $ entre 0 et 2 est de :

\begin{enumerate}

\Faux{environ $4405,1$}

\Vrai{environ $2 202,5$}

\Faux{environ $22 025,5$}

\end{enumerate}

\end{QCM}

\newpage

\subsection*{Problème}

Alain Raffletou cherche à connaitre le nombre moyen de clients journaliers dans une de ses boutiques sur l'année en cours. Pour cela, il modélise le nombre de clients par une fonction donnant le nombre de clients $c\left(t\right)$ en fonction du temps $t$, exprimé en jours. On prends donc $t\in\left[0;365\right]$.

\begin{QCM}

En remarquant que le nombre de clients ne peut être négatif, déterminé, parmi les propositions ci-dessous, laquelle donne la formule de $c\left(t\right)$ en fonction de $t$ :

\begin{enumerate}

\Vrai{$c\left(t\right)=5\cos\left(\frac{4\pi t}{365}\right)+14$}

\Faux{$c\left(t\right)=5\cos\left(\frac{4\pi t}{365}-14\right)$}

\Faux{$c\left(t\right)=5\sin\left(\frac{4\pi t}{365}+14\right)$}

\Faux{$c\left(t\right)=5\sin\left(\frac{4\pi t}{365}\right)-14$}

\end{enumerate}

\end{QCM}

\begin{QCM}

On a alors que les valeurs de $c\left(t\right)$ sont comprises entre :

\begin{enumerate}

\Faux{$0$ et $14$}

\Faux{$0$ et $19$ }

\Vrai{$9$ et $19$}

\Faux{$0$ et $5$ }

\end{enumerate}

\end{QCM}

Dès lors, il peut calculer le nombre moyen de clients en appliquant la formule

$\frac{1}{365}\int\limits _{0}^{365}C\left(t\right)dt$ , où $C$ désigne une primitive de $c$.

\begin{QCM}

Parmi les propositions ci-dessous, laquelle donne une formule possible C~?

\begin{enumerate}

\Faux{$C\left(t\right)=-5\sin\left(\frac{4\pi t}{365}\right)$}

\Faux{$C\left(t\right)=\frac{20}{365}\pi\sin\left(\frac{4\pi t}{365}\right)-14t+5$}

\Vrai{$C\left(t\right)=\frac{1825}{2}\pi\sin\left(\frac{4\pi t}{365}\right)+14t$}

\Faux{$C\left(t\right)=\frac{20}{365}\pi\cos\left(\frac{4\pi t}{365}\right)-14t-13$}

\end{enumerate}

\end{QCM}

\begin{QCM}

On obtient ainsi que le nombre moyen de clients journaliers sur l'année en cours est donné par :

\begin{enumerate}

\Faux{$14\sin\left(4\pi\right)$ }

\Vrai{$14$}

\Faux{$5$}

\Faux{$5\cos\left(4\pi\right)$}

\end{enumerate}

\end{QCM}

\end{Sujet}

\begin{Sujet}

\subsection*{Exercice 1}

Pour chacune des propositions de cet exercice, cochez A si elle est vraie\par et B si elle est fausse.

On s'intéresse à la fonction f définie par $f:x\mapsto\left(5x+4\right)e^{5x}$.

\begin{VraiFaux}

\Faux{ L'ensemble de définition de f est $\left]0,+\infty\right[$.}

\end{VraiFaux}

\begin{VraiFaux}

\Vrai{ La dérivée de f est donnée par $f^{\prime}:x\mapsto\left(25x+25\right)e^{5x}$.}

\end{VraiFaux}

On admet que son approximation quadratique est donné par\par $\qquad f\left(x\right)\approx 4 + 25 x + 75 x^2$.

\begin{VraiFaux}

\Vrai{ L'équation de la tangente $T$ à la courbe représentative $\mathscr C$ de $f$ en son point d'abscisse $0$ est donnée par $y=25 x +4 $.}

\end{VraiFaux}

\begin{VraiFaux}

\Faux{ $T$ est au-dessus de $\mathscr C$.}

\end{VraiFaux}

\subsection*{Exercice 2}

Cet exercice doit être rédigé au verso de la grille réponse.

\subsection*{Exercice 3}

\begin{QCM}

Parmi ces fonctions, laquelle est une primitive de la fonction $f$\par définie par $ f(x)=x \cdot \cos (3 x+1)$ ?

\begin{enumerate}

\Faux{ $G:x \mapsto x \cdot \sin (3 x+1)$}

\Faux{ $F:x \mapsto \cos (3 x+1)-3 x \cdot \sin (3 x+1)$}

\Vrai{ $H:x \mapsto \frac{x \cdot \sin (3 x+1)}{3}+\frac{\cos (3 x+1)}{9}$}

\end{enumerate}

\end{QCM}

\begin{QCM}

$\int\limits_{0}^{2}\left(5 t^{3}+3 t^{2}-t+7\right) d t=$

\begin{enumerate}

\Faux{ 71}

\Faux{ $15 t^{2}+6 t-1$}

\Vrai{ 40}

\end{enumerate}

\end{QCM}

\begin{QCM}

La valeur moyenne de la fonction $g: x \rightarrow e^{5 x} $ entre 0 et 2 est de :

\begin{enumerate}

\Faux{environ $4405,1$}

\Vrai{environ $2 202,5$}

\Faux{environ $22 025,5$}

\end{enumerate}

\end{QCM}

\newpage

\subsection*{Problème}

Alain Raffletou cherche à connaitre le nombre moyen de clients journaliers dans une de ses boutiques sur l'année en cours. Pour cela, il modélise le nombre de clients par une fonction donnant le nombre de clients $c\left(t\right)$ en fonction du temps $t$, exprimé en jours. On prends donc $t\in\left[0;365\right]$.

\begin{QCM}

En remarquant que le nombre de clients ne peut être négatif, déterminé, parmi les propositions ci-dessous, laquelle donne la formule de $c\left(t\right)$ en fonction de $t$ :

\begin{enumerate}

\Faux{$c\left(t\right)=3\cos\left(\frac{6\pi t}{365}-14\right)$}

\Vrai{$c\left(t\right)=3\cos\left(\frac{6\pi t}{365}\right)+14$}

\Faux{$c\left(t\right)=3\sin\left(\frac{6\pi t}{365}+14\right)$}

\Faux{$c\left(t\right)=3\sin\left(\frac{6\pi t}{365}\right)-14$}

\end{enumerate}

\end{QCM}

\begin{QCM}

On a alors que les valeurs de $c\left(t\right)$ sont comprises entre :

\begin{enumerate}

\Vrai{$11$ et $17$}

\Faux{$0$ et $17$ }

\Faux{$0$ et $14$}

\Faux{$0$ et $3$ }

\end{enumerate}

\end{QCM}

Dès lors, il peut calculer le nombre moyen de clients en appliquant la formule

$\frac{1}{365}\int\limits _{0}^{365}C\left(t\right)dt$ , où $C$ désigne une primitive de $c$.

\begin{QCM}

Parmi les propositions ci-dessous, laquelle donne une formule possible C~?

\begin{enumerate}

\Faux{$C\left(t\right)=-3\sin\left(\frac{6\pi t}{365}\right)$}

\Vrai{$C\left(t\right)=\frac{1095}{2}\pi\sin\left(\frac{6\pi t}{365}\right)+14t$}

\Faux{$C\left(t\right)=\frac{18}{365}\pi\sin\left(\frac{6\pi t}{365}\right)-14t+9$}

\Faux{$C\left(t\right)=\frac{18}{365}\pi\cos\left(\frac{6\pi t}{365}\right)-14t-19$}

\end{enumerate}

\end{QCM}

\begin{QCM}

On obtient ainsi que le nombre moyen de clients journaliers sur l'année en cours est donné par :

\begin{enumerate}

\Vrai{$14$}

\Faux{$3\cos\left(6\pi\right)$}

\Faux{$3$}

\Faux{$14\sin\left(6\pi\right)$ }

\end{enumerate}

\end{QCM}

\end{Sujet}

\begin{Sujet}

\subsection*{Exercice 1}

Pour chacune des propositions de cet exercice, cochez A si elle est vraie\par et B si elle est fausse.

On s'intéresse à la fonction f définie par $f:x\mapsto\left(6x+5\right)e^{3x}$.

\begin{VraiFaux}

\Faux{ L'ensemble de définition de f est $\left]0,+\infty\right[$.}

\end{VraiFaux}

\begin{VraiFaux}

\Faux{ La dérivée de f est donnée par $f^{\prime}:x\mapsto\left(18\right)e^{cx}$.}

\end{VraiFaux}

On admet que son approximation quadratique est donné par\par $\qquad f\left(x\right)\approx 5 + 36 x + 126 x^2$.

\begin{VraiFaux}

\Vrai{ L'équation de la tangente $T$ à la courbe représentative $\mathscr C$ de $f$ en son point d'abscisse $0$ est donnée par $y=36 x +5 $.}

\end{VraiFaux}

\begin{VraiFaux}

\Faux{ $T$ est au-dessus de $\mathscr C$.}

\end{VraiFaux}

\subsection*{Exercice 2}

Cet exercice doit être rédigé au verso de la grille réponse.

\subsection*{Exercice 3}

\begin{QCM}

Parmi ces fonctions, laquelle est une primitive de la fonction $f$\par définie par $ f(x)=x \cdot \cos (3 x+1)$ ?

\begin{enumerate}

\Vrai{ $H:x \mapsto \frac{x \cdot \sin (3 x+1)}{3}+\frac{\cos (3 x+1)}{9}$}

\Faux{ $G:x \mapsto x \cdot \sin (3 x+1)$}

\Faux{ $F:x \mapsto \cos (3 x+1)-3 x \cdot \sin (3 x+1)$}

\end{enumerate}

\end{QCM}

\begin{QCM}

$\int\limits_{0}^{2}\left(5 t^{3}+3 t^{2}-t+7\right) d t=$

\begin{enumerate}

\Vrai{ 40}

\Faux{ $15 t^{2}+6 t-1$}

\Faux{ 71}

\end{enumerate}

\end{QCM}

\begin{QCM}

La valeur moyenne de la fonction $g: x \rightarrow e^{5 x} $ entre 0 et 2 est de :

\begin{enumerate}

\Faux{environ $22 025,5$}

\Faux{environ $4405,1$}

\Vrai{environ $2 202,5$}

\end{enumerate}

\end{QCM}

\newpage

\subsection*{Problème}

Alain Raffletou cherche à connaitre le nombre moyen de clients journaliers dans une de ses boutiques sur l'année en cours. Pour cela, il modélise le nombre de clients par une fonction donnant le nombre de clients $c\left(t\right)$ en fonction du temps $t$, exprimé en jours. On prends donc $t\in\left[0;365\right]$.

\begin{QCM}

En remarquant que le nombre de clients ne peut être négatif, déterminé, parmi les propositions ci-dessous, laquelle donne la formule de $c\left(t\right)$ en fonction de $t$ :

\begin{enumerate}

\Vrai{$c\left(t\right)=4\cos\left(\frac{4\pi t}{365}\right)+27$}

\Faux{$c\left(t\right)=4\cos\left(\frac{4\pi t}{365}-27\right)$}

\Faux{$c\left(t\right)=4\sin\left(\frac{4\pi t}{365}+27\right)$}

\Faux{$c\left(t\right)=4\sin\left(\frac{4\pi t}{365}\right)-27$}

\end{enumerate}

\end{QCM}

\begin{QCM}

On a alors que les valeurs de $c\left(t\right)$ sont comprises entre :

\begin{enumerate}

\Vrai{$23$ et $31$}

\Faux{$0$ et $27$}

\Faux{$0$ et $31$ }

\Faux{$0$ et $4$ }

\end{enumerate}

\end{QCM}

Dès lors, il peut calculer le nombre moyen de clients en appliquant la formule

$\frac{1}{365}\int\limits _{0}^{365}C\left(t\right)dt$ , où $C$ désigne une primitive de $c$.

\begin{QCM}

Parmi les propositions ci-dessous, laquelle donne une formule possible C~?

\begin{enumerate}

\Faux{$C\left(t\right)=\frac{16}{365}\pi\sin\left(\frac{4\pi t}{365}\right)-27t+20$}

\Faux{$C\left(t\right)=-4\sin\left(\frac{4\pi t}{365}\right)$}

\Faux{$C\left(t\right)=\frac{16}{365}\pi\cos\left(\frac{4\pi t}{365}\right)-27t-42$}

\Vrai{$C\left(t\right)=\frac{1460}{2}\pi\sin\left(\frac{4\pi t}{365}\right)+27t$}

\end{enumerate}

\end{QCM}

\begin{QCM}

On obtient ainsi que le nombre moyen de clients journaliers sur l'année en cours est donné par :

\begin{enumerate}

\Faux{$27\sin\left(4\pi\right)$ }

\Faux{$4$}

\Faux{$4\cos\left(4\pi\right)$}

\Vrai{$27$}

\end{enumerate}

\end{QCM}

\end{Sujet}

\begin{Sujet}

\subsection*{Exercice 1}

Pour chacune des propositions de cet exercice, cochez A si elle est vraie\par et B si elle est fausse.

On s'intéresse à la fonction f définie par $f:x\mapsto\left(3x+3\right)e^{5x}$.

\begin{VraiFaux}

\Faux{ L'ensemble de définition de f est $\left]0,+\infty\right[$.}

\end{VraiFaux}

\begin{VraiFaux}

\Vrai{ La dérivée de f est donnée par $f^{\prime}:x\mapsto\left(15x+18\right)e^{5x}$.}

\end{VraiFaux}

On admet que son approximation quadratique est donné par\par $\qquad f\left(x\right)\approx 3 + 12 x + \frac{45}{2} x^2$.

\begin{VraiFaux}

\Vrai{ L'équation de la tangente $T$ à la courbe représentative $\mathscr C$ de $f$ en son point d'abscisse $0$ est donnée par $y=12 x +3 $.}

\end{VraiFaux}

\begin{VraiFaux}

\Vrai{ $T$ est au-dessous de $\mathscr C$.}

\end{VraiFaux}

\subsection*{Exercice 2}

Cet exercice doit être rédigé au verso de la grille réponse.

\subsection*{Exercice 3}

\begin{QCM}

Parmi ces fonctions, laquelle est une primitive de la fonction $f$\par définie par $ f(x)=x \cdot \cos (3 x+1)$ ?

\begin{enumerate}

\Vrai{ $H:x \mapsto \frac{x \cdot \sin (3 x+1)}{3}+\frac{\cos (3 x+1)}{9}$}

\Faux{ $G:x \mapsto x \cdot \sin (3 x+1)$}

\Faux{ $F:x \mapsto \cos (3 x+1)-3 x \cdot \sin (3 x+1)$}

\end{enumerate}

\end{QCM}

\begin{QCM}

$\int\limits_{0}^{2}\left(5 t^{3}+3 t^{2}-t+7\right) d t=$

\begin{enumerate}

\Faux{ 71}

\Faux{ $15 t^{2}+6 t-1$}

\Vrai{ 40}

\end{enumerate}

\end{QCM}

\begin{QCM}

La valeur moyenne de la fonction $g: x \rightarrow e^{5 x} $ entre 0 et 2 est de :

\begin{enumerate}

\Faux{environ $4405,1$}

\Vrai{environ $2 202,5$}

\Faux{environ $22 025,5$}

\end{enumerate}

\end{QCM}

\newpage

\subsection*{Problème}

Alain Raffletou cherche à connaitre le nombre moyen de clients journaliers dans une de ses boutiques sur l'année en cours. Pour cela, il modélise le nombre de clients par une fonction donnant le nombre de clients $c\left(t\right)$ en fonction du temps $t$, exprimé en jours. On prends donc $t\in\left[0;365\right]$.

\begin{QCM}

En remarquant que le nombre de clients ne peut être négatif, déterminé, parmi les propositions ci-dessous, laquelle donne la formule de $c\left(t\right)$ en fonction de $t$ :

\begin{enumerate}

\Vrai{$c\left(t\right)=3\cos\left(\frac{6\pi t}{365}\right)+30$}

\Faux{$c\left(t\right)=3\sin\left(\frac{6\pi t}{365}+30\right)$}

\Faux{$c\left(t\right)=3\cos\left(\frac{6\pi t}{365}-30\right)$}

\Faux{$c\left(t\right)=3\sin\left(\frac{6\pi t}{365}\right)-30$}

\end{enumerate}

\end{QCM}

\begin{QCM}

On a alors que les valeurs de $c\left(t\right)$ sont comprises entre :

\begin{enumerate}

\Faux{$0$ et $30$}

\Faux{$0$ et $33$ }

\Vrai{$27$ et $33$}

\Faux{$0$ et $3$ }

\end{enumerate}

\end{QCM}

Dès lors, il peut calculer le nombre moyen de clients en appliquant la formule

$\frac{1}{365}\int\limits _{0}^{365}C\left(t\right)dt$ , où $C$ désigne une primitive de $c$.

\begin{QCM}

Parmi les propositions ci-dessous, laquelle donne une formule possible C~?

\begin{enumerate}

\Vrai{$C\left(t\right)=\frac{1095}{2}\pi\sin\left(\frac{6\pi t}{365}\right)+30t$}

\Faux{$C\left(t\right)=\frac{18}{365}\pi\cos\left(\frac{6\pi t}{365}\right)-30t-51$}

\Faux{$C\left(t\right)=-3\sin\left(\frac{6\pi t}{365}\right)$}

\Faux{$C\left(t\right)=\frac{18}{365}\pi\sin\left(\frac{6\pi t}{365}\right)-30t+25$}

\end{enumerate}

\end{QCM}

\begin{QCM}

On obtient ainsi que le nombre moyen de clients journaliers sur l'année en cours est donné par :

\begin{enumerate}

\Vrai{$30$}

\Faux{$3$}

\Faux{$30\sin\left(6\pi\right)$ }

\Faux{$3\cos\left(6\pi\right)$}

\end{enumerate}

\end{QCM}

\end{Sujet}

\begin{Sujet}

\subsection*{Exercice 1}

Pour chacune des propositions de cet exercice, cochez A si elle est vraie\par et B si elle est fausse.

On s'intéresse à la fonction f définie par $f:x\mapsto\left(7x+4\right)e^{4x}$.

\begin{VraiFaux}

\Vrai{ L'ensemble de définition de f est $\left]-\infty,+\infty\right[$.}

\end{VraiFaux}

\begin{VraiFaux}

\Vrai{ La dérivée de f est donnée par $f^{\prime}:x\mapsto\left(28x+23\right)e^{4x}$.}

\end{VraiFaux}

On admet que son approximation quadratique est donné par\par $\qquad f\left(x\right)\approx 4 + 35 x + 147 x^2$.

\begin{VraiFaux}

\Vrai{ L'équation de la tangente $T$ à la courbe représentative $\mathscr C$ de $f$ en son point d'abscisse $0$ est donnée par $y=35 x +4 $.}

\end{VraiFaux}

\begin{VraiFaux}

\Vrai{ $T$ est au-dessous de $\mathscr C$.}

\end{VraiFaux}

\subsection*{Exercice 2}

Cet exercice doit être rédigé au verso de la grille réponse.

\subsection*{Exercice 3}

\begin{QCM}

Parmi ces fonctions, laquelle est une primitive de la fonction $f$\par définie par $ f(x)=x \cdot \cos (3 x+1)$ ?

\begin{enumerate}

\Vrai{ $H:x \mapsto \frac{x \cdot \sin (3 x+1)}{3}+\frac{\cos (3 x+1)}{9}$}

\Faux{ $F:x \mapsto \cos (3 x+1)-3 x \cdot \sin (3 x+1)$}

\Faux{ $G:x \mapsto x \cdot \sin (3 x+1)$}

\end{enumerate}

\end{QCM}

\begin{QCM}

$\int\limits_{0}^{2}\left(5 t^{3}+3 t^{2}-t+7\right) d t=$

\begin{enumerate}

\Vrai{ 40}

\Faux{ $15 t^{2}+6 t-1$}

\Faux{ 71}

\end{enumerate}

\end{QCM}

\begin{QCM}

La valeur moyenne de la fonction $g: x \rightarrow e^{5 x} $ entre 0 et 2 est de :

\begin{enumerate}

\Faux{environ $4405,1$}

\Faux{environ $22 025,5$}

\Vrai{environ $2 202,5$}

\end{enumerate}

\end{QCM}

\newpage

\subsection*{Problème}

Alain Raffletou cherche à connaitre le nombre moyen de clients journaliers dans une de ses boutiques sur l'année en cours. Pour cela, il modélise le nombre de clients par une fonction donnant le nombre de clients $c\left(t\right)$ en fonction du temps $t$, exprimé en jours. On prends donc $t\in\left[0;365\right]$.

\begin{QCM}

En remarquant que le nombre de clients ne peut être négatif, déterminé, parmi les propositions ci-dessous, laquelle donne la formule de $c\left(t\right)$ en fonction de $t$ :

\begin{enumerate}

\Faux{$c\left(t\right)=5\cos\left(\frac{8\pi t}{365}-22\right)$}

\Faux{$c\left(t\right)=5\sin\left(\frac{8\pi t}{365}\right)-22$}

\Faux{$c\left(t\right)=5\sin\left(\frac{8\pi t}{365}+22\right)$}

\Vrai{$c\left(t\right)=5\cos\left(\frac{8\pi t}{365}\right)+22$}

\end{enumerate}

\end{QCM}

\begin{QCM}

On a alors que les valeurs de $c\left(t\right)$ sont comprises entre :

\begin{enumerate}

\Faux{$0$ et $22$}

\Vrai{$17$ et $27$}

\Faux{$0$ et $27$ }

\Faux{$0$ et $5$ }

\end{enumerate}

\end{QCM}

Dès lors, il peut calculer le nombre moyen de clients en appliquant la formule

$\frac{1}{365}\int\limits _{0}^{365}C\left(t\right)dt$ , où $C$ désigne une primitive de $c$.

\begin{QCM}

Parmi les propositions ci-dessous, laquelle donne une formule possible C~?

\begin{enumerate}

\Faux{$C\left(t\right)=-5\sin\left(\frac{8\pi t}{365}\right)$}

\Faux{$C\left(t\right)=\frac{40}{365}\pi\sin\left(\frac{8\pi t}{365}\right)-22t+13$}

\Vrai{$C\left(t\right)=\frac{1825}{2}\pi\sin\left(\frac{8\pi t}{365}\right)+22t$}

\Faux{$C\left(t\right)=\frac{40}{365}\pi\cos\left(\frac{8\pi t}{365}\right)-22t-29$}

\end{enumerate}

\end{QCM}

\begin{QCM}

On obtient ainsi que le nombre moyen de clients journaliers sur l'année en cours est donné par :

\begin{enumerate}

\Faux{$5$}

\Faux{$22\sin\left(8\pi\right)$ }

\Faux{$5\cos\left(8\pi\right)$}

\Vrai{$22$}

\end{enumerate}

\end{QCM}

\end{Sujet}

\begin{Sujet}

\subsection*{Exercice 1}

Pour chacune des propositions de cet exercice, cochez A si elle est vraie\par et B si elle est fausse.

On s'intéresse à la fonction f définie par $f:x\mapsto\left(6x+3\right)e^{4x}$.

\begin{VraiFaux}

\Vrai{ L'ensemble de définition de f est $\left]-\infty,+\infty\right[$.}

\end{VraiFaux}

\begin{VraiFaux}

\Vrai{ La dérivée de f est donnée par $f^{\prime}:x\mapsto\left(24x+18\right)e^{4x}$.}

\end{VraiFaux}

On admet que son approximation quadratique est donné par\par $\qquad f\left(x\right)\approx 3 + 24 x + 90 x^2$.

\begin{VraiFaux}

\Faux{ L'équation de la tangente $T$ à la courbe représentative $\mathscr C$ de $f$ en son point d'abscisse $0$ est donnée par $y=3x + 24 $.}

\end{VraiFaux}

\begin{VraiFaux}

\Vrai{ $T$ est au-dessous de $\mathscr C$.}

\end{VraiFaux}

\subsection*{Exercice 2}

Cet exercice doit être rédigé au verso de la grille réponse.

\subsection*{Exercice 3}

\begin{QCM}

Parmi ces fonctions, laquelle est une primitive de la fonction $f$\par définie par $ f(x)=x \cdot \cos (3 x+1)$ ?

\begin{enumerate}

\Faux{ $G:x \mapsto x \cdot \sin (3 x+1)$}

\Vrai{ $H:x \mapsto \frac{x \cdot \sin (3 x+1)}{3}+\frac{\cos (3 x+1)}{9}$}

\Faux{ $F:x \mapsto \cos (3 x+1)-3 x \cdot \sin (3 x+1)$}

\end{enumerate}

\end{QCM}

\begin{QCM}

$\int\limits_{0}^{2}\left(5 t^{3}+3 t^{2}-t+7\right) d t=$

\begin{enumerate}

\Vrai{ 40}

\Faux{ $15 t^{2}+6 t-1$}

\Faux{ 71}

\end{enumerate}

\end{QCM}

\begin{QCM}

La valeur moyenne de la fonction $g: x \rightarrow e^{5 x} $ entre 0 et 2 est de :

\begin{enumerate}

\Faux{environ $22 025,5$}

\Vrai{environ $2 202,5$}

\Faux{environ $4405,1$}

\end{enumerate}

\end{QCM}

\newpage

\subsection*{Problème}

Alain Raffletou cherche à connaitre le nombre moyen de clients journaliers dans une de ses boutiques sur l'année en cours. Pour cela, il modélise le nombre de clients par une fonction donnant le nombre de clients $c\left(t\right)$ en fonction du temps $t$, exprimé en jours. On prends donc $t\in\left[0;365\right]$.

\begin{QCM}

En remarquant que le nombre de clients ne peut être négatif, déterminé, parmi les propositions ci-dessous, laquelle donne la formule de $c\left(t\right)$ en fonction de $t$ :

\begin{enumerate}

\Faux{$c\left(t\right)=4\cos\left(\frac{4\pi t}{365}-10\right)$}

\Faux{$c\left(t\right)=4\sin\left(\frac{4\pi t}{365}\right)-10$}

\Vrai{$c\left(t\right)=4\cos\left(\frac{4\pi t}{365}\right)+10$}

\Faux{$c\left(t\right)=4\sin\left(\frac{4\pi t}{365}+10\right)$}

\end{enumerate}

\end{QCM}

\begin{QCM}

On a alors que les valeurs de $c\left(t\right)$ sont comprises entre :

\begin{enumerate}

\Faux{$0$ et $4$ }

\Faux{$0$ et $14$ }

\Vrai{$6$ et $14$}

\Faux{$0$ et $10$}

\end{enumerate}

\end{QCM}

Dès lors, il peut calculer le nombre moyen de clients en appliquant la formule

$\frac{1}{365}\int\limits _{0}^{365}C\left(t\right)dt$ , où $C$ désigne une primitive de $c$.

\begin{QCM}

Parmi les propositions ci-dessous, laquelle donne une formule possible C~?

\begin{enumerate}

\Faux{$C\left(t\right)=\frac{16}{365}\pi\cos\left(\frac{4\pi t}{365}\right)-10t-8$}

\Faux{$C\left(t\right)=\frac{16}{365}\pi\sin\left(\frac{4\pi t}{365}\right)-10t+3$}

\Faux{$C\left(t\right)=-4\sin\left(\frac{4\pi t}{365}\right)$}

\Vrai{$C\left(t\right)=\frac{1460}{2}\pi\sin\left(\frac{4\pi t}{365}\right)+10t$}

\end{enumerate}

\end{QCM}

\begin{QCM}

On obtient ainsi que le nombre moyen de clients journaliers sur l'année en cours est donné par :

\begin{enumerate}

\Faux{$4\cos\left(4\pi\right)$}

\Vrai{$10$}

\Faux{$4$}

\Faux{$10\sin\left(4\pi\right)$ }

\end{enumerate}

\end{QCM}

\end{Sujet}

\begin{Sujet}

\subsection*{Exercice 1}

Pour chacune des propositions de cet exercice, cochez A si elle est vraie\par et B si elle est fausse.

On s'intéresse à la fonction f définie par $f:x\mapsto\left(2x+2\right)e^{6x}$.

\begin{VraiFaux}

\Vrai{ L'ensemble de définition de f est $\left]-\infty,+\infty\right[$.}

\end{VraiFaux}

\begin{VraiFaux}

\Faux{ La dérivée de f est donnée par $f^{\prime}:x\mapsto\left(12\right)e^{cx}$.}

\end{VraiFaux}

On admet que son approximation quadratique est donné par\par $\qquad f\left(x\right)\approx 2 + 6 x + 8 x^2$.

\begin{VraiFaux}

\Vrai{ L'équation de la tangente $T$ à la courbe représentative $\mathscr C$ de $f$ en son point d'abscisse $0$ est donnée par $y=6 x +2 $.}

\end{VraiFaux}

\begin{VraiFaux}

\Vrai{ $T$ est au-dessous de $\mathscr C$.}

\end{VraiFaux}

\subsection*{Exercice 2}

Cet exercice doit être rédigé au verso de la grille réponse.

\subsection*{Exercice 3}

\begin{QCM}

Parmi ces fonctions, laquelle est une primitive de la fonction $f$\par définie par $ f(x)=x \cdot \cos (3 x+1)$ ?

\begin{enumerate}

\Vrai{ $H:x \mapsto \frac{x \cdot \sin (3 x+1)}{3}+\frac{\cos (3 x+1)}{9}$}

\Faux{ $F:x \mapsto \cos (3 x+1)-3 x \cdot \sin (3 x+1)$}

\Faux{ $G:x \mapsto x \cdot \sin (3 x+1)$}

\end{enumerate}

\end{QCM}

\begin{QCM}

$\int\limits_{0}^{2}\left(5 t^{3}+3 t^{2}-t+7\right) d t=$

\begin{enumerate}

\Vrai{ 40}

\Faux{ 71}

\Faux{ $15 t^{2}+6 t-1$}

\end{enumerate}

\end{QCM}

\begin{QCM}

La valeur moyenne de la fonction $g: x \rightarrow e^{5 x} $ entre 0 et 2 est de :

\begin{enumerate}

\Vrai{environ $2 202,5$}

\Faux{environ $22 025,5$}

\Faux{environ $4405,1$}

\end{enumerate}

\end{QCM}

\newpage

\subsection*{Problème}

Alain Raffletou cherche à connaitre le nombre moyen de clients journaliers dans une de ses boutiques sur l'année en cours. Pour cela, il modélise le nombre de clients par une fonction donnant le nombre de clients $c\left(t\right)$ en fonction du temps $t$, exprimé en jours. On prends donc $t\in\left[0;365\right]$.

\begin{QCM}

En remarquant que le nombre de clients ne peut être négatif, déterminé, parmi les propositions ci-dessous, laquelle donne la formule de $c\left(t\right)$ en fonction de $t$ :

\begin{enumerate}

\Faux{$c\left(t\right)=3\cos\left(\frac{4\pi t}{365}-15\right)$}

\Faux{$c\left(t\right)=3\sin\left(\frac{4\pi t}{365}+15\right)$}

\Faux{$c\left(t\right)=3\sin\left(\frac{4\pi t}{365}\right)-15$}

\Vrai{$c\left(t\right)=3\cos\left(\frac{4\pi t}{365}\right)+15$}

\end{enumerate}

\end{QCM}

\begin{QCM}

On a alors que les valeurs de $c\left(t\right)$ sont comprises entre :

\begin{enumerate}

\Faux{$0$ et $15$}

\Vrai{$12$ et $18$}

\Faux{$0$ et $3$ }

\Faux{$0$ et $18$ }

\end{enumerate}

\end{QCM}

Dès lors, il peut calculer le nombre moyen de clients en appliquant la formule

$\frac{1}{365}\int\limits _{0}^{365}C\left(t\right)dt$ , où $C$ désigne une primitive de $c$.

\begin{QCM}

Parmi les propositions ci-dessous, laquelle donne une formule possible C~?

\begin{enumerate}

\Vrai{$C\left(t\right)=\frac{1095}{2}\pi\sin\left(\frac{4\pi t}{365}\right)+15t$}

\Faux{$C\left(t\right)=-3\sin\left(\frac{4\pi t}{365}\right)$}

\Faux{$C\left(t\right)=\frac{12}{365}\pi\sin\left(\frac{4\pi t}{365}\right)-15t+10$}

\Faux{$C\left(t\right)=\frac{12}{365}\pi\cos\left(\frac{4\pi t}{365}\right)-15t-21$}

\end{enumerate}

\end{QCM}

\begin{QCM}

On obtient ainsi que le nombre moyen de clients journaliers sur l'année en cours est donné par :

\begin{enumerate}

\Faux{$3\cos\left(4\pi\right)$}

\Faux{$3$}

\Vrai{$15$}

\Faux{$15\sin\left(4\pi\right)$ }

\end{enumerate}

\end{QCM}

\end{Sujet}

\begin{Sujet}

\subsection*{Exercice 1}

Pour chacune des propositions de cet exercice, cochez A si elle est vraie\par et B si elle est fausse.

On s'intéresse à la fonction f définie par $f:x\mapsto\left(5x+5\right)e^{3x}$.

\begin{VraiFaux}

\Vrai{ L'ensemble de définition de f est $\left]-\infty,+\infty\right[$.}

\end{VraiFaux}

\begin{VraiFaux}

\Faux{ La dérivée de f est donnée par $f^{\prime}:x\mapsto\left(15\right)e^{cx}$.}

\end{VraiFaux}

On admet que son approximation quadratique est donné par\par $\qquad f\left(x\right)\approx 5 + 30 x + \frac{175}{2} x^2$.

\begin{VraiFaux}

\Vrai{ L'équation de la tangente $T$ à la courbe représentative $\mathscr C$ de $f$ en son point d'abscisse $0$ est donnée par $y=30 x +5 $.}

\end{VraiFaux}

\begin{VraiFaux}

\Faux{ $T$ est au-dessus de $\mathscr C$.}

\end{VraiFaux}

\subsection*{Exercice 2}

Cet exercice doit être rédigé au verso de la grille réponse.

\subsection*{Exercice 3}

\begin{QCM}

Parmi ces fonctions, laquelle est une primitive de la fonction $f$\par définie par $ f(x)=x \cdot \cos (3 x+1)$ ?

\begin{enumerate}

\Faux{ $G:x \mapsto x \cdot \sin (3 x+1)$}

\Faux{ $F:x \mapsto \cos (3 x+1)-3 x \cdot \sin (3 x+1)$}

\Vrai{ $H:x \mapsto \frac{x \cdot \sin (3 x+1)}{3}+\frac{\cos (3 x+1)}{9}$}

\end{enumerate}

\end{QCM}

\begin{QCM}

$\int\limits_{0}^{2}\left(5 t^{3}+3 t^{2}-t+7\right) d t=$

\begin{enumerate}

\Vrai{ 40}

\Faux{ $15 t^{2}+6 t-1$}

\Faux{ 71}

\end{enumerate}

\end{QCM}

\begin{QCM}

La valeur moyenne de la fonction $g: x \rightarrow e^{5 x} $ entre 0 et 2 est de :

\begin{enumerate}

\Vrai{environ $2 202,5$}

\Faux{environ $22 025,5$}

\Faux{environ $4405,1$}

\end{enumerate}

\end{QCM}

\newpage

\subsection*{Problème}

Alain Raffletou cherche à connaitre le nombre moyen de clients journaliers dans une de ses boutiques sur l'année en cours. Pour cela, il modélise le nombre de clients par une fonction donnant le nombre de clients $c\left(t\right)$ en fonction du temps $t$, exprimé en jours. On prends donc $t\in\left[0;365\right]$.

\begin{QCM}

En remarquant que le nombre de clients ne peut être négatif, déterminé, parmi les propositions ci-dessous, laquelle donne la formule de $c\left(t\right)$ en fonction de $t$ :

\begin{enumerate}

\Faux{$c\left(t\right)=5\sin\left(\frac{8\pi t}{365}\right)-15$}

\Vrai{$c\left(t\right)=5\cos\left(\frac{8\pi t}{365}\right)+15$}

\Faux{$c\left(t\right)=5\cos\left(\frac{8\pi t}{365}-15\right)$}

\Faux{$c\left(t\right)=5\sin\left(\frac{8\pi t}{365}+15\right)$}

\end{enumerate}

\end{QCM}

\begin{QCM}

On a alors que les valeurs de $c\left(t\right)$ sont comprises entre :

\begin{enumerate}

\Vrai{$10$ et $20$}

\Faux{$0$ et $5$ }

\Faux{$0$ et $15$}

\Faux{$0$ et $20$ }

\end{enumerate}

\end{QCM}

Dès lors, il peut calculer le nombre moyen de clients en appliquant la formule

$\frac{1}{365}\int\limits _{0}^{365}C\left(t\right)dt$ , où $C$ désigne une primitive de $c$.

\begin{QCM}

Parmi les propositions ci-dessous, laquelle donne une formule possible C~?

\begin{enumerate}

\Faux{$C\left(t\right)=-5\sin\left(\frac{8\pi t}{365}\right)$}

\Vrai{$C\left(t\right)=\frac{1825}{2}\pi\sin\left(\frac{8\pi t}{365}\right)+15t$}

\Faux{$C\left(t\right)=\frac{40}{365}\pi\cos\left(\frac{8\pi t}{365}\right)-15t-15$}

\Faux{$C\left(t\right)=\frac{40}{365}\pi\sin\left(\frac{8\pi t}{365}\right)-15t+6$}

\end{enumerate}

\end{QCM}

\begin{QCM}

On obtient ainsi que le nombre moyen de clients journaliers sur l'année en cours est donné par :

\begin{enumerate}

\Vrai{$15$}

\Faux{$5$}

\Faux{$5\cos\left(8\pi\right)$}

\Faux{$15\sin\left(8\pi\right)$ }

\end{enumerate}

\end{QCM}

\end{Sujet}

\begin{Sujet}

\subsection*{Exercice 1}

Pour chacune des propositions de cet exercice, cochez A si elle est vraie\par et B si elle est fausse.

On s'intéresse à la fonction f définie par $f:x\mapsto\left(6x+5\right)e^{5x}$.

\begin{VraiFaux}

\Faux{ L'ensemble de définition de f est $\left]0,+\infty\right[$.}

\end{VraiFaux}

\begin{VraiFaux}

\Vrai{ La dérivée de f est donnée par $f^{\prime}:x\mapsto\left(30x+31\right)e^{5x}$.}

\end{VraiFaux}

On admet que son approximation quadratique est donné par\par $\qquad f\left(x\right)\approx 5 + 36 x + 126 x^2$.

\begin{VraiFaux}

\Faux{ L'équation de la tangente $T$ à la courbe représentative $\mathscr C$ de $f$ en son point d'abscisse $0$ est donnée par $y=5x + 36 $.}

\end{VraiFaux}

\begin{VraiFaux}

\Vrai{ $T$ est au-dessous de $\mathscr C$.}

\end{VraiFaux}

\subsection*{Exercice 2}

Cet exercice doit être rédigé au verso de la grille réponse.

\subsection*{Exercice 3}

\begin{QCM}

Parmi ces fonctions, laquelle est une primitive de la fonction $f$\par définie par $ f(x)=x \cdot \cos (3 x+1)$ ?

\begin{enumerate}

\Faux{ $F:x \mapsto \cos (3 x+1)-3 x \cdot \sin (3 x+1)$}

\Vrai{ $H:x \mapsto \frac{x \cdot \sin (3 x+1)}{3}+\frac{\cos (3 x+1)}{9}$}

\Faux{ $G:x \mapsto x \cdot \sin (3 x+1)$}

\end{enumerate}

\end{QCM}

\begin{QCM}

$\int\limits_{0}^{2}\left(5 t^{3}+3 t^{2}-t+7\right) d t=$

\begin{enumerate}

\Faux{ 71}

\Vrai{ 40}

\Faux{ $15 t^{2}+6 t-1$}

\end{enumerate}

\end{QCM}

\begin{QCM}

La valeur moyenne de la fonction $g: x \rightarrow e^{5 x} $ entre 0 et 2 est de :

\begin{enumerate}

\Faux{environ $22 025,5$}

\Faux{environ $4405,1$}

\Vrai{environ $2 202,5$}

\end{enumerate}

\end{QCM}

\newpage

\subsection*{Problème}

Alain Raffletou cherche à connaitre le nombre moyen de clients journaliers dans une de ses boutiques sur l'année en cours. Pour cela, il modélise le nombre de clients par une fonction donnant le nombre de clients $c\left(t\right)$ en fonction du temps $t$, exprimé en jours. On prends donc $t\in\left[0;365\right]$.

\begin{QCM}

En remarquant que le nombre de clients ne peut être négatif, déterminé, parmi les propositions ci-dessous, laquelle donne la formule de $c\left(t\right)$ en fonction de $t$ :

\begin{enumerate}

\Faux{$c\left(t\right)=5\sin\left(\frac{6\pi t}{365}+15\right)$}

\Faux{$c\left(t\right)=5\sin\left(\frac{6\pi t}{365}\right)-15$}

\Faux{$c\left(t\right)=5\cos\left(\frac{6\pi t}{365}-15\right)$}

\Vrai{$c\left(t\right)=5\cos\left(\frac{6\pi t}{365}\right)+15$}

\end{enumerate}

\end{QCM}

\begin{QCM}

On a alors que les valeurs de $c\left(t\right)$ sont comprises entre :

\begin{enumerate}

\Faux{$0$ et $20$ }

\Vrai{$10$ et $20$}

\Faux{$0$ et $5$ }

\Faux{$0$ et $15$}

\end{enumerate}

\end{QCM}

Dès lors, il peut calculer le nombre moyen de clients en appliquant la formule

$\frac{1}{365}\int\limits _{0}^{365}C\left(t\right)dt$ , où $C$ désigne une primitive de $c$.

\begin{QCM}

Parmi les propositions ci-dessous, laquelle donne une formule possible C~?

\begin{enumerate}

\Faux{$C\left(t\right)=\frac{30}{365}\pi\cos\left(\frac{6\pi t}{365}\right)-15t-15$}

\Vrai{$C\left(t\right)=\frac{1825}{2}\pi\sin\left(\frac{6\pi t}{365}\right)+15t$}

\Faux{$C\left(t\right)=\frac{30}{365}\pi\sin\left(\frac{6\pi t}{365}\right)-15t+6$}

\Faux{$C\left(t\right)=-5\sin\left(\frac{6\pi t}{365}\right)$}

\end{enumerate}

\end{QCM}

\begin{QCM}

On obtient ainsi que le nombre moyen de clients journaliers sur l'année en cours est donné par :

\begin{enumerate}

\Faux{$5$}

\Vrai{$15$}

\Faux{$5\cos\left(6\pi\right)$}

\Faux{$15\sin\left(6\pi\right)$ }

\end{enumerate}

\end{QCM}

\end{Sujet}

\begin{Sujet}

\subsection*{Exercice 1}

Pour chacune des propositions de cet exercice, cochez A si elle est vraie\par et B si elle est fausse.

On s'intéresse à la fonction f définie par $f:x\mapsto\left(3x+4\right)e^{5x}$.

\begin{VraiFaux}

\Faux{ L'ensemble de définition de f est $\left]0,+\infty\right[$.}

\end{VraiFaux}

\begin{VraiFaux}

\Faux{ La dérivée de f est donnée par $f^{\prime}:x\mapsto\left(15\right)e^{cx}$.}

\end{VraiFaux}

On admet que son approximation quadratique est donné par\par $\qquad f\left(x\right)\approx 4 + 15 x + 27 x^2$.

\begin{VraiFaux}

\Faux{ L'équation de la tangente $T$ à la courbe représentative $\mathscr C$ de $f$ en son point d'abscisse $0$ est donnée par $y=4x + 15 $.}

\end{VraiFaux}

\begin{VraiFaux}

\Vrai{ $T$ est au-dessous de $\mathscr C$.}

\end{VraiFaux}

\subsection*{Exercice 2}

Cet exercice doit être rédigé au verso de la grille réponse.

\subsection*{Exercice 3}

\begin{QCM}

Parmi ces fonctions, laquelle est une primitive de la fonction $f$\par définie par $ f(x)=x \cdot \cos (3 x+1)$ ?

\begin{enumerate}

\Faux{ $F:x \mapsto \cos (3 x+1)-3 x \cdot \sin (3 x+1)$}

\Vrai{ $H:x \mapsto \frac{x \cdot \sin (3 x+1)}{3}+\frac{\cos (3 x+1)}{9}$}

\Faux{ $G:x \mapsto x \cdot \sin (3 x+1)$}

\end{enumerate}

\end{QCM}

\begin{QCM}

$\int\limits_{0}^{2}\left(5 t^{3}+3 t^{2}-t+7\right) d t=$

\begin{enumerate}

\Faux{ $15 t^{2}+6 t-1$}

\Faux{ 71}

\Vrai{ 40}

\end{enumerate}

\end{QCM}

\begin{QCM}

La valeur moyenne de la fonction $g: x \rightarrow e^{5 x} $ entre 0 et 2 est de :

\begin{enumerate}

\Faux{environ $22 025,5$}

\Vrai{environ $2 202,5$}

\Faux{environ $4405,1$}

\end{enumerate}

\end{QCM}

\newpage

\subsection*{Problème}

Alain Raffletou cherche à connaitre le nombre moyen de clients journaliers dans une de ses boutiques sur l'année en cours. Pour cela, il modélise le nombre de clients par une fonction donnant le nombre de clients $c\left(t\right)$ en fonction du temps $t$, exprimé en jours. On prends donc $t\in\left[0;365\right]$.

\begin{QCM}

En remarquant que le nombre de clients ne peut être négatif, déterminé, parmi les propositions ci-dessous, laquelle donne la formule de $c\left(t\right)$ en fonction de $t$ :

\begin{enumerate}

\Faux{$c\left(t\right)=3\sin\left(\frac{8\pi t}{365}+16\right)$}

\Vrai{$c\left(t\right)=3\cos\left(\frac{8\pi t}{365}\right)+16$}

\Faux{$c\left(t\right)=3\sin\left(\frac{8\pi t}{365}\right)-16$}

\Faux{$c\left(t\right)=3\cos\left(\frac{8\pi t}{365}-16\right)$}

\end{enumerate}

\end{QCM}

\begin{QCM}

On a alors que les valeurs de $c\left(t\right)$ sont comprises entre :

\begin{enumerate}

\Faux{$0$ et $16$}

\Faux{$0$ et $19$ }

\Faux{$0$ et $3$ }

\Vrai{$13$ et $19$}

\end{enumerate}

\end{QCM}

Dès lors, il peut calculer le nombre moyen de clients en appliquant la formule

$\frac{1}{365}\int\limits _{0}^{365}C\left(t\right)dt$ , où $C$ désigne une primitive de $c$.

\begin{QCM}

Parmi les propositions ci-dessous, laquelle donne une formule possible C~?

\begin{enumerate}

\Faux{$C\left(t\right)=-3\sin\left(\frac{8\pi t}{365}\right)$}

\Faux{$C\left(t\right)=\frac{24}{365}\pi\cos\left(\frac{8\pi t}{365}\right)-16t-23$}

\Faux{$C\left(t\right)=\frac{24}{365}\pi\sin\left(\frac{8\pi t}{365}\right)-16t+11$}

\Vrai{$C\left(t\right)=\frac{1095}{2}\pi\sin\left(\frac{8\pi t}{365}\right)+16t$}

\end{enumerate}

\end{QCM}

\begin{QCM}

On obtient ainsi que le nombre moyen de clients journaliers sur l'année en cours est donné par :

\begin{enumerate}

\Faux{$3\cos\left(8\pi\right)$}

\Faux{$16\sin\left(8\pi\right)$ }

\Faux{$3$}

\Vrai{$16$}

\end{enumerate}

\end{QCM}

\end{Sujet}

\begin{Sujet}

\subsection*{Exercice 1}

Pour chacune des propositions de cet exercice, cochez A si elle est vraie\par et B si elle est fausse.

On s'intéresse à la fonction f définie par $f:x\mapsto\left(6x+7\right)e^{5x}$.

\begin{VraiFaux}

\Faux{ L'ensemble de définition de f est $\left]0,+\infty\right[$.}

\end{VraiFaux}

\begin{VraiFaux}

\Vrai{ La dérivée de f est donnée par $f^{\prime}:x\mapsto\left(30x+41\right)e^{5x}$.}

\end{VraiFaux}

On admet que son approximation quadratique est donné par\par $\qquad f\left(x\right)\approx 7 + 48 x + 162 x^2$.

\begin{VraiFaux}

\Faux{ L'équation de la tangente $T$ à la courbe représentative $\mathscr C$ de $f$ en son point d'abscisse $0$ est donnée par $y=7x + 48 $.}

\end{VraiFaux}

\begin{VraiFaux}

\Vrai{ $T$ est au-dessous de $\mathscr C$.}

\end{VraiFaux}

\subsection*{Exercice 2}

Cet exercice doit être rédigé au verso de la grille réponse.

\subsection*{Exercice 3}

\begin{QCM}

Parmi ces fonctions, laquelle est une primitive de la fonction $f$\par définie par $ f(x)=x \cdot \cos (3 x+1)$ ?

\begin{enumerate}

\Faux{ $F:x \mapsto \cos (3 x+1)-3 x \cdot \sin (3 x+1)$}

\Vrai{ $H:x \mapsto \frac{x \cdot \sin (3 x+1)}{3}+\frac{\cos (3 x+1)}{9}$}

\Faux{ $G:x \mapsto x \cdot \sin (3 x+1)$}

\end{enumerate}

\end{QCM}

\begin{QCM}

$\int\limits_{0}^{2}\left(5 t^{3}+3 t^{2}-t+7\right) d t=$

\begin{enumerate}

\Faux{ 71}

\Vrai{ 40}

\Faux{ $15 t^{2}+6 t-1$}

\end{enumerate}

\end{QCM}

\begin{QCM}

La valeur moyenne de la fonction $g: x \rightarrow e^{5 x} $ entre 0 et 2 est de :

\begin{enumerate}

\Faux{environ $22 025,5$}

\Faux{environ $4405,1$}

\Vrai{environ $2 202,5$}

\end{enumerate}

\end{QCM}

\newpage

\subsection*{Problème}

Alain Raffletou cherche à connaitre le nombre moyen de clients journaliers dans une de ses boutiques sur l'année en cours. Pour cela, il modélise le nombre de clients par une fonction donnant le nombre de clients $c\left(t\right)$ en fonction du temps $t$, exprimé en jours. On prends donc $t\in\left[0;365\right]$.

\begin{QCM}

En remarquant que le nombre de clients ne peut être négatif, déterminé, parmi les propositions ci-dessous, laquelle donne la formule de $c\left(t\right)$ en fonction de $t$ :

\begin{enumerate}

\Vrai{$c\left(t\right)=3\cos\left(\frac{4\pi t}{365}\right)+18$}

\Faux{$c\left(t\right)=3\sin\left(\frac{4\pi t}{365}\right)-18$}

\Faux{$c\left(t\right)=3\cos\left(\frac{4\pi t}{365}-18\right)$}

\Faux{$c\left(t\right)=3\sin\left(\frac{4\pi t}{365}+18\right)$}

\end{enumerate}

\end{QCM}

\begin{QCM}

On a alors que les valeurs de $c\left(t\right)$ sont comprises entre :

\begin{enumerate}

\Faux{$0$ et $18$}

\Faux{$0$ et $3$ }

\Vrai{$15$ et $21$}

\Faux{$0$ et $21$ }

\end{enumerate}

\end{QCM}

Dès lors, il peut calculer le nombre moyen de clients en appliquant la formule

$\frac{1}{365}\int\limits _{0}^{365}C\left(t\right)dt$ , où $C$ désigne une primitive de $c$.

\begin{QCM}

Parmi les propositions ci-dessous, laquelle donne une formule possible C~?

\begin{enumerate}

\Faux{$C\left(t\right)=\frac{12}{365}\pi\sin\left(\frac{4\pi t}{365}\right)-18t+13$}

\Vrai{$C\left(t\right)=\frac{1095}{2}\pi\sin\left(\frac{4\pi t}{365}\right)+18t$}

\Faux{$C\left(t\right)=\frac{12}{365}\pi\cos\left(\frac{4\pi t}{365}\right)-18t-27$}

\Faux{$C\left(t\right)=-3\sin\left(\frac{4\pi t}{365}\right)$}

\end{enumerate}

\end{QCM}

\begin{QCM}

On obtient ainsi que le nombre moyen de clients journaliers sur l'année en cours est donné par :

\begin{enumerate}

\Faux{$3$}

\Vrai{$18$}

\Faux{$18\sin\left(4\pi\right)$ }

\Faux{$3\cos\left(4\pi\right)$}

\end{enumerate}

\end{QCM}

\end{Sujet}

\begin{Sujet}

\subsection*{Exercice 1}

Pour chacune des propositions de cet exercice, cochez A si elle est vraie\par et B si elle est fausse.

On s'intéresse à la fonction f définie par $f:x\mapsto\left(2x+4\right)e^{5x}$.

\begin{VraiFaux}

\Faux{ L'ensemble de définition de f est $\left]0,+\infty\right[$.}

\end{VraiFaux}

\begin{VraiFaux}

\Vrai{ La dérivée de f est donnée par $f^{\prime}:x\mapsto\left(10x+22\right)e^{5x}$.}

\end{VraiFaux}

On admet que son approximation quadratique est donné par\par $\qquad f\left(x\right)\approx 4 + 10 x + 12 x^2$.

\begin{VraiFaux}

\Faux{ L'équation de la tangente $T$ à la courbe représentative $\mathscr C$ de $f$ en son point d'abscisse $0$ est donnée par $y=4x + 10 $.}

\end{VraiFaux}

\begin{VraiFaux}

\Vrai{ $T$ est au-dessous de $\mathscr C$.}

\end{VraiFaux}

\subsection*{Exercice 2}

Cet exercice doit être rédigé au verso de la grille réponse.

\subsection*{Exercice 3}

\begin{QCM}

Parmi ces fonctions, laquelle est une primitive de la fonction $f$\par définie par $ f(x)=x \cdot \cos (3 x+1)$ ?

\begin{enumerate}

\Faux{ $F:x \mapsto \cos (3 x+1)-3 x \cdot \sin (3 x+1)$}

\Faux{ $G:x \mapsto x \cdot \sin (3 x+1)$}

\Vrai{ $H:x \mapsto \frac{x \cdot \sin (3 x+1)}{3}+\frac{\cos (3 x+1)}{9}$}

\end{enumerate}

\end{QCM}

\begin{QCM}

$\int\limits_{0}^{2}\left(5 t^{3}+3 t^{2}-t+7\right) d t=$

\begin{enumerate}

\Faux{ $15 t^{2}+6 t-1$}

\Faux{ 71}

\Vrai{ 40}

\end{enumerate}

\end{QCM}

\begin{QCM}

La valeur moyenne de la fonction $g: x \rightarrow e^{5 x} $ entre 0 et 2 est de :

\begin{enumerate}

\Faux{environ $22 025,5$}

\Faux{environ $4405,1$}

\Vrai{environ $2 202,5$}

\end{enumerate}

\end{QCM}

\newpage

\subsection*{Problème}

Alain Raffletou cherche à connaitre le nombre moyen de clients journaliers dans une de ses boutiques sur l'année en cours. Pour cela, il modélise le nombre de clients par une fonction donnant le nombre de clients $c\left(t\right)$ en fonction du temps $t$, exprimé en jours. On prends donc $t\in\left[0;365\right]$.

\begin{QCM}

En remarquant que le nombre de clients ne peut être négatif, déterminé, parmi les propositions ci-dessous, laquelle donne la formule de $c\left(t\right)$ en fonction de $t$ :

\begin{enumerate}

\Vrai{$c\left(t\right)=3\cos\left(\frac{6\pi t}{365}\right)+19$}

\Faux{$c\left(t\right)=3\cos\left(\frac{6\pi t}{365}-19\right)$}

\Faux{$c\left(t\right)=3\sin\left(\frac{6\pi t}{365}\right)-19$}

\Faux{$c\left(t\right)=3\sin\left(\frac{6\pi t}{365}+19\right)$}

\end{enumerate}

\end{QCM}

\begin{QCM}

On a alors que les valeurs de $c\left(t\right)$ sont comprises entre :

\begin{enumerate}

\Faux{$0$ et $19$}

\Vrai{$16$ et $22$}

\Faux{$0$ et $3$ }

\Faux{$0$ et $22$ }

\end{enumerate}

\end{QCM}

Dès lors, il peut calculer le nombre moyen de clients en appliquant la formule

$\frac{1}{365}\int\limits _{0}^{365}C\left(t\right)dt$ , où $C$ désigne une primitive de $c$.

\begin{QCM}

Parmi les propositions ci-dessous, laquelle donne une formule possible C~?

\begin{enumerate}

\Vrai{$C\left(t\right)=\frac{1095}{2}\pi\sin\left(\frac{6\pi t}{365}\right)+19t$}

\Faux{$C\left(t\right)=\frac{18}{365}\pi\cos\left(\frac{6\pi t}{365}\right)-19t-29$}

\Faux{$C\left(t\right)=-3\sin\left(\frac{6\pi t}{365}\right)$}

\Faux{$C\left(t\right)=\frac{18}{365}\pi\sin\left(\frac{6\pi t}{365}\right)-19t+14$}

\end{enumerate}

\end{QCM}

\begin{QCM}

On obtient ainsi que le nombre moyen de clients journaliers sur l'année en cours est donné par :

\begin{enumerate}

\Faux{$3$}

\Vrai{$19$}

\Faux{$19\sin\left(6\pi\right)$ }

\Faux{$3\cos\left(6\pi\right)$}

\end{enumerate}

\end{QCM}

\end{Sujet}

\begin{Sujet}

\subsection*{Exercice 1}

Pour chacune des propositions de cet exercice, cochez A si elle est vraie\par et B si elle est fausse.

On s'intéresse à la fonction f définie par $f:x\mapsto\left(6x+6\right)e^{6x}$.

\begin{VraiFaux}

\Vrai{ L'ensemble de définition de f est $\left]-\infty,+\infty\right[$.}

\end{VraiFaux}

\begin{VraiFaux}

\Faux{ La dérivée de f est donnée par $f^{\prime}:x\mapsto\left(36\right)e^{cx}$.}

\end{VraiFaux}

On admet que son approximation quadratique est donné par\par $\qquad f\left(x\right)\approx 6 + 42 x + 144 x^2$.

\begin{VraiFaux}

\Vrai{ L'équation de la tangente $T$ à la courbe représentative $\mathscr C$ de $f$ en son point d'abscisse $0$ est donnée par $y=42 x +6 $.}

\end{VraiFaux}

\begin{VraiFaux}

\Vrai{ $T$ est au-dessous de $\mathscr C$.}

\end{VraiFaux}

\subsection*{Exercice 2}

Cet exercice doit être rédigé au verso de la grille réponse.

\subsection*{Exercice 3}

\begin{QCM}

Parmi ces fonctions, laquelle est une primitive de la fonction $f$\par définie par $ f(x)=x \cdot \cos (3 x+1)$ ?

\begin{enumerate}

\Faux{ $G:x \mapsto x \cdot \sin (3 x+1)$}

\Faux{ $F:x \mapsto \cos (3 x+1)-3 x \cdot \sin (3 x+1)$}

\Vrai{ $H:x \mapsto \frac{x \cdot \sin (3 x+1)}{3}+\frac{\cos (3 x+1)}{9}$}

\end{enumerate}

\end{QCM}

\begin{QCM}

$\int\limits_{0}^{2}\left(5 t^{3}+3 t^{2}-t+7\right) d t=$

\begin{enumerate}

\Faux{ $15 t^{2}+6 t-1$}

\Faux{ 71}

\Vrai{ 40}

\end{enumerate}

\end{QCM}

\begin{QCM}

La valeur moyenne de la fonction $g: x \rightarrow e^{5 x} $ entre 0 et 2 est de :

\begin{enumerate}

\Faux{environ $4405,1$}

\Vrai{environ $2 202,5$}

\Faux{environ $22 025,5$}

\end{enumerate}

\end{QCM}

\newpage

\subsection*{Problème}

Alain Raffletou cherche à connaitre le nombre moyen de clients journaliers dans une de ses boutiques sur l'année en cours. Pour cela, il modélise le nombre de clients par une fonction donnant le nombre de clients $c\left(t\right)$ en fonction du temps $t$, exprimé en jours. On prends donc $t\in\left[0;365\right]$.

\begin{QCM}

En remarquant que le nombre de clients ne peut être négatif, déterminé, parmi les propositions ci-dessous, laquelle donne la formule de $c\left(t\right)$ en fonction de $t$ :

\begin{enumerate}

\Vrai{$c\left(t\right)=3\cos\left(\frac{8\pi t}{365}\right)+16$}

\Faux{$c\left(t\right)=3\sin\left(\frac{8\pi t}{365}\right)-16$}

\Faux{$c\left(t\right)=3\cos\left(\frac{8\pi t}{365}-16\right)$}

\Faux{$c\left(t\right)=3\sin\left(\frac{8\pi t}{365}+16\right)$}

\end{enumerate}

\end{QCM}

\begin{QCM}

On a alors que les valeurs de $c\left(t\right)$ sont comprises entre :

\begin{enumerate}

\Vrai{$13$ et $19$}

\Faux{$0$ et $3$ }

\Faux{$0$ et $16$}

\Faux{$0$ et $19$ }

\end{enumerate}

\end{QCM}

Dès lors, il peut calculer le nombre moyen de clients en appliquant la formule

$\frac{1}{365}\int\limits _{0}^{365}C\left(t\right)dt$ , où $C$ désigne une primitive de $c$.

\begin{QCM}

Parmi les propositions ci-dessous, laquelle donne une formule possible C~?

\begin{enumerate}

\Faux{$C\left(t\right)=\frac{24}{365}\pi\sin\left(\frac{8\pi t}{365}\right)-16t+11$}

\Faux{$C\left(t\right)=-3\sin\left(\frac{8\pi t}{365}\right)$}

\Faux{$C\left(t\right)=\frac{24}{365}\pi\cos\left(\frac{8\pi t}{365}\right)-16t-23$}

\Vrai{$C\left(t\right)=\frac{1095}{2}\pi\sin\left(\frac{8\pi t}{365}\right)+16t$}

\end{enumerate}

\end{QCM}

\begin{QCM}

On obtient ainsi que le nombre moyen de clients journaliers sur l'année en cours est donné par :

\begin{enumerate}

\Faux{$16\sin\left(8\pi\right)$ }

\Vrai{$16$}

\Faux{$3\cos\left(8\pi\right)$}

\Faux{$3$}

\end{enumerate}

\end{QCM}

\end{Sujet}

\begin{Sujet}

\subsection*{Exercice 1}

Pour chacune des propositions de cet exercice, cochez A si elle est vraie\par et B si elle est fausse.

On s'intéresse à la fonction f définie par $f:x\mapsto\left(5x+7\right)e^{7x}$.

\begin{VraiFaux}

\Vrai{ L'ensemble de définition de f est $\left]-\infty,+\infty\right[$.}

\end{VraiFaux}

\begin{VraiFaux}

\Vrai{ La dérivée de f est donnée par $f^{\prime}:x\mapsto\left(35x+54\right)e^{7x}$.}

\end{VraiFaux}

On admet que son approximation quadratique est donné par\par $\qquad f\left(x\right)\approx 7 + 40 x + \frac{225}{2} x^2$.

\begin{VraiFaux}

\Vrai{ L'équation de la tangente $T$ à la courbe représentative $\mathscr C$ de $f$ en son point d'abscisse $0$ est donnée par $y=40 x +7 $.}

\end{VraiFaux}

\begin{VraiFaux}

\Vrai{ $T$ est au-dessous de $\mathscr C$.}

\end{VraiFaux}

\subsection*{Exercice 2}

Cet exercice doit être rédigé au verso de la grille réponse.

\subsection*{Exercice 3}

\begin{QCM}

Parmi ces fonctions, laquelle est une primitive de la fonction $f$\par définie par $ f(x)=x \cdot \cos (3 x+1)$ ?

\begin{enumerate}

\Vrai{ $H:x \mapsto \frac{x \cdot \sin (3 x+1)}{3}+\frac{\cos (3 x+1)}{9}$}

\Faux{ $F:x \mapsto \cos (3 x+1)-3 x \cdot \sin (3 x+1)$}

\Faux{ $G:x \mapsto x \cdot \sin (3 x+1)$}

\end{enumerate}

\end{QCM}

\begin{QCM}

$\int\limits_{0}^{2}\left(5 t^{3}+3 t^{2}-t+7\right) d t=$

\begin{enumerate}

\Faux{ $15 t^{2}+6 t-1$}

\Vrai{ 40}

\Faux{ 71}

\end{enumerate}

\end{QCM}

\begin{QCM}

La valeur moyenne de la fonction $g: x \rightarrow e^{5 x} $ entre 0 et 2 est de :

\begin{enumerate}

\Faux{environ $4405,1$}

\Faux{environ $22 025,5$}

\Vrai{environ $2 202,5$}

\end{enumerate}

\end{QCM}

\newpage

\subsection*{Problème}

Alain Raffletou cherche à connaitre le nombre moyen de clients journaliers dans une de ses boutiques sur l'année en cours. Pour cela, il modélise le nombre de clients par une fonction donnant le nombre de clients $c\left(t\right)$ en fonction du temps $t$, exprimé en jours. On prends donc $t\in\left[0;365\right]$.

\begin{QCM}

En remarquant que le nombre de clients ne peut être négatif, déterminé, parmi les propositions ci-dessous, laquelle donne la formule de $c\left(t\right)$ en fonction de $t$ :

\begin{enumerate}

\Faux{$c\left(t\right)=3\cos\left(\frac{4\pi t}{365}-22\right)$}

\Vrai{$c\left(t\right)=3\cos\left(\frac{4\pi t}{365}\right)+22$}

\Faux{$c\left(t\right)=3\sin\left(\frac{4\pi t}{365}\right)-22$}

\Faux{$c\left(t\right)=3\sin\left(\frac{4\pi t}{365}+22\right)$}

\end{enumerate}

\end{QCM}

\begin{QCM}

On a alors que les valeurs de $c\left(t\right)$ sont comprises entre :

\begin{enumerate}

\Faux{$0$ et $22$}

\Vrai{$19$ et $25$}

\Faux{$0$ et $3$ }

\Faux{$0$ et $25$ }

\end{enumerate}

\end{QCM}

Dès lors, il peut calculer le nombre moyen de clients en appliquant la formule

$\frac{1}{365}\int\limits _{0}^{365}C\left(t\right)dt$ , où $C$ désigne une primitive de $c$.

\begin{QCM}

Parmi les propositions ci-dessous, laquelle donne une formule possible C~?

\begin{enumerate}

\Faux{$C\left(t\right)=\frac{12}{365}\pi\sin\left(\frac{4\pi t}{365}\right)-22t+17$}

\Faux{$C\left(t\right)=-3\sin\left(\frac{4\pi t}{365}\right)$}

\Vrai{$C\left(t\right)=\frac{1095}{2}\pi\sin\left(\frac{4\pi t}{365}\right)+22t$}

\Faux{$C\left(t\right)=\frac{12}{365}\pi\cos\left(\frac{4\pi t}{365}\right)-22t-35$}

\end{enumerate}

\end{QCM}

\begin{QCM}

On obtient ainsi que le nombre moyen de clients journaliers sur l'année en cours est donné par :

\begin{enumerate}

\Faux{$3$}

\Faux{$3\cos\left(4\pi\right)$}

\Faux{$22\sin\left(4\pi\right)$ }

\Vrai{$22$}

\end{enumerate}

\end{QCM}

\end{Sujet}

\begin{Sujet}

\subsection*{Exercice 1}

Pour chacune des propositions de cet exercice, cochez A si elle est vraie\par et B si elle est fausse.

On s'intéresse à la fonction f définie par $f:x\mapsto\left(4x+5\right)e^{7x}$.

\begin{VraiFaux}

\Vrai{ L'ensemble de définition de f est $\left]-\infty,+\infty\right[$.}

\end{VraiFaux}

\begin{VraiFaux}

\Faux{ La dérivée de f est donnée par $f^{\prime}:x\mapsto\left(28\right)e^{cx}$.}

\end{VraiFaux}

On admet que son approximation quadratique est donné par\par $\qquad f\left(x\right)\approx 5 + 24 x + 56 x^2$.

\begin{VraiFaux}

\Vrai{ L'équation de la tangente $T$ à la courbe représentative $\mathscr C$ de $f$ en son point d'abscisse $0$ est donnée par $y=24 x +5 $.}

\end{VraiFaux}

\begin{VraiFaux}

\Vrai{ $T$ est au-dessous de $\mathscr C$.}

\end{VraiFaux}

\subsection*{Exercice 2}

Cet exercice doit être rédigé au verso de la grille réponse.

\subsection*{Exercice 3}

\begin{QCM}

Parmi ces fonctions, laquelle est une primitive de la fonction $f$\par définie par $ f(x)=x \cdot \cos (3 x+1)$ ?

\begin{enumerate}

\Faux{ $F:x \mapsto \cos (3 x+1)-3 x \cdot \sin (3 x+1)$}

\Faux{ $G:x \mapsto x \cdot \sin (3 x+1)$}

\Vrai{ $H:x \mapsto \frac{x \cdot \sin (3 x+1)}{3}+\frac{\cos (3 x+1)}{9}$}

\end{enumerate}

\end{QCM}

\begin{QCM}

$\int\limits_{0}^{2}\left(5 t^{3}+3 t^{2}-t+7\right) d t=$

\begin{enumerate}

\Faux{ 71}

\Faux{ $15 t^{2}+6 t-1$}

\Vrai{ 40}

\end{enumerate}

\end{QCM}

\begin{QCM}

La valeur moyenne de la fonction $g: x \rightarrow e^{5 x} $ entre 0 et 2 est de :

\begin{enumerate}

\Faux{environ $4405,1$}

\Vrai{environ $2 202,5$}

\Faux{environ $22 025,5$}

\end{enumerate}

\end{QCM}

\newpage

\subsection*{Problème}

Alain Raffletou cherche à connaitre le nombre moyen de clients journaliers dans une de ses boutiques sur l'année en cours. Pour cela, il modélise le nombre de clients par une fonction donnant le nombre de clients $c\left(t\right)$ en fonction du temps $t$, exprimé en jours. On prends donc $t\in\left[0;365\right]$.

\begin{QCM}

En remarquant que le nombre de clients ne peut être négatif, déterminé, parmi les propositions ci-dessous, laquelle donne la formule de $c\left(t\right)$ en fonction de $t$ :

\begin{enumerate}

\Faux{$c\left(t\right)=5\sin\left(\frac{8\pi t}{365}\right)-13$}

\Faux{$c\left(t\right)=5\sin\left(\frac{8\pi t}{365}+13\right)$}

\Vrai{$c\left(t\right)=5\cos\left(\frac{8\pi t}{365}\right)+13$}

\Faux{$c\left(t\right)=5\cos\left(\frac{8\pi t}{365}-13\right)$}

\end{enumerate}

\end{QCM}

\begin{QCM}

On a alors que les valeurs de $c\left(t\right)$ sont comprises entre :

\begin{enumerate}

\Faux{$0$ et $18$ }

\Faux{$0$ et $5$ }

\Vrai{$8$ et $18$}

\Faux{$0$ et $13$}

\end{enumerate}

\end{QCM}

Dès lors, il peut calculer le nombre moyen de clients en appliquant la formule

$\frac{1}{365}\int\limits _{0}^{365}C\left(t\right)dt$ , où $C$ désigne une primitive de $c$.

\begin{QCM}

Parmi les propositions ci-dessous, laquelle donne une formule possible C~?

\begin{enumerate}

\Faux{$C\left(t\right)=-5\sin\left(\frac{8\pi t}{365}\right)$}

\Faux{$C\left(t\right)=\frac{40}{365}\pi\cos\left(\frac{8\pi t}{365}\right)-13t-11$}

\Vrai{$C\left(t\right)=\frac{1825}{2}\pi\sin\left(\frac{8\pi t}{365}\right)+13t$}

\Faux{$C\left(t\right)=\frac{40}{365}\pi\sin\left(\frac{8\pi t}{365}\right)-13t+4$}

\end{enumerate}

\end{QCM}

\begin{QCM}

On obtient ainsi que le nombre moyen de clients journaliers sur l'année en cours est donné par :

\begin{enumerate}

\Faux{$5$}

\Faux{$13\sin\left(8\pi\right)$ }

\Faux{$5\cos\left(8\pi\right)$}

\Vrai{$13$}

\end{enumerate}

\end{QCM}

\end{Sujet}

\begin{Sujet}

\subsection*{Exercice 1}

Pour chacune des propositions de cet exercice, cochez A si elle est vraie\par et B si elle est fausse.

On s'intéresse à la fonction f définie par $f:x\mapsto\left(6x+7\right)e^{7x}$.

\begin{VraiFaux}

\Faux{ L'ensemble de définition de f est $\left]0,+\infty\right[$.}

\end{VraiFaux}

\begin{VraiFaux}

\Faux{ La dérivée de f est donnée par $f^{\prime}:x\mapsto\left(42\right)e^{cx}$.}

\end{VraiFaux}

On admet que son approximation quadratique est donné par\par $\qquad f\left(x\right)\approx 7 + 48 x + 162 x^2$.

\begin{VraiFaux}

\Faux{ L'équation de la tangente $T$ à la courbe représentative $\mathscr C$ de $f$ en son point d'abscisse $0$ est donnée par $y=7x + 48 $.}

\end{VraiFaux}

\begin{VraiFaux}

\Vrai{ $T$ est au-dessous de $\mathscr C$.}

\end{VraiFaux}

\subsection*{Exercice 2}

Cet exercice doit être rédigé au verso de la grille réponse.

\subsection*{Exercice 3}

\begin{QCM}

Parmi ces fonctions, laquelle est une primitive de la fonction $f$\par définie par $ f(x)=x \cdot \cos (3 x+1)$ ?

\begin{enumerate}

\Vrai{ $H:x \mapsto \frac{x \cdot \sin (3 x+1)}{3}+\frac{\cos (3 x+1)}{9}$}

\Faux{ $G:x \mapsto x \cdot \sin (3 x+1)$}

\Faux{ $F:x \mapsto \cos (3 x+1)-3 x \cdot \sin (3 x+1)$}

\end{enumerate}

\end{QCM}

\begin{QCM}

$\int\limits_{0}^{2}\left(5 t^{3}+3 t^{2}-t+7\right) d t=$

\begin{enumerate}

\Vrai{ 40}

\Faux{ 71}

\Faux{ $15 t^{2}+6 t-1$}

\end{enumerate}

\end{QCM}

\begin{QCM}

La valeur moyenne de la fonction $g: x \rightarrow e^{5 x} $ entre 0 et 2 est de :

\begin{enumerate}

\Faux{environ $22 025,5$}

\Faux{environ $4405,1$}

\Vrai{environ $2 202,5$}

\end{enumerate}

\end{QCM}

\newpage

\subsection*{Problème}

Alain Raffletou cherche à connaitre le nombre moyen de clients journaliers dans une de ses boutiques sur l'année en cours. Pour cela, il modélise le nombre de clients par une fonction donnant le nombre de clients $c\left(t\right)$ en fonction du temps $t$, exprimé en jours. On prends donc $t\in\left[0;365\right]$.

\begin{QCM}

En remarquant que le nombre de clients ne peut être négatif, déterminé, parmi les propositions ci-dessous, laquelle donne la formule de $c\left(t\right)$ en fonction de $t$ :

\begin{enumerate}

\Faux{$c\left(t\right)=3\sin\left(\frac{6\pi t}{365}\right)-20$}

\Vrai{$c\left(t\right)=3\cos\left(\frac{6\pi t}{365}\right)+20$}

\Faux{$c\left(t\right)=3\sin\left(\frac{6\pi t}{365}+20\right)$}

\Faux{$c\left(t\right)=3\cos\left(\frac{6\pi t}{365}-20\right)$}

\end{enumerate}

\end{QCM}

\begin{QCM}

On a alors que les valeurs de $c\left(t\right)$ sont comprises entre :

\begin{enumerate}

\Faux{$0$ et $3$ }

\Vrai{$17$ et $23$}

\Faux{$0$ et $23$ }

\Faux{$0$ et $20$}

\end{enumerate}

\end{QCM}

Dès lors, il peut calculer le nombre moyen de clients en appliquant la formule

$\frac{1}{365}\int\limits _{0}^{365}C\left(t\right)dt$ , où $C$ désigne une primitive de $c$.

\begin{QCM}

Parmi les propositions ci-dessous, laquelle donne une formule possible C~?

\begin{enumerate}

\Faux{$C\left(t\right)=\frac{18}{365}\pi\sin\left(\frac{6\pi t}{365}\right)-20t+15$}

\Faux{$C\left(t\right)=\frac{18}{365}\pi\cos\left(\frac{6\pi t}{365}\right)-20t-31$}

\Vrai{$C\left(t\right)=\frac{1095}{2}\pi\sin\left(\frac{6\pi t}{365}\right)+20t$}

\Faux{$C\left(t\right)=-3\sin\left(\frac{6\pi t}{365}\right)$}

\end{enumerate}

\end{QCM}

\begin{QCM}

On obtient ainsi que le nombre moyen de clients journaliers sur l'année en cours est donné par :

\begin{enumerate}

\Faux{$3\cos\left(6\pi\right)$}

\Vrai{$20$}

\Faux{$3$}

\Faux{$20\sin\left(6\pi\right)$ }

\end{enumerate}

\end{QCM}

\end{Sujet}

\begin{Sujet}

\subsection*{Exercice 1}

Pour chacune des propositions de cet exercice, cochez A si elle est vraie\par et B si elle est fausse.

On s'intéresse à la fonction f définie par $f:x\mapsto\left(2x+6\right)e^{5x}$.

\begin{VraiFaux}

\Vrai{ L'ensemble de définition de f est $\left]-\infty,+\infty\right[$.}

\end{VraiFaux}

\begin{VraiFaux}

\Faux{ La dérivée de f est donnée par $f^{\prime}:x\mapsto\left(10\right)e^{cx}$.}

\end{VraiFaux}

On admet que son approximation quadratique est donné par\par $\qquad f\left(x\right)\approx 6 + 14 x + 16 x^2$.

\begin{VraiFaux}

\Vrai{ L'équation de la tangente $T$ à la courbe représentative $\mathscr C$ de $f$ en son point d'abscisse $0$ est donnée par $y=14 x +6 $.}

\end{VraiFaux}

\begin{VraiFaux}

\Vrai{ $T$ est au-dessous de $\mathscr C$.}

\end{VraiFaux}

\subsection*{Exercice 2}

Cet exercice doit être rédigé au verso de la grille réponse.

\subsection*{Exercice 3}

\begin{QCM}

Parmi ces fonctions, laquelle est une primitive de la fonction $f$\par définie par $ f(x)=x \cdot \cos (3 x+1)$ ?

\begin{enumerate}

\Vrai{ $H:x \mapsto \frac{x \cdot \sin (3 x+1)}{3}+\frac{\cos (3 x+1)}{9}$}

\Faux{ $F:x \mapsto \cos (3 x+1)-3 x \cdot \sin (3 x+1)$}

\Faux{ $G:x \mapsto x \cdot \sin (3 x+1)$}

\end{enumerate}

\end{QCM}

\begin{QCM}

$\int\limits_{0}^{2}\left(5 t^{3}+3 t^{2}-t+7\right) d t=$

\begin{enumerate}

\Vrai{ 40}

\Faux{ $15 t^{2}+6 t-1$}

\Faux{ 71}

\end{enumerate}

\end{QCM}

\begin{QCM}

La valeur moyenne de la fonction $g: x \rightarrow e^{5 x} $ entre 0 et 2 est de :

\begin{enumerate}

\Vrai{environ $2 202,5$}

\Faux{environ $4405,1$}

\Faux{environ $22 025,5$}

\end{enumerate}

\end{QCM}

\newpage

\subsection*{Problème}

Alain Raffletou cherche à connaitre le nombre moyen de clients journaliers dans une de ses boutiques sur l'année en cours. Pour cela, il modélise le nombre de clients par une fonction donnant le nombre de clients $c\left(t\right)$ en fonction du temps $t$, exprimé en jours. On prends donc $t\in\left[0;365\right]$.

\begin{QCM}

En remarquant que le nombre de clients ne peut être négatif, déterminé, parmi les propositions ci-dessous, laquelle donne la formule de $c\left(t\right)$ en fonction de $t$ :

\begin{enumerate}

\Vrai{$c\left(t\right)=5\cos\left(\frac{8\pi t}{365}\right)+14$}

\Faux{$c\left(t\right)=5\cos\left(\frac{8\pi t}{365}-14\right)$}

\Faux{$c\left(t\right)=5\sin\left(\frac{8\pi t}{365}\right)-14$}

\Faux{$c\left(t\right)=5\sin\left(\frac{8\pi t}{365}+14\right)$}

\end{enumerate}

\end{QCM}

\begin{QCM}

On a alors que les valeurs de $c\left(t\right)$ sont comprises entre :

\begin{enumerate}

\Faux{$0$ et $5$ }

\Faux{$0$ et $19$ }

\Faux{$0$ et $14$}

\Vrai{$9$ et $19$}

\end{enumerate}

\end{QCM}

Dès lors, il peut calculer le nombre moyen de clients en appliquant la formule

$\frac{1}{365}\int\limits _{0}^{365}C\left(t\right)dt$ , où $C$ désigne une primitive de $c$.

\begin{QCM}

Parmi les propositions ci-dessous, laquelle donne une formule possible C~?

\begin{enumerate}

\Faux{$C\left(t\right)=\frac{40}{365}\pi\cos\left(\frac{8\pi t}{365}\right)-14t-13$}

\Vrai{$C\left(t\right)=\frac{1825}{2}\pi\sin\left(\frac{8\pi t}{365}\right)+14t$}

\Faux{$C\left(t\right)=\frac{40}{365}\pi\sin\left(\frac{8\pi t}{365}\right)-14t+5$}

\Faux{$C\left(t\right)=-5\sin\left(\frac{8\pi t}{365}\right)$}

\end{enumerate}

\end{QCM}

\begin{QCM}

On obtient ainsi que le nombre moyen de clients journaliers sur l'année en cours est donné par :

\begin{enumerate}

\Faux{$5$}

\Faux{$5\cos\left(8\pi\right)$}

\Vrai{$14$}

\Faux{$14\sin\left(8\pi\right)$ }

\end{enumerate}

\end{QCM}

\end{Sujet}

\begin{Sujet}

\subsection*{Exercice 1}

Pour chacune des propositions de cet exercice, cochez A si elle est vraie\par et B si elle est fausse.

On s'intéresse à la fonction f définie par $f:x\mapsto\left(3x+3\right)e^{5x}$.

\begin{VraiFaux}

\Vrai{ L'ensemble de définition de f est $\left]-\infty,+\infty\right[$.}

\end{VraiFaux}

\begin{VraiFaux}

\Vrai{ La dérivée de f est donnée par $f^{\prime}:x\mapsto\left(15x+18\right)e^{5x}$.}

\end{VraiFaux}

On admet que son approximation quadratique est donné par\par $\qquad f\left(x\right)\approx 3 + 12 x + \frac{45}{2} x^2$.

\begin{VraiFaux}

\Vrai{ L'équation de la tangente $T$ à la courbe représentative $\mathscr C$ de $f$ en son point d'abscisse $0$ est donnée par $y=12 x +3 $.}

\end{VraiFaux}

\begin{VraiFaux}

\Faux{ $T$ est au-dessus de $\mathscr C$.}

\end{VraiFaux}

\subsection*{Exercice 2}

Cet exercice doit être rédigé au verso de la grille réponse.

\subsection*{Exercice 3}

\begin{QCM}

Parmi ces fonctions, laquelle est une primitive de la fonction $f$\par définie par $ f(x)=x \cdot \cos (3 x+1)$ ?

\begin{enumerate}

\Faux{ $F:x \mapsto \cos (3 x+1)-3 x \cdot \sin (3 x+1)$}

\Vrai{ $H:x \mapsto \frac{x \cdot \sin (3 x+1)}{3}+\frac{\cos (3 x+1)}{9}$}

\Faux{ $G:x \mapsto x \cdot \sin (3 x+1)$}

\end{enumerate}

\end{QCM}

\begin{QCM}

$\int\limits_{0}^{2}\left(5 t^{3}+3 t^{2}-t+7\right) d t=$

\begin{enumerate}

\Faux{ $15 t^{2}+6 t-1$}

\Vrai{ 40}

\Faux{ 71}

\end{enumerate}

\end{QCM}

\begin{QCM}

La valeur moyenne de la fonction $g: x \rightarrow e^{5 x} $ entre 0 et 2 est de :

\begin{enumerate}

\Faux{environ $22 025,5$}

\Faux{environ $4405,1$}

\Vrai{environ $2 202,5$}

\end{enumerate}

\end{QCM}

\newpage

\subsection*{Problème}

Alain Raffletou cherche à connaitre le nombre moyen de clients journaliers dans une de ses boutiques sur l'année en cours. Pour cela, il modélise le nombre de clients par une fonction donnant le nombre de clients $c\left(t\right)$ en fonction du temps $t$, exprimé en jours. On prends donc $t\in\left[0;365\right]$.

\begin{QCM}

En remarquant que le nombre de clients ne peut être négatif, déterminé, parmi les propositions ci-dessous, laquelle donne la formule de $c\left(t\right)$ en fonction de $t$ :

\begin{enumerate}

\Faux{$c\left(t\right)=4\sin\left(\frac{8\pi t}{365}\right)-23$}

\Faux{$c\left(t\right)=4\sin\left(\frac{8\pi t}{365}+23\right)$}

\Faux{$c\left(t\right)=4\cos\left(\frac{8\pi t}{365}-23\right)$}

\Vrai{$c\left(t\right)=4\cos\left(\frac{8\pi t}{365}\right)+23$}

\end{enumerate}

\end{QCM}

\begin{QCM}

On a alors que les valeurs de $c\left(t\right)$ sont comprises entre :

\begin{enumerate}

\Faux{$0$ et $23$}

\Vrai{$19$ et $27$}

\Faux{$0$ et $27$ }

\Faux{$0$ et $4$ }

\end{enumerate}

\end{QCM}

Dès lors, il peut calculer le nombre moyen de clients en appliquant la formule

$\frac{1}{365}\int\limits _{0}^{365}C\left(t\right)dt$ , où $C$ désigne une primitive de $c$.

\begin{QCM}

Parmi les propositions ci-dessous, laquelle donne une formule possible C~?

\begin{enumerate}

\Faux{$C\left(t\right)=\frac{32}{365}\pi\sin\left(\frac{8\pi t}{365}\right)-23t+16$}

\Faux{$C\left(t\right)=-4\sin\left(\frac{8\pi t}{365}\right)$}

\Vrai{$C\left(t\right)=\frac{1460}{2}\pi\sin\left(\frac{8\pi t}{365}\right)+23t$}

\Faux{$C\left(t\right)=\frac{32}{365}\pi\cos\left(\frac{8\pi t}{365}\right)-23t-34$}

\end{enumerate}

\end{QCM}

\begin{QCM}

On obtient ainsi que le nombre moyen de clients journaliers sur l'année en cours est donné par :

\begin{enumerate}

\Vrai{$23$}

\Faux{$23\sin\left(8\pi\right)$ }

\Faux{$4$}

\Faux{$4\cos\left(8\pi\right)$}

\end{enumerate}

\end{QCM}

\end{Sujet}

\begin{Sujet}

\subsection*{Exercice 1}

Pour chacune des propositions de cet exercice, cochez A si elle est vraie\par et B si elle est fausse.

On s'intéresse à la fonction f définie par $f:x\mapsto\left(2x+2\right)e^{4x}$.

\begin{VraiFaux}

\Vrai{ L'ensemble de définition de f est $\left]-\infty,+\infty\right[$.}

\end{VraiFaux}

\begin{VraiFaux}

\Vrai{ La dérivée de f est donnée par $f^{\prime}:x\mapsto\left(8x+10\right)e^{4x}$.}

\end{VraiFaux}

On admet que son approximation quadratique est donné par\par $\qquad f\left(x\right)\approx 2 + 6 x + 8 x^2$.

\begin{VraiFaux}

\Faux{ L'équation de la tangente $T$ à la courbe représentative $\mathscr C$ de $f$ en son point d'abscisse $0$ est donnée par $y=2x + 6 $.}

\end{VraiFaux}

\begin{VraiFaux}

\Faux{ $T$ est au-dessus de $\mathscr C$.}

\end{VraiFaux}

\subsection*{Exercice 2}

Cet exercice doit être rédigé au verso de la grille réponse.

\subsection*{Exercice 3}

\begin{QCM}

Parmi ces fonctions, laquelle est une primitive de la fonction $f$\par définie par $ f(x)=x \cdot \cos (3 x+1)$ ?

\begin{enumerate}

\Vrai{ $H:x \mapsto \frac{x \cdot \sin (3 x+1)}{3}+\frac{\cos (3 x+1)}{9}$}

\Faux{ $G:x \mapsto x \cdot \sin (3 x+1)$}

\Faux{ $F:x \mapsto \cos (3 x+1)-3 x \cdot \sin (3 x+1)$}

\end{enumerate}

\end{QCM}

\begin{QCM}

$\int\limits_{0}^{2}\left(5 t^{3}+3 t^{2}-t+7\right) d t=$

\begin{enumerate}

\Faux{ 71}

\Vrai{ 40}

\Faux{ $15 t^{2}+6 t-1$}

\end{enumerate}

\end{QCM}

\begin{QCM}

La valeur moyenne de la fonction $g: x \rightarrow e^{5 x} $ entre 0 et 2 est de :

\begin{enumerate}

\Faux{environ $4405,1$}

\Vrai{environ $2 202,5$}

\Faux{environ $22 025,5$}

\end{enumerate}

\end{QCM}

\newpage

\subsection*{Problème}

Alain Raffletou cherche à connaitre le nombre moyen de clients journaliers dans une de ses boutiques sur l'année en cours. Pour cela, il modélise le nombre de clients par une fonction donnant le nombre de clients $c\left(t\right)$ en fonction du temps $t$, exprimé en jours. On prends donc $t\in\left[0;365\right]$.

\begin{QCM}

En remarquant que le nombre de clients ne peut être négatif, déterminé, parmi les propositions ci-dessous, laquelle donne la formule de $c\left(t\right)$ en fonction de $t$ :

\begin{enumerate}

\Vrai{$c\left(t\right)=4\cos\left(\frac{4\pi t}{365}\right)+10$}

\Faux{$c\left(t\right)=4\sin\left(\frac{4\pi t}{365}+10\right)$}

\Faux{$c\left(t\right)=4\sin\left(\frac{4\pi t}{365}\right)-10$}

\Faux{$c\left(t\right)=4\cos\left(\frac{4\pi t}{365}-10\right)$}

\end{enumerate}

\end{QCM}

\begin{QCM}

On a alors que les valeurs de $c\left(t\right)$ sont comprises entre :

\begin{enumerate}

\Faux{$0$ et $10$}

\Vrai{$6$ et $14$}

\Faux{$0$ et $14$ }

\Faux{$0$ et $4$ }

\end{enumerate}

\end{QCM}

Dès lors, il peut calculer le nombre moyen de clients en appliquant la formule

$\frac{1}{365}\int\limits _{0}^{365}C\left(t\right)dt$ , où $C$ désigne une primitive de $c$.

\begin{QCM}

Parmi les propositions ci-dessous, laquelle donne une formule possible C~?

\begin{enumerate}

\Vrai{$C\left(t\right)=\frac{1460}{2}\pi\sin\left(\frac{4\pi t}{365}\right)+10t$}

\Faux{$C\left(t\right)=\frac{16}{365}\pi\sin\left(\frac{4\pi t}{365}\right)-10t+3$}

\Faux{$C\left(t\right)=\frac{16}{365}\pi\cos\left(\frac{4\pi t}{365}\right)-10t-8$}

\Faux{$C\left(t\right)=-4\sin\left(\frac{4\pi t}{365}\right)$}

\end{enumerate}

\end{QCM}

\begin{QCM}

On obtient ainsi que le nombre moyen de clients journaliers sur l'année en cours est donné par :

\begin{enumerate}

\Faux{$4\cos\left(4\pi\right)$}

\Faux{$10\sin\left(4\pi\right)$ }

\Vrai{$10$}

\Faux{$4$}

\end{enumerate}

\end{QCM}

\end{Sujet}

\begin{Sujet}

\subsection*{Exercice 1}

Pour chacune des propositions de cet exercice, cochez A si elle est vraie\par et B si elle est fausse.

On s'intéresse à la fonction f définie par $f:x\mapsto\left(3x+5\right)e^{3x}$.

\begin{VraiFaux}

\Vrai{ L'ensemble de définition de f est $\left]-\infty,+\infty\right[$.}

\end{VraiFaux}

\begin{VraiFaux}

\Faux{ La dérivée de f est donnée par $f^{\prime}:x\mapsto\left(9\right)e^{cx}$.}

\end{VraiFaux}

On admet que son approximation quadratique est donné par\par $\qquad f\left(x\right)\approx 5 + 18 x + \frac{63}{2} x^2$.

\begin{VraiFaux}

\Faux{ L'équation de la tangente $T$ à la courbe représentative $\mathscr C$ de $f$ en son point d'abscisse $0$ est donnée par $y=5x + 18 $.}

\end{VraiFaux}

\begin{VraiFaux}

\Vrai{ $T$ est au-dessous de $\mathscr C$.}

\end{VraiFaux}

\subsection*{Exercice 2}

Cet exercice doit être rédigé au verso de la grille réponse.

\subsection*{Exercice 3}

\begin{QCM}

Parmi ces fonctions, laquelle est une primitive de la fonction $f$\par définie par $ f(x)=x \cdot \cos (3 x+1)$ ?

\begin{enumerate}

\Faux{ $F:x \mapsto \cos (3 x+1)-3 x \cdot \sin (3 x+1)$}

\Vrai{ $H:x \mapsto \frac{x \cdot \sin (3 x+1)}{3}+\frac{\cos (3 x+1)}{9}$}

\Faux{ $G:x \mapsto x \cdot \sin (3 x+1)$}

\end{enumerate}

\end{QCM}

\begin{QCM}

$\int\limits_{0}^{2}\left(5 t^{3}+3 t^{2}-t+7\right) d t=$

\begin{enumerate}

\Faux{ $15 t^{2}+6 t-1$}

\Vrai{ 40}

\Faux{ 71}

\end{enumerate}

\end{QCM}

\begin{QCM}

La valeur moyenne de la fonction $g: x \rightarrow e^{5 x} $ entre 0 et 2 est de :

\begin{enumerate}

\Faux{environ $4405,1$}

\Vrai{environ $2 202,5$}

\Faux{environ $22 025,5$}

\end{enumerate}

\end{QCM}

\newpage

\subsection*{Problème}

Alain Raffletou cherche à connaitre le nombre moyen de clients journaliers dans une de ses boutiques sur l'année en cours. Pour cela, il modélise le nombre de clients par une fonction donnant le nombre de clients $c\left(t\right)$ en fonction du temps $t$, exprimé en jours. On prends donc $t\in\left[0;365\right]$.

\begin{QCM}

En remarquant que le nombre de clients ne peut être négatif, déterminé, parmi les propositions ci-dessous, laquelle donne la formule de $c\left(t\right)$ en fonction de $t$ :

\begin{enumerate}

\Faux{$c\left(t\right)=4\cos\left(\frac{8\pi t}{365}-17\right)$}

\Vrai{$c\left(t\right)=4\cos\left(\frac{8\pi t}{365}\right)+17$}

\Faux{$c\left(t\right)=4\sin\left(\frac{8\pi t}{365}+17\right)$}

\Faux{$c\left(t\right)=4\sin\left(\frac{8\pi t}{365}\right)-17$}

\end{enumerate}

\end{QCM}

\begin{QCM}

On a alors que les valeurs de $c\left(t\right)$ sont comprises entre :

\begin{enumerate}

\Vrai{$13$ et $21$}

\Faux{$0$ et $4$ }

\Faux{$0$ et $21$ }

\Faux{$0$ et $17$}

\end{enumerate}

\end{QCM}

Dès lors, il peut calculer le nombre moyen de clients en appliquant la formule

$\frac{1}{365}\int\limits _{0}^{365}C\left(t\right)dt$ , où $C$ désigne une primitive de $c$.

\begin{QCM}

Parmi les propositions ci-dessous, laquelle donne une formule possible C~?

\begin{enumerate}

\Faux{$C\left(t\right)=\frac{32}{365}\pi\cos\left(\frac{8\pi t}{365}\right)-17t-22$}

\Vrai{$C\left(t\right)=\frac{1460}{2}\pi\sin\left(\frac{8\pi t}{365}\right)+17t$}

\Faux{$C\left(t\right)=\frac{32}{365}\pi\sin\left(\frac{8\pi t}{365}\right)-17t+10$}

\Faux{$C\left(t\right)=-4\sin\left(\frac{8\pi t}{365}\right)$}

\end{enumerate}

\end{QCM}

\begin{QCM}

On obtient ainsi que le nombre moyen de clients journaliers sur l'année en cours est donné par :

\begin{enumerate}

\Faux{$4$}

\Faux{$17\sin\left(8\pi\right)$ }

\Faux{$4\cos\left(8\pi\right)$}

\Vrai{$17$}

\end{enumerate}

\end{QCM}

\end{Sujet}

\begin{Sujet}

\subsection*{Exercice 1}

Pour chacune des propositions de cet exercice, cochez A si elle est vraie\par et B si elle est fausse.

On s'intéresse à la fonction f définie par $f:x\mapsto\left(4x+5\right)e^{7x}$.

\begin{VraiFaux}

\Faux{ L'ensemble de définition de f est $\left]0,+\infty\right[$.}

\end{VraiFaux}

\begin{VraiFaux}

\Vrai{ La dérivée de f est donnée par $f^{\prime}:x\mapsto\left(28x+39\right)e^{7x}$.}

\end{VraiFaux}

On admet que son approximation quadratique est donné par\par $\qquad f\left(x\right)\approx 5 + 24 x + 56 x^2$.

\begin{VraiFaux}

\Faux{ L'équation de la tangente $T$ à la courbe représentative $\mathscr C$ de $f$ en son point d'abscisse $0$ est donnée par $y=5x + 24 $.}

\end{VraiFaux}

\begin{VraiFaux}

\Vrai{ $T$ est au-dessous de $\mathscr C$.}

\end{VraiFaux}

\subsection*{Exercice 2}

Cet exercice doit être rédigé au verso de la grille réponse.

\subsection*{Exercice 3}

\begin{QCM}

Parmi ces fonctions, laquelle est une primitive de la fonction $f$\par définie par $ f(x)=x \cdot \cos (3 x+1)$ ?

\begin{enumerate}

\Vrai{ $H:x \mapsto \frac{x \cdot \sin (3 x+1)}{3}+\frac{\cos (3 x+1)}{9}$}

\Faux{ $G:x \mapsto x \cdot \sin (3 x+1)$}

\Faux{ $F:x \mapsto \cos (3 x+1)-3 x \cdot \sin (3 x+1)$}

\end{enumerate}

\end{QCM}

\begin{QCM}

$\int\limits_{0}^{2}\left(5 t^{3}+3 t^{2}-t+7\right) d t=$

\begin{enumerate}

\Faux{ 71}

\Vrai{ 40}

\Faux{ $15 t^{2}+6 t-1$}

\end{enumerate}

\end{QCM}

\begin{QCM}

La valeur moyenne de la fonction $g: x \rightarrow e^{5 x} $ entre 0 et 2 est de :

\begin{enumerate}

\Vrai{environ $2 202,5$}

\Faux{environ $22 025,5$}

\Faux{environ $4405,1$}

\end{enumerate}

\end{QCM}

\newpage

\subsection*{Problème}

Alain Raffletou cherche à connaitre le nombre moyen de clients journaliers dans une de ses boutiques sur l'année en cours. Pour cela, il modélise le nombre de clients par une fonction donnant le nombre de clients $c\left(t\right)$ en fonction du temps $t$, exprimé en jours. On prends donc $t\in\left[0;365\right]$.

\begin{QCM}

En remarquant que le nombre de clients ne peut être négatif, déterminé, parmi les propositions ci-dessous, laquelle donne la formule de $c\left(t\right)$ en fonction de $t$ :

\begin{enumerate}

\Vrai{$c\left(t\right)=3\cos\left(\frac{6\pi t}{365}\right)+17$}

\Faux{$c\left(t\right)=3\sin\left(\frac{6\pi t}{365}+17\right)$}

\Faux{$c\left(t\right)=3\sin\left(\frac{6\pi t}{365}\right)-17$}

\Faux{$c\left(t\right)=3\cos\left(\frac{6\pi t}{365}-17\right)$}

\end{enumerate}

\end{QCM}

\begin{QCM}

On a alors que les valeurs de $c\left(t\right)$ sont comprises entre :

\begin{enumerate}

\Vrai{$14$ et $20$}

\Faux{$0$ et $3$ }

\Faux{$0$ et $17$}

\Faux{$0$ et $20$ }

\end{enumerate}

\end{QCM}

Dès lors, il peut calculer le nombre moyen de clients en appliquant la formule

$\frac{1}{365}\int\limits _{0}^{365}C\left(t\right)dt$ , où $C$ désigne une primitive de $c$.

\begin{QCM}

Parmi les propositions ci-dessous, laquelle donne une formule possible C~?

\begin{enumerate}

\Faux{$C\left(t\right)=\frac{18}{365}\pi\sin\left(\frac{6\pi t}{365}\right)-17t+12$}

\Faux{$C\left(t\right)=\frac{18}{365}\pi\cos\left(\frac{6\pi t}{365}\right)-17t-25$}

\Faux{$C\left(t\right)=-3\sin\left(\frac{6\pi t}{365}\right)$}

\Vrai{$C\left(t\right)=\frac{1095}{2}\pi\sin\left(\frac{6\pi t}{365}\right)+17t$}

\end{enumerate}

\end{QCM}

\begin{QCM}

On obtient ainsi que le nombre moyen de clients journaliers sur l'année en cours est donné par :

\begin{enumerate}

\Faux{$17\sin\left(6\pi\right)$ }

\Vrai{$17$}

\Faux{$3$}

\Faux{$3\cos\left(6\pi\right)$}

\end{enumerate}

\end{QCM}

\end{Sujet}

\begin{Sujet}

\subsection*{Exercice 1}

Pour chacune des propositions de cet exercice, cochez A si elle est vraie\par et B si elle est fausse.

On s'intéresse à la fonction f définie par $f:x\mapsto\left(7x+3\right)e^{4x}$.

\begin{VraiFaux}

\Faux{ L'ensemble de définition de f est $\left]0,+\infty\right[$.}

\end{VraiFaux}

\begin{VraiFaux}

\Vrai{ La dérivée de f est donnée par $f^{\prime}:x\mapsto\left(28x+19\right)e^{4x}$.}

\end{VraiFaux}

On admet que son approximation quadratique est donné par\par $\qquad f\left(x\right)\approx 3 + 28 x + \frac{245}{2} x^2$.

\begin{VraiFaux}

\Faux{ L'équation de la tangente $T$ à la courbe représentative $\mathscr C$ de $f$ en son point d'abscisse $0$ est donnée par $y=3x + 28 $.}

\end{VraiFaux}

\begin{VraiFaux}

\Faux{ $T$ est au-dessus de $\mathscr C$.}

\end{VraiFaux}

\subsection*{Exercice 2}

Cet exercice doit être rédigé au verso de la grille réponse.

\subsection*{Exercice 3}

\begin{QCM}

Parmi ces fonctions, laquelle est une primitive de la fonction $f$\par définie par $ f(x)=x \cdot \cos (3 x+1)$ ?

\begin{enumerate}

\Faux{ $F:x \mapsto \cos (3 x+1)-3 x \cdot \sin (3 x+1)$}

\Vrai{ $H:x \mapsto \frac{x \cdot \sin (3 x+1)}{3}+\frac{\cos (3 x+1)}{9}$}

\Faux{ $G:x \mapsto x \cdot \sin (3 x+1)$}

\end{enumerate}

\end{QCM}

\begin{QCM}

$\int\limits_{0}^{2}\left(5 t^{3}+3 t^{2}-t+7\right) d t=$

\begin{enumerate}

\Faux{ $15 t^{2}+6 t-1$}

\Vrai{ 40}

\Faux{ 71}

\end{enumerate}

\end{QCM}

\begin{QCM}

La valeur moyenne de la fonction $g: x \rightarrow e^{5 x} $ entre 0 et 2 est de :

\begin{enumerate}

\Faux{environ $22 025,5$}

\Vrai{environ $2 202,5$}

\Faux{environ $4405,1$}

\end{enumerate}

\end{QCM}

\newpage

\subsection*{Problème}

Alain Raffletou cherche à connaitre le nombre moyen de clients journaliers dans une de ses boutiques sur l'année en cours. Pour cela, il modélise le nombre de clients par une fonction donnant le nombre de clients $c\left(t\right)$ en fonction du temps $t$, exprimé en jours. On prends donc $t\in\left[0;365\right]$.

\begin{QCM}

En remarquant que le nombre de clients ne peut être négatif, déterminé, parmi les propositions ci-dessous, laquelle donne la formule de $c\left(t\right)$ en fonction de $t$ :

\begin{enumerate}

\Faux{$c\left(t\right)=3\sin\left(\frac{6\pi t}{365}+16\right)$}

\Faux{$c\left(t\right)=3\sin\left(\frac{6\pi t}{365}\right)-16$}

\Faux{$c\left(t\right)=3\cos\left(\frac{6\pi t}{365}-16\right)$}

\Vrai{$c\left(t\right)=3\cos\left(\frac{6\pi t}{365}\right)+16$}

\end{enumerate}

\end{QCM}

\begin{QCM}

On a alors que les valeurs de $c\left(t\right)$ sont comprises entre :

\begin{enumerate}

\Vrai{$13$ et $19$}

\Faux{$0$ et $16$}

\Faux{$0$ et $19$ }

\Faux{$0$ et $3$ }

\end{enumerate}

\end{QCM}

Dès lors, il peut calculer le nombre moyen de clients en appliquant la formule

$\frac{1}{365}\int\limits _{0}^{365}C\left(t\right)dt$ , où $C$ désigne une primitive de $c$.

\begin{QCM}

Parmi les propositions ci-dessous, laquelle donne une formule possible C~?

\begin{enumerate}

\Faux{$C\left(t\right)=-3\sin\left(\frac{6\pi t}{365}\right)$}

\Faux{$C\left(t\right)=\frac{18}{365}\pi\cos\left(\frac{6\pi t}{365}\right)-16t-23$}

\Vrai{$C\left(t\right)=\frac{1095}{2}\pi\sin\left(\frac{6\pi t}{365}\right)+16t$}

\Faux{$C\left(t\right)=\frac{18}{365}\pi\sin\left(\frac{6\pi t}{365}\right)-16t+11$}

\end{enumerate}

\end{QCM}

\begin{QCM}

On obtient ainsi que le nombre moyen de clients journaliers sur l'année en cours est donné par :

\begin{enumerate}

\Vrai{$16$}

\Faux{$3\cos\left(6\pi\right)$}

\Faux{$16\sin\left(6\pi\right)$ }

\Faux{$3$}

\end{enumerate}

\end{QCM}

\end{Sujet}

\begin{Sujet}

\subsection*{Exercice 1}

Pour chacune des propositions de cet exercice, cochez A si elle est vraie\par et B si elle est fausse.

On s'intéresse à la fonction f définie par $f:x\mapsto\left(6x+7\right)e^{7x}$.

\begin{VraiFaux}

\Faux{ L'ensemble de définition de f est $\left]0,+\infty\right[$.}

\end{VraiFaux}

\begin{VraiFaux}

\Vrai{ La dérivée de f est donnée par $f^{\prime}:x\mapsto\left(42x+55\right)e^{7x}$.}

\end{VraiFaux}

On admet que son approximation quadratique est donné par\par $\qquad f\left(x\right)\approx 7 + 48 x + 162 x^2$.

\begin{VraiFaux}

\Faux{ L'équation de la tangente $T$ à la courbe représentative $\mathscr C$ de $f$ en son point d'abscisse $0$ est donnée par $y=7x + 48 $.}

\end{VraiFaux}

\begin{VraiFaux}

\Vrai{ $T$ est au-dessous de $\mathscr C$.}

\end{VraiFaux}

\subsection*{Exercice 2}

Cet exercice doit être rédigé au verso de la grille réponse.

\subsection*{Exercice 3}

\begin{QCM}

Parmi ces fonctions, laquelle est une primitive de la fonction $f$\par définie par $ f(x)=x \cdot \cos (3 x+1)$ ?

\begin{enumerate}

\Faux{ $F:x \mapsto \cos (3 x+1)-3 x \cdot \sin (3 x+1)$}

\Vrai{ $H:x \mapsto \frac{x \cdot \sin (3 x+1)}{3}+\frac{\cos (3 x+1)}{9}$}

\Faux{ $G:x \mapsto x \cdot \sin (3 x+1)$}

\end{enumerate}

\end{QCM}

\begin{QCM}

$\int\limits_{0}^{2}\left(5 t^{3}+3 t^{2}-t+7\right) d t=$

\begin{enumerate}

\Vrai{ 40}

\Faux{ $15 t^{2}+6 t-1$}

\Faux{ 71}

\end{enumerate}

\end{QCM}

\begin{QCM}

La valeur moyenne de la fonction $g: x \rightarrow e^{5 x} $ entre 0 et 2 est de :

\begin{enumerate}

\Faux{environ $22 025,5$}

\Faux{environ $4405,1$}

\Vrai{environ $2 202,5$}

\end{enumerate}

\end{QCM}

\newpage

\subsection*{Problème}

Alain Raffletou cherche à connaitre le nombre moyen de clients journaliers dans une de ses boutiques sur l'année en cours. Pour cela, il modélise le nombre de clients par une fonction donnant le nombre de clients $c\left(t\right)$ en fonction du temps $t$, exprimé en jours. On prends donc $t\in\left[0;365\right]$.

\begin{QCM}

En remarquant que le nombre de clients ne peut être négatif, déterminé, parmi les propositions ci-dessous, laquelle donne la formule de $c\left(t\right)$ en fonction de $t$ :

\begin{enumerate}

\Faux{$c\left(t\right)=4\sin\left(\frac{6\pi t}{365}+25\right)$}

\Faux{$c\left(t\right)=4\cos\left(\frac{6\pi t}{365}-25\right)$}

\Faux{$c\left(t\right)=4\sin\left(\frac{6\pi t}{365}\right)-25$}

\Vrai{$c\left(t\right)=4\cos\left(\frac{6\pi t}{365}\right)+25$}

\end{enumerate}

\end{QCM}

\begin{QCM}

On a alors que les valeurs de $c\left(t\right)$ sont comprises entre :

\begin{enumerate}

\Faux{$0$ et $4$ }

\Faux{$0$ et $29$ }

\Vrai{$21$ et $29$}

\Faux{$0$ et $25$}

\end{enumerate}

\end{QCM}

Dès lors, il peut calculer le nombre moyen de clients en appliquant la formule

$\frac{1}{365}\int\limits _{0}^{365}C\left(t\right)dt$ , où $C$ désigne une primitive de $c$.

\begin{QCM}

Parmi les propositions ci-dessous, laquelle donne une formule possible C~?

\begin{enumerate}

\Faux{$C\left(t\right)=-4\sin\left(\frac{6\pi t}{365}\right)$}

\Faux{$C\left(t\right)=\frac{24}{365}\pi\sin\left(\frac{6\pi t}{365}\right)-25t+18$}

\Faux{$C\left(t\right)=\frac{24}{365}\pi\cos\left(\frac{6\pi t}{365}\right)-25t-38$}

\Vrai{$C\left(t\right)=\frac{1460}{2}\pi\sin\left(\frac{6\pi t}{365}\right)+25t$}

\end{enumerate}

\end{QCM}

\begin{QCM}

On obtient ainsi que le nombre moyen de clients journaliers sur l'année en cours est donné par :

\begin{enumerate}

\Faux{$25\sin\left(6\pi\right)$ }

\Faux{$4$}

\Faux{$4\cos\left(6\pi\right)$}

\Vrai{$25$}

\end{enumerate}

\end{QCM}

\end{Sujet}

\begin{Sujet}

\subsection*{Exercice 1}

Pour chacune des propositions de cet exercice, cochez A si elle est vraie\par et B si elle est fausse.

On s'intéresse à la fonction f définie par $f:x\mapsto\left(2x+3\right)e^{4x}$.

\begin{VraiFaux}

\Faux{ L'ensemble de définition de f est $\left]0,+\infty\right[$.}

\end{VraiFaux}

\begin{VraiFaux}

\Faux{ La dérivée de f est donnée par $f^{\prime}:x\mapsto\left(8\right)e^{cx}$.}

\end{VraiFaux}

On admet que son approximation quadratique est donné par\par $\qquad f\left(x\right)\approx 3 + 8 x + 10 x^2$.

\begin{VraiFaux}

\Faux{ L'équation de la tangente $T$ à la courbe représentative $\mathscr C$ de $f$ en son point d'abscisse $0$ est donnée par $y=3x + 8 $.}

\end{VraiFaux}

\begin{VraiFaux}

\Faux{ $T$ est au-dessus de $\mathscr C$.}

\end{VraiFaux}

\subsection*{Exercice 2}

Cet exercice doit être rédigé au verso de la grille réponse.

\subsection*{Exercice 3}

\begin{QCM}

Parmi ces fonctions, laquelle est une primitive de la fonction $f$\par définie par $ f(x)=x \cdot \cos (3 x+1)$ ?

\begin{enumerate}

\Faux{ $F:x \mapsto \cos (3 x+1)-3 x \cdot \sin (3 x+1)$}

\Faux{ $G:x \mapsto x \cdot \sin (3 x+1)$}

\Vrai{ $H:x \mapsto \frac{x \cdot \sin (3 x+1)}{3}+\frac{\cos (3 x+1)}{9}$}

\end{enumerate}

\end{QCM}

\begin{QCM}

$\int\limits_{0}^{2}\left(5 t^{3}+3 t^{2}-t+7\right) d t=$

\begin{enumerate}

\Faux{ 71}

\Faux{ $15 t^{2}+6 t-1$}

\Vrai{ 40}

\end{enumerate}

\end{QCM}

\begin{QCM}

La valeur moyenne de la fonction $g: x \rightarrow e^{5 x} $ entre 0 et 2 est de :

\begin{enumerate}

\Faux{environ $4405,1$}

\Faux{environ $22 025,5$}

\Vrai{environ $2 202,5$}

\end{enumerate}

\end{QCM}

\newpage

\subsection*{Problème}

Alain Raffletou cherche à connaitre le nombre moyen de clients journaliers dans une de ses boutiques sur l'année en cours. Pour cela, il modélise le nombre de clients par une fonction donnant le nombre de clients $c\left(t\right)$ en fonction du temps $t$, exprimé en jours. On prends donc $t\in\left[0;365\right]$.

\begin{QCM}

En remarquant que le nombre de clients ne peut être négatif, déterminé, parmi les propositions ci-dessous, laquelle donne la formule de $c\left(t\right)$ en fonction de $t$ :

\begin{enumerate}

\Faux{$c\left(t\right)=4\sin\left(\frac{6\pi t}{365}+30\right)$}

\Vrai{$c\left(t\right)=4\cos\left(\frac{6\pi t}{365}\right)+30$}

\Faux{$c\left(t\right)=4\cos\left(\frac{6\pi t}{365}-30\right)$}

\Faux{$c\left(t\right)=4\sin\left(\frac{6\pi t}{365}\right)-30$}

\end{enumerate}

\end{QCM}

\begin{QCM}

On a alors que les valeurs de $c\left(t\right)$ sont comprises entre :

\begin{enumerate}

\Vrai{$26$ et $34$}

\Faux{$0$ et $4$ }

\Faux{$0$ et $34$ }

\Faux{$0$ et $30$}

\end{enumerate}

\end{QCM}

Dès lors, il peut calculer le nombre moyen de clients en appliquant la formule

$\frac{1}{365}\int\limits _{0}^{365}C\left(t\right)dt$ , où $C$ désigne une primitive de $c$.

\begin{QCM}

Parmi les propositions ci-dessous, laquelle donne une formule possible C~?

\begin{enumerate}

\Vrai{$C\left(t\right)=\frac{1460}{2}\pi\sin\left(\frac{6\pi t}{365}\right)+30t$}

\Faux{$C\left(t\right)=\frac{24}{365}\pi\sin\left(\frac{6\pi t}{365}\right)-30t+23$}

\Faux{$C\left(t\right)=\frac{24}{365}\pi\cos\left(\frac{6\pi t}{365}\right)-30t-48$}

\Faux{$C\left(t\right)=-4\sin\left(\frac{6\pi t}{365}\right)$}

\end{enumerate}

\end{QCM}

\begin{QCM}

On obtient ainsi que le nombre moyen de clients journaliers sur l'année en cours est donné par :

\begin{enumerate}

\Faux{$4\cos\left(6\pi\right)$}

\Vrai{$30$}

\Faux{$30\sin\left(6\pi\right)$ }

\Faux{$4$}

\end{enumerate}

\end{QCM}

\end{Sujet}

\begin{Sujet}

\subsection*{Exercice 1}

Pour chacune des propositions de cet exercice, cochez A si elle est vraie\par et B si elle est fausse.

On s'intéresse à la fonction f définie par $f:x\mapsto\left(2x+2\right)e^{2x}$.

\begin{VraiFaux}

\Vrai{ L'ensemble de définition de f est $\left]-\infty,+\infty\right[$.}

\end{VraiFaux}

\begin{VraiFaux}

\Faux{ La dérivée de f est donnée par $f^{\prime}:x\mapsto\left(4\right)e^{cx}$.}

\end{VraiFaux}

On admet que son approximation quadratique est donné par\par $\qquad f\left(x\right)\approx 2 + 6 x + 8 x^2$.

\begin{VraiFaux}

\Vrai{ L'équation de la tangente $T$ à la courbe représentative $\mathscr C$ de $f$ en son point d'abscisse $0$ est donnée par $y=6 x +2 $.}

\end{VraiFaux}

\begin{VraiFaux}

\Faux{ $T$ est au-dessus de $\mathscr C$.}

\end{VraiFaux}

\subsection*{Exercice 2}

Cet exercice doit être rédigé au verso de la grille réponse.

\subsection*{Exercice 3}

\begin{QCM}

Parmi ces fonctions, laquelle est une primitive de la fonction $f$\par définie par $ f(x)=x \cdot \cos (3 x+1)$ ?

\begin{enumerate}

\Vrai{ $H:x \mapsto \frac{x \cdot \sin (3 x+1)}{3}+\frac{\cos (3 x+1)}{9}$}

\Faux{ $F:x \mapsto \cos (3 x+1)-3 x \cdot \sin (3 x+1)$}

\Faux{ $G:x \mapsto x \cdot \sin (3 x+1)$}

\end{enumerate}

\end{QCM}

\begin{QCM}

$\int\limits_{0}^{2}\left(5 t^{3}+3 t^{2}-t+7\right) d t=$

\begin{enumerate}

\Faux{ 71}

\Vrai{ 40}

\Faux{ $15 t^{2}+6 t-1$}

\end{enumerate}

\end{QCM}

\begin{QCM}

La valeur moyenne de la fonction $g: x \rightarrow e^{5 x} $ entre 0 et 2 est de :

\begin{enumerate}

\Faux{environ $4405,1$}

\Vrai{environ $2 202,5$}

\Faux{environ $22 025,5$}

\end{enumerate}

\end{QCM}

\newpage

\subsection*{Problème}

Alain Raffletou cherche à connaitre le nombre moyen de clients journaliers dans une de ses boutiques sur l'année en cours. Pour cela, il modélise le nombre de clients par une fonction donnant le nombre de clients $c\left(t\right)$ en fonction du temps $t$, exprimé en jours. On prends donc $t\in\left[0;365\right]$.

\begin{QCM}

En remarquant que le nombre de clients ne peut être négatif, déterminé, parmi les propositions ci-dessous, laquelle donne la formule de $c\left(t\right)$ en fonction de $t$ :

\begin{enumerate}

\Faux{$c\left(t\right)=3\sin\left(\frac{4\pi t}{365}+25\right)$}

\Vrai{$c\left(t\right)=3\cos\left(\frac{4\pi t}{365}\right)+25$}

\Faux{$c\left(t\right)=3\sin\left(\frac{4\pi t}{365}\right)-25$}

\Faux{$c\left(t\right)=3\cos\left(\frac{4\pi t}{365}-25\right)$}

\end{enumerate}

\end{QCM}

\begin{QCM}

On a alors que les valeurs de $c\left(t\right)$ sont comprises entre :

\begin{enumerate}

\Faux{$0$ et $3$ }

\Faux{$0$ et $28$ }

\Vrai{$22$ et $28$}

\Faux{$0$ et $25$}

\end{enumerate}

\end{QCM}

Dès lors, il peut calculer le nombre moyen de clients en appliquant la formule

$\frac{1}{365}\int\limits _{0}^{365}C\left(t\right)dt$ , où $C$ désigne une primitive de $c$.

\begin{QCM}

Parmi les propositions ci-dessous, laquelle donne une formule possible C~?

\begin{enumerate}

\Faux{$C\left(t\right)=\frac{12}{365}\pi\cos\left(\frac{4\pi t}{365}\right)-25t-41$}

\Faux{$C\left(t\right)=-3\sin\left(\frac{4\pi t}{365}\right)$}

\Faux{$C\left(t\right)=\frac{12}{365}\pi\sin\left(\frac{4\pi t}{365}\right)-25t+20$}

\Vrai{$C\left(t\right)=\frac{1095}{2}\pi\sin\left(\frac{4\pi t}{365}\right)+25t$}

\end{enumerate}

\end{QCM}

\begin{QCM}

On obtient ainsi que le nombre moyen de clients journaliers sur l'année en cours est donné par :

\begin{enumerate}

\Vrai{$25$}

\Faux{$3$}

\Faux{$3\cos\left(4\pi\right)$}

\Faux{$25\sin\left(4\pi\right)$ }

\end{enumerate}

\end{QCM}

\end{Sujet}

\begin{Sujet}

\subsection*{Exercice 1}

Pour chacune des propositions de cet exercice, cochez A si elle est vraie\par et B si elle est fausse.

On s'intéresse à la fonction f définie par $f:x\mapsto\left(7x+7\right)e^{3x}$.

\begin{VraiFaux}

\Vrai{ L'ensemble de définition de f est $\left]-\infty,+\infty\right[$.}

\end{VraiFaux}

\begin{VraiFaux}

\Vrai{ La dérivée de f est donnée par $f^{\prime}:x\mapsto\left(21x+28\right)e^{3x}$.}

\end{VraiFaux}

On admet que son approximation quadratique est donné par\par $\qquad f\left(x\right)\approx 7 + 56 x + \frac{441}{2} x^2$.

\begin{VraiFaux}

\Vrai{ L'équation de la tangente $T$ à la courbe représentative $\mathscr C$ de $f$ en son point d'abscisse $0$ est donnée par $y=56 x +7 $.}

\end{VraiFaux}

\begin{VraiFaux}

\Vrai{ $T$ est au-dessous de $\mathscr C$.}

\end{VraiFaux}

\subsection*{Exercice 2}

Cet exercice doit être rédigé au verso de la grille réponse.

\subsection*{Exercice 3}

\begin{QCM}

Parmi ces fonctions, laquelle est une primitive de la fonction $f$\par définie par $ f(x)=x \cdot \cos (3 x+1)$ ?

\begin{enumerate}

\Faux{ $G:x \mapsto x \cdot \sin (3 x+1)$}

\Vrai{ $H:x \mapsto \frac{x \cdot \sin (3 x+1)}{3}+\frac{\cos (3 x+1)}{9}$}

\Faux{ $F:x \mapsto \cos (3 x+1)-3 x \cdot \sin (3 x+1)$}

\end{enumerate}

\end{QCM}

\begin{QCM}

$\int\limits_{0}^{2}\left(5 t^{3}+3 t^{2}-t+7\right) d t=$

\begin{enumerate}

\Vrai{ 40}

\Faux{ $15 t^{2}+6 t-1$}

\Faux{ 71}

\end{enumerate}

\end{QCM}

\begin{QCM}

La valeur moyenne de la fonction $g: x \rightarrow e^{5 x} $ entre 0 et 2 est de :

\begin{enumerate}

\Faux{environ $22 025,5$}

\Faux{environ $4405,1$}

\Vrai{environ $2 202,5$}

\end{enumerate}

\end{QCM}

\newpage

\subsection*{Problème}

Alain Raffletou cherche à connaitre le nombre moyen de clients journaliers dans une de ses boutiques sur l'année en cours. Pour cela, il modélise le nombre de clients par une fonction donnant le nombre de clients $c\left(t\right)$ en fonction du temps $t$, exprimé en jours. On prends donc $t\in\left[0;365\right]$.

\begin{QCM}

En remarquant que le nombre de clients ne peut être négatif, déterminé, parmi les propositions ci-dessous, laquelle donne la formule de $c\left(t\right)$ en fonction de $t$ :

\begin{enumerate}

\Faux{$c\left(t\right)=4\sin\left(\frac{4\pi t}{365}\right)-22$}

\Vrai{$c\left(t\right)=4\cos\left(\frac{4\pi t}{365}\right)+22$}

\Faux{$c\left(t\right)=4\sin\left(\frac{4\pi t}{365}+22\right)$}

\Faux{$c\left(t\right)=4\cos\left(\frac{4\pi t}{365}-22\right)$}

\end{enumerate}

\end{QCM}

\begin{QCM}

On a alors que les valeurs de $c\left(t\right)$ sont comprises entre :

\begin{enumerate}

\Faux{$0$ et $22$}

\Vrai{$18$ et $26$}

\Faux{$0$ et $4$ }

\Faux{$0$ et $26$ }

\end{enumerate}

\end{QCM}

Dès lors, il peut calculer le nombre moyen de clients en appliquant la formule

$\frac{1}{365}\int\limits _{0}^{365}C\left(t\right)dt$ , où $C$ désigne une primitive de $c$.

\begin{QCM}

Parmi les propositions ci-dessous, laquelle donne une formule possible C~?

\begin{enumerate}

\Faux{$C\left(t\right)=\frac{16}{365}\pi\cos\left(\frac{4\pi t}{365}\right)-22t-32$}

\Vrai{$C\left(t\right)=\frac{1460}{2}\pi\sin\left(\frac{4\pi t}{365}\right)+22t$}

\Faux{$C\left(t\right)=\frac{16}{365}\pi\sin\left(\frac{4\pi t}{365}\right)-22t+15$}

\Faux{$C\left(t\right)=-4\sin\left(\frac{4\pi t}{365}\right)$}

\end{enumerate}

\end{QCM}

\begin{QCM}

On obtient ainsi que le nombre moyen de clients journaliers sur l'année en cours est donné par :

\begin{enumerate}

\Vrai{$22$}

\Faux{$4\cos\left(4\pi\right)$}

\Faux{$4$}

\Faux{$22\sin\left(4\pi\right)$ }

\end{enumerate}

\end{QCM}

\end{Sujet}

\begin{Sujet}

\subsection*{Exercice 1}

Pour chacune des propositions de cet exercice, cochez A si elle est vraie\par et B si elle est fausse.

On s'intéresse à la fonction f définie par $f:x\mapsto\left(7x+5\right)e^{5x}$.

\begin{VraiFaux}

\Vrai{ L'ensemble de définition de f est $\left]-\infty,+\infty\right[$.}

\end{VraiFaux}

\begin{VraiFaux}

\Vrai{ La dérivée de f est donnée par $f^{\prime}:x\mapsto\left(35x+32\right)e^{5x}$.}

\end{VraiFaux}

On admet que son approximation quadratique est donné par\par $\qquad f\left(x\right)\approx 5 + 42 x + \frac{343}{2} x^2$.

\begin{VraiFaux}

\Faux{ L'équation de la tangente $T$ à la courbe représentative $\mathscr C$ de $f$ en son point d'abscisse $0$ est donnée par $y=5x + 42 $.}

\end{VraiFaux}

\begin{VraiFaux}

\Faux{ $T$ est au-dessus de $\mathscr C$.}

\end{VraiFaux}

\subsection*{Exercice 2}

Cet exercice doit être rédigé au verso de la grille réponse.

\subsection*{Exercice 3}

\begin{QCM}

Parmi ces fonctions, laquelle est une primitive de la fonction $f$\par définie par $ f(x)=x \cdot \cos (3 x+1)$ ?

\begin{enumerate}

\Faux{ $F:x \mapsto \cos (3 x+1)-3 x \cdot \sin (3 x+1)$}

\Faux{ $G:x \mapsto x \cdot \sin (3 x+1)$}

\Vrai{ $H:x \mapsto \frac{x \cdot \sin (3 x+1)}{3}+\frac{\cos (3 x+1)}{9}$}

\end{enumerate}

\end{QCM}

\begin{QCM}

$\int\limits_{0}^{2}\left(5 t^{3}+3 t^{2}-t+7\right) d t=$

\begin{enumerate}

\Faux{ 71}

\Vrai{ 40}

\Faux{ $15 t^{2}+6 t-1$}

\end{enumerate}

\end{QCM}

\begin{QCM}

La valeur moyenne de la fonction $g: x \rightarrow e^{5 x} $ entre 0 et 2 est de :

\begin{enumerate}

\Faux{environ $22 025,5$}

\Faux{environ $4405,1$}

\Vrai{environ $2 202,5$}

\end{enumerate}

\end{QCM}

\newpage

\subsection*{Problème}

Alain Raffletou cherche à connaitre le nombre moyen de clients journaliers dans une de ses boutiques sur l'année en cours. Pour cela, il modélise le nombre de clients par une fonction donnant le nombre de clients $c\left(t\right)$ en fonction du temps $t$, exprimé en jours. On prends donc $t\in\left[0;365\right]$.

\begin{QCM}

En remarquant que le nombre de clients ne peut être négatif, déterminé, parmi les propositions ci-dessous, laquelle donne la formule de $c\left(t\right)$ en fonction de $t$ :

\begin{enumerate}

\Faux{$c\left(t\right)=5\cos\left(\frac{8\pi t}{365}-27\right)$}

\Vrai{$c\left(t\right)=5\cos\left(\frac{8\pi t}{365}\right)+27$}

\Faux{$c\left(t\right)=5\sin\left(\frac{8\pi t}{365}+27\right)$}

\Faux{$c\left(t\right)=5\sin\left(\frac{8\pi t}{365}\right)-27$}

\end{enumerate}

\end{QCM}

\begin{QCM}

On a alors que les valeurs de $c\left(t\right)$ sont comprises entre :

\begin{enumerate}

\Vrai{$22$ et $32$}

\Faux{$0$ et $27$}

\Faux{$0$ et $5$ }

\Faux{$0$ et $32$ }

\end{enumerate}

\end{QCM}

Dès lors, il peut calculer le nombre moyen de clients en appliquant la formule

$\frac{1}{365}\int\limits _{0}^{365}C\left(t\right)dt$ , où $C$ désigne une primitive de $c$.

\begin{QCM}

Parmi les propositions ci-dessous, laquelle donne une formule possible C~?

\begin{enumerate}

\Faux{$C\left(t\right)=\frac{40}{365}\pi\cos\left(\frac{8\pi t}{365}\right)-27t-39$}

\Faux{$C\left(t\right)=\frac{40}{365}\pi\sin\left(\frac{8\pi t}{365}\right)-27t+18$}

\Faux{$C\left(t\right)=-5\sin\left(\frac{8\pi t}{365}\right)$}

\Vrai{$C\left(t\right)=\frac{1825}{2}\pi\sin\left(\frac{8\pi t}{365}\right)+27t$}

\end{enumerate}

\end{QCM}

\begin{QCM}

On obtient ainsi que le nombre moyen de clients journaliers sur l'année en cours est donné par :

\begin{enumerate}

\Vrai{$27$}

\Faux{$5\cos\left(8\pi\right)$}

\Faux{$27\sin\left(8\pi\right)$ }

\Faux{$5$}

\end{enumerate}

\end{QCM}

\end{Sujet}

\begin{Sujet}

\subsection*{Exercice 1}

Pour chacune des propositions de cet exercice, cochez A si elle est vraie\par et B si elle est fausse.

On s'intéresse à la fonction f définie par $f:x\mapsto\left(2x+2\right)e^{6x}$.

\begin{VraiFaux}

\Faux{ L'ensemble de définition de f est $\left]0,+\infty\right[$.}

\end{VraiFaux}

\begin{VraiFaux}

\Faux{ La dérivée de f est donnée par $f^{\prime}:x\mapsto\left(12\right)e^{cx}$.}

\end{VraiFaux}

On admet que son approximation quadratique est donné par\par $\qquad f\left(x\right)\approx 2 + 6 x + 8 x^2$.

\begin{VraiFaux}

\Vrai{ L'équation de la tangente $T$ à la courbe représentative $\mathscr C$ de $f$ en son point d'abscisse $0$ est donnée par $y=6 x +2 $.}

\end{VraiFaux}

\begin{VraiFaux}

\Faux{ $T$ est au-dessus de $\mathscr C$.}

\end{VraiFaux}

\subsection*{Exercice 2}

Cet exercice doit être rédigé au verso de la grille réponse.

\subsection*{Exercice 3}

\begin{QCM}

Parmi ces fonctions, laquelle est une primitive de la fonction $f$\par définie par $ f(x)=x \cdot \cos (3 x+1)$ ?

\begin{enumerate}

\Faux{ $F:x \mapsto \cos (3 x+1)-3 x \cdot \sin (3 x+1)$}

\Faux{ $G:x \mapsto x \cdot \sin (3 x+1)$}

\Vrai{ $H:x \mapsto \frac{x \cdot \sin (3 x+1)}{3}+\frac{\cos (3 x+1)}{9}$}

\end{enumerate}

\end{QCM}

\begin{QCM}

$\int\limits_{0}^{2}\left(5 t^{3}+3 t^{2}-t+7\right) d t=$

\begin{enumerate}

\Faux{ 71}

\Faux{ $15 t^{2}+6 t-1$}

\Vrai{ 40}

\end{enumerate}

\end{QCM}

\begin{QCM}

La valeur moyenne de la fonction $g: x \rightarrow e^{5 x} $ entre 0 et 2 est de :

\begin{enumerate}

\Vrai{environ $2 202,5$}

\Faux{environ $22 025,5$}

\Faux{environ $4405,1$}

\end{enumerate}

\end{QCM}

\newpage

\subsection*{Problème}

Alain Raffletou cherche à connaitre le nombre moyen de clients journaliers dans une de ses boutiques sur l'année en cours. Pour cela, il modélise le nombre de clients par une fonction donnant le nombre de clients $c\left(t\right)$ en fonction du temps $t$, exprimé en jours. On prends donc $t\in\left[0;365\right]$.

\begin{QCM}

En remarquant que le nombre de clients ne peut être négatif, déterminé, parmi les propositions ci-dessous, laquelle donne la formule de $c\left(t\right)$ en fonction de $t$ :

\begin{enumerate}

\Faux{$c\left(t\right)=5\sin\left(\frac{8\pi t}{365}+13\right)$}

\Vrai{$c\left(t\right)=5\cos\left(\frac{8\pi t}{365}\right)+13$}

\Faux{$c\left(t\right)=5\sin\left(\frac{8\pi t}{365}\right)-13$}

\Faux{$c\left(t\right)=5\cos\left(\frac{8\pi t}{365}-13\right)$}

\end{enumerate}

\end{QCM}

\begin{QCM}

On a alors que les valeurs de $c\left(t\right)$ sont comprises entre :

\begin{enumerate}

\Faux{$0$ et $18$ }

\Faux{$0$ et $13$}

\Vrai{$8$ et $18$}

\Faux{$0$ et $5$ }

\end{enumerate}

\end{QCM}

Dès lors, il peut calculer le nombre moyen de clients en appliquant la formule

$\frac{1}{365}\int\limits _{0}^{365}C\left(t\right)dt$ , où $C$ désigne une primitive de $c$.

\begin{QCM}

Parmi les propositions ci-dessous, laquelle donne une formule possible C~?

\begin{enumerate}

\Faux{$C\left(t\right)=-5\sin\left(\frac{8\pi t}{365}\right)$}

\Faux{$C\left(t\right)=\frac{40}{365}\pi\cos\left(\frac{8\pi t}{365}\right)-13t-11$}

\Faux{$C\left(t\right)=\frac{40}{365}\pi\sin\left(\frac{8\pi t}{365}\right)-13t+4$}

\Vrai{$C\left(t\right)=\frac{1825}{2}\pi\sin\left(\frac{8\pi t}{365}\right)+13t$}

\end{enumerate}

\end{QCM}

\begin{QCM}

On obtient ainsi que le nombre moyen de clients journaliers sur l'année en cours est donné par :

\begin{enumerate}

\Vrai{$13$}

\Faux{$5\cos\left(8\pi\right)$}

\Faux{$13\sin\left(8\pi\right)$ }

\Faux{$5$}

\end{enumerate}

\end{QCM}

\end{Sujet}

\begin{Sujet}

\subsection*{Exercice 1}

Pour chacune des propositions de cet exercice, cochez A si elle est vraie\par et B si elle est fausse.

On s'intéresse à la fonction f définie par $f:x\mapsto\left(6x+6\right)e^{5x}$.

\begin{VraiFaux}

\Faux{ L'ensemble de définition de f est $\left]0,+\infty\right[$.}

\end{VraiFaux}

\begin{VraiFaux}

\Faux{ La dérivée de f est donnée par $f^{\prime}:x\mapsto\left(30\right)e^{cx}$.}

\end{VraiFaux}

On admet que son approximation quadratique est donné par\par $\qquad f\left(x\right)\approx 6 + 42 x + 144 x^2$.

\begin{VraiFaux}

\Faux{ L'équation de la tangente $T$ à la courbe représentative $\mathscr C$ de $f$ en son point d'abscisse $0$ est donnée par $y=6x + 42 $.}

\end{VraiFaux}

\begin{VraiFaux}

\Vrai{ $T$ est au-dessous de $\mathscr C$.}

\end{VraiFaux}

\subsection*{Exercice 2}

Cet exercice doit être rédigé au verso de la grille réponse.

\subsection*{Exercice 3}

\begin{QCM}

Parmi ces fonctions, laquelle est une primitive de la fonction $f$\par définie par $ f(x)=x \cdot \cos (3 x+1)$ ?

\begin{enumerate}

\Vrai{ $H:x \mapsto \frac{x \cdot \sin (3 x+1)}{3}+\frac{\cos (3 x+1)}{9}$}

\Faux{ $F:x \mapsto \cos (3 x+1)-3 x \cdot \sin (3 x+1)$}

\Faux{ $G:x \mapsto x \cdot \sin (3 x+1)$}

\end{enumerate}

\end{QCM}

\begin{QCM}

$\int\limits_{0}^{2}\left(5 t^{3}+3 t^{2}-t+7\right) d t=$

\begin{enumerate}

\Faux{ $15 t^{2}+6 t-1$}

\Vrai{ 40}

\Faux{ 71}

\end{enumerate}

\end{QCM}

\begin{QCM}

La valeur moyenne de la fonction $g: x \rightarrow e^{5 x} $ entre 0 et 2 est de :

\begin{enumerate}

\Vrai{environ $2 202,5$}

\Faux{environ $4405,1$}

\Faux{environ $22 025,5$}

\end{enumerate}

\end{QCM}

\newpage

\subsection*{Problème}

Alain Raffletou cherche à connaitre le nombre moyen de clients journaliers dans une de ses boutiques sur l'année en cours. Pour cela, il modélise le nombre de clients par une fonction donnant le nombre de clients $c\left(t\right)$ en fonction du temps $t$, exprimé en jours. On prends donc $t\in\left[0;365\right]$.

\begin{QCM}

En remarquant que le nombre de clients ne peut être négatif, déterminé, parmi les propositions ci-dessous, laquelle donne la formule de $c\left(t\right)$ en fonction de $t$ :

\begin{enumerate}

\Faux{$c\left(t\right)=3\sin\left(\frac{6\pi t}{365}\right)-23$}

\Faux{$c\left(t\right)=3\sin\left(\frac{6\pi t}{365}+23\right)$}

\Faux{$c\left(t\right)=3\cos\left(\frac{6\pi t}{365}-23\right)$}

\Vrai{$c\left(t\right)=3\cos\left(\frac{6\pi t}{365}\right)+23$}

\end{enumerate}

\end{QCM}

\begin{QCM}

On a alors que les valeurs de $c\left(t\right)$ sont comprises entre :

\begin{enumerate}

\Vrai{$20$ et $26$}

\Faux{$0$ et $23$}

\Faux{$0$ et $3$ }

\Faux{$0$ et $26$ }

\end{enumerate}

\end{QCM}

Dès lors, il peut calculer le nombre moyen de clients en appliquant la formule

$\frac{1}{365}\int\limits _{0}^{365}C\left(t\right)dt$ , où $C$ désigne une primitive de $c$.

\begin{QCM}

Parmi les propositions ci-dessous, laquelle donne une formule possible C~?

\begin{enumerate}

\Faux{$C\left(t\right)=\frac{18}{365}\pi\sin\left(\frac{6\pi t}{365}\right)-23t+18$}

\Faux{$C\left(t\right)=\frac{18}{365}\pi\cos\left(\frac{6\pi t}{365}\right)-23t-37$}

\Faux{$C\left(t\right)=-3\sin\left(\frac{6\pi t}{365}\right)$}

\Vrai{$C\left(t\right)=\frac{1095}{2}\pi\sin\left(\frac{6\pi t}{365}\right)+23t$}

\end{enumerate}

\end{QCM}

\begin{QCM}

On obtient ainsi que le nombre moyen de clients journaliers sur l'année en cours est donné par :

\begin{enumerate}

\Vrai{$23$}

\Faux{$3$}

\Faux{$23\sin\left(6\pi\right)$ }

\Faux{$3\cos\left(6\pi\right)$}

\end{enumerate}

\end{QCM}

\end{Sujet}

\begin{Sujet}

\subsection*{Exercice 1}

Pour chacune des propositions de cet exercice, cochez A si elle est vraie\par et B si elle est fausse.

On s'intéresse à la fonction f définie par $f:x\mapsto\left(7x+3\right)e^{6x}$.

\begin{VraiFaux}

\Faux{ L'ensemble de définition de f est $\left]0,+\infty\right[$.}

\end{VraiFaux}

\begin{VraiFaux}

\Faux{ La dérivée de f est donnée par $f^{\prime}:x\mapsto\left(42\right)e^{cx}$.}

\end{VraiFaux}

On admet que son approximation quadratique est donné par\par $\qquad f\left(x\right)\approx 3 + 28 x + \frac{245}{2} x^2$.

\begin{VraiFaux}

\Vrai{ L'équation de la tangente $T$ à la courbe représentative $\mathscr C$ de $f$ en son point d'abscisse $0$ est donnée par $y=28 x +3 $.}

\end{VraiFaux}

\begin{VraiFaux}

\Vrai{ $T$ est au-dessous de $\mathscr C$.}

\end{VraiFaux}

\subsection*{Exercice 2}

Cet exercice doit être rédigé au verso de la grille réponse.

\subsection*{Exercice 3}

\begin{QCM}

Parmi ces fonctions, laquelle est une primitive de la fonction $f$\par définie par $ f(x)=x \cdot \cos (3 x+1)$ ?

\begin{enumerate}

\Faux{ $F:x \mapsto \cos (3 x+1)-3 x \cdot \sin (3 x+1)$}

\Faux{ $G:x \mapsto x \cdot \sin (3 x+1)$}

\Vrai{ $H:x \mapsto \frac{x \cdot \sin (3 x+1)}{3}+\frac{\cos (3 x+1)}{9}$}

\end{enumerate}

\end{QCM}

\begin{QCM}

$\int\limits_{0}^{2}\left(5 t^{3}+3 t^{2}-t+7\right) d t=$

\begin{enumerate}

\Faux{ $15 t^{2}+6 t-1$}

\Vrai{ 40}

\Faux{ 71}

\end{enumerate}

\end{QCM}

\begin{QCM}

La valeur moyenne de la fonction $g: x \rightarrow e^{5 x} $ entre 0 et 2 est de :

\begin{enumerate}

\Faux{environ $22 025,5$}

\Vrai{environ $2 202,5$}

\Faux{environ $4405,1$}

\end{enumerate}

\end{QCM}

\newpage

\subsection*{Problème}

Alain Raffletou cherche à connaitre le nombre moyen de clients journaliers dans une de ses boutiques sur l'année en cours. Pour cela, il modélise le nombre de clients par une fonction donnant le nombre de clients $c\left(t\right)$ en fonction du temps $t$, exprimé en jours. On prends donc $t\in\left[0;365\right]$.

\begin{QCM}

En remarquant que le nombre de clients ne peut être négatif, déterminé, parmi les propositions ci-dessous, laquelle donne la formule de $c\left(t\right)$ en fonction de $t$ :

\begin{enumerate}

\Faux{$c\left(t\right)=5\sin\left(\frac{4\pi t}{365}\right)-27$}

\Faux{$c\left(t\right)=5\cos\left(\frac{4\pi t}{365}-27\right)$}

\Vrai{$c\left(t\right)=5\cos\left(\frac{4\pi t}{365}\right)+27$}

\Faux{$c\left(t\right)=5\sin\left(\frac{4\pi t}{365}+27\right)$}

\end{enumerate}

\end{QCM}

\begin{QCM}

On a alors que les valeurs de $c\left(t\right)$ sont comprises entre :

\begin{enumerate}

\Faux{$0$ et $32$ }

\Faux{$0$ et $5$ }

\Faux{$0$ et $27$}

\Vrai{$22$ et $32$}

\end{enumerate}

\end{QCM}

Dès lors, il peut calculer le nombre moyen de clients en appliquant la formule

$\frac{1}{365}\int\limits _{0}^{365}C\left(t\right)dt$ , où $C$ désigne une primitive de $c$.

\begin{QCM}

Parmi les propositions ci-dessous, laquelle donne une formule possible C~?

\begin{enumerate}

\Faux{$C\left(t\right)=\frac{20}{365}\pi\cos\left(\frac{4\pi t}{365}\right)-27t-39$}

\Faux{$C\left(t\right)=\frac{20}{365}\pi\sin\left(\frac{4\pi t}{365}\right)-27t+18$}

\Faux{$C\left(t\right)=-5\sin\left(\frac{4\pi t}{365}\right)$}

\Vrai{$C\left(t\right)=\frac{1825}{2}\pi\sin\left(\frac{4\pi t}{365}\right)+27t$}

\end{enumerate}

\end{QCM}

\begin{QCM}

On obtient ainsi que le nombre moyen de clients journaliers sur l'année en cours est donné par :

\begin{enumerate}

\Faux{$5\cos\left(4\pi\right)$}

\Faux{$5$}

\Faux{$27\sin\left(4\pi\right)$ }

\Vrai{$27$}

\end{enumerate}

\end{QCM}

\end{Sujet}

\begin{Sujet}

\subsection*{Exercice 1}

Pour chacune des propositions de cet exercice, cochez A si elle est vraie\par et B si elle est fausse.

On s'intéresse à la fonction f définie par $f:x\mapsto\left(4x+6\right)e^{6x}$.

\begin{VraiFaux}

\Faux{ L'ensemble de définition de f est $\left]0,+\infty\right[$.}

\end{VraiFaux}

\begin{VraiFaux}

\Faux{ La dérivée de f est donnée par $f^{\prime}:x\mapsto\left(24\right)e^{cx}$.}

\end{VraiFaux}

On admet que son approximation quadratique est donné par\par $\qquad f\left(x\right)\approx 6 + 28 x + 64 x^2$.

\begin{VraiFaux}

\Vrai{ L'équation de la tangente $T$ à la courbe représentative $\mathscr C$ de $f$ en son point d'abscisse $0$ est donnée par $y=28 x +6 $.}

\end{VraiFaux}

\begin{VraiFaux}

\Faux{ $T$ est au-dessus de $\mathscr C$.}

\end{VraiFaux}

\subsection*{Exercice 2}

Cet exercice doit être rédigé au verso de la grille réponse.

\subsection*{Exercice 3}

\begin{QCM}

Parmi ces fonctions, laquelle est une primitive de la fonction $f$\par définie par $ f(x)=x \cdot \cos (3 x+1)$ ?

\begin{enumerate}

\Vrai{ $H:x \mapsto \frac{x \cdot \sin (3 x+1)}{3}+\frac{\cos (3 x+1)}{9}$}

\Faux{ $G:x \mapsto x \cdot \sin (3 x+1)$}

\Faux{ $F:x \mapsto \cos (3 x+1)-3 x \cdot \sin (3 x+1)$}

\end{enumerate}

\end{QCM}

\begin{QCM}

$\int\limits_{0}^{2}\left(5 t^{3}+3 t^{2}-t+7\right) d t=$

\begin{enumerate}

\Faux{ 71}

\Faux{ $15 t^{2}+6 t-1$}

\Vrai{ 40}

\end{enumerate}

\end{QCM}

\begin{QCM}

La valeur moyenne de la fonction $g: x \rightarrow e^{5 x} $ entre 0 et 2 est de :

\begin{enumerate}

\Faux{environ $22 025,5$}

\Vrai{environ $2 202,5$}

\Faux{environ $4405,1$}

\end{enumerate}

\end{QCM}

\newpage

\subsection*{Problème}

Alain Raffletou cherche à connaitre le nombre moyen de clients journaliers dans une de ses boutiques sur l'année en cours. Pour cela, il modélise le nombre de clients par une fonction donnant le nombre de clients $c\left(t\right)$ en fonction du temps $t$, exprimé en jours. On prends donc $t\in\left[0;365\right]$.

\begin{QCM}

En remarquant que le nombre de clients ne peut être négatif, déterminé, parmi les propositions ci-dessous, laquelle donne la formule de $c\left(t\right)$ en fonction de $t$ :

\begin{enumerate}

\Faux{$c\left(t\right)=5\sin\left(\frac{6\pi t}{365}+25\right)$}

\Faux{$c\left(t\right)=5\cos\left(\frac{6\pi t}{365}-25\right)$}

\Vrai{$c\left(t\right)=5\cos\left(\frac{6\pi t}{365}\right)+25$}

\Faux{$c\left(t\right)=5\sin\left(\frac{6\pi t}{365}\right)-25$}

\end{enumerate}

\end{QCM}

\begin{QCM}

On a alors que les valeurs de $c\left(t\right)$ sont comprises entre :

\begin{enumerate}

\Vrai{$20$ et $30$}

\Faux{$0$ et $5$ }

\Faux{$0$ et $25$}

\Faux{$0$ et $30$ }

\end{enumerate}

\end{QCM}

Dès lors, il peut calculer le nombre moyen de clients en appliquant la formule

$\frac{1}{365}\int\limits _{0}^{365}C\left(t\right)dt$ , où $C$ désigne une primitive de $c$.

\begin{QCM}

Parmi les propositions ci-dessous, laquelle donne une formule possible C~?

\begin{enumerate}

\Faux{$C\left(t\right)=\frac{30}{365}\pi\sin\left(\frac{6\pi t}{365}\right)-25t+16$}

\Faux{$C\left(t\right)=\frac{30}{365}\pi\cos\left(\frac{6\pi t}{365}\right)-25t-35$}

\Vrai{$C\left(t\right)=\frac{1825}{2}\pi\sin\left(\frac{6\pi t}{365}\right)+25t$}

\Faux{$C\left(t\right)=-5\sin\left(\frac{6\pi t}{365}\right)$}

\end{enumerate}

\end{QCM}

\begin{QCM}

On obtient ainsi que le nombre moyen de clients journaliers sur l'année en cours est donné par :

\begin{enumerate}

\Faux{$25\sin\left(6\pi\right)$ }

\Faux{$5$}

\Vrai{$25$}

\Faux{$5\cos\left(6\pi\right)$}

\end{enumerate}

\end{QCM}

\end{Sujet}

\begin{Sujet}

\subsection*{Exercice 1}

Pour chacune des propositions de cet exercice, cochez A si elle est vraie\par et B si elle est fausse.

On s'intéresse à la fonction f définie par $f:x\mapsto\left(3x+5\right)e^{3x}$.

\begin{VraiFaux}

\Faux{ L'ensemble de définition de f est $\left]0,+\infty\right[$.}

\end{VraiFaux}

\begin{VraiFaux}

\Vrai{ La dérivée de f est donnée par $f^{\prime}:x\mapsto\left(9x+18\right)e^{3x}$.}

\end{VraiFaux}

On admet que son approximation quadratique est donné par\par $\qquad f\left(x\right)\approx 5 + 18 x + \frac{63}{2} x^2$.

\begin{VraiFaux}

\Vrai{ L'équation de la tangente $T$ à la courbe représentative $\mathscr C$ de $f$ en son point d'abscisse $0$ est donnée par $y=18 x +5 $.}

\end{VraiFaux}

\begin{VraiFaux}

\Vrai{ $T$ est au-dessous de $\mathscr C$.}

\end{VraiFaux}

\subsection*{Exercice 2}

Cet exercice doit être rédigé au verso de la grille réponse.

\subsection*{Exercice 3}

\begin{QCM}

Parmi ces fonctions, laquelle est une primitive de la fonction $f$\par définie par $ f(x)=x \cdot \cos (3 x+1)$ ?

\begin{enumerate}

\Faux{ $G:x \mapsto x \cdot \sin (3 x+1)$}

\Vrai{ $H:x \mapsto \frac{x \cdot \sin (3 x+1)}{3}+\frac{\cos (3 x+1)}{9}$}

\Faux{ $F:x \mapsto \cos (3 x+1)-3 x \cdot \sin (3 x+1)$}

\end{enumerate}

\end{QCM}

\begin{QCM}

$\int\limits_{0}^{2}\left(5 t^{3}+3 t^{2}-t+7\right) d t=$

\begin{enumerate}

\Faux{ $15 t^{2}+6 t-1$}

\Vrai{ 40}

\Faux{ 71}

\end{enumerate}

\end{QCM}

\begin{QCM}

La valeur moyenne de la fonction $g: x \rightarrow e^{5 x} $ entre 0 et 2 est de :

\begin{enumerate}

\Faux{environ $22 025,5$}

\Faux{environ $4405,1$}

\Vrai{environ $2 202,5$}

\end{enumerate}

\end{QCM}

\newpage

\subsection*{Problème}

Alain Raffletou cherche à connaitre le nombre moyen de clients journaliers dans une de ses boutiques sur l'année en cours. Pour cela, il modélise le nombre de clients par une fonction donnant le nombre de clients $c\left(t\right)$ en fonction du temps $t$, exprimé en jours. On prends donc $t\in\left[0;365\right]$.

\begin{QCM}

En remarquant que le nombre de clients ne peut être négatif, déterminé, parmi les propositions ci-dessous, laquelle donne la formule de $c\left(t\right)$ en fonction de $t$ :

\begin{enumerate}

\Faux{$c\left(t\right)=4\cos\left(\frac{6\pi t}{365}-17\right)$}

\Faux{$c\left(t\right)=4\sin\left(\frac{6\pi t}{365}\right)-17$}

\Faux{$c\left(t\right)=4\sin\left(\frac{6\pi t}{365}+17\right)$}

\Vrai{$c\left(t\right)=4\cos\left(\frac{6\pi t}{365}\right)+17$}

\end{enumerate}

\end{QCM}

\begin{QCM}

On a alors que les valeurs de $c\left(t\right)$ sont comprises entre :

\begin{enumerate}

\Faux{$0$ et $21$ }

\Faux{$0$ et $4$ }

\Faux{$0$ et $17$}

\Vrai{$13$ et $21$}

\end{enumerate}

\end{QCM}

Dès lors, il peut calculer le nombre moyen de clients en appliquant la formule

$\frac{1}{365}\int\limits _{0}^{365}C\left(t\right)dt$ , où $C$ désigne une primitive de $c$.

\begin{QCM}

Parmi les propositions ci-dessous, laquelle donne une formule possible C~?

\begin{enumerate}

\Faux{$C\left(t\right)=-4\sin\left(\frac{6\pi t}{365}\right)$}

\Faux{$C\left(t\right)=\frac{24}{365}\pi\sin\left(\frac{6\pi t}{365}\right)-17t+10$}

\Faux{$C\left(t\right)=\frac{24}{365}\pi\cos\left(\frac{6\pi t}{365}\right)-17t-22$}

\Vrai{$C\left(t\right)=\frac{1460}{2}\pi\sin\left(\frac{6\pi t}{365}\right)+17t$}

\end{enumerate}

\end{QCM}

\begin{QCM}

On obtient ainsi que le nombre moyen de clients journaliers sur l'année en cours est donné par :

\begin{enumerate}

\Faux{$4$}

\Faux{$4\cos\left(6\pi\right)$}

\Faux{$17\sin\left(6\pi\right)$ }

\Vrai{$17$}

\end{enumerate}

\end{QCM}

\end{Sujet}

\begin{Sujet}

\subsection*{Exercice 1}

Pour chacune des propositions de cet exercice, cochez A si elle est vraie\par et B si elle est fausse.

On s'intéresse à la fonction f définie par $f:x\mapsto\left(7x+5\right)e^{3x}$.

\begin{VraiFaux}

\Vrai{ L'ensemble de définition de f est $\left]-\infty,+\infty\right[$.}

\end{VraiFaux}

\begin{VraiFaux}

\Faux{ La dérivée de f est donnée par $f^{\prime}:x\mapsto\left(21\right)e^{cx}$.}

\end{VraiFaux}

On admet que son approximation quadratique est donné par\par $\qquad f\left(x\right)\approx 5 + 42 x + \frac{343}{2} x^2$.

\begin{VraiFaux}

\Vrai{ L'équation de la tangente $T$ à la courbe représentative $\mathscr C$ de $f$ en son point d'abscisse $0$ est donnée par $y=42 x +5 $.}

\end{VraiFaux}

\begin{VraiFaux}

\Vrai{ $T$ est au-dessous de $\mathscr C$.}

\end{VraiFaux}

\subsection*{Exercice 2}

Cet exercice doit être rédigé au verso de la grille réponse.

\subsection*{Exercice 3}

\begin{QCM}

Parmi ces fonctions, laquelle est une primitive de la fonction $f$\par définie par $ f(x)=x \cdot \cos (3 x+1)$ ?

\begin{enumerate}

\Faux{ $G:x \mapsto x \cdot \sin (3 x+1)$}

\Vrai{ $H:x \mapsto \frac{x \cdot \sin (3 x+1)}{3}+\frac{\cos (3 x+1)}{9}$}

\Faux{ $F:x \mapsto \cos (3 x+1)-3 x \cdot \sin (3 x+1)$}

\end{enumerate}

\end{QCM}

\begin{QCM}

$\int\limits_{0}^{2}\left(5 t^{3}+3 t^{2}-t+7\right) d t=$

\begin{enumerate}

\Faux{ 71}

\Vrai{ 40}

\Faux{ $15 t^{2}+6 t-1$}

\end{enumerate}

\end{QCM}

\begin{QCM}

La valeur moyenne de la fonction $g: x \rightarrow e^{5 x} $ entre 0 et 2 est de :

\begin{enumerate}

\Faux{environ $4405,1$}

\Faux{environ $22 025,5$}

\Vrai{environ $2 202,5$}

\end{enumerate}

\end{QCM}

\newpage

\subsection*{Problème}

Alain Raffletou cherche à connaitre le nombre moyen de clients journaliers dans une de ses boutiques sur l'année en cours. Pour cela, il modélise le nombre de clients par une fonction donnant le nombre de clients $c\left(t\right)$ en fonction du temps $t$, exprimé en jours. On prends donc $t\in\left[0;365\right]$.

\begin{QCM}

En remarquant que le nombre de clients ne peut être négatif, déterminé, parmi les propositions ci-dessous, laquelle donne la formule de $c\left(t\right)$ en fonction de $t$ :

\begin{enumerate}

\Faux{$c\left(t\right)=3\cos\left(\frac{8\pi t}{365}-28\right)$}

\Vrai{$c\left(t\right)=3\cos\left(\frac{8\pi t}{365}\right)+28$}

\Faux{$c\left(t\right)=3\sin\left(\frac{8\pi t}{365}+28\right)$}

\Faux{$c\left(t\right)=3\sin\left(\frac{8\pi t}{365}\right)-28$}

\end{enumerate}

\end{QCM}

\begin{QCM}

On a alors que les valeurs de $c\left(t\right)$ sont comprises entre :

\begin{enumerate}

\Faux{$0$ et $31$ }

\Vrai{$25$ et $31$}

\Faux{$0$ et $28$}

\Faux{$0$ et $3$ }

\end{enumerate}

\end{QCM}

Dès lors, il peut calculer le nombre moyen de clients en appliquant la formule

$\frac{1}{365}\int\limits _{0}^{365}C\left(t\right)dt$ , où $C$ désigne une primitive de $c$.

\begin{QCM}

Parmi les propositions ci-dessous, laquelle donne une formule possible C~?

\begin{enumerate}

\Faux{$C\left(t\right)=-3\sin\left(\frac{8\pi t}{365}\right)$}

\Faux{$C\left(t\right)=\frac{24}{365}\pi\sin\left(\frac{8\pi t}{365}\right)-28t+23$}

\Vrai{$C\left(t\right)=\frac{1095}{2}\pi\sin\left(\frac{8\pi t}{365}\right)+28t$}

\Faux{$C\left(t\right)=\frac{24}{365}\pi\cos\left(\frac{8\pi t}{365}\right)-28t-47$}

\end{enumerate}

\end{QCM}

\begin{QCM}

On obtient ainsi que le nombre moyen de clients journaliers sur l'année en cours est donné par :

\begin{enumerate}

\Vrai{$28$}

\Faux{$3\cos\left(8\pi\right)$}

\Faux{$28\sin\left(8\pi\right)$ }

\Faux{$3$}

\end{enumerate}

\end{QCM}

\end{Sujet}

\begin{Sujet}

\subsection*{Exercice 1}

Pour chacune des propositions de cet exercice, cochez A si elle est vraie\par et B si elle est fausse.

On s'intéresse à la fonction f définie par $f:x\mapsto\left(5x+4\right)e^{5x}$.

\begin{VraiFaux}

\Faux{ L'ensemble de définition de f est $\left]0,+\infty\right[$.}

\end{VraiFaux}

\begin{VraiFaux}

\Vrai{ La dérivée de f est donnée par $f^{\prime}:x\mapsto\left(25x+25\right)e^{5x}$.}

\end{VraiFaux}

On admet que son approximation quadratique est donné par\par $\qquad f\left(x\right)\approx 4 + 25 x + 75 x^2$.

\begin{VraiFaux}

\Vrai{ L'équation de la tangente $T$ à la courbe représentative $\mathscr C$ de $f$ en son point d'abscisse $0$ est donnée par $y=25 x +4 $.}

\end{VraiFaux}

\begin{VraiFaux}

\Vrai{ $T$ est au-dessous de $\mathscr C$.}

\end{VraiFaux}

\subsection*{Exercice 2}

Cet exercice doit être rédigé au verso de la grille réponse.

\subsection*{Exercice 3}

\begin{QCM}

Parmi ces fonctions, laquelle est une primitive de la fonction $f$\par définie par $ f(x)=x \cdot \cos (3 x+1)$ ?

\begin{enumerate}

\Vrai{ $H:x \mapsto \frac{x \cdot \sin (3 x+1)}{3}+\frac{\cos (3 x+1)}{9}$}

\Faux{ $G:x \mapsto x \cdot \sin (3 x+1)$}

\Faux{ $F:x \mapsto \cos (3 x+1)-3 x \cdot \sin (3 x+1)$}

\end{enumerate}

\end{QCM}

\begin{QCM}

$\int\limits_{0}^{2}\left(5 t^{3}+3 t^{2}-t+7\right) d t=$

\begin{enumerate}

\Vrai{ 40}

\Faux{ $15 t^{2}+6 t-1$}

\Faux{ 71}

\end{enumerate}

\end{QCM}

\begin{QCM}

La valeur moyenne de la fonction $g: x \rightarrow e^{5 x} $ entre 0 et 2 est de :

\begin{enumerate}

\Faux{environ $22 025,5$}

\Vrai{environ $2 202,5$}

\Faux{environ $4405,1$}

\end{enumerate}

\end{QCM}

\newpage

\subsection*{Problème}

Alain Raffletou cherche à connaitre le nombre moyen de clients journaliers dans une de ses boutiques sur l'année en cours. Pour cela, il modélise le nombre de clients par une fonction donnant le nombre de clients $c\left(t\right)$ en fonction du temps $t$, exprimé en jours. On prends donc $t\in\left[0;365\right]$.

\begin{QCM}

En remarquant que le nombre de clients ne peut être négatif, déterminé, parmi les propositions ci-dessous, laquelle donne la formule de $c\left(t\right)$ en fonction de $t$ :

\begin{enumerate}

\Faux{$c\left(t\right)=5\sin\left(\frac{6\pi t}{365}\right)-21$}

\Faux{$c\left(t\right)=5\cos\left(\frac{6\pi t}{365}-21\right)$}

\Vrai{$c\left(t\right)=5\cos\left(\frac{6\pi t}{365}\right)+21$}

\Faux{$c\left(t\right)=5\sin\left(\frac{6\pi t}{365}+21\right)$}

\end{enumerate}

\end{QCM}

\begin{QCM}

On a alors que les valeurs de $c\left(t\right)$ sont comprises entre :

\begin{enumerate}

\Faux{$0$ et $5$ }

\Vrai{$16$ et $26$}

\Faux{$0$ et $26$ }

\Faux{$0$ et $21$}

\end{enumerate}

\end{QCM}

Dès lors, il peut calculer le nombre moyen de clients en appliquant la formule

$\frac{1}{365}\int\limits _{0}^{365}C\left(t\right)dt$ , où $C$ désigne une primitive de $c$.

\begin{QCM}

Parmi les propositions ci-dessous, laquelle donne une formule possible C~?

\begin{enumerate}

\Faux{$C\left(t\right)=\frac{30}{365}\pi\sin\left(\frac{6\pi t}{365}\right)-21t+12$}

\Faux{$C\left(t\right)=-5\sin\left(\frac{6\pi t}{365}\right)$}

\Vrai{$C\left(t\right)=\frac{1825}{2}\pi\sin\left(\frac{6\pi t}{365}\right)+21t$}

\Faux{$C\left(t\right)=\frac{30}{365}\pi\cos\left(\frac{6\pi t}{365}\right)-21t-27$}

\end{enumerate}

\end{QCM}

\begin{QCM}

On obtient ainsi que le nombre moyen de clients journaliers sur l'année en cours est donné par :

\begin{enumerate}

\Faux{$21\sin\left(6\pi\right)$ }

\Vrai{$21$}

\Faux{$5$}

\Faux{$5\cos\left(6\pi\right)$}

\end{enumerate}

\end{QCM}

\end{Sujet}

\begin{Sujet}

\subsection*{Exercice 1}

Pour chacune des propositions de cet exercice, cochez A si elle est vraie\par et B si elle est fausse.

On s'intéresse à la fonction f définie par $f:x\mapsto\left(4x+7\right)e^{6x}$.

\begin{VraiFaux}

\Faux{ L'ensemble de définition de f est $\left]0,+\infty\right[$.}

\end{VraiFaux}

\begin{VraiFaux}

\Faux{ La dérivée de f est donnée par $f^{\prime}:x\mapsto\left(24\right)e^{cx}$.}

\end{VraiFaux}

On admet que son approximation quadratique est donné par\par $\qquad f\left(x\right)\approx 7 + 32 x + 72 x^2$.

\begin{VraiFaux}

\Vrai{ L'équation de la tangente $T$ à la courbe représentative $\mathscr C$ de $f$ en son point d'abscisse $0$ est donnée par $y=32 x +7 $.}

\end{VraiFaux}

\begin{VraiFaux}

\Faux{ $T$ est au-dessus de $\mathscr C$.}

\end{VraiFaux}

\subsection*{Exercice 2}

Cet exercice doit être rédigé au verso de la grille réponse.

\subsection*{Exercice 3}

\begin{QCM}

Parmi ces fonctions, laquelle est une primitive de la fonction $f$\par définie par $ f(x)=x \cdot \cos (3 x+1)$ ?

\begin{enumerate}

\Vrai{ $H:x \mapsto \frac{x \cdot \sin (3 x+1)}{3}+\frac{\cos (3 x+1)}{9}$}

\Faux{ $F:x \mapsto \cos (3 x+1)-3 x \cdot \sin (3 x+1)$}

\Faux{ $G:x \mapsto x \cdot \sin (3 x+1)$}

\end{enumerate}

\end{QCM}

\begin{QCM}

$\int\limits_{0}^{2}\left(5 t^{3}+3 t^{2}-t+7\right) d t=$

\begin{enumerate}

\Faux{ $15 t^{2}+6 t-1$}

\Vrai{ 40}

\Faux{ 71}

\end{enumerate}

\end{QCM}

\begin{QCM}

La valeur moyenne de la fonction $g: x \rightarrow e^{5 x} $ entre 0 et 2 est de :

\begin{enumerate}

\Faux{environ $22 025,5$}

\Faux{environ $4405,1$}

\Vrai{environ $2 202,5$}

\end{enumerate}

\end{QCM}

\newpage

\subsection*{Problème}

Alain Raffletou cherche à connaitre le nombre moyen de clients journaliers dans une de ses boutiques sur l'année en cours. Pour cela, il modélise le nombre de clients par une fonction donnant le nombre de clients $c\left(t\right)$ en fonction du temps $t$, exprimé en jours. On prends donc $t\in\left[0;365\right]$.

\begin{QCM}

En remarquant que le nombre de clients ne peut être négatif, déterminé, parmi les propositions ci-dessous, laquelle donne la formule de $c\left(t\right)$ en fonction de $t$ :

\begin{enumerate}

\Faux{$c\left(t\right)=3\cos\left(\frac{8\pi t}{365}-22\right)$}

\Faux{$c\left(t\right)=3\sin\left(\frac{8\pi t}{365}\right)-22$}

\Faux{$c\left(t\right)=3\sin\left(\frac{8\pi t}{365}+22\right)$}

\Vrai{$c\left(t\right)=3\cos\left(\frac{8\pi t}{365}\right)+22$}

\end{enumerate}

\end{QCM}

\begin{QCM}

On a alors que les valeurs de $c\left(t\right)$ sont comprises entre :

\begin{enumerate}

\Faux{$0$ et $22$}

\Faux{$0$ et $25$ }

\Faux{$0$ et $3$ }

\Vrai{$19$ et $25$}

\end{enumerate}

\end{QCM}

Dès lors, il peut calculer le nombre moyen de clients en appliquant la formule

$\frac{1}{365}\int\limits _{0}^{365}C\left(t\right)dt$ , où $C$ désigne une primitive de $c$.

\begin{QCM}

Parmi les propositions ci-dessous, laquelle donne une formule possible C~?

\begin{enumerate}

\Faux{$C\left(t\right)=\frac{24}{365}\pi\sin\left(\frac{8\pi t}{365}\right)-22t+17$}

\Vrai{$C\left(t\right)=\frac{1095}{2}\pi\sin\left(\frac{8\pi t}{365}\right)+22t$}

\Faux{$C\left(t\right)=-3\sin\left(\frac{8\pi t}{365}\right)$}

\Faux{$C\left(t\right)=\frac{24}{365}\pi\cos\left(\frac{8\pi t}{365}\right)-22t-35$}

\end{enumerate}

\end{QCM}

\begin{QCM}

On obtient ainsi que le nombre moyen de clients journaliers sur l'année en cours est donné par :

\begin{enumerate}

\Faux{$3\cos\left(8\pi\right)$}

\Vrai{$22$}

\Faux{$3$}

\Faux{$22\sin\left(8\pi\right)$ }

\end{enumerate}

\end{QCM}

\end{Sujet}

\begin{Sujet}

\subsection*{Exercice 1}

Pour chacune des propositions de cet exercice, cochez A si elle est vraie\par et B si elle est fausse.

On s'intéresse à la fonction f définie par $f:x\mapsto\left(3x+3\right)e^{5x}$.

\begin{VraiFaux}

\Vrai{ L'ensemble de définition de f est $\left]-\infty,+\infty\right[$.}

\end{VraiFaux}

\begin{VraiFaux}

\Faux{ La dérivée de f est donnée par $f^{\prime}:x\mapsto\left(15\right)e^{cx}$.}

\end{VraiFaux}

On admet que son approximation quadratique est donné par\par $\qquad f\left(x\right)\approx 3 + 12 x + \frac{45}{2} x^2$.

\begin{VraiFaux}

\Vrai{ L'équation de la tangente $T$ à la courbe représentative $\mathscr C$ de $f$ en son point d'abscisse $0$ est donnée par $y=12 x +3 $.}

\end{VraiFaux}

\begin{VraiFaux}

\Vrai{ $T$ est au-dessous de $\mathscr C$.}

\end{VraiFaux}

\subsection*{Exercice 2}

Cet exercice doit être rédigé au verso de la grille réponse.

\subsection*{Exercice 3}

\begin{QCM}

Parmi ces fonctions, laquelle est une primitive de la fonction $f$\par définie par $ f(x)=x \cdot \cos (3 x+1)$ ?

\begin{enumerate}

\Faux{ $G:x \mapsto x \cdot \sin (3 x+1)$}

\Faux{ $F:x \mapsto \cos (3 x+1)-3 x \cdot \sin (3 x+1)$}

\Vrai{ $H:x \mapsto \frac{x \cdot \sin (3 x+1)}{3}+\frac{\cos (3 x+1)}{9}$}

\end{enumerate}

\end{QCM}

\begin{QCM}

$\int\limits_{0}^{2}\left(5 t^{3}+3 t^{2}-t+7\right) d t=$

\begin{enumerate}

\Vrai{ 40}

\Faux{ 71}

\Faux{ $15 t^{2}+6 t-1$}

\end{enumerate}

\end{QCM}

\begin{QCM}

La valeur moyenne de la fonction $g: x \rightarrow e^{5 x} $ entre 0 et 2 est de :

\begin{enumerate}

\Vrai{environ $2 202,5$}

\Faux{environ $4405,1$}

\Faux{environ $22 025,5$}

\end{enumerate}

\end{QCM}

\newpage

\subsection*{Problème}

Alain Raffletou cherche à connaitre le nombre moyen de clients journaliers dans une de ses boutiques sur l'année en cours. Pour cela, il modélise le nombre de clients par une fonction donnant le nombre de clients $c\left(t\right)$ en fonction du temps $t$, exprimé en jours. On prends donc $t\in\left[0;365\right]$.

\begin{QCM}

En remarquant que le nombre de clients ne peut être négatif, déterminé, parmi les propositions ci-dessous, laquelle donne la formule de $c\left(t\right)$ en fonction de $t$ :

\begin{enumerate}

\Faux{$c\left(t\right)=3\sin\left(\frac{8\pi t}{365}\right)-18$}

\Faux{$c\left(t\right)=3\cos\left(\frac{8\pi t}{365}-18\right)$}

\Vrai{$c\left(t\right)=3\cos\left(\frac{8\pi t}{365}\right)+18$}

\Faux{$c\left(t\right)=3\sin\left(\frac{8\pi t}{365}+18\right)$}

\end{enumerate}

\end{QCM}

\begin{QCM}

On a alors que les valeurs de $c\left(t\right)$ sont comprises entre :

\begin{enumerate}

\Faux{$0$ et $18$}

\Vrai{$15$ et $21$}

\Faux{$0$ et $21$ }

\Faux{$0$ et $3$ }

\end{enumerate}

\end{QCM}

Dès lors, il peut calculer le nombre moyen de clients en appliquant la formule

$\frac{1}{365}\int\limits _{0}^{365}C\left(t\right)dt$ , où $C$ désigne une primitive de $c$.

\begin{QCM}

Parmi les propositions ci-dessous, laquelle donne une formule possible C~?

\begin{enumerate}

\Faux{$C\left(t\right)=\frac{24}{365}\pi\cos\left(\frac{8\pi t}{365}\right)-18t-27$}

\Faux{$C\left(t\right)=\frac{24}{365}\pi\sin\left(\frac{8\pi t}{365}\right)-18t+13$}

\Vrai{$C\left(t\right)=\frac{1095}{2}\pi\sin\left(\frac{8\pi t}{365}\right)+18t$}

\Faux{$C\left(t\right)=-3\sin\left(\frac{8\pi t}{365}\right)$}

\end{enumerate}

\end{QCM}

\begin{QCM}

On obtient ainsi que le nombre moyen de clients journaliers sur l'année en cours est donné par :

\begin{enumerate}

\Faux{$3\cos\left(8\pi\right)$}

\Faux{$18\sin\left(8\pi\right)$ }

\Vrai{$18$}

\Faux{$3$}

\end{enumerate}

\end{QCM}

\end{Sujet}

\begin{Sujet}

\subsection*{Exercice 1}

Pour chacune des propositions de cet exercice, cochez A si elle est vraie\par et B si elle est fausse.

On s'intéresse à la fonction f définie par $f:x\mapsto\left(6x+4\right)e^{5x}$.

\begin{VraiFaux}

\Vrai{ L'ensemble de définition de f est $\left]-\infty,+\infty\right[$.}

\end{VraiFaux}

\begin{VraiFaux}

\Vrai{ La dérivée de f est donnée par $f^{\prime}:x\mapsto\left(30x+26\right)e^{5x}$.}

\end{VraiFaux}

On admet que son approximation quadratique est donné par\par $\qquad f\left(x\right)\approx 4 + 30 x + 108 x^2$.

\begin{VraiFaux}

\Faux{ L'équation de la tangente $T$ à la courbe représentative $\mathscr C$ de $f$ en son point d'abscisse $0$ est donnée par $y=4x + 30 $.}

\end{VraiFaux}

\begin{VraiFaux}

\Vrai{ $T$ est au-dessous de $\mathscr C$.}

\end{VraiFaux}

\subsection*{Exercice 2}

Cet exercice doit être rédigé au verso de la grille réponse.

\subsection*{Exercice 3}

\begin{QCM}

Parmi ces fonctions, laquelle est une primitive de la fonction $f$\par définie par $ f(x)=x \cdot \cos (3 x+1)$ ?

\begin{enumerate}

\Faux{ $F:x \mapsto \cos (3 x+1)-3 x \cdot \sin (3 x+1)$}

\Faux{ $G:x \mapsto x \cdot \sin (3 x+1)$}

\Vrai{ $H:x \mapsto \frac{x \cdot \sin (3 x+1)}{3}+\frac{\cos (3 x+1)}{9}$}

\end{enumerate}

\end{QCM}

\begin{QCM}

$\int\limits_{0}^{2}\left(5 t^{3}+3 t^{2}-t+7\right) d t=$

\begin{enumerate}

\Faux{ 71}

\Faux{ $15 t^{2}+6 t-1$}

\Vrai{ 40}

\end{enumerate}

\end{QCM}

\begin{QCM}

La valeur moyenne de la fonction $g: x \rightarrow e^{5 x} $ entre 0 et 2 est de :

\begin{enumerate}

\Vrai{environ $2 202,5$}

\Faux{environ $22 025,5$}

\Faux{environ $4405,1$}

\end{enumerate}

\end{QCM}

\newpage

\subsection*{Problème}

Alain Raffletou cherche à connaitre le nombre moyen de clients journaliers dans une de ses boutiques sur l'année en cours. Pour cela, il modélise le nombre de clients par une fonction donnant le nombre de clients $c\left(t\right)$ en fonction du temps $t$, exprimé en jours. On prends donc $t\in\left[0;365\right]$.

\begin{QCM}

En remarquant que le nombre de clients ne peut être négatif, déterminé, parmi les propositions ci-dessous, laquelle donne la formule de $c\left(t\right)$ en fonction de $t$ :

\begin{enumerate}

\Faux{$c\left(t\right)=3\sin\left(\frac{8\pi t}{365}\right)-12$}

\Vrai{$c\left(t\right)=3\cos\left(\frac{8\pi t}{365}\right)+12$}

\Faux{$c\left(t\right)=3\cos\left(\frac{8\pi t}{365}-12\right)$}

\Faux{$c\left(t\right)=3\sin\left(\frac{8\pi t}{365}+12\right)$}

\end{enumerate}

\end{QCM}

\begin{QCM}

On a alors que les valeurs de $c\left(t\right)$ sont comprises entre :

\begin{enumerate}

\Faux{$0$ et $3$ }

\Faux{$0$ et $12$}

\Vrai{$9$ et $15$}

\Faux{$0$ et $15$ }

\end{enumerate}

\end{QCM}

Dès lors, il peut calculer le nombre moyen de clients en appliquant la formule

$\frac{1}{365}\int\limits _{0}^{365}C\left(t\right)dt$ , où $C$ désigne une primitive de $c$.

\begin{QCM}

Parmi les propositions ci-dessous, laquelle donne une formule possible C~?

\begin{enumerate}

\Faux{$C\left(t\right)=-3\sin\left(\frac{8\pi t}{365}\right)$}

\Faux{$C\left(t\right)=\frac{24}{365}\pi\sin\left(\frac{8\pi t}{365}\right)-12t+7$}

\Vrai{$C\left(t\right)=\frac{1095}{2}\pi\sin\left(\frac{8\pi t}{365}\right)+12t$}

\Faux{$C\left(t\right)=\frac{24}{365}\pi\cos\left(\frac{8\pi t}{365}\right)-12t-15$}

\end{enumerate}

\end{QCM}

\begin{QCM}

On obtient ainsi que le nombre moyen de clients journaliers sur l'année en cours est donné par :

\begin{enumerate}

\Faux{$3\cos\left(8\pi\right)$}

\Vrai{$12$}

\Faux{$3$}

\Faux{$12\sin\left(8\pi\right)$ }

\end{enumerate}

\end{QCM}

\end{Sujet}

\begin{Sujet}

\subsection*{Exercice 1}

Pour chacune des propositions de cet exercice, cochez A si elle est vraie\par et B si elle est fausse.

On s'intéresse à la fonction f définie par $f:x\mapsto\left(2x+2\right)e^{4x}$.

\begin{VraiFaux}

\Faux{ L'ensemble de définition de f est $\left]0,+\infty\right[$.}

\end{VraiFaux}

\begin{VraiFaux}

\Vrai{ La dérivée de f est donnée par $f^{\prime}:x\mapsto\left(8x+10\right)e^{4x}$.}

\end{VraiFaux}

On admet que son approximation quadratique est donné par\par $\qquad f\left(x\right)\approx 2 + 6 x + 8 x^2$.

\begin{VraiFaux}

\Vrai{ L'équation de la tangente $T$ à la courbe représentative $\mathscr C$ de $f$ en son point d'abscisse $0$ est donnée par $y=6 x +2 $.}

\end{VraiFaux}

\begin{VraiFaux}

\Vrai{ $T$ est au-dessous de $\mathscr C$.}

\end{VraiFaux}

\subsection*{Exercice 2}

Cet exercice doit être rédigé au verso de la grille réponse.

\subsection*{Exercice 3}

\begin{QCM}

Parmi ces fonctions, laquelle est une primitive de la fonction $f$\par définie par $ f(x)=x \cdot \cos (3 x+1)$ ?

\begin{enumerate}

\Faux{ $G:x \mapsto x \cdot \sin (3 x+1)$}

\Faux{ $F:x \mapsto \cos (3 x+1)-3 x \cdot \sin (3 x+1)$}

\Vrai{ $H:x \mapsto \frac{x \cdot \sin (3 x+1)}{3}+\frac{\cos (3 x+1)}{9}$}

\end{enumerate}

\end{QCM}

\begin{QCM}

$\int\limits_{0}^{2}\left(5 t^{3}+3 t^{2}-t+7\right) d t=$

\begin{enumerate}

\Faux{ $15 t^{2}+6 t-1$}

\Faux{ 71}

\Vrai{ 40}

\end{enumerate}

\end{QCM}

\begin{QCM}

La valeur moyenne de la fonction $g: x \rightarrow e^{5 x} $ entre 0 et 2 est de :

\begin{enumerate}

\Vrai{environ $2 202,5$}

\Faux{environ $4405,1$}

\Faux{environ $22 025,5$}

\end{enumerate}

\end{QCM}

\newpage

\subsection*{Problème}

Alain Raffletou cherche à connaitre le nombre moyen de clients journaliers dans une de ses boutiques sur l'année en cours. Pour cela, il modélise le nombre de clients par une fonction donnant le nombre de clients $c\left(t\right)$ en fonction du temps $t$, exprimé en jours. On prends donc $t\in\left[0;365\right]$.

\begin{QCM}

En remarquant que le nombre de clients ne peut être négatif, déterminé, parmi les propositions ci-dessous, laquelle donne la formule de $c\left(t\right)$ en fonction de $t$ :

\begin{enumerate}

\Faux{$c\left(t\right)=5\cos\left(\frac{6\pi t}{365}-28\right)$}

\Faux{$c\left(t\right)=5\sin\left(\frac{6\pi t}{365}+28\right)$}

\Faux{$c\left(t\right)=5\sin\left(\frac{6\pi t}{365}\right)-28$}

\Vrai{$c\left(t\right)=5\cos\left(\frac{6\pi t}{365}\right)+28$}

\end{enumerate}

\end{QCM}

\begin{QCM}

On a alors que les valeurs de $c\left(t\right)$ sont comprises entre :

\begin{enumerate}

\Faux{$0$ et $5$ }

\Faux{$0$ et $33$ }

\Vrai{$23$ et $33$}

\Faux{$0$ et $28$}

\end{enumerate}

\end{QCM}

Dès lors, il peut calculer le nombre moyen de clients en appliquant la formule

$\frac{1}{365}\int\limits _{0}^{365}C\left(t\right)dt$ , où $C$ désigne une primitive de $c$.

\begin{QCM}

Parmi les propositions ci-dessous, laquelle donne une formule possible C~?

\begin{enumerate}

\Faux{$C\left(t\right)=\frac{30}{365}\pi\cos\left(\frac{6\pi t}{365}\right)-28t-41$}

\Vrai{$C\left(t\right)=\frac{1825}{2}\pi\sin\left(\frac{6\pi t}{365}\right)+28t$}

\Faux{$C\left(t\right)=\frac{30}{365}\pi\sin\left(\frac{6\pi t}{365}\right)-28t+19$}

\Faux{$C\left(t\right)=-5\sin\left(\frac{6\pi t}{365}\right)$}

\end{enumerate}

\end{QCM}

\begin{QCM}

On obtient ainsi que le nombre moyen de clients journaliers sur l'année en cours est donné par :

\begin{enumerate}

\Faux{$28\sin\left(6\pi\right)$ }

\Vrai{$28$}

\Faux{$5\cos\left(6\pi\right)$}

\Faux{$5$}

\end{enumerate}

\end{QCM}

\end{Sujet}

\begin{Sujet}

\subsection*{Exercice 1}

Pour chacune des propositions de cet exercice, cochez A si elle est vraie\par et B si elle est fausse.

On s'intéresse à la fonction f définie par $f:x\mapsto\left(2x+2\right)e^{5x}$.

\begin{VraiFaux}

\Faux{ L'ensemble de définition de f est $\left]0,+\infty\right[$.}

\end{VraiFaux}

\begin{VraiFaux}

\Faux{ La dérivée de f est donnée par $f^{\prime}:x\mapsto\left(10\right)e^{cx}$.}

\end{VraiFaux}

On admet que son approximation quadratique est donné par\par $\qquad f\left(x\right)\approx 2 + 6 x + 8 x^2$.

\begin{VraiFaux}

\Faux{ L'équation de la tangente $T$ à la courbe représentative $\mathscr C$ de $f$ en son point d'abscisse $0$ est donnée par $y=2x + 6 $.}

\end{VraiFaux}

\begin{VraiFaux}

\Vrai{ $T$ est au-dessous de $\mathscr C$.}

\end{VraiFaux}

\subsection*{Exercice 2}

Cet exercice doit être rédigé au verso de la grille réponse.

\subsection*{Exercice 3}

\begin{QCM}

Parmi ces fonctions, laquelle est une primitive de la fonction $f$\par définie par $ f(x)=x \cdot \cos (3 x+1)$ ?

\begin{enumerate}

\Vrai{ $H:x \mapsto \frac{x \cdot \sin (3 x+1)}{3}+\frac{\cos (3 x+1)}{9}$}

\Faux{ $F:x \mapsto \cos (3 x+1)-3 x \cdot \sin (3 x+1)$}

\Faux{ $G:x \mapsto x \cdot \sin (3 x+1)$}

\end{enumerate}

\end{QCM}

\begin{QCM}

$\int\limits_{0}^{2}\left(5 t^{3}+3 t^{2}-t+7\right) d t=$

\begin{enumerate}

\Faux{ 71}

\Vrai{ 40}

\Faux{ $15 t^{2}+6 t-1$}

\end{enumerate}

\end{QCM}

\begin{QCM}

La valeur moyenne de la fonction $g: x \rightarrow e^{5 x} $ entre 0 et 2 est de :

\begin{enumerate}

\Faux{environ $4405,1$}

\Faux{environ $22 025,5$}

\Vrai{environ $2 202,5$}

\end{enumerate}

\end{QCM}

\newpage

\subsection*{Problème}

Alain Raffletou cherche à connaitre le nombre moyen de clients journaliers dans une de ses boutiques sur l'année en cours. Pour cela, il modélise le nombre de clients par une fonction donnant le nombre de clients $c\left(t\right)$ en fonction du temps $t$, exprimé en jours. On prends donc $t\in\left[0;365\right]$.

\begin{QCM}

En remarquant que le nombre de clients ne peut être négatif, déterminé, parmi les propositions ci-dessous, laquelle donne la formule de $c\left(t\right)$ en fonction de $t$ :

\begin{enumerate}

\Faux{$c\left(t\right)=4\sin\left(\frac{4\pi t}{365}+10\right)$}

\Vrai{$c\left(t\right)=4\cos\left(\frac{4\pi t}{365}\right)+10$}

\Faux{$c\left(t\right)=4\sin\left(\frac{4\pi t}{365}\right)-10$}

\Faux{$c\left(t\right)=4\cos\left(\frac{4\pi t}{365}-10\right)$}

\end{enumerate}

\end{QCM}

\begin{QCM}

On a alors que les valeurs de $c\left(t\right)$ sont comprises entre :

\begin{enumerate}

\Faux{$0$ et $4$ }

\Faux{$0$ et $14$ }

\Faux{$0$ et $10$}

\Vrai{$6$ et $14$}

\end{enumerate}

\end{QCM}

Dès lors, il peut calculer le nombre moyen de clients en appliquant la formule

$\frac{1}{365}\int\limits _{0}^{365}C\left(t\right)dt$ , où $C$ désigne une primitive de $c$.

\begin{QCM}

Parmi les propositions ci-dessous, laquelle donne une formule possible C~?

\begin{enumerate}

\Vrai{$C\left(t\right)=\frac{1460}{2}\pi\sin\left(\frac{4\pi t}{365}\right)+10t$}

\Faux{$C\left(t\right)=\frac{16}{365}\pi\cos\left(\frac{4\pi t}{365}\right)-10t-8$}

\Faux{$C\left(t\right)=\frac{16}{365}\pi\sin\left(\frac{4\pi t}{365}\right)-10t+3$}

\Faux{$C\left(t\right)=-4\sin\left(\frac{4\pi t}{365}\right)$}

\end{enumerate}

\end{QCM}

\begin{QCM}

On obtient ainsi que le nombre moyen de clients journaliers sur l'année en cours est donné par :

\begin{enumerate}

\Faux{$4$}

\Faux{$10\sin\left(4\pi\right)$ }

\Faux{$4\cos\left(4\pi\right)$}

\Vrai{$10$}

\end{enumerate}

\end{QCM}

\end{Sujet}

\begin{Sujet}

\subsection*{Exercice 1}

Pour chacune des propositions de cet exercice, cochez A si elle est vraie\par et B si elle est fausse.

On s'intéresse à la fonction f définie par $f:x\mapsto\left(6x+6\right)e^{7x}$.

\begin{VraiFaux}

\Faux{ L'ensemble de définition de f est $\left]0,+\infty\right[$.}

\end{VraiFaux}

\begin{VraiFaux}

\Faux{ La dérivée de f est donnée par $f^{\prime}:x\mapsto\left(42\right)e^{cx}$.}

\end{VraiFaux}

On admet que son approximation quadratique est donné par\par $\qquad f\left(x\right)\approx 6 + 42 x + 144 x^2$.

\begin{VraiFaux}

\Vrai{ L'équation de la tangente $T$ à la courbe représentative $\mathscr C$ de $f$ en son point d'abscisse $0$ est donnée par $y=42 x +6 $.}

\end{VraiFaux}

\begin{VraiFaux}

\Vrai{ $T$ est au-dessous de $\mathscr C$.}

\end{VraiFaux}

\subsection*{Exercice 2}

Cet exercice doit être rédigé au verso de la grille réponse.

\subsection*{Exercice 3}

\begin{QCM}

Parmi ces fonctions, laquelle est une primitive de la fonction $f$\par définie par $ f(x)=x \cdot \cos (3 x+1)$ ?

\begin{enumerate}

\Faux{ $G:x \mapsto x \cdot \sin (3 x+1)$}

\Vrai{ $H:x \mapsto \frac{x \cdot \sin (3 x+1)}{3}+\frac{\cos (3 x+1)}{9}$}

\Faux{ $F:x \mapsto \cos (3 x+1)-3 x \cdot \sin (3 x+1)$}

\end{enumerate}

\end{QCM}

\begin{QCM}

$\int\limits_{0}^{2}\left(5 t^{3}+3 t^{2}-t+7\right) d t=$

\begin{enumerate}

\Faux{ $15 t^{2}+6 t-1$}

\Vrai{ 40}

\Faux{ 71}

\end{enumerate}

\end{QCM}

\begin{QCM}

La valeur moyenne de la fonction $g: x \rightarrow e^{5 x} $ entre 0 et 2 est de :

\begin{enumerate}

\Vrai{environ $2 202,5$}

\Faux{environ $4405,1$}

\Faux{environ $22 025,5$}

\end{enumerate}

\end{QCM}

\newpage

\subsection*{Problème}

Alain Raffletou cherche à connaitre le nombre moyen de clients journaliers dans une de ses boutiques sur l'année en cours. Pour cela, il modélise le nombre de clients par une fonction donnant le nombre de clients $c\left(t\right)$ en fonction du temps $t$, exprimé en jours. On prends donc $t\in\left[0;365\right]$.

\begin{QCM}

En remarquant que le nombre de clients ne peut être négatif, déterminé, parmi les propositions ci-dessous, laquelle donne la formule de $c\left(t\right)$ en fonction de $t$ :

\begin{enumerate}

\Faux{$c\left(t\right)=3\cos\left(\frac{6\pi t}{365}-20\right)$}

\Faux{$c\left(t\right)=3\sin\left(\frac{6\pi t}{365}\right)-20$}

\Faux{$c\left(t\right)=3\sin\left(\frac{6\pi t}{365}+20\right)$}

\Vrai{$c\left(t\right)=3\cos\left(\frac{6\pi t}{365}\right)+20$}

\end{enumerate}

\end{QCM}

\begin{QCM}

On a alors que les valeurs de $c\left(t\right)$ sont comprises entre :

\begin{enumerate}

\Faux{$0$ et $3$ }

\Faux{$0$ et $20$}

\Vrai{$17$ et $23$}

\Faux{$0$ et $23$ }

\end{enumerate}

\end{QCM}

Dès lors, il peut calculer le nombre moyen de clients en appliquant la formule

$\frac{1}{365}\int\limits _{0}^{365}C\left(t\right)dt$ , où $C$ désigne une primitive de $c$.

\begin{QCM}

Parmi les propositions ci-dessous, laquelle donne une formule possible C~?

\begin{enumerate}

\Faux{$C\left(t\right)=\frac{18}{365}\pi\sin\left(\frac{6\pi t}{365}\right)-20t+15$}

\Faux{$C\left(t\right)=\frac{18}{365}\pi\cos\left(\frac{6\pi t}{365}\right)-20t-31$}

\Vrai{$C\left(t\right)=\frac{1095}{2}\pi\sin\left(\frac{6\pi t}{365}\right)+20t$}

\Faux{$C\left(t\right)=-3\sin\left(\frac{6\pi t}{365}\right)$}

\end{enumerate}

\end{QCM}

\begin{QCM}

On obtient ainsi que le nombre moyen de clients journaliers sur l'année en cours est donné par :

\begin{enumerate}

\Faux{$20\sin\left(6\pi\right)$ }

\Vrai{$20$}

\Faux{$3\cos\left(6\pi\right)$}

\Faux{$3$}

\end{enumerate}

\end{QCM}

\end{Sujet}

\begin{Sujet}

\subsection*{Exercice 1}

Pour chacune des propositions de cet exercice, cochez A si elle est vraie\par et B si elle est fausse.

On s'intéresse à la fonction f définie par $f:x\mapsto\left(3x+3\right)e^{7x}$.

\begin{VraiFaux}

\Vrai{ L'ensemble de définition de f est $\left]-\infty,+\infty\right[$.}

\end{VraiFaux}

\begin{VraiFaux}

\Faux{ La dérivée de f est donnée par $f^{\prime}:x\mapsto\left(21\right)e^{cx}$.}

\end{VraiFaux}

On admet que son approximation quadratique est donné par\par $\qquad f\left(x\right)\approx 3 + 12 x + \frac{45}{2} x^2$.

\begin{VraiFaux}

\Vrai{ L'équation de la tangente $T$ à la courbe représentative $\mathscr C$ de $f$ en son point d'abscisse $0$ est donnée par $y=12 x +3 $.}

\end{VraiFaux}

\begin{VraiFaux}

\Faux{ $T$ est au-dessus de $\mathscr C$.}

\end{VraiFaux}

\subsection*{Exercice 2}

Cet exercice doit être rédigé au verso de la grille réponse.

\subsection*{Exercice 3}

\begin{QCM}

Parmi ces fonctions, laquelle est une primitive de la fonction $f$\par définie par $ f(x)=x \cdot \cos (3 x+1)$ ?

\begin{enumerate}

\Faux{ $F:x \mapsto \cos (3 x+1)-3 x \cdot \sin (3 x+1)$}

\Faux{ $G:x \mapsto x \cdot \sin (3 x+1)$}

\Vrai{ $H:x \mapsto \frac{x \cdot \sin (3 x+1)}{3}+\frac{\cos (3 x+1)}{9}$}

\end{enumerate}

\end{QCM}

\begin{QCM}

$\int\limits_{0}^{2}\left(5 t^{3}+3 t^{2}-t+7\right) d t=$

\begin{enumerate}

\Vrai{ 40}

\Faux{ 71}

\Faux{ $15 t^{2}+6 t-1$}

\end{enumerate}

\end{QCM}

\begin{QCM}

La valeur moyenne de la fonction $g: x \rightarrow e^{5 x} $ entre 0 et 2 est de :

\begin{enumerate}

\Faux{environ $4405,1$}

\Vrai{environ $2 202,5$}

\Faux{environ $22 025,5$}

\end{enumerate}

\end{QCM}

\newpage

\subsection*{Problème}

Alain Raffletou cherche à connaitre le nombre moyen de clients journaliers dans une de ses boutiques sur l'année en cours. Pour cela, il modélise le nombre de clients par une fonction donnant le nombre de clients $c\left(t\right)$ en fonction du temps $t$, exprimé en jours. On prends donc $t\in\left[0;365\right]$.

\begin{QCM}

En remarquant que le nombre de clients ne peut être négatif, déterminé, parmi les propositions ci-dessous, laquelle donne la formule de $c\left(t\right)$ en fonction de $t$ :

\begin{enumerate}

\Faux{$c\left(t\right)=4\cos\left(\frac{6\pi t}{365}-12\right)$}

\Vrai{$c\left(t\right)=4\cos\left(\frac{6\pi t}{365}\right)+12$}

\Faux{$c\left(t\right)=4\sin\left(\frac{6\pi t}{365}+12\right)$}

\Faux{$c\left(t\right)=4\sin\left(\frac{6\pi t}{365}\right)-12$}

\end{enumerate}

\end{QCM}

\begin{QCM}

On a alors que les valeurs de $c\left(t\right)$ sont comprises entre :

\begin{enumerate}

\Faux{$0$ et $4$ }

\Faux{$0$ et $12$}

\Vrai{$8$ et $16$}

\Faux{$0$ et $16$ }

\end{enumerate}

\end{QCM}

Dès lors, il peut calculer le nombre moyen de clients en appliquant la formule

$\frac{1}{365}\int\limits _{0}^{365}C\left(t\right)dt$ , où $C$ désigne une primitive de $c$.

\begin{QCM}

Parmi les propositions ci-dessous, laquelle donne une formule possible C~?

\begin{enumerate}

\Faux{$C\left(t\right)=\frac{24}{365}\pi\sin\left(\frac{6\pi t}{365}\right)-12t+5$}

\Faux{$C\left(t\right)=\frac{24}{365}\pi\cos\left(\frac{6\pi t}{365}\right)-12t-12$}

\Faux{$C\left(t\right)=-4\sin\left(\frac{6\pi t}{365}\right)$}

\Vrai{$C\left(t\right)=\frac{1460}{2}\pi\sin\left(\frac{6\pi t}{365}\right)+12t$}

\end{enumerate}

\end{QCM}

\begin{QCM}

On obtient ainsi que le nombre moyen de clients journaliers sur l'année en cours est donné par :

\begin{enumerate}

\Faux{$4$}

\Faux{$12\sin\left(6\pi\right)$ }

\Vrai{$12$}

\Faux{$4\cos\left(6\pi\right)$}

\end{enumerate}

\end{QCM}

\end{Sujet}

\begin{Sujet}

\subsection*{Exercice 1}

Pour chacune des propositions de cet exercice, cochez A si elle est vraie\par et B si elle est fausse.

On s'intéresse à la fonction f définie par $f:x\mapsto\left(4x+3\right)e^{3x}$.

\begin{VraiFaux}

\Faux{ L'ensemble de définition de f est $\left]0,+\infty\right[$.}

\end{VraiFaux}

\begin{VraiFaux}

\Vrai{ La dérivée de f est donnée par $f^{\prime}:x\mapsto\left(12x+13\right)e^{3x}$.}

\end{VraiFaux}

On admet que son approximation quadratique est donné par\par $\qquad f\left(x\right)\approx 3 + 16 x + 40 x^2$.

\begin{VraiFaux}

\Vrai{ L'équation de la tangente $T$ à la courbe représentative $\mathscr C$ de $f$ en son point d'abscisse $0$ est donnée par $y=16 x +3 $.}

\end{VraiFaux}

\begin{VraiFaux}

\Faux{ $T$ est au-dessus de $\mathscr C$.}

\end{VraiFaux}

\subsection*{Exercice 2}

Cet exercice doit être rédigé au verso de la grille réponse.

\subsection*{Exercice 3}

\begin{QCM}

Parmi ces fonctions, laquelle est une primitive de la fonction $f$\par définie par $ f(x)=x \cdot \cos (3 x+1)$ ?

\begin{enumerate}

\Faux{ $G:x \mapsto x \cdot \sin (3 x+1)$}

\Faux{ $F:x \mapsto \cos (3 x+1)-3 x \cdot \sin (3 x+1)$}

\Vrai{ $H:x \mapsto \frac{x \cdot \sin (3 x+1)}{3}+\frac{\cos (3 x+1)}{9}$}

\end{enumerate}

\end{QCM}

\begin{QCM}

$\int\limits_{0}^{2}\left(5 t^{3}+3 t^{2}-t+7\right) d t=$

\begin{enumerate}

\Vrai{ 40}

\Faux{ 71}

\Faux{ $15 t^{2}+6 t-1$}

\end{enumerate}

\end{QCM}

\begin{QCM}

La valeur moyenne de la fonction $g: x \rightarrow e^{5 x} $ entre 0 et 2 est de :

\begin{enumerate}

\Faux{environ $22 025,5$}

\Faux{environ $4405,1$}

\Vrai{environ $2 202,5$}

\end{enumerate}

\end{QCM}

\newpage

\subsection*{Problème}

Alain Raffletou cherche à connaitre le nombre moyen de clients journaliers dans une de ses boutiques sur l'année en cours. Pour cela, il modélise le nombre de clients par une fonction donnant le nombre de clients $c\left(t\right)$ en fonction du temps $t$, exprimé en jours. On prends donc $t\in\left[0;365\right]$.

\begin{QCM}

En remarquant que le nombre de clients ne peut être négatif, déterminé, parmi les propositions ci-dessous, laquelle donne la formule de $c\left(t\right)$ en fonction de $t$ :

\begin{enumerate}

\Vrai{$c\left(t\right)=3\cos\left(\frac{8\pi t}{365}\right)+18$}

\Faux{$c\left(t\right)=3\cos\left(\frac{8\pi t}{365}-18\right)$}

\Faux{$c\left(t\right)=3\sin\left(\frac{8\pi t}{365}\right)-18$}

\Faux{$c\left(t\right)=3\sin\left(\frac{8\pi t}{365}+18\right)$}

\end{enumerate}

\end{QCM}

\begin{QCM}

On a alors que les valeurs de $c\left(t\right)$ sont comprises entre :

\begin{enumerate}

\Faux{$0$ et $18$}

\Faux{$0$ et $21$ }

\Vrai{$15$ et $21$}

\Faux{$0$ et $3$ }

\end{enumerate}

\end{QCM}

Dès lors, il peut calculer le nombre moyen de clients en appliquant la formule

$\frac{1}{365}\int\limits _{0}^{365}C\left(t\right)dt$ , où $C$ désigne une primitive de $c$.

\begin{QCM}

Parmi les propositions ci-dessous, laquelle donne une formule possible C~?

\begin{enumerate}

\Faux{$C\left(t\right)=\frac{24}{365}\pi\cos\left(\frac{8\pi t}{365}\right)-18t-27$}

\Faux{$C\left(t\right)=\frac{24}{365}\pi\sin\left(\frac{8\pi t}{365}\right)-18t+13$}

\Vrai{$C\left(t\right)=\frac{1095}{2}\pi\sin\left(\frac{8\pi t}{365}\right)+18t$}

\Faux{$C\left(t\right)=-3\sin\left(\frac{8\pi t}{365}\right)$}

\end{enumerate}

\end{QCM}

\begin{QCM}

On obtient ainsi que le nombre moyen de clients journaliers sur l'année en cours est donné par :

\begin{enumerate}

\Faux{$3\cos\left(8\pi\right)$}

\Vrai{$18$}

\Faux{$3$}

\Faux{$18\sin\left(8\pi\right)$ }

\end{enumerate}

\end{QCM}

\end{Sujet}

\begin{Sujet}

\subsection*{Exercice 1}

Pour chacune des propositions de cet exercice, cochez A si elle est vraie\par et B si elle est fausse.

On s'intéresse à la fonction f définie par $f:x\mapsto\left(5x+7\right)e^{7x}$.

\begin{VraiFaux}

\Faux{ L'ensemble de définition de f est $\left]0,+\infty\right[$.}

\end{VraiFaux}

\begin{VraiFaux}

\Faux{ La dérivée de f est donnée par $f^{\prime}:x\mapsto\left(35\right)e^{cx}$.}

\end{VraiFaux}

On admet que son approximation quadratique est donné par\par $\qquad f\left(x\right)\approx 7 + 40 x + \frac{225}{2} x^2$.

\begin{VraiFaux}

\Faux{ L'équation de la tangente $T$ à la courbe représentative $\mathscr C$ de $f$ en son point d'abscisse $0$ est donnée par $y=7x + 40 $.}

\end{VraiFaux}

\begin{VraiFaux}

\Vrai{ $T$ est au-dessous de $\mathscr C$.}

\end{VraiFaux}

\subsection*{Exercice 2}

Cet exercice doit être rédigé au verso de la grille réponse.

\subsection*{Exercice 3}

\begin{QCM}

Parmi ces fonctions, laquelle est une primitive de la fonction $f$\par définie par $ f(x)=x \cdot \cos (3 x+1)$ ?

\begin{enumerate}

\Faux{ $F:x \mapsto \cos (3 x+1)-3 x \cdot \sin (3 x+1)$}

\Faux{ $G:x \mapsto x \cdot \sin (3 x+1)$}

\Vrai{ $H:x \mapsto \frac{x \cdot \sin (3 x+1)}{3}+\frac{\cos (3 x+1)}{9}$}

\end{enumerate}

\end{QCM}

\begin{QCM}

$\int\limits_{0}^{2}\left(5 t^{3}+3 t^{2}-t+7\right) d t=$

\begin{enumerate}

\Vrai{ 40}

\Faux{ $15 t^{2}+6 t-1$}

\Faux{ 71}

\end{enumerate}

\end{QCM}

\begin{QCM}

La valeur moyenne de la fonction $g: x \rightarrow e^{5 x} $ entre 0 et 2 est de :

\begin{enumerate}

\Faux{environ $4405,1$}

\Vrai{environ $2 202,5$}

\Faux{environ $22 025,5$}

\end{enumerate}

\end{QCM}

\newpage

\subsection*{Problème}

Alain Raffletou cherche à connaitre le nombre moyen de clients journaliers dans une de ses boutiques sur l'année en cours. Pour cela, il modélise le nombre de clients par une fonction donnant le nombre de clients $c\left(t\right)$ en fonction du temps $t$, exprimé en jours. On prends donc $t\in\left[0;365\right]$.

\begin{QCM}

En remarquant que le nombre de clients ne peut être négatif, déterminé, parmi les propositions ci-dessous, laquelle donne la formule de $c\left(t\right)$ en fonction de $t$ :

\begin{enumerate}

\Faux{$c\left(t\right)=4\sin\left(\frac{4\pi t}{365}+18\right)$}

\Faux{$c\left(t\right)=4\sin\left(\frac{4\pi t}{365}\right)-18$}

\Faux{$c\left(t\right)=4\cos\left(\frac{4\pi t}{365}-18\right)$}

\Vrai{$c\left(t\right)=4\cos\left(\frac{4\pi t}{365}\right)+18$}

\end{enumerate}

\end{QCM}

\begin{QCM}

On a alors que les valeurs de $c\left(t\right)$ sont comprises entre :

\begin{enumerate}

\Faux{$0$ et $4$ }

\Faux{$0$ et $18$}

\Vrai{$14$ et $22$}

\Faux{$0$ et $22$ }

\end{enumerate}

\end{QCM}

Dès lors, il peut calculer le nombre moyen de clients en appliquant la formule

$\frac{1}{365}\int\limits _{0}^{365}C\left(t\right)dt$ , où $C$ désigne une primitive de $c$.

\begin{QCM}

Parmi les propositions ci-dessous, laquelle donne une formule possible C~?

\begin{enumerate}

\Faux{$C\left(t\right)=\frac{16}{365}\pi\sin\left(\frac{4\pi t}{365}\right)-18t+11$}

\Faux{$C\left(t\right)=-4\sin\left(\frac{4\pi t}{365}\right)$}

\Faux{$C\left(t\right)=\frac{16}{365}\pi\cos\left(\frac{4\pi t}{365}\right)-18t-24$}

\Vrai{$C\left(t\right)=\frac{1460}{2}\pi\sin\left(\frac{4\pi t}{365}\right)+18t$}

\end{enumerate}

\end{QCM}

\begin{QCM}

On obtient ainsi que le nombre moyen de clients journaliers sur l'année en cours est donné par :

\begin{enumerate}

\Faux{$4\cos\left(4\pi\right)$}

\Faux{$4$}

\Faux{$18\sin\left(4\pi\right)$ }

\Vrai{$18$}

\end{enumerate}

\end{QCM}

\end{Sujet}

\begin{Sujet}

\subsection*{Exercice 1}

Pour chacune des propositions de cet exercice, cochez A si elle est vraie\par et B si elle est fausse.

On s'intéresse à la fonction f définie par $f:x\mapsto\left(2x+4\right)e^{6x}$.

\begin{VraiFaux}

\Faux{ L'ensemble de définition de f est $\left]0,+\infty\right[$.}

\end{VraiFaux}

\begin{VraiFaux}

\Faux{ La dérivée de f est donnée par $f^{\prime}:x\mapsto\left(12\right)e^{cx}$.}

\end{VraiFaux}

On admet que son approximation quadratique est donné par\par $\qquad f\left(x\right)\approx 4 + 10 x + 12 x^2$.

\begin{VraiFaux}

\Faux{ L'équation de la tangente $T$ à la courbe représentative $\mathscr C$ de $f$ en son point d'abscisse $0$ est donnée par $y=4x + 10 $.}

\end{VraiFaux}

\begin{VraiFaux}

\Faux{ $T$ est au-dessus de $\mathscr C$.}

\end{VraiFaux}

\subsection*{Exercice 2}

Cet exercice doit être rédigé au verso de la grille réponse.

\subsection*{Exercice 3}

\begin{QCM}

Parmi ces fonctions, laquelle est une primitive de la fonction $f$\par définie par $ f(x)=x \cdot \cos (3 x+1)$ ?

\begin{enumerate}

\Vrai{ $H:x \mapsto \frac{x \cdot \sin (3 x+1)}{3}+\frac{\cos (3 x+1)}{9}$}

\Faux{ $G:x \mapsto x \cdot \sin (3 x+1)$}

\Faux{ $F:x \mapsto \cos (3 x+1)-3 x \cdot \sin (3 x+1)$}

\end{enumerate}

\end{QCM}

\begin{QCM}

$\int\limits_{0}^{2}\left(5 t^{3}+3 t^{2}-t+7\right) d t=$

\begin{enumerate}

\Vrai{ 40}

\Faux{ $15 t^{2}+6 t-1$}

\Faux{ 71}

\end{enumerate}

\end{QCM}

\begin{QCM}

La valeur moyenne de la fonction $g: x \rightarrow e^{5 x} $ entre 0 et 2 est de :

\begin{enumerate}

\Faux{environ $4405,1$}

\Faux{environ $22 025,5$}

\Vrai{environ $2 202,5$}

\end{enumerate}

\end{QCM}

\newpage

\subsection*{Problème}

Alain Raffletou cherche à connaitre le nombre moyen de clients journaliers dans une de ses boutiques sur l'année en cours. Pour cela, il modélise le nombre de clients par une fonction donnant le nombre de clients $c\left(t\right)$ en fonction du temps $t$, exprimé en jours. On prends donc $t\in\left[0;365\right]$.

\begin{QCM}

En remarquant que le nombre de clients ne peut être négatif, déterminé, parmi les propositions ci-dessous, laquelle donne la formule de $c\left(t\right)$ en fonction de $t$ :

\begin{enumerate}

\Vrai{$c\left(t\right)=5\cos\left(\frac{6\pi t}{365}\right)+17$}

\Faux{$c\left(t\right)=5\sin\left(\frac{6\pi t}{365}+17\right)$}

\Faux{$c\left(t\right)=5\cos\left(\frac{6\pi t}{365}-17\right)$}

\Faux{$c\left(t\right)=5\sin\left(\frac{6\pi t}{365}\right)-17$}

\end{enumerate}

\end{QCM}

\begin{QCM}

On a alors que les valeurs de $c\left(t\right)$ sont comprises entre :

\begin{enumerate}

\Faux{$0$ et $5$ }

\Faux{$0$ et $17$}

\Faux{$0$ et $22$ }

\Vrai{$12$ et $22$}

\end{enumerate}

\end{QCM}

Dès lors, il peut calculer le nombre moyen de clients en appliquant la formule

$\frac{1}{365}\int\limits _{0}^{365}C\left(t\right)dt$ , où $C$ désigne une primitive de $c$.

\begin{QCM}

Parmi les propositions ci-dessous, laquelle donne une formule possible C~?

\begin{enumerate}

\Vrai{$C\left(t\right)=\frac{1825}{2}\pi\sin\left(\frac{6\pi t}{365}\right)+17t$}

\Faux{$C\left(t\right)=\frac{30}{365}\pi\sin\left(\frac{6\pi t}{365}\right)-17t+8$}

\Faux{$C\left(t\right)=\frac{30}{365}\pi\cos\left(\frac{6\pi t}{365}\right)-17t-19$}

\Faux{$C\left(t\right)=-5\sin\left(\frac{6\pi t}{365}\right)$}

\end{enumerate}

\end{QCM}

\begin{QCM}

On obtient ainsi que le nombre moyen de clients journaliers sur l'année en cours est donné par :

\begin{enumerate}

\Faux{$5\cos\left(6\pi\right)$}

\Faux{$17\sin\left(6\pi\right)$ }

\Faux{$5$}

\Vrai{$17$}

\end{enumerate}

\end{QCM}

\end{Sujet}

\begin{Sujet}

\subsection*{Exercice 1}

Pour chacune des propositions de cet exercice, cochez A si elle est vraie\par et B si elle est fausse.

On s'intéresse à la fonction f définie par $f:x\mapsto\left(4x+2\right)e^{2x}$.

\begin{VraiFaux}

\Vrai{ L'ensemble de définition de f est $\left]-\infty,+\infty\right[$.}

\end{VraiFaux}

\begin{VraiFaux}

\Vrai{ La dérivée de f est donnée par $f^{\prime}:x\mapsto\left(8x+8\right)e^{2x}$.}

\end{VraiFaux}

On admet que son approximation quadratique est donné par\par $\qquad f\left(x\right)\approx 2 + 12 x + 32 x^2$.

\begin{VraiFaux}

\Vrai{ L'équation de la tangente $T$ à la courbe représentative $\mathscr C$ de $f$ en son point d'abscisse $0$ est donnée par $y=12 x +2 $.}

\end{VraiFaux}

\begin{VraiFaux}

\Faux{ $T$ est au-dessus de $\mathscr C$.}

\end{VraiFaux}

\subsection*{Exercice 2}

Cet exercice doit être rédigé au verso de la grille réponse.

\subsection*{Exercice 3}

\begin{QCM}

Parmi ces fonctions, laquelle est une primitive de la fonction $f$\par définie par $ f(x)=x \cdot \cos (3 x+1)$ ?

\begin{enumerate}

\Vrai{ $H:x \mapsto \frac{x \cdot \sin (3 x+1)}{3}+\frac{\cos (3 x+1)}{9}$}

\Faux{ $G:x \mapsto x \cdot \sin (3 x+1)$}

\Faux{ $F:x \mapsto \cos (3 x+1)-3 x \cdot \sin (3 x+1)$}

\end{enumerate}

\end{QCM}

\begin{QCM}

$\int\limits_{0}^{2}\left(5 t^{3}+3 t^{2}-t+7\right) d t=$

\begin{enumerate}

\Faux{ $15 t^{2}+6 t-1$}

\Vrai{ 40}

\Faux{ 71}

\end{enumerate}

\end{QCM}

\begin{QCM}

La valeur moyenne de la fonction $g: x \rightarrow e^{5 x} $ entre 0 et 2 est de :

\begin{enumerate}

\Vrai{environ $2 202,5$}

\Faux{environ $4405,1$}

\Faux{environ $22 025,5$}

\end{enumerate}

\end{QCM}

\newpage

\subsection*{Problème}

Alain Raffletou cherche à connaitre le nombre moyen de clients journaliers dans une de ses boutiques sur l'année en cours. Pour cela, il modélise le nombre de clients par une fonction donnant le nombre de clients $c\left(t\right)$ en fonction du temps $t$, exprimé en jours. On prends donc $t\in\left[0;365\right]$.

\begin{QCM}

En remarquant que le nombre de clients ne peut être négatif, déterminé, parmi les propositions ci-dessous, laquelle donne la formule de $c\left(t\right)$ en fonction de $t$ :

\begin{enumerate}

\Faux{$c\left(t\right)=5\sin\left(\frac{4\pi t}{365}\right)-29$}

\Faux{$c\left(t\right)=5\cos\left(\frac{4\pi t}{365}-29\right)$}

\Faux{$c\left(t\right)=5\sin\left(\frac{4\pi t}{365}+29\right)$}

\Vrai{$c\left(t\right)=5\cos\left(\frac{4\pi t}{365}\right)+29$}

\end{enumerate}

\end{QCM}

\begin{QCM}

On a alors que les valeurs de $c\left(t\right)$ sont comprises entre :

\begin{enumerate}

\Faux{$0$ et $5$ }

\Faux{$0$ et $34$ }

\Faux{$0$ et $29$}

\Vrai{$24$ et $34$}

\end{enumerate}

\end{QCM}

Dès lors, il peut calculer le nombre moyen de clients en appliquant la formule

$\frac{1}{365}\int\limits _{0}^{365}C\left(t\right)dt$ , où $C$ désigne une primitive de $c$.

\begin{QCM}

Parmi les propositions ci-dessous, laquelle donne une formule possible C~?

\begin{enumerate}

\Vrai{$C\left(t\right)=\frac{1825}{2}\pi\sin\left(\frac{4\pi t}{365}\right)+29t$}

\Faux{$C\left(t\right)=\frac{20}{365}\pi\sin\left(\frac{4\pi t}{365}\right)-29t+20$}

\Faux{$C\left(t\right)=\frac{20}{365}\pi\cos\left(\frac{4\pi t}{365}\right)-29t-43$}

\Faux{$C\left(t\right)=-5\sin\left(\frac{4\pi t}{365}\right)$}

\end{enumerate}

\end{QCM}

\begin{QCM}

On obtient ainsi que le nombre moyen de clients journaliers sur l'année en cours est donné par :

\begin{enumerate}

\Faux{$5$}

\Faux{$5\cos\left(4\pi\right)$}

\Vrai{$29$}

\Faux{$29\sin\left(4\pi\right)$ }

\end{enumerate}

\end{QCM}

\end{Sujet}

\begin{Sujet}

\subsection*{Exercice 1}

Pour chacune des propositions de cet exercice, cochez A si elle est vraie\par et B si elle est fausse.

On s'intéresse à la fonction f définie par $f:x\mapsto\left(6x+2\right)e^{2x}$.

\begin{VraiFaux}

\Faux{ L'ensemble de définition de f est $\left]0,+\infty\right[$.}

\end{VraiFaux}

\begin{VraiFaux}

\Faux{ La dérivée de f est donnée par $f^{\prime}:x\mapsto\left(12\right)e^{cx}$.}

\end{VraiFaux}

On admet que son approximation quadratique est donné par\par $\qquad f\left(x\right)\approx 2 + 18 x + 72 x^2$.

\begin{VraiFaux}

\Faux{ L'équation de la tangente $T$ à la courbe représentative $\mathscr C$ de $f$ en son point d'abscisse $0$ est donnée par $y=2x + 18 $.}

\end{VraiFaux}

\begin{VraiFaux}

\Vrai{ $T$ est au-dessous de $\mathscr C$.}

\end{VraiFaux}

\subsection*{Exercice 2}

Cet exercice doit être rédigé au verso de la grille réponse.

\subsection*{Exercice 3}

\begin{QCM}

Parmi ces fonctions, laquelle est une primitive de la fonction $f$\par définie par $ f(x)=x \cdot \cos (3 x+1)$ ?

\begin{enumerate}

\Vrai{ $H:x \mapsto \frac{x \cdot \sin (3 x+1)}{3}+\frac{\cos (3 x+1)}{9}$}

\Faux{ $G:x \mapsto x \cdot \sin (3 x+1)$}

\Faux{ $F:x \mapsto \cos (3 x+1)-3 x \cdot \sin (3 x+1)$}

\end{enumerate}

\end{QCM}

\begin{QCM}

$\int\limits_{0}^{2}\left(5 t^{3}+3 t^{2}-t+7\right) d t=$

\begin{enumerate}

\Faux{ $15 t^{2}+6 t-1$}

\Faux{ 71}

\Vrai{ 40}

\end{enumerate}

\end{QCM}

\begin{QCM}

La valeur moyenne de la fonction $g: x \rightarrow e^{5 x} $ entre 0 et 2 est de :

\begin{enumerate}

\Faux{environ $22 025,5$}

\Faux{environ $4405,1$}

\Vrai{environ $2 202,5$}

\end{enumerate}

\end{QCM}

\newpage

\subsection*{Problème}

Alain Raffletou cherche à connaitre le nombre moyen de clients journaliers dans une de ses boutiques sur l'année en cours. Pour cela, il modélise le nombre de clients par une fonction donnant le nombre de clients $c\left(t\right)$ en fonction du temps $t$, exprimé en jours. On prends donc $t\in\left[0;365\right]$.

\begin{QCM}

En remarquant que le nombre de clients ne peut être négatif, déterminé, parmi les propositions ci-dessous, laquelle donne la formule de $c\left(t\right)$ en fonction de $t$ :

\begin{enumerate}

\Vrai{$c\left(t\right)=3\cos\left(\frac{8\pi t}{365}\right)+17$}

\Faux{$c\left(t\right)=3\cos\left(\frac{8\pi t}{365}-17\right)$}

\Faux{$c\left(t\right)=3\sin\left(\frac{8\pi t}{365}+17\right)$}

\Faux{$c\left(t\right)=3\sin\left(\frac{8\pi t}{365}\right)-17$}

\end{enumerate}

\end{QCM}

\begin{QCM}

On a alors que les valeurs de $c\left(t\right)$ sont comprises entre :

\begin{enumerate}

\Faux{$0$ et $3$ }

\Faux{$0$ et $17$}

\Vrai{$14$ et $20$}

\Faux{$0$ et $20$ }

\end{enumerate}

\end{QCM}

Dès lors, il peut calculer le nombre moyen de clients en appliquant la formule

$\frac{1}{365}\int\limits _{0}^{365}C\left(t\right)dt$ , où $C$ désigne une primitive de $c$.

\begin{QCM}

Parmi les propositions ci-dessous, laquelle donne une formule possible C~?

\begin{enumerate}

\Faux{$C\left(t\right)=\frac{24}{365}\pi\sin\left(\frac{8\pi t}{365}\right)-17t+12$}

\Vrai{$C\left(t\right)=\frac{1095}{2}\pi\sin\left(\frac{8\pi t}{365}\right)+17t$}

\Faux{$C\left(t\right)=-3\sin\left(\frac{8\pi t}{365}\right)$}

\Faux{$C\left(t\right)=\frac{24}{365}\pi\cos\left(\frac{8\pi t}{365}\right)-17t-25$}

\end{enumerate}

\end{QCM}

\begin{QCM}

On obtient ainsi que le nombre moyen de clients journaliers sur l'année en cours est donné par :

\begin{enumerate}

\Faux{$3$}

\Vrai{$17$}

\Faux{$17\sin\left(8\pi\right)$ }

\Faux{$3\cos\left(8\pi\right)$}

\end{enumerate}

\end{QCM}

\end{Sujet}

\begin{Sujet}

\subsection*{Exercice 1}

Pour chacune des propositions de cet exercice, cochez A si elle est vraie\par et B si elle est fausse.

On s'intéresse à la fonction f définie par $f:x\mapsto\left(7x+4\right)e^{7x}$.

\begin{VraiFaux}

\Faux{ L'ensemble de définition de f est $\left]0,+\infty\right[$.}

\end{VraiFaux}

\begin{VraiFaux}

\Faux{ La dérivée de f est donnée par $f^{\prime}:x\mapsto\left(49\right)e^{cx}$.}

\end{VraiFaux}

On admet que son approximation quadratique est donné par\par $\qquad f\left(x\right)\approx 4 + 35 x + 147 x^2$.

\begin{VraiFaux}

\Faux{ L'équation de la tangente $T$ à la courbe représentative $\mathscr C$ de $f$ en son point d'abscisse $0$ est donnée par $y=4x + 35 $.}

\end{VraiFaux}

\begin{VraiFaux}

\Faux{ $T$ est au-dessus de $\mathscr C$.}

\end{VraiFaux}

\subsection*{Exercice 2}

Cet exercice doit être rédigé au verso de la grille réponse.

\subsection*{Exercice 3}

\begin{QCM}

Parmi ces fonctions, laquelle est une primitive de la fonction $f$\par définie par $ f(x)=x \cdot \cos (3 x+1)$ ?

\begin{enumerate}

\Faux{ $F:x \mapsto \cos (3 x+1)-3 x \cdot \sin (3 x+1)$}

\Vrai{ $H:x \mapsto \frac{x \cdot \sin (3 x+1)}{3}+\frac{\cos (3 x+1)}{9}$}

\Faux{ $G:x \mapsto x \cdot \sin (3 x+1)$}

\end{enumerate}

\end{QCM}

\begin{QCM}

$\int\limits_{0}^{2}\left(5 t^{3}+3 t^{2}-t+7\right) d t=$

\begin{enumerate}

\Faux{ 71}

\Vrai{ 40}

\Faux{ $15 t^{2}+6 t-1$}

\end{enumerate}

\end{QCM}

\begin{QCM}

La valeur moyenne de la fonction $g: x \rightarrow e^{5 x} $ entre 0 et 2 est de :

\begin{enumerate}

\Vrai{environ $2 202,5$}

\Faux{environ $22 025,5$}

\Faux{environ $4405,1$}

\end{enumerate}

\end{QCM}

\newpage

\subsection*{Problème}

Alain Raffletou cherche à connaitre le nombre moyen de clients journaliers dans une de ses boutiques sur l'année en cours. Pour cela, il modélise le nombre de clients par une fonction donnant le nombre de clients $c\left(t\right)$ en fonction du temps $t$, exprimé en jours. On prends donc $t\in\left[0;365\right]$.

\begin{QCM}

En remarquant que le nombre de clients ne peut être négatif, déterminé, parmi les propositions ci-dessous, laquelle donne la formule de $c\left(t\right)$ en fonction de $t$ :

\begin{enumerate}

\Faux{$c\left(t\right)=3\cos\left(\frac{8\pi t}{365}-24\right)$}

\Vrai{$c\left(t\right)=3\cos\left(\frac{8\pi t}{365}\right)+24$}

\Faux{$c\left(t\right)=3\sin\left(\frac{8\pi t}{365}+24\right)$}

\Faux{$c\left(t\right)=3\sin\left(\frac{8\pi t}{365}\right)-24$}

\end{enumerate}

\end{QCM}

\begin{QCM}

On a alors que les valeurs de $c\left(t\right)$ sont comprises entre :

\begin{enumerate}

\Faux{$0$ et $27$ }

\Vrai{$21$ et $27$}

\Faux{$0$ et $3$ }

\Faux{$0$ et $24$}

\end{enumerate}

\end{QCM}

Dès lors, il peut calculer le nombre moyen de clients en appliquant la formule

$\frac{1}{365}\int\limits _{0}^{365}C\left(t\right)dt$ , où $C$ désigne une primitive de $c$.

\begin{QCM}

Parmi les propositions ci-dessous, laquelle donne une formule possible C~?

\begin{enumerate}

\Faux{$C\left(t\right)=-3\sin\left(\frac{8\pi t}{365}\right)$}

\Faux{$C\left(t\right)=\frac{24}{365}\pi\sin\left(\frac{8\pi t}{365}\right)-24t+19$}

\Faux{$C\left(t\right)=\frac{24}{365}\pi\cos\left(\frac{8\pi t}{365}\right)-24t-39$}

\Vrai{$C\left(t\right)=\frac{1095}{2}\pi\sin\left(\frac{8\pi t}{365}\right)+24t$}

\end{enumerate}

\end{QCM}

\begin{QCM}

On obtient ainsi que le nombre moyen de clients journaliers sur l'année en cours est donné par :

\begin{enumerate}

\Faux{$3$}

\Faux{$24\sin\left(8\pi\right)$ }

\Faux{$3\cos\left(8\pi\right)$}

\Vrai{$24$}

\end{enumerate}

\end{QCM}

\end{Sujet}

\begin{Sujet}

\subsection*{Exercice 1}

Pour chacune des propositions de cet exercice, cochez A si elle est vraie\par et B si elle est fausse.

On s'intéresse à la fonction f définie par $f:x\mapsto\left(7x+4\right)e^{3x}$.

\begin{VraiFaux}

\Faux{ L'ensemble de définition de f est $\left]0,+\infty\right[$.}

\end{VraiFaux}

\begin{VraiFaux}

\Vrai{ La dérivée de f est donnée par $f^{\prime}:x\mapsto\left(21x+19\right)e^{3x}$.}

\end{VraiFaux}

On admet que son approximation quadratique est donné par\par $\qquad f\left(x\right)\approx 4 + 35 x + 147 x^2$.

\begin{VraiFaux}

\Faux{ L'équation de la tangente $T$ à la courbe représentative $\mathscr C$ de $f$ en son point d'abscisse $0$ est donnée par $y=4x + 35 $.}

\end{VraiFaux}

\begin{VraiFaux}

\Vrai{ $T$ est au-dessous de $\mathscr C$.}

\end{VraiFaux}

\subsection*{Exercice 2}

Cet exercice doit être rédigé au verso de la grille réponse.

\subsection*{Exercice 3}

\begin{QCM}

Parmi ces fonctions, laquelle est une primitive de la fonction $f$\par définie par $ f(x)=x \cdot \cos (3 x+1)$ ?

\begin{enumerate}

\Vrai{ $H:x \mapsto \frac{x \cdot \sin (3 x+1)}{3}+\frac{\cos (3 x+1)}{9}$}

\Faux{ $G:x \mapsto x \cdot \sin (3 x+1)$}

\Faux{ $F:x \mapsto \cos (3 x+1)-3 x \cdot \sin (3 x+1)$}

\end{enumerate}

\end{QCM}

\begin{QCM}

$\int\limits_{0}^{2}\left(5 t^{3}+3 t^{2}-t+7\right) d t=$

\begin{enumerate}

\Faux{ 71}

\Faux{ $15 t^{2}+6 t-1$}

\Vrai{ 40}

\end{enumerate}

\end{QCM}

\begin{QCM}

La valeur moyenne de la fonction $g: x \rightarrow e^{5 x} $ entre 0 et 2 est de :

\begin{enumerate}

\Faux{environ $4405,1$}

\Faux{environ $22 025,5$}

\Vrai{environ $2 202,5$}

\end{enumerate}

\end{QCM}

\newpage

\subsection*{Problème}

Alain Raffletou cherche à connaitre le nombre moyen de clients journaliers dans une de ses boutiques sur l'année en cours. Pour cela, il modélise le nombre de clients par une fonction donnant le nombre de clients $c\left(t\right)$ en fonction du temps $t$, exprimé en jours. On prends donc $t\in\left[0;365\right]$.

\begin{QCM}

En remarquant que le nombre de clients ne peut être négatif, déterminé, parmi les propositions ci-dessous, laquelle donne la formule de $c\left(t\right)$ en fonction de $t$ :

\begin{enumerate}

\Vrai{$c\left(t\right)=3\cos\left(\frac{8\pi t}{365}\right)+15$}

\Faux{$c\left(t\right)=3\sin\left(\frac{8\pi t}{365}\right)-15$}

\Faux{$c\left(t\right)=3\sin\left(\frac{8\pi t}{365}+15\right)$}

\Faux{$c\left(t\right)=3\cos\left(\frac{8\pi t}{365}-15\right)$}

\end{enumerate}

\end{QCM}

\begin{QCM}

On a alors que les valeurs de $c\left(t\right)$ sont comprises entre :

\begin{enumerate}

\Faux{$0$ et $3$ }

\Faux{$0$ et $15$}

\Faux{$0$ et $18$ }

\Vrai{$12$ et $18$}

\end{enumerate}

\end{QCM}

Dès lors, il peut calculer le nombre moyen de clients en appliquant la formule

$\frac{1}{365}\int\limits _{0}^{365}C\left(t\right)dt$ , où $C$ désigne une primitive de $c$.

\begin{QCM}

Parmi les propositions ci-dessous, laquelle donne une formule possible C~?

\begin{enumerate}

\Faux{$C\left(t\right)=\frac{24}{365}\pi\sin\left(\frac{8\pi t}{365}\right)-15t+10$}

\Faux{$C\left(t\right)=\frac{24}{365}\pi\cos\left(\frac{8\pi t}{365}\right)-15t-21$}

\Vrai{$C\left(t\right)=\frac{1095}{2}\pi\sin\left(\frac{8\pi t}{365}\right)+15t$}

\Faux{$C\left(t\right)=-3\sin\left(\frac{8\pi t}{365}\right)$}

\end{enumerate}

\end{QCM}

\begin{QCM}

On obtient ainsi que le nombre moyen de clients journaliers sur l'année en cours est donné par :

\begin{enumerate}

\Faux{$3$}

\Faux{$3\cos\left(8\pi\right)$}

\Faux{$15\sin\left(8\pi\right)$ }

\Vrai{$15$}

\end{enumerate}

\end{QCM}

\end{Sujet}

\begin{Sujet}

\subsection*{Exercice 1}

Pour chacune des propositions de cet exercice, cochez A si elle est vraie\par et B si elle est fausse.

On s'intéresse à la fonction f définie par $f:x\mapsto\left(6x+2\right)e^{3x}$.

\begin{VraiFaux}

\Vrai{ L'ensemble de définition de f est $\left]-\infty,+\infty\right[$.}

\end{VraiFaux}

\begin{VraiFaux}

\Faux{ La dérivée de f est donnée par $f^{\prime}:x\mapsto\left(18\right)e^{cx}$.}

\end{VraiFaux}

On admet que son approximation quadratique est donné par\par $\qquad f\left(x\right)\approx 2 + 18 x + 72 x^2$.

\begin{VraiFaux}

\Faux{ L'équation de la tangente $T$ à la courbe représentative $\mathscr C$ de $f$ en son point d'abscisse $0$ est donnée par $y=2x + 18 $.}

\end{VraiFaux}

\begin{VraiFaux}

\Faux{ $T$ est au-dessus de $\mathscr C$.}

\end{VraiFaux}

\subsection*{Exercice 2}

Cet exercice doit être rédigé au verso de la grille réponse.

\subsection*{Exercice 3}

\begin{QCM}

Parmi ces fonctions, laquelle est une primitive de la fonction $f$\par définie par $ f(x)=x \cdot \cos (3 x+1)$ ?

\begin{enumerate}

\Faux{ $F:x \mapsto \cos (3 x+1)-3 x \cdot \sin (3 x+1)$}

\Faux{ $G:x \mapsto x \cdot \sin (3 x+1)$}

\Vrai{ $H:x \mapsto \frac{x \cdot \sin (3 x+1)}{3}+\frac{\cos (3 x+1)}{9}$}

\end{enumerate}

\end{QCM}

\begin{QCM}

$\int\limits_{0}^{2}\left(5 t^{3}+3 t^{2}-t+7\right) d t=$

\begin{enumerate}

\Faux{ $15 t^{2}+6 t-1$}

\Vrai{ 40}

\Faux{ 71}

\end{enumerate}

\end{QCM}

\begin{QCM}

La valeur moyenne de la fonction $g: x \rightarrow e^{5 x} $ entre 0 et 2 est de :

\begin{enumerate}

\Faux{environ $22 025,5$}

\Vrai{environ $2 202,5$}

\Faux{environ $4405,1$}

\end{enumerate}

\end{QCM}

\newpage

\subsection*{Problème}

Alain Raffletou cherche à connaitre le nombre moyen de clients journaliers dans une de ses boutiques sur l'année en cours. Pour cela, il modélise le nombre de clients par une fonction donnant le nombre de clients $c\left(t\right)$ en fonction du temps $t$, exprimé en jours. On prends donc $t\in\left[0;365\right]$.

\begin{QCM}

En remarquant que le nombre de clients ne peut être négatif, déterminé, parmi les propositions ci-dessous, laquelle donne la formule de $c\left(t\right)$ en fonction de $t$ :

\begin{enumerate}

\Vrai{$c\left(t\right)=3\cos\left(\frac{6\pi t}{365}\right)+22$}

\Faux{$c\left(t\right)=3\sin\left(\frac{6\pi t}{365}+22\right)$}

\Faux{$c\left(t\right)=3\cos\left(\frac{6\pi t}{365}-22\right)$}

\Faux{$c\left(t\right)=3\sin\left(\frac{6\pi t}{365}\right)-22$}

\end{enumerate}

\end{QCM}

\begin{QCM}

On a alors que les valeurs de $c\left(t\right)$ sont comprises entre :

\begin{enumerate}

\Faux{$0$ et $25$ }

\Faux{$0$ et $22$}

\Vrai{$19$ et $25$}

\Faux{$0$ et $3$ }

\end{enumerate}

\end{QCM}

Dès lors, il peut calculer le nombre moyen de clients en appliquant la formule

$\frac{1}{365}\int\limits _{0}^{365}C\left(t\right)dt$ , où $C$ désigne une primitive de $c$.

\begin{QCM}

Parmi les propositions ci-dessous, laquelle donne une formule possible C~?

\begin{enumerate}

\Vrai{$C\left(t\right)=\frac{1095}{2}\pi\sin\left(\frac{6\pi t}{365}\right)+22t$}

\Faux{$C\left(t\right)=\frac{18}{365}\pi\sin\left(\frac{6\pi t}{365}\right)-22t+17$}

\Faux{$C\left(t\right)=\frac{18}{365}\pi\cos\left(\frac{6\pi t}{365}\right)-22t-35$}

\Faux{$C\left(t\right)=-3\sin\left(\frac{6\pi t}{365}\right)$}

\end{enumerate}

\end{QCM}

\begin{QCM}

On obtient ainsi que le nombre moyen de clients journaliers sur l'année en cours est donné par :

\begin{enumerate}

\Faux{$22\sin\left(6\pi\right)$ }

\Faux{$3\cos\left(6\pi\right)$}

\Vrai{$22$}

\Faux{$3$}

\end{enumerate}

\end{QCM}

\end{Sujet}
\end{document}
