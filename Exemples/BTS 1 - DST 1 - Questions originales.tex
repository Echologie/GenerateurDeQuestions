# BTS 1 - DST 1 - Questions originales

a : -2,-3,-4,-5
b : 2,4,6,8
c : 1,3,5,7,9

\item L'ensemble des solutions de l'inéquation $#a# x+#b#<#c#$ est :
\begin{enumerate}
\item\MauvaiseReponse $\mathbb{R}$
\item\BonneReponse $]#(c-b)/a#,+\infty[$
\item\MauvaiseReponse $]-\infty, #(b-c)/a#]$
\end{enumerate}


a : -2,-3,-4,-5
b : 2,4,6,8
c : 1,3,5,7,9

\item L'ensemble des solutions de l'inéquation $#a#x+#b# \geqslant #c#$
\begin{enumerate}
\item\MauvaiseReponse $\mathbb{R}$
\item\MauvaiseReponse $[#(c-b)/a#,+\infty[$
\item\BonneReponse $]-\infty,#(c-b)/a#]$
\end{enumerate}


a : 2,3,4
b : 2,4,6,8
c : 1,3,5,7,9

\item Le nombre $#1/b#$
\begin{enumerate}
\item\MauvaiseReponse est solution de l'équation $#b#x+1=0$
\item\MauvaiseReponse est solution de l'équation $x+#b#=0$
\item\BonneReponse est solution de l'inéquation $#a#x+#c#>0$
\end{enumerate}


a : 3,5,6,7
b : 4,5,6
c : 1,2,3,5

\item Le nombre $\sqrt{#a#}$
\begin{enumerate}
\item\MauvaiseReponse est solution de l'équation $x^{2}+#a#=0$
\item\BonneReponse est solution de l'équation $x^{3}-#a#x=0$
\item\MauvaiseReponse est solution de l'inéquation $-#b#x+#c#>0$
\end{enumerate}




a : 8,9,10,11,12
b : 2,4,6,8
c : 1,3,5,7

\item Le nombre $#1/a#$
\begin{enumerate}
\item\MauvaiseReponse est solution de l'équation $#a-1#x+1=0$
\item\BonneReponse est solution de l'équation $#a#x-1=0$
\item\MauvaiseReponse est solution de l'inéquation $#b#x+#c#<0$
\end{enumerate}


a : -2,-3,-4,2,3,5
b : 2,4,-2,3,-3
c : 3,5
d : 2,5
e : 6,8

\item Le couple solution du système $\left\{\begin{array}{c}#c#x+#d#y=#c*a+d*b# \\ x-#e#y=#a-e*b#\end{array}\right.$ est
\begin{enumerate}
\item\MauvaiseReponse $(#-2*a#;#b#)$
\item\MauvaiseReponse $(#b/2# ; #a#)$
\item\BonneReponse $(#a# ;#b#)$
\end{enumerate}





a : -2,-3,-4,2,3,5
b : 2,4,3

\item $u$ est la suite définie pour tout entier $n \geqslant 1$ par $u_{n}=\frac{#a#n^{2}-#b#}{n^{2}}$.
\begin{enumerate}
\item\MauvaiseReponse $u_{3}=3$
\item\BonneReponse $u_{3}=#(a* 9-b)/9#$
\item\MauvaiseReponse $u_{3} =#((a* 3)^2-b)/9#$
\end{enumerate}


a : 3,5
b : 1,2,3
c : 2,3,4

Soit $\left(u_{n}\right)$ la suite définie par: $u_{0}=#c#$ et, pour tout entier naturel $n$,
\item $u_{n+1}=#a# u_{n}-#b#$, alors :
\begin{enumerate}
\item\MauvaiseReponse $u_{3}=#c#$
\item\MauvaiseReponse $u_{3}=#b/(a-1)+(c-b/(a-1))* a^2#$
\item\BonneReponse $u_{3}=#b/(a-1)+(c-b/(a-1))* a^3#$
\end{enumerate}




a : 3,4,5
b : -1,-2,-3

\item $v$ est la suite définie par $v_{0}=#b#$ et la relation de récurrence $v_{n+1}=#a#n-v_{n}$, alors :
\begin{enumerate}
\item\BonneReponse $v_{2}=#a+b#$
\item\MauvaiseReponse $v_{2}=#b-3#$
\item\MauvaiseReponse $v_{2}=#a+1#$
\end{enumerate}






a : -3/4,-1/2,-2/3,-1/4,-1/3
b : -1,-2,-3,-4,-5

\item On considère la suite arithmétique $\left(u_{n}\right)$ de premier terme $u_{0}=#b#$ et de raison $r=#a#$.
\begin{enumerate}
\item\MauvaiseReponse $u_{3}=#b#$
\item\MauvaiseReponse $u_{3}=#b+2* a#$
\item\BonneReponse $u_{3}=#b+3* a#$
\end{enumerate}




a : 3/4,1/2,2/3,1/4,1/3
b : 5/8,3/7,4/9,2/5

Soit $\left(u_{n}\right)$ une suite géométrique de raison $q=#a#$.
\item Sachant que $u_{3}=#b#$, le premier terme $\mathrm{u}_ {0}$ est :
\begin{enumerate}
\item\BonneReponse $#b*(a)^ -3#$
\item\MauvaiseReponse $#b* (a)^ 3#$
\item\MauvaiseReponse $#a* (b)^ -3#$
\end{enumerate}






a : 2,3,4,5,6,8,9
b : 10,12,15,18

Soit la suite $\left(u_{n}\right)$ définie par $u_{n+1}=#a/b#u_n$.
\item Cette suite est :
\begin{enumerate}
\item\MauvaiseReponse croissante
\item\BonneReponse décroissante
\item\MauvaiseReponse constante
\end{enumerate}

Soit la suite $\left(u_{n}\right)$ définie par $u_{n+1}=#b/a#u_n$.
\item Cette suite est :
\begin{enumerate}
\item\BonneReponse croissante
\item\MauvaiseReponse décroissante
\item\MauvaiseReponse constante
\end{enumerate}


a : 2,4
b : 3,5,8
c : 6,7,9

\item On considère l'équation $#a#x^{2}-#b# x-#c#=0$ alors le discriminant $\Delta$ est égal à :
\begin{enumerate}
\item\BonneReponse $#b^2+4*a*c#$
\item\MauvaiseReponse $#b^2-4*a*c#$
\item\MauvaiseReponse $#-(b^2)+4* c#$
\end{enumerate}



a : 2,3
b : 1,2,3,4,5,6,7,8,9

\item Le nombre de solutions de l'équation $#a#x^{2}-#2*a*b# x+#a* b^2#=0$ est
\begin{enumerate}
\item\MauvaiseReponse $0$
\item\BonneReponse $1$
\item\MauvaiseReponse $2$
\end{enumerate}



a : 2,3,4,5
b : 6,7,8,9

\item L'ensemble des solutions de l'équation $x^{2}+#b-a# x-#a* b#=0$ est :
\begin{enumerate}
\item\MauvaiseReponse $\emptyset$
\item\BonneReponse $\{#a# ;-#b#\}$
\item\MauvaiseReponse $\{#a# ; #b#\}$
\end{enumerate}


